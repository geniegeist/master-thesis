\chapter{Applications}

\section{Negative Hölder Spaces}

In Chapter~\ref{chapter:notation} we introduced the space of locally \( \alpha  \)-Hölder functions \( \mathcal{C}^\alpha \) for positive exponents \( \alpha > 0 \). We drop the adjective \emph{locally} and simply call it the space of \( \alpha \)-Hölder functions. We also say that \( f \) is \( \alpha \)-Hölder continuous if \( f \in \mathcal{C}^\alpha \). 

Our goal is to extend \( \mathcal{C}^\alpha \) to non-positive exponents \( \alpha \leq 0 \). In this case, \( \mathcal{C}^\alpha \) is no longer a space of continuously differentiable functions but a space of \emph{distributions}. Then, we will see that the reconstruction \( \mathcal{R}F \) of a \( \gamma \)-coherent \( F \) germ with negative \( \gamma < 0\) lies in \( \mathcal{C}^\gamma \).

Hölder spaces \( \mathcal{C}^\alpha \) play an important role in stochastic processes. The Hölder exponent \( \alpha \) encodes the ``regularity'' or ``roughness'' of a process. That is the smaller \( \alpha \) gets, the more irregular the process becomes. Take the \emph{Brownian Motion}, which is almost surely locally \( \alpha \)-Hölder continuous for \( \alpha \in (0, \frac{1}{2}) \). In case of \( \alpha = 1 \), we obtain Lipschitz continuity. Hence, Hölder continuity is a generalization of Lipschitz continuity. For \( \alpha > 1 \), the only \( \alpha \)-Hölder continuous functions in \( \mathbb{R} \) are constant functions. Clearly, constant functions are the most regular functions we can think of. 

Let us extend \( \mathcal{C}^\alpha \) to negative exponents \( \alpha < 0 \). It then becomes a space of \emph{distributions}.

\begin{definition}[Negative Hölder Space]
  Let \( \alpha \leq 0 \) and let \( r_\alpha = \min\left\{ n \in \mathbb{N} : r > -\alpha \right\} \). We define the Hölder space \( C^\alpha \) as the space of all distributions \( T \) such that for any compact set \( K \subset \mathbb{R}^d \) there exists \( C < \infty \) with 
  \begin{gather}
    |T(\psi^\epsilon_x)| \leq C \epsilon^\alpha \label{eq:negative-hoelder-spaces} \\
    \text{for all \(x \in K,  \epsilon \in (0,1] \) and \( \psi \in \mathcal{B}_{r_\alpha} \)}. \nonumber 
  \end{gather}
  The semi-norm \( \lVert \cdot \rVert_{\mathcal{C}^\alpha(K)} \) is defined as 
  \begin{align}\label{tennisarm}
    \lVert T \rVert_{\mathcal{C}^\alpha(K)} 
    = \sup_{
      \substack{
        x \in K,\\
        \lambda \in (0,1],\\
        \psi \in \mathcal{B}_{r_\alpha}
      }
    } \frac{|T(\psi^\lambda_x)|}{\lambda^\alpha} \hspace{4em} \forall T \in \mathcal{D}'.
  \end{align}
  Clearly, \( T \in \mathcal{C}^\alpha \iff \lVert T \rVert_{\mathcal{C}^\alpha(K)} < \infty \) for all compact sets \( K \subset \mathbb{R}^d \).
\end{definition}

We present our main result. A distribution \( T \in \mathcal{D}' \) lies in \( \mathcal{C}^\alpha \) if inequality~\eqref{eq:negative-hoelder-spaces} holds for a \emph{single} arbitrary test function \( \varphi \in \mathcal{D} \) rather for all \( \psi \in \mathcal{B}_{r_\alpha} \). This characterization of negative Hölder spaces is obtained from the following theorem, which we prove similarly as in the proof of the Reconstruction Theorem.

\begin{theorem}\label{proposition:spiderman}
  Let \( T \in \mathcal{D}' \). If there exists a set \( K \subset \mathbb{R}^d \) and a test function \( \varphi \in \mathcal{D}, \int \varphi(x) \, \mathrm{d}x \neq 0 \) such that 
  \begin{gather}\label{equation:spiderman}
    \forall x \in \enlarg{K}{2}, \epsilon \in \left\{ 2^{k} \right\}_{k \in \mathbb{N}}: \quad |T(\varphi^\epsilon_x)| \leq \epsilon^\alpha f(\epsilon, x) \\
    \text{for some \( \alpha \leq 0 \) and \( f: (0,1] \times \enlarg{K}{2} \to [0, \infty) \)}, \nonumber
  \end{gather}
  then the above inequality~\eqref{equation:spiderman} also holds for \emph{every} test function \( \psi \in \mathcal{B}_r \) and integer \( r > -\alpha \) in the following sense: for every \( \psi \in \mathcal{B}_r \) and \(r > -\alpha\) there exists a constant such that
  \begin{align*}
    \forall x \in K, \epsilon \in (0,1]: \quad |T(\psi^\epsilon_x)| \leq \{ \mathrm{constant} \} \cdot \epsilon^\alpha \sup_{\substack{\epsilon' \in (0, \epsilon]\\x' \in B(x, 2\epsilon)}} f(\epsilon', x').
  \end{align*}
\end{theorem}

\begin{proof}
  Let \( T, \varphi, K, \alpha \) and \( f \) be as above. Fix an integer \( r > -\alpha \). As in the second step of the proof of the Reconstruction Theorem (see~\eqref{definition:tweakedvarphi}), we define the tweaked test function \( \hat \varphi = T_\varphi \) for \(a = \frac{1}{\int \varphi(x) \, \mathrm{d}x}\) and \(\lambda_i = \frac{2^{-(i+1)}}{1+R_\varphi}\), \(i = 0,\ldots,r-1\). 
  
  We claim that \( \hat \varphi \) satisfies a slightly modified inequality~\eqref{equation:spiderman}, i.e.\ for all \( x \in \enlarg{K}{2}, \epsilon = 2^{-k} \):
  \begin{align}\label{krankenhaushelikopter}
    | T(\hat \varphi^\epsilon_x)| \leq C \epsilon^\alpha \sup\limits_{\epsilon' \in (0, \epsilon]} f(\epsilon', x) \quad \text{ with } \quad C = \frac{e^2 r}{ \int \varphi(x) \, \mathrm{d}x} \left( \frac{2^{-r-1}}{1 + R_\varphi} \right)^\alpha.
  \end{align}
  This part is also very similar to the second step of the proof of the Reconstruction Theorem (see Lemma~\ref{lemma:hat-phi-satisfies-table}).
  Let \( \epsilon \in \left\{ 2^{-k} \right\}_{k \in \mathbb{N}} \) and \( x \in \enlarg{K}{2} \). By definition of \( \hat \varphi \) we have
  \begin{align*} 
    |T( \hat \varphi^\epsilon_x )| 
    \leq \frac{1}{\int \varphi(x) \, \mathrm{d}x} \sum^{r-1}_{i=0} |c_i| |T(\varphi^{\epsilon\lambda_i}_{x})|
    &\overset{\eqref{equation:spiderman}}{\leq} \frac{1}{\int \varphi(x) \, \mathrm{d}x} \sum^{r-1}_{i=0} |c_i| (\epsilon \lambda_i)^{\alpha} f(\epsilon, x) \\
    &\Downarrow \text{Note that \( \frac{2^{-r - 1}}{1 + R_\varphi} \)} \leq \lambda_i \text{ and } |c_i| \leq e^2 \text{ see~\eqref{jsknfjkewfwhiru}} \\
    &\leq \frac{e^2r}{\int \varphi(x) \, \mathrm{d}x}  \left( \frac{2^{-r-1}}{1+R_\varphi} \right)^\alpha \epsilon^{\alpha} f(\epsilon, x)
  \end{align*}
  This proves the claim~\eqref{krankenhaushelikopter}.

  Next, we define our usual mollifier \( \rho = \hat \varphi^2 * \hat \varphi  \) and \( \epsilon_k = 2^{-k} \). Recall from Step 2: Tweaking that we have the crucial property \( \rho^{\epsilon_{k + 1}} - \rho^{\epsilon_k} = \hat \varphi^{\epsilon_k} * \check \varphi^{\epsilon_k} \) with \( \check \varphi = \hat \varphi^{\frac{1}{2}} - \hat \varphi^2 \). This allows us to write \( T(\psi^\lambda_x)  = \lim\limits_{n \to \infty} T (\rho^{\epsilon_n} * \psi^\lambda_x)\). Furthermore, we define \( N = \min\left\{ k \in \mathbb{N} : \epsilon_k \leq \lambda \right\} \). Hence, \( \frac{1}{2} \lambda < \epsilon_N \leq \lambda \). Fix \( x \in K, \lambda \in (0,1] \) and \( \psi \in \mathcal{B}_r \). Then, we write 
  \begin{align*}
    T(\psi^\lambda_x) = \underbrace{T(\rho^{\epsilon_N} * \psi^\lambda_x)}_{=A} + \underbrace{(T(\psi^\lambda_x) - T(\rho^{\epsilon_N} * \psi^\lambda_x))}_{=B}.
  \end{align*}

  \begin{itemize}
    \item We estimate \( A \). We begin with \( A = T(\rho^{\epsilon_N} * \psi^\lambda_x) = \iint T(\hat \varphi^{\epsilon_N}_y) \hat \varphi^{2\epsilon_N}(y-z)\psi^\lambda_x(z) \, \mathrm{d}y \, \mathrm{d}z \), where for the last equality we used Corollary~\ref{cor:minosokoad} twice. The last expression can also be written as a convolution \( \int T(\hat \varphi^{\epsilon_N}_y) (\hat \varphi^{2\epsilon_N} * \psi^\lambda_x) (y) \, \mathrm{d}y \). We apply the first inequality of Lemma~\ref{lemma:WALFANGER} and obtain \( |A| \leq  2^d \lVert \hat \varphi \rVert_{L^1} \sup\limits_{y \in B(x, \lambda + \epsilon_N)}|T(\hat \varphi^{\epsilon_N}_y)| \). Then, we use our claim~\eqref{krankenhaushelikopter} to get (recall that \( \alpha \leq 0, \epsilon_N \leq \lambda, \lambda + 2\epsilon_N \leq 2 \lambda \) and \( \epsilon_N \geq \frac{\lambda}{2} \)):
    \begin{align*}
      |A| \leq 2^d \lVert \hat \varphi \rVert_{L^1}  C \epsilon_N^\alpha \sup_{\substack{x' \in B(x, 2\lambda)\\ \epsilon' \in (0, \lambda ] }} f(\epsilon', x') \leq \left\{ 2^{d - \alpha} C \lVert \hat \varphi \rVert_{L^1}\right \} \lambda^\alpha \sup_{\substack{x' \in B(x, 2\lambda)\\ \epsilon' \in (0, \lambda ] }} f(\epsilon', x').
    \end{align*}

    \item We estimate \( B \). First, we define a sequence \( (B_k)_{k \geq {N}} \) such that \( B = \sum^\infty_{k = N} B_k \). We set \( B_k = T(\rho^{\epsilon_{k+1}} * \psi^\lambda_x) - T (\rho^{\epsilon_k} * \psi^\lambda_x) \overset{\eqref{lemma:mollified-distribution}}{=} \iint T(\hat \varphi^{\epsilon_k}_y) (\check \varphi^{\epsilon_k}(y-z) \psi^\lambda_x(z)) \, \mathrm{d}y \, \mathrm{d}z \). Again, the last expression can be written as a convolution \( \int  T(\hat \varphi^{\epsilon_k}_y) (\check \varphi^{\epsilon_k} * \psi^\lambda_x)(y) \, \mathrm{d}y \) so that we can use Lemma~\ref{lemma:WALFANGER}. Applying the second inequality of Lemma~\ref{lemma:WALFANGER} yields 
    \begin{align*}
      |B_k| \leq 4^d \lVert \check \varphi \rVert_{L^1} \epsilon_k^r \lambda^{-r} \sup_{y \in B(x, \lambda + \epsilon_k)}|T(\hat \varphi^{\epsilon_k}_y)| 
      \overset{\eqref{krankenhaushelikopter}}{\leq}
      4^d \lVert \check \varphi \rVert_{L^1} \epsilon_k^{\alpha + r} \lambda^{-r} C \sup_{\substack{x' \in B(x, 2\lambda)\\ \epsilon' \in (0, \lambda ] }} f(\epsilon', x').
    \end{align*}
    We see that \( \sum^\infty_{k = N} |B_k| < \infty \) because \( \sum^\infty_{k=N} \epsilon_k^{\alpha + r} = \frac{\epsilon_N^{\alpha + r}}{1-2^{-\alpha - r}} \) for \( \alpha + r > 0 \). So, 
    \begin{align*}
      |B| \leq \sum^\infty_{k = N} |b_k| \leq \frac{C 4^d \lVert \check \varphi \rVert_{L^1}}{1-2^{-\alpha -r}}\lambda^{-r} \epsilon_N^{\alpha + r}\sup_{\substack{x' \in B(x, 2\lambda)\\ \epsilon' \in (0, \lambda ] }} f(\epsilon', x').
    \end{align*}
  \end{itemize}

  We are ready to estimate \( T(\psi^\lambda_x) \). Recall that \( \lVert \check \varphi \rVert_{L^1} \leq 2 \lVert \hat \varphi \rVert_{L^1} \). Then, we have
  \begin{align*}
    |T(\psi^\lambda_x)| \leq \left \{\frac{4^{d-\alpha+1}}{1-2^{-\alpha - r}} \lVert \hat \varphi \rVert_{L^1} C \right \}\lambda^\alpha \sup_{\substack{x' \in B(x, 2\lambda)\\ \epsilon' \in (0, \lambda ] }} f(\epsilon', x').
  \end{align*}
  Using~\eqref{krankenhaushelikopter} and~\eqref{tweaked-l1-norm} we explicitly state the constant:
  \begin{align}\label{constant:freiburg}
    \left\{ \frac{4^{d - \alpha + 1}}{1-2^{-\alpha - r}} \left( \frac{e^2r}{|\int \varphi(x) \, \mathrm{d}x|} \right)^2 \left( \frac{2^{-r-1}}{1 + R_\varphi} \right)^\alpha \lVert \varphi \rVert_{L^1} \right\}.
  \end{align}
\end{proof}

As a corollary, we get the following characterization of negative Hölder spaces, where testing for a \emph{single} test function \( \varphi \in \mathcal{D} \) suffices to show that a distribution lies in a negative Hölder space.

\begin{corollary}[Characterization of Negative Hölder Spaces]\label{corollary:negative-hoelder-spaces}
  Let \( \alpha \leq 0 \) and \( T \in \mathcal{D}' \). Then, the following conditions are equivalent 
  \begin{enumerate}
    \item \( T \in \mathcal{C}^\alpha \);
    \item There exists an integer \( r > - \alpha \) such that~\eqref{eq:negative-hoelder-spaces} holds for all test functions \( \psi \in \mathcal{B}_r \);
    \item There exists \( \varphi \in \mathcal{D} \) with \( \int \varphi(x) \, \mathrm{d}x  \neq 0 \) such that for any compact set \( K \subset \mathbb{R}^d \) there exists a constant \( \tilde C < \infty \) with 
    \begin{gather*}
      |T(\varphi^\epsilon_x)| \leq \tilde C \epsilon^\alpha \\
      \text{for all \( x \in K \)and \( \epsilon \in \left\{ 2^{-k} \right\}_{k \in \mathbb{N}} \)}.
    \end{gather*}
  \end{enumerate}
\end{corollary}

\begin{proof}\(  \)

  \begin{itemize}
    \item \( 1. \implies 2. \) This holds because \( \mathcal{B}_r \subset \mathcal{B}_{r_\alpha} \) for \( r \geq r_\alpha \).
    \item \( 2. \implies 3. \) Choose any \( \varphi \in \mathcal{B}_{r} \) with \( \int \varphi(x) \, \mathrm{d}x \).
    \item \( 3. \implies 1. \) Apply Theorem~\ref{proposition:spiderman} with \( f \equiv \tilde C \).
  \end{itemize}
\end{proof}

We further exploit that negative Hölder-continuity is based on a single test function \( \varphi \in \mathcal{D} \). We estimate the \( \lVert \cdot \rVert_{\mathcal{C}^\alpha(K)} \)-norm using \( \varphi \).

\begin{corollary}\label{corollary:c-alpha-norm-estimate}
  Let \( \alpha \leq 0 \) and \( T \in \mathcal{D}' \). Let \( \varphi \in \mathcal{D} \) be the test function as in Corollary~\ref{corollary:negative-hoelder-spaces}. Then, we have for any compact set \( K \subset \mathbb{R}^d \)
  \begin{align*}
    \lVert T \rVert_{\mathcal{C}^\alpha(K)} \leq \{ \mathrm{constant} \} \cdot \sup_{\substack{x \in \enlarg{K}{2} \\ \epsilon \in (0,1]}} 
    \frac{|T(\varphi^{\epsilon}_x)|}{\epsilon^\alpha},
  \end{align*}
  where the constant is given as in~\eqref{constant:freiburg}.
\end{corollary}

\begin{proof}
  By Corollary~\ref{corollary:negative-hoelder-spaces} there exists a test function \( \varphi \in \mathcal{D} \) such that \( |T(\varphi^\epsilon_x)| \leq \tilde C \epsilon^\alpha  \) uniformly for \( x \) in compact sets and \( \epsilon \in \left\{ 2^{-k} \right\}_{k \in \mathbb{N}} \) with some constant \( \tilde C \). Thus, we apply Theorem~\ref{proposition:spiderman} with \( \tilde C = \sup\limits_{\substack{x \in \enlarg{K}{2} \\ \epsilon \in (0,1]}} \frac{|T(\varphi^{\epsilon}_x)|}{\epsilon^\alpha} \), and we obtain

  \begin{align*}
    \lVert T \rVert_{\mathcal{C}^\alpha(K)} 
    = \sup_{
      \substack{
        x \in K,\\
        \epsilon \in (0,1],\\
        \psi \in \mathcal{B}_{r_\alpha}
      }
    } \frac{|T(\psi^\epsilon_x)|}{\epsilon^\alpha}
    \leq 
    \frac{
      \{ \mathrm{constant} \} \cdot \epsilon^\alpha
    }{
      \epsilon^\alpha
    }
    \sup\limits_{
        \substack{
          \epsilon' \in (0, \epsilon]\\
          x' \in B(x, 2\epsilon)
        }
      } f(\epsilon', x') = \{ \mathrm{constant} \} \cdot \sup_{\substack{x \in \enlarg{K}{2} \\ \epsilon \in (0,1]}} 
      \frac{|T(\varphi^{\epsilon}_x)|}{\epsilon^\alpha}.
  \end{align*}
\end{proof}

To the end of this chapter, we prove that  \( \mathcal{R}F \) lies in a negative Hölder-space if the germ \( F \) has \emph{global} homogeneity bound. Local homogeneity bound, that we get for free by the coherence condition (Lemma~\ref{lemma:homogeneity-bound}), does not suffice. This fact is later important to see why the Sewing Lemma is slightly more general than the Reconstruction Theorem.

\begin{theorem}[Reconstruction Theorem and Hölder Spaces]\label{theorem:rec-and-negative-hoelder}
  Let \( F = (F_x)_{x \in \mathbb{R}^d} \) be \( (\alpha, \gamma) \)-coherent germ with global homogeneity bound \( \beta < \gamma \). If \( \beta > 0 \), then \( \mathcal{R}F = 0 \). If \( \beta \leq 0 \), then \( \mathcal{R}F \in \mathcal{C}^\beta \). Additionally, in case of \( \beta \leq 0 \), the reconstruction operator \( \mathcal{R} \) which maps coherent germs \( F \) to their reconstruction \( \mathcal{R}F \) is continuous in the following sense: there exists a constant such that for every compact set \( K \subset \mathbb{R}^d \) the operator \( \mathcal{R} \) satisfies
  \begin{gather}\label{equation:rec-and-negative-hoelder}
    \lVert \mathcal{R}F \rVert_{\mathcal{C}^\beta(K)} \leq \left\{ \mathrm{constant} \right\} \cdot \left( 
        \vertiii{F}^{\mathrm{coh}}_{\enlarg{K}{4}, \varphi, \alpha, \gamma} + \vertiii{F}^{\mathrm{\hom}}_{\enlarg{K}{2}, \varphi, \beta}
     \right) \quad \text{for all germs \( F = (F_x)_{x \in \mathbb{R}^d} \)}.
  \end{gather}
\end{theorem}

\begin{proof}
  Let \( F = (F_x)_{x \in \mathbb{R}^d} \) be a \( (\alpha, \gamma) \)-coherent germ with homogeneity bound \( \beta > 0 \). Then, \( f \equiv 0 \) satisfies \( \lim_{\lambda \to 0} |(f-F_x)(\varphi^\lambda_x)| = 0 \) uniformly for \( x \) in compact sets. Theorem~\ref{theorem:uniqueness-reconstruction} guarantees the uniqueness of a reconstruction \( \mathcal{R}F \). Hence, \( \mathcal{R}F = 0 \).

  Let \( \beta \leq 0 \). Fix a compact set \( K \subset \mathbb{R}^d \). To show that \( \mathcal{R}F \) lies in \( \mathcal{C}^\beta \), it suffices to show~\eqref{equation:rec-and-negative-hoelder} because \( \mathcal{R}F \in \mathcal{C}^\beta \iff \lVert \mathcal{R}F \rVert_{\mathcal{C}^\beta(K)} < \infty \). So, we compute the \( \mathcal{C}^\beta \)-norm by Corollary~\ref{corollary:c-alpha-norm-estimate}
  \begin{align*}
    \lVert \mathcal{R}F \rVert_{\mathcal{C}^\beta(K)}  
    \leq  
    \{ \mathrm{constant} \} \cdot \sup_{\substack{x \in \enlarg{K}{2} \\ \lambda \in (0,1]}} 
    \frac{|\mathcal{R}F(\varphi^{\lambda}_x)|}{\lambda^\beta}.
  \end{align*}
  Let \( f = \mathcal{R}F \) and \( \varphi \) be the test function in the coherence condition. We claim that 
  \begin{align*}
    \sup_{\substack{x \in \enlarg{K}{2} \\ \lambda \in (0,1]}} 
    \frac{|\mathcal{R}F(\varphi^{\lambda}_x)|}{\lambda^\beta} \leq \left\{  \mathrm{constant} \right\} \left( \vertiii{F}^{\mathrm{coh}}_{\enlarg{K}{4}, \varphi, \alpha, \gamma} + \vertiii{F}^{\mathrm{\hom}}_{\enlarg{K}{2}, \varphi, \beta} \right).
  \end{align*}
  We choose \( \bar r = \min\left\{ r \in \mathbb{N} : r > \max\left\{ -\alpha, -\beta \right\}\right\} \), and apply the Reconstruction Theorem (Theorem~\ref{theorem:reconstruction-theorem}) for \( r = \bar r \) and \( K = \enlarg{K}{2} \). Let \( x \in \enlarg{K}{2} \) and \( \lambda \in (0,1] \). Note that \( \varphi \in \mathcal{D} \) does not necessarily lie in \( \mathcal{B}_{\bar r} \). However, it is easy to find parameters \( c, z \) and \( \eta \) such that the test function \( \xi: x \mapsto c \varphi^{\eta}(x - z) \) lies in \( \mathcal{B}_{\bar r} \); the parameter \( c \) scales the \( C^{\bar r} \)-norm to one, and \( \eta \) and \( z \) shift the compact support of \( \varphi \) to \( B(0,1) \). Using the Reconstruction Theorem, we obtain
  \begin{align*}
    |(f - F_x)(\varphi^\lambda_x)| \leq \frac{1}{c}|(f - F_x)(\xi^{\eta^{-1}\lambda}_{x - z})| &\leq \frac{1}{c}(|(f - F_{x- z})(\xi^{\eta^{-1}\lambda}_{x - z})| + |(F_{x -z} - F_x)(\xi^{\eta^{-1}\lambda}_{x - z})|) \\
    &\leq \frac{1}{c}\left\{ \mathrm{constant} \right\} \vertiii{F}^{\mathrm{coh}}_{\enlarg{K}{4}, \varphi, \alpha, \gamma} \begin{cases}
      (\eta^{-1}\lambda)^\gamma  \quad &\text{if \( \gamma \neq 0 \)}\\
      1 + |\log(\eta^{-1}\lambda)| & \text{if \( \gamma = 0 \)}
    \end{cases} \\ 
     &\hspace{1em}+ \frac{1}{c}|(F_{x -z} - F_x)(\xi^{\eta^{-1}\lambda}_{x - z})|
  \end{align*}
  Since \( \eta, z \) and \( c \) only depend on \( \varphi \), we encode \( \eta, z \) and \( c \) in the multiplicative constant factor. The second summand is estimated with the coherence condition. Then, we get
  \begin{align*}
    |(f - F_x)(\varphi^\lambda_x)| \leq \left\{ \mathrm{constant} \right\} \cdot \vertiii{F}^{\mathrm{coh}}_{\enlarg{K}{4}, \varphi, \alpha, \gamma}\begin{cases}
      \lambda^\gamma  \quad &\text{if \( \gamma \neq 0 \)}\\
      1 + |\log(\lambda)| & \text{if \( \gamma = 0 \)}
    \end{cases}.
  \end{align*}
  Next, observe that \( \lambda^\gamma < \lambda^\beta \) because \( \gamma > \beta \). Moreover, a lengthy computation shows \( 1 + |\log(\lambda)| \leq \left\{ -\beta^{-1}e^{-(1 + \beta)} \right\}\lambda^\beta \) for all \( \lambda \in (0,1] \) and \( \beta \leq 0 \), see Theorem 12.7 in~\cite{caravenna2021hairer}. Hence,
  \begin{align*}
    |(f - F_x)(\varphi^\lambda_x)| \leq \left\{ \mathrm{constant} \right\} \cdot (1 + \left\{ -\beta^{-1}e^{-(1 + \beta)}\right \}) \cdot \vertiii{F}^{\mathrm{coh}}_{\enlarg{K}{4}, \varphi, \alpha, \gamma}  \lambda^\beta.
  \end{align*}
  Finally, we use the above estimate and the homogeneity semi-norm~\eqref{definition:semi-norm-homogeneity} to bound
  \begin{align*}
    \sup_{\substack{x \in \enlarg{K}{2} \\ \lambda \in (0,1]}} 
    \frac{|\mathcal{R}F(\varphi^{\lambda}_x)|}{\lambda^\beta} 
    &\leq 
    \sup_{\substack{x \in \enlarg{K}{2} \\ \lambda \in (0,1]}} 
    \frac{|(f - F_x)(\varphi^{\lambda}_x)| + |F_x(\varphi^\lambda_x)|}{\lambda^\beta} \\
    &\leq \left\{ \mathrm{constant} \right\} \vertiii{F}^{\mathrm{coh}}_{\enlarg{K}{4}, \varphi, \alpha, \gamma} + \vertiii{F}^{\mathrm{\hom}}_{\enlarg{K}{2}, \varphi, \beta} .
  \end{align*}
  We proved the theorem.
\end{proof}

It follows \( \mathcal{R}F \in \mathcal{C}^\gamma \) because \( \mathcal{C}^\beta \subset \mathcal{C}^\gamma \) for \( \beta < \gamma \) (to see this note that \( \lambda^\beta < \lambda^\gamma \) for \( \lambda \in (0,1] \)).  Moreover, \( \mathcal{R}F \) is unique up to an element of \( \mathcal{C}^\gamma \) as we show next.

\begin{corollary}[Non uniqueness]
  Let \( F = (F_x)_{x \in \mathbb{R}^d} \) be a \( \gamma \)-coherent germ for some \( \gamma < 0 \). Then, the reconstruction \( \mathcal{R}F \) is unique up to an element of \( \mathcal{C}^\gamma \).
\end{corollary}

\begin{proof}
  Let \( f \) and \( g \) be two reconstructions of a  \( \gamma \)-coherent germ \( F \) with negative \( \gamma < 0 \). Let \( K \subset \mathbb{R}^d \) be a compact set. Then,
  \begin{align*}
    |(f - g)(\psi^\lambda_x)| \leq |(f - F)(\psi^\lambda_x)| + |(F - g)(\psi^\lambda_x)| \leq \left\{ \mathrm{constant} \right\} \lambda^\gamma
  \end{align*}
  holds uniformly for all test functions \( \psi \in \mathcal{B}_r, x \in K \) and \( \lambda \in (0,1] \). Thus, \( f - g \in \mathcal{C}^\gamma \) by Corollary~\ref{corollary:negative-hoelder-spaces}.

  On the other hand let \( g \in \mathcal{C}^\gamma \). Then, \( \mathcal{R}F + g \) is a reconstruction because 
  \begin{align*}
    |(\{\mathcal{R}F + g \} - F_x)(\psi^\lambda_x)| \leq |(\mathcal{R}F - F_x)(\psi^\lambda_x)| + |g(\psi^\lambda_x)| \leq \left\{ \mathrm{constant} \right\} \lambda^\gamma.
  \end{align*}
\end{proof}

To conclude, we defined negative Hölder spaces. We also showed that \( \mathcal{R}F \) lives in a negative Hölder space and is not unique.

\section{Enhanced Coherence and Homogeneity}

We return to the coherence condition and homogeneity bound. In the previous chapter, we saw that it suffices to test for a single test function \( \varphi \in \mathcal{D} \) if we want some distribution to lie in a negative Hölder-space. We will prove something similar for coherence and homogeneity. If a test function \( \varphi \in \mathcal{D} \) satisfies the inequality~\eqref{coherence} in the coherence condition, then so do all test functions in \( \mathcal{B}_r \). Hence, our definition of coherence is quite powerful; testing a germ against a single test function \( \varphi \) allows us to inspect the germ's behavior for the whole class of test functions in \( \mathcal{B}_r \).

\begin{theorem}[Enhanced Coherence]
  Let \( F = (F_x)_{x \in \mathbb{R}^d} \) be an \( (\bm{\alpha}, \gamma) \)-coherent germ. For any compact set \( K \subset \mathbb{R}^d \) and \( r > - \alpha_{\enlarg{K}{2}} \) there exists a constant \( C < \infty \) such that
  \begin{gather*}
    |(F_z - F_y)(\psi^\lambda_y)| \leq C \lambda^{\alpha_{\enlarg{K}{2}}} (|z - y| + \lambda)^{\gamma - \alpha_{\enlarg{K}{2}}}\\
    \text{uniformly for \( z,y \in K, \lambda \in (0,1] \) and \( \psi \in \mathcal{B}_r \).}
  \end{gather*}
\end{theorem}

We stress that our definition of coherence requires only a single test function \( \varphi \in \mathcal{D} \) with \( \int \varphi(x) \, \mathrm{d}x  \neq 0\) to be tested.

\begin{proof}
  The proof is straightforward and uses Theorem~\ref{proposition:spiderman}.

  Let \( K \subset \mathbb{R}^d \) be a compact set and \( z,y \in K \). We choose a third point \( x \in \enlarg{K}{2} \) just for the purpose of applying Theorem~\ref{proposition:spiderman}. Using the triangle inequality, we have for all \( \lambda \in (0,1] \) that
  \begin{align*}
    |(F_z - F_y)(\varphi^\lambda_x)| &\leq |(F_z - F_x)(\varphi^\lambda_x)|  + |(F_x - F_y)(\varphi^\lambda_x)| \\
    &\Downarrow \text{Apply coherence for \( K = \enlarg{K}{2} \)} \\
    &\leq \left\{ \mathrm{constant} \right\} \lambda^{\alpha_{\enlarg{K}{2}}}(|z - x| + |y - x| + \lambda)^{\gamma - \alpha_{\enlarg{K}{2}}}.
  \end{align*}
  The constant only depends on \( K \) and \( \varphi \); this fact is important because the inequality must hold uniformly for points in compact sets. Next, we apply Theorem~\ref{proposition:spiderman} for \( T = F_z - F_y \) and \( f(\lambda, x) = (|z - x| + |y-x| + \lambda)^{\gamma - \alpha_{\enlarg{K}{2}}} \), which yields 
  \begin{gather*}
    |(F_z - F_y)(\psi^\lambda_x)| \leq \left\{ \mathrm{constant} \right\} \lambda^{\alpha_{\enlarg{K}{2}}}(|z - x| + |x-y| + 5\lambda)^{\gamma - \alpha_{\enlarg{K}{2}}} \\
    \text{for any \( r > -\alpha_{\enlarg{K}{2}}, \lambda \in (0,1] \) and \( \psi \in \mathcal{B}_r \)}.
  \end{gather*}
  Note that the multiplicative constant still only depends on \( K \) and \( \varphi \); hence the above inequality holds uniformly for \( x,y \in K, \lambda \in (0,1] \) and \( \psi \in \mathcal{B}_r \). We set \( x = y \), which ends the proof.
\end{proof}

In the proof we replaced \( \alpha_{K} \) by \( \alpha_{\enlarg{K}{2}} \) to prove an enhanced coherence condition. If we replace \( \beta_{K} \) by \( \beta_{\enlarg{K}{2}} \), we obtain an enhanced homogeneity bound.

\begin{theorem}[Enhanced Local Homogeneity]
  Let \( F = (F_x)_{x \in \mathbb{R}^d} \) be an \( (\bm{\alpha}, \gamma) \)-coherent germ with local homogeneity bound \( \bm{\beta} \). For any compact set \( K \subset \mathbb{R}^d \) and \( r > - \max\{ \alpha_{\enlarg{K}{2}}, - \beta_{\enlarg{K}{2}} \} \) there exists a constant \( C < \infty \) such that
  \begin{gather*}
    |F_x(\psi^\lambda_x)| \leq C \lambda^{\beta_{\enlarg{K}{2}}}
    \\
    \text{uniformly for \(x \in K, \lambda \in (0,1] \) and \( \psi \in \mathcal{B}_r \).}
  \end{gather*}
\end{theorem}

\begin{proof}
  Let \( K \subset \mathbb{R}^d \) be a compact set and \( r > - \max\{ \alpha_{\enlarg{K}{2}}, - \beta_{\enlarg{K}{2}} \} \). Using the triangle inequality, we have \( |F_x(\psi^\lambda_x)| \leq |(F_x -\mathcal{R}F)(\psi^\lambda_x)| + |\mathcal{R}F(\psi^\lambda_x)| \). The first summand is estimated with the Reconstruction Theorem, which yields  \( |(F_x -\mathcal{R}F)(\psi^\lambda_x)| \leq \left\{ \mathrm{constant} \right \} \lambda^\gamma \leq \left\{ \mathrm{constant} \right\} \lambda^\beta \) uniformly for \( x \in K, \lambda \in (0,1], \psi \in \mathcal{B}_r \). For the second summand, we use that \( \mathcal{R}F \in \mathcal{C}^\beta \), see~\eqref{eq:negative-hoelder-spaces}. Hence, \( |\mathcal{R}F(\psi^\lambda_x) | \leq \left\{ \mathrm{constant} \right\} \lambda^\beta \)  uniformly for \( x \in K, \lambda \in (0,1], \psi \in \mathcal{B}_r \). This ends the proof.
\end{proof}

Clearly, enhanced coherence implies coherence, and enhanced homogeneity implies homogeneity. At first sight, the converse seems false. However, we proved that our initial definition of coherence and homogeneity offers as much as information as the enhanced versions. The proof was simple and relied on the Reconstruction Theorem.

\section{Sewing Lemma}\label{section:sewing-lemma}

The Sewing Lemma is closely related to the Reconstruction Theorem; often it is called the \( 1 \)-dimensional analogue of the Reconstruction Theorem. Originally, Gubinelli introduced the Sewing Lemma~\cite{gubinelli2004controlling} to study rough paths.

Our goal is to prove the Sewing Lemma with the Reconstruction Theorem. We closely follow the proof of Broux and Zambotti~\cite[Chapter 5]{broux2021sewing}. We will see that the Sewing Lemma is slightly more general than the Reconstruction Theorem in the one-dimensional setting \( \mathbb{R} \). That is, it is no hurdle to show that Sewing implies the Reconstruction Theorem. Proving the converse is nontrivial without additional assumptions. We need to assume a \emph{global homogeneity bound}. To date, a proof that the Sewing Lemma implies the Reconstruction Theorem without any additional assumption is still missing. 

Yet, the Sewing Lemma has a drawback. It is not applicable in the multidimensional setting \( \mathbb{R}^d \) in contrast to the Reconstruction Theorem. Here, the distributional view begins to shine.

We state the Sewing Lemma. Recall that in Chapter~\ref{chapter:first-peek-at-reconstruction}, we have already introduced the Sewing Lemma by Lemma~\ref{first-sewing-lemma}. We were trying to find a suitable assumption for the Reconstruction Theorem, which ultimately led to coherence.

\begin{lemma}[Sewing Lemma for \( \gamma > 1 \),~\cite{broux2021sewing}]
  Let \(\gamma > 1\). Define \( \Delta = \left \{ (s,t) :  0 \leq s \leq t \leq T\right \} \) for some fixed \(T > 0\). Let \(A: \Delta \to \mathbb{R}\) be a continuous function such that there exists \( C <\infty \) with
  \begin{gather*}
      \delta A_{s,u,t} \coloneqq |A_{s,t} - A_{s,u} - A_{u,t}| \leq C  {|t-s|}^\gamma \\
      \text{uniformly for \(0 \leq s \leq u \leq t \leq T\)}. \nonumber
  \end{gather*} 
  Then, there exists a unique function \(I: [0,T] \to \mathbb{R}\) and \(\tilde C < {\infty}\)  such that \(I_0 = 0\) and 
  \begin{gather*}
      |I_t - I_s - A_{s,t}| \leq \tilde C|t-s|^\gamma \\
      \text{uniformly for \(0 \leq s \leq t \leq T\).}
  \end{gather*}  
  Furthermore, \(I\) is the limit of Riemann-type sums. That is 
  \begin{align*}
    I_t =  \lim\limits_{|\pi| \to 0} \sum\limits_{i=0}^{\# \pi - 1} A_{t_i,t_{i+1}}
  \end{align*}
  where the limit is taken over partitions \( \pi \) of \( [0,T] \).
\end{lemma}

This Sewing Lemma holds for \( \gamma > 1 \). Recently, the Sewing Lemma was extended to \( \gamma \in (0,1] \) by Broux and Zambotti~\cite{broux2021sewing}. 

\begin{lemma}[Sewing Lemma for \( 0 < \gamma \leq 1  \),~\cite{broux2021sewing}]
  Let \(0 < \gamma \leq 1\). Let \( A: \Delta \to \mathbb{R} \) be a continuous function that satisfies 
  \begin{gather*}
      \delta A_{s,u,t} \leq C {|t-s|}^\gamma\\
      \text{uniformly for \(0 \leq s \leq u \leq t \leq T\)}. \nonumber
  \end{gather*} 
  Then, there exists a (non-unique) function \(I: [0,T] \to \mathbb{R}\) and \(\tilde C < {\infty}\)  such that \(I_0 = 0\) and 
  \begin{gather*}
      |I_t - I_s - A_{s,t}| \leq \tilde C \begin{cases}
        |t-s|^\gamma \quad & \text{if \( 0 < \gamma < 1, \)}\\
        |t-s|(1 + |\log(|t-s|)|) & \text{if \( \gamma = 1 \)},
      \end{cases}\\
      \text{uniformly for \( 0 \leq s \leq t \leq T \)}.
  \end{gather*}  
\end{lemma}

We are going to prove both Sewing Lemmas using the Reconstruction Theorem. Let's briefly discuss the idea of the proof.

\begin{itemize}
  \item Given a family of \( (A_{s,t}) \) we construct a coherent germ \( F = (F_x)_{x \in \mathbb{R}^d} \).
  \item We apply the Reconstruction Theorem on \( F \) and obtain \( \mathcal{R}F \).
  \item We find a primitive \( I \) of \( \mathcal{R}F \). That is we find a function \( I \) such that \( I' = \mathcal{R}F \). Then, \( I \) is our wanted function.
\end{itemize}

Finding a primitive \( I \) is nontrivial if we only assume \( A_{s,t} \) to satisfy the sewing condition
\begin{gather}\label{sewing-condition-2}
  \delta A_{s,u,t} \leq C {|t-s|}^\gamma \\
  \text{uniformly for \(0 \leq s \leq u \leq t \leq T\)}. \nonumber
\end{gather}
However, if we further assume that \( A_{s,t} \) satisfies a \emph{global} homogeneity bound \( \beta \), that is
\begin{align*}
  |A_{s,t}| \leq \{ \mathrm{constant} \} \cdot |t-s|^\beta,
\end{align*}
then the constructed germ \( F \) also has a \emph{global} homogeneity bound \( \beta - 1 \). By Theorem~\ref{theorem:rec-and-negative-hoelder}, the reconstruction \( \mathcal{R}F \) is in \( \mathcal{C}^{\beta - 1} \). Then, there exists \( I \in \mathcal{C}^\beta \) such that \( I_0 = 0 \) and \( I' = \mathcal{R}F \). This is a well-known fact, see~\cite[Lemma~3.10]{brault2019solving}. In other words, the global homogeneity bound gives us enough information about the regularity of \( \mathcal{R}F \); then knowing the regularity allows us to claim the existence of a primitive of \( \mathcal{R}F \). 

We emphasize that the constructed germ must have a \emph{global} homogeneity bound. It trivially has a \emph{local} homogeneity bound by Lemma~\ref{lemma:homogeneity-bound} but not necessarily a global bound.

We begin with a naive proof of the Sewing Lemma for \( \gamma > 1 \) and \( 0 < \gamma \leq 1 \). Assume \( A \in C([0,1] \times [0,1]) \) is a continuous function satisfying the sewing condition~\eqref{sewing-condition-2}. Let \( \gamma > 0 \). First, we extend \( A \) to the entire domain \( \mathbb{R}^2 \). We set \
\begin{align*}
  p(s) &= \max\left\{ 0, \min\left\{ s, 1 \right\} \right\}  \\ A_{s,t} &= A_{p(s), p(t)}.
\end{align*}
Next, we define a germ \( F = (F_s)_{s \in \mathbb{R}^d} \) by differentiating \( A_{\cdot, \cdot} \) with respect to its second variable; here we mean the distributional derivative. So, 
\begin{align*}
  F_s(\varphi) = - \int_{\mathbb{R}^d} A_{s,t} \varphi'(t) \, \mathrm{d}t \qquad \forall \varphi \in \mathcal{D}(\mathbb{R}^d).
\end{align*}
We claim that \( F = (F_s)_{s \in \mathbb{R}^d} \) is a \( (-1, \gamma - 1) \)-coherent germ.

\begin{proof}
  Fix any test function \( \varphi \in \mathcal{D}(\mathbb{R}^d) \). Let \( s,u,t \in \mathbb{R} \) and \( \lambda \in (0,1] \). Then,
  \begin{align*}
    |(F_t - F_s)(\varphi^\lambda_s)| 
    &= \left |-\lambda^{-1} \int_{\mathbb{R}^d} A_{p(t), p(x)} \varphi'^{\lambda}_s(x) \, \mathrm{d}x + \lambda^{-1}\int_{\mathbb{R}^d} A_{p(s), p(x)} \varphi'^{\lambda}_s(x) \, \mathrm{d}x \right | \\
    &= \left | \lambda^{-1} \int_{\mathbb{R}^d} (A_{p(t), p(x)} - A_{p(s), p(x)}) \varphi'^\lambda_s(x) \, \mathrm{d}x \right | \\
    &\Downarrow \text{Substitution \( v = \frac{x -s}{\lambda} \)} \\
    &= \lambda^{-1} \left | \int_{\mathbb{R}^d} \delta A_{p(t), p(s), p(s + \lambda v)} \varphi'(v) \, \mathrm{d}v \right |.
  \end{align*}
  For the last equation we also used that \( \int_{\mathbb{R}^d} A_{p(t), p(s)} \varphi'(v) \, \mathrm{d}v = 0 \); this follows from integration by parts. Since \( \varphi \in \mathcal{D}(\mathbb{R}^d) \), the integral \( |\int \varphi'(v) \, \mathrm{d}v |  \) is bounded from above by some constant. So, we use the sewing condition~\eqref{sewing-lemma-condition} to estimate 
  \begin{align*}
    |(F_t - F_s)(\varphi^\lambda_s)|  &\leq \left\{ \mathrm{constant} \right\} \lambda^{-1}\left( 
      \max\left\{ 
        |p(s + \lambda v) - p(s)|,
        |p(s) - p(t)|
       \right\}
     \right)^\gamma \\
     &\Downarrow \text{\( p \) is Lipschitz-continuous} \\
     &\leq \left\{ \mathrm{constant} \right\} \lambda^{-1}(\lambda + |t-s|)^\gamma. 
  \end{align*}
  This proves that \( F \) is a \( (-1, \gamma - 1) \)-coherent germ.
\end{proof}

We apply the Reconstruction Theorem. Then, there exists \( \mathcal{R}F \) such that for any compact set \( K \subset \mathbb{R}\) we have 
\begin{gather}
  |(F_s - \mathcal{R}F)(\psi^\lambda_s)| \leq \left\{ \mathrm{constant} \right\} \lambda^{\gamma - 1} \label{starbucks}\\
  \text{uniformly for \( s \in K, \lambda \in (0,1] \) and \( \psi \in \mathcal{B}_{r_K} \)}.\nonumber
\end{gather}
Naively, we assume that there exists a primitive \( I \) of \( \mathcal{R}F \). Then, \( \mathcal{R}F(1_{[s,s + \lambda]}) = -I_{s, s + \lambda} \). Also, we have \( F_s(1_{[s,s + \lambda]}) = -A_{s, s + \lambda} \). Then, we write \( 1_{[s, s + \lambda]}= \lambda (1_{[0,1]})^\lambda_s \) so that we can apply the Reconstruction Theorem:
\begin{align*}
  |(\mathcal{R}F  - F_s)(1_{[s,s + \lambda]})| = |(I - A)_{s, s + \lambda}| = |I_t - I_s - A_{s,t}| \overset{\eqref{starbucks}}{\leq} \left\{ \mathrm{constant} \right\} \lambda^{\gamma}.
\end{align*}
This \emph{would} prove the Sewing Lemma \emph{if} \( 1_{(0,1)} \) were a test function (it is clearly not smooth), and if we knew that a primitive \( I \) exists in the first place. We fix both issues.

\subsection*{Approximating an Indicator Function}

The indicator function \( 1_{(0,1)} \) is not a test function. However, we find smooth functions \( \varphi_n \) and \( \psi_n \) that approximate \( 1_{(0,1)} \).

\begin{lemma}[Dyadic Approximation of Indicator Functions]
  There exist \( \varphi_n, \psi_n \in \mathcal{D} \) for \( n \in \mathbb{N}_0 \) such that 
  \begin{itemize}
    \item \( \mathrm{\sup}(\varphi_n) \subset [\frac{1}{16}2^{-n}, \frac{15}{16}2^{-n}] \) and \( \mathrm{\sup}(\psi_n) \subset [1 - \frac{15}{16}2^{-n}, 1- \frac{1}{16}2^{-n}] \) for all \( n \in \mathbb{N}_0 \),
    \item \( \sup\limits_{n \in \mathbb{N}_0} \sup\limits_{0 \leq k \leq r, k \in \mathbb{N}} \frac{\lVert \partial^k \varphi_n \rVert_{\infty} + \lVert \partial^k \psi_n \rVert_{\infty}}{2^{kn}} < \infty\), and 
    \item \( 1_{0,1} = \sum\limits_{n \geq 0} \varphi_n + \psi_n \).
  \end{itemize}
\end{lemma}

\begin{proof}
  We skip the proof to turn our attention to the interesting part; instead we refer to~\cite[Lemma 5.2]{broux2021sewing}.
\end{proof}

We approximate \( 1_{0,1} \) with the following theorem.

\begin{theorem}
  Assume that \( I \) is a primitive of \( \mathcal{R}F \) in the sense of distributions, and \( I_0 = 0 \). Define for \( n \in \mathbb{N}, s \in [0,T] \) and \( \lambda > 0 \)
  \begin{align*}
    \Delta^N_{s, \lambda} = \sum^N_{n=0}(\mathcal{R}F - F_s)(\lambda(\varphi_n + \psi_n)^\lambda_s) .
  \end{align*}
  Then, we have for all \( s \in \mathbb{R} \) and \( \lambda > 0 \)
  \begin{align*}
    \lim_{N \to \infty} \Delta^N_{s, \lambda} = (I - A)_{s, s + \lambda}.
  \end{align*}
  Additionally, for any compact set \( K \subset \mathbb{R} \) there exists a constant \( C < \infty \) such that 
  \begin{align*}
    |\Delta^N_{s, \lambda}| \leq C \lambda^\gamma
  \end{align*}
  uniformly for \( s \in K, N \in \mathbb{N} \) and \( \lambda \in(0,1] \).
\end{theorem}

We give a proof but omit minor technical details to keep this section concise; again we refer to~\cite{broux2021sewing} for the full proof. 

\begin{proof}[Sketch of proof]
  Let \( N \in \mathbb{N}_0 \). Then, 
  \begin{align*}
    \Delta^N_{s, \lambda} = \sum^N_{n=0} (\mathcal{R}F - F_s)(\lambda(\varphi_n + \psi_n)_s^\lambda) = - \int_{\mathbb{R}} (I_u - A_{s,u}) \sum^N_{n=0} \left( (\varphi_n + \psi_n)' \right)^\lambda_s (u) \, \mathrm{d}u.
  \end{align*}
  We define for \( N \in \mathbb{N} \) and \( u \in \mathbb{R} \)
  \begin{align*}
    \eta_N(u) &= 2^{-N} \sum^N_{n=0}(\varphi_n + \psi_n)'(2^{-N}u )1_{[0,T]}(u) \\
    \tilde \eta_N(u) &= -2^{-N} \sum^N_{n=0}(\varphi_n + \psi_n)'(1 + 2^{-N}u )1_{[-1,0]}(u).
  \end{align*}
  Then, by substitution and rearranging the terms we get
  \begin{align*}
    \sum^N_{n=0}(\varphi_n + \psi_n)' = (\eta_N)^{\frac{1}{2^N}} - (\tilde \eta_N)_1^{\frac{1}{2^N}}.
  \end{align*}
  After a substitution \( \frac{u-s}{\lambda} \leadsto u \), we write 
  \begin{align*}
    \Delta^N_{s,\lambda} = - \int I_{s + \lambda2^{-N}u}\eta_N(u) \, \mathrm{d}u + \int A_{s,s + \lambda2^{-N}u} \eta_N(u) \, \mathrm{d}u \\
    + \int I_{s + \lambda + \lambda2^{-N}u}\tilde \eta_N(u) \, \mathrm{d}u
    - \int  A_{s, s + \lambda + \lambda2^{-N}u} \tilde \eta_N(u) \, \mathrm{d}u 
  \end{align*}
  We show that the first summand converges to \( -I_s \). We have 
  \begin{align*}
    \int I_{s + \lambda2^{-N}u}\eta_N(u) \, \mathrm{d}u = I_s + \int (I_{s + \lambda2^{-N}u} - I_s)\eta_N(u) \, \mathrm{d}u.
  \end{align*}
  By assumption, \( I \) is continuous. Hence, the integrand \( (I_{s + \lambda2^{-N} \cdot} - I_s)\eta_N(\cdot) \to 0 \) converges point wise to \( 0 \) as \( N \to \infty \) if \( \eta_N \) is continuous. If we further assume that \( \eta_N \in \mathcal{D}(B(0,1)) \), then the integrand is bounded by a constant. We can then apply the dominated convergence theorem, which shows that the first summand converges to \( -I_s \).  It remains to prove that \( \eta_N \in \mathcal{D}(B(0,1)) \), for which we use the properties of the dyadic approximation. It is not difficult to show, but we omit this part.

  Similarly, we show that the other three summands converge to \( I_{s + \lambda} \) or \( A_{s, s+\lambda} \), or they vanish. We then obtain 
  \begin{align*}
    \Delta^N_{s, \lambda} = I_{s + \lambda} - I_s - A_{s, s+ \lambda} + o_{N \to \infty}(1).
  \end{align*}
  This proves our first claim.

  Next, we want to bound \( |\Delta^N_{s,\lambda}| \). This is the step where we use the Reconstruction Theorem. We define 
  \begin{align*}
    \eta_n(x) = \varphi_n(2^{-n}x) \quad \text{and} \quad \tilde \eta_n(x) = \psi_n(2^{-n} + 1).
  \end{align*}
  Thus, we have \( (\varphi_n)^\lambda_s = 2^{-n} (\eta_n)_s^{2^{-n}\lambda} \) and \( (\psi_n)^\lambda_s = 2^{-n} (\tilde \eta_n)^{2^{-n}\lambda}_{s + \lambda}\). Writing 
  \begin{align*}
    \Delta^N_{s, \lambda} &= \sum^N_{n=0} (\mathcal{R}F - F_s)(\lambda 2^{-n}(\eta_n)_s^{\lambda^{2^{-n}}}) 
    + \sum^N_{n=0} (\mathcal{R}F - F_{s + \lambda})(\lambda 2^{-n}(\tilde \eta_n)_{s + \lambda}^{\lambda^{2^{-n}}}) \\
     &\qquad + \sum^N_{n=0} (F_{s + \lambda} - F_s)(\lambda 2^{-n}(\tilde \eta_n)_{s + \lambda}^{\lambda^{2^{-n}}}),
  \end{align*}
  we apply the Reconstruction Theorem to estimate the first two summands. The first summand is then bounded from above by \( \leq \left\{ \mathrm{constant} \right\} \sum^N_{n=0}\lambda2^{-n} (\lambda 2^{-n})^{\gamma - 1}  \leq \left\{ \mathrm{constant} \right\} \lambda^\gamma\). Similarly for the second summand we obtain a bound \( \leq \left\{ \mathrm{constant} \right\} \lambda^\gamma \). 

  Finally, it remains to bound the third summand. By definition of \( F_s \) we write 
  \begin{align*}
    \sum^N_{n=0} (F_{s + \lambda} - F_s)(\lambda 2^{-n}(\tilde \eta_n)_{s + \lambda}^{\lambda^{2^{-n}}}) = -\lambda \int_{\mathbb{R}} \delta A_{s + \lambda,s , u} \sum^N_{n=0}2^{-n}\left( (\tilde \eta_n)^{\lambda 2^{-n}}_{s + \lambda} \right)' (u) \, \mathrm{d}u,
  \end{align*}
  where we again used the fact that \( \int A_{s + \lambda, s} \, \tilde \eta'(u) \, \mathrm{d}u \) vanishes by integration by parts. Since \( \tilde \eta_n \in \mathcal{D}(B(0,1)) \), we have \( \mathrm{supp}((\tilde \eta)_n)_{s + \lambda}^{\lambda 2^{-n}} \subset [s + \lambda, s + \lambda + 2^{-n}\lambda] \subset [s + \lambda, s + 2\lambda]\). Hence, 
  \begin{align*}
    |\sum^N_{n=0} (F_{s + \lambda} - F_s)(\lambda 2^{-n}(\tilde \eta_n)_{s + \lambda}^{\lambda^{2^{-n}}})| \leq \left( \sup_{u \in [s + \lambda, s + 2\lambda]} |\delta A_{s + \lambda, s, u} | \right) \int_{\mathbb{R}} |\sum^N_{n=0} \lambda 2^{-n} ((\tilde \eta_n)_{s + \lambda}^{\lambda 2^{-n}})'(u)| \, \mathrm{d}u.
  \end{align*}
  We use the Sewing condition to estimate \( \delta A_{s + \lambda, s, u} \leq (|u-s| + |s - (s + \lambda)|)^\gamma \leq \left\{ \mathrm{constant}  \right\} \lambda^\gamma \).

  To find a constant that bounds the integral \( \int_{\mathbb{R}} |\sum^N_{n=0} \lambda 2^{-n} ((\tilde \eta_n)_{s + \lambda}^{\lambda 2^{-n}})'(u)| \, \mathrm{d}u \) we use the properties of the dyadic approximation. The details are found in~\cite[Proposition 5.3]{broux2021sewing}. This finishes the proof of \( |\Delta^N_{s, \lambda}| \leq \left\{ \mathrm{constant} \right\} \lambda^\gamma \).
\end{proof}

This theorem establishes our wanted function \( I \) in the Sewing Lemma. Hence, we prove the Sewing Lemma with the Reconstruction Theorem \emph{as long as} there exists a primitive \( I \) of \( \mathcal{R}F \).

\section*{Existence of a Primitive}

We assume there exists \( 0 < \beta < \min\left\{ 1, \gamma \right\} \) such that there exists a constant \( C < \infty \) with
\begin{align*}
  |A_{s,t} | \leq C |t-s|^\beta
\end{align*}
uniformly for \( s,t \in [0,1] \). We claim that the constructed germ \( F \) has a \emph{global} homogeneity bound \( \beta - 1 \). 

\begin{proof}
  Let \( \varphi \in \mathcal{D}(\mathbb{R}) \) be any test function. Let \( s,t \in \mathbb{R} \) and \( \lambda \in (0,1] \). Then, 
  \begin{align*}
    |F_s(\varphi^\lambda_s)| = \lambda^{-1} \int_{\mathbb{R}} |A_{p(s), p(s + \lambda v)} \varphi'(v) \, \mathrm{d}v| \leq \left\{ \mathrm{constant} \right\} \lambda^{\beta - 1},
  \end{align*}
  where we again used the Lipschitz-continuity of \( p \); see the proof that \( F \) is a \( (-1, \gamma - 1) \)-coherent germ at the beginning of this chapter. 
\end{proof}

Hence, \( \mathcal{R}F \in \mathcal{C}^{\beta - 1} \). This proves the existence of a primitive \( I \); that is \( I_0 = 0 \) and \( I' = \mathcal{R}F \). We summarize our insights in a theorem.

\begin{theorem}[Sewing via the Reconstruction Theorem]
  Let \( \beta, \gamma > 0 \) with \( \beta < 1 \). Let \( A \in C([0,1] \times [0,1]) \) be a function that satisfies 
  \begin{align*}
    |\delta A_{s,u,t}| &\leq \left\{ \mathrm{constant} \right\} (\max\left\{  |t-u|, |u-s|\right\})^\gamma,\\
    |A_{s,t}| &\leq \{ \mathrm{constant} \} |t-s|^\beta
  \end{align*}
  uniformly for \( s,u,t \in [0,T] \). Then, there exists \( I \in \mathcal{C}^\beta \) such that 
  \begin{align*}
    |I_t - I_s -A_{s,t}| \leq \left\{ \mathrm{constant} \right\} \begin{cases}
      |t-s|^\gamma \quad & \text{if \( \gamma \neq 1 \)}\\
      |t-s|(1 + |\log(|t - s|)|) &\text{if \( \gamma = 1 \)}
    \end{cases}
  \end{align*}
  uniformly for \( s,t \in [0,T] \).
\end{theorem}