%% thesis.tex 2014/04/11
%
% Based on sample files of unknown authorship.
%
% The Current Maintainer of this work is Paul Vojta.

\documentclass[masters]{ucbthesis}

\usepackage{biblatex}
\usepackage{rotating} % provides sidewaystable and sidewaysfigure

\usepackage[utf8]{inputenc}
\usepackage{amsmath,amsthm,amssymb}

\usepackage{mathtools}
\usepackage{bbm}
\usepackage{marvosym}
\usepackage[hidelinks]{hyperref}
\usepackage{framed}
\usepackage{enumitem}
\usepackage{float}
\usepackage{bm}
\usepackage{tabularx}
\usepackage{booktabs}
\usepackage{url}


\newtheorem{theorem}{Theorem}[chapter]
\newtheorem{proposition}[theorem]{Proposition}
\newtheorem{lemma}[theorem]{Lemma}
\newtheorem{corollary}[theorem]{Corollary}
\newtheorem{conjecture}[theorem]{Conjecture}
\newtheorem{postulate}[theorem]{Postulate}
\theoremstyle{definition}
\newtheorem{definition}[theorem]{Definition}
\newtheorem{example}[theorem]{Example}
\newtheorem{observation}{Observation}
\newtheorem{remark}[theorem]{Remark}

\newcommand{\vertiii}[1]{{\left\vert\kern-0.25ex\left\vert\kern-0.25ex\left\vert#1 
    \right\vert\kern-0.25ex\right\vert\kern-0.25ex\right\vert}}

\newcommand*\enlarg[2]{\bar{#1}_{#2}}
\newcommand*\cohnorm[1]{\vertiii{#1}^{\mathrm{coh}}_{K,\varphi,\alpha,\gamma}}

% To compile this file, run "latex thesis", then "biber thesis"
% (or "bibtex thesis", if the output from latex asks for that instead),
% and then "latex thesis" (without the quotes in each case).

% Double spacing, if you want it.  Do not use for the final copy.
% \def\dsp{\def\baselinestretch{2.0}\large\normalsize}
% \dsp

% If the Grad. Division insists that the first paragraph of a section
% be indented (like the others), then include this line:
% \usepackage{indentfirst}

\addtolength{\abovecaptionskip}{\baselineskip}

\bibliography{references}

\hyphenation{mar-gin-al-ia}
\hyphenation{bra-va-do}

\begin{document}

% Declarations for Front Matter

\title{On the Hairer-Caravenna-Zambotti Reconstruction}
\author{Viet-Duc Nguyen}
\degreesemester{Summer}
\degreeyear{2022}
\degree{Bachelor of Science}
\chair{Professor Peter K. Friz}
\othermembers{Professor Peter Bank}
% For a co-chair who is subordinate to the \chair listed above
% \cochair{Professor Benedict Francis Pope}
% For two co-chairs of equal standing (do not use \chair with this one)
% \cochairs{Professor Richard Francis Sony}{Professor Benedict Francis Pope}
\numberofmembers{3}
% Previous degrees are no longer to be listed on the title page.
% \prevdegrees{B.A. (University of Northern South Dakota at Hoople) 1978 \\
%   M.S. (Ed's School of Quantum Mechanics and Muffler Repair) 1989}
\field{Mathematics}
% Designated Emphasis -- this is optional, and rare
% \emphasis{Colloidal Telemetry}
% This is optional, and rare
% \jointinstitution{University of Western Maryland}
% This is optional (default is Berkeley)
% \campus{Berkeley}

% For a masters thesis, replace the above \documentclass line with
% \documentclass[masters]{ucbthesis}
% This affects the title and approval pages, which by default calls this
% document a "dissertation", not a "thesis".

\maketitle
% Delete (or comment out) the \approvalpage line for the final version.
% \approvalpage
\copyrightpage

\begin{alwayssingle}
\pagenumbering{gobble}
\section*{Zusammenfassung in deutscher Sprache}

Gegeben sei eine Familie von Distributionen \( (F_x)_{x \in \mathbb{R}^d} \). Gesucht ist eine Distribution, die für jeden Punkt \( x \in \mathbb{R}^d \) durch \( F_x \) lokal gut approximiert wird. Wir stellen das \emph{Reconstruction Theorem} vor, welches die Existenz sowie die Eindeutigkeit solch einer Distribution sichert. 

Das Problem ist von großer Bedeutung für die Behandlung stochastischer Differentialgleichungen; genauer ist es das zentrale Theorem der Theorie der \emph{Regularity Structures} von Martin Hairer, welche auch in diesem Zusammenhang zum ersten Mal bewiesen wurde. Wir geben einen alternativen Zugang zum Reconstruction Theorem, die ohne Regularity Structures auskommt. Stattdessen betten wir das Reconstruction Theorem in die Theorie der Distributionen ein.
Wir finden eine hinreichende und notwendige Bedingung für das Reconstruction Theorem, die wir \emph{coherence} nennen. Der Beweis des Reconstruction Theorems bildet das Kernstück dieser Arbeit. 

Als Anwendung des Reconstruction Theorems führen wir negative Hölderräume ein und beweisen das Sewing Lemma, ein wichtiges Hilfsmittel in der Theorie der rauen Pfade. Das Sewing Lemma wird oft als das eindimensionale Analogon des Reconstruction Theorems betrachtet. Wir überprüfen diese Aussage.

\end{alwayssingle}

% (This file is included by thesis.tex; you do not latex it by itself.)

\begin{abstract}

% The text of the abstract goes here.  If you need to use a \section
% command you will need to use \section*, \subsection*, etc. so that
% you don't get any numbering.  You probably won't be using any of
% these commands in the abstract anyway.


This paper continues the research of Arthur Bik and Orlando Marigliano on the classification of one-dimensional discrete statistical models with rational maximum likelihood estimators using fundamental models. We determine the number of fundamental models in the simplex \( \Delta_6 \) with a maximum degree of eleven, a result that was previously unknown. Moreover, we reduce the number of cases to consider for proving the finite number of fundamental models in the simplex \( \Delta_5 \) with a maximum degree of eleven from 300,000 to 12,000, making a proof far more feasible in the future. The algorithm underpinning these key results is embedded in the framework of solving non-trivial hyperfield linear systems, which we have developed specifically for this thesis. All the code is publicly available on GitHub.


\end{abstract}


\begin{frontmatter}

% You can delete the \clearpage lines if you don't want these to start on
% separate pages.

\tableofcontents
\clearpage
%\listoffigures
%\clearpage
%\listoftables


\end{frontmatter}

\pagestyle{headings}

% (Optional) \part{First Part}
\pagenumbering{arabic}

\chapter{Introduction} 

In statistics, we come across various collections of probability distributions, such as the normal distribution, Poisson distribution, and binomial distribution. These distributions are used to model random variables in applications and are referred to as \emph{statistical models}. Precisely, a statistical model is just a set of probability distributions. If the set contains only discrete distributions, we call it a \emph{discrete statistical model}. In this case, discrete statistical models are just subsets of the probability simplex \( \Delta_n \coloneqq \left\{ p \in \mathbb{R}^{n + 1} \mid \sum p_i = 1 \right\} \). 

A discrete distribution \( p \in \mathcal{M} \subset \Delta_n \) from a discrete statistical model encapsulates the probabilities of observing the states \( 0, \dots, n \), i.e. if \( X \in \left\{ 0, \dots, n \right\} \) is a discrete random variable, then the state \( X = i \) occurs with probability \( p_i \) for all \( i = 0, \dots, n \). Say we have a binomial random variable \( X \) with \( n + 1 \) states, then \( p_i = \binom{n}{i} \theta^i (1-\theta)^{n-i} \) computes the probability of observing \( i \) successes in \( n \) trials with success probability \( \theta \in [0,1] \). The set \( \mathcal{M} \) of all probability distributions of that form, i.e. \( \mathcal{M} = \left\{ (\binom{n}{i} \theta^i (1-\theta)^{n-i})_{i=0}^n \mid \theta \in [0,1] \right\} \), is our first example of a discrete statistical model, and is known as the \emph{binomial model}.

\begin{figure}
    \centering
    \includegraphics[width=0.4\textwidth]{assets/binom-discrete-model.png}
    \caption{This figure shows the probability simplex \( \Delta_2 \) with the binomial model (red curve). Every point on the curve is a binomial distribution.}
    \label{fig:binom-discrete-model}
\end{figure}

Given a statistical model \( \mathcal{M} \subset \Delta_n \) and data \( u \in \mathbb{N}^{n+1} \), a typical problem in statistics is to find a distribution from a statistical model that best describes the data. ``Best'' can mean a lot of things, but in \emph{maximum likelihood estimation} it means finding the distribution that maximizes the probability of observing the data; the map \( \Phi: \Delta_n \to \mathcal{M}, u \mapsto \hat p \) that assigns the data \( u \) to a distribution \( \hat p \in \mathcal{M}\) from the statistical model is called the \emph{maximum likelihood estimator (MLE)}. This map is characterized by the property that \( \hat p \) maximizes the log-likelihood function \( \ell(p) = \sum u_i \log p_i \) for all \( p \in \mathcal{M} \). 

We focus on \textbf{{one-dimensional {discrete} {statistical} {models} with rational MLE}}. These are models \( \mathcal{M} \) satisfying 
\begin{itemize}
    \item \( \mathcal{M} = \mathrm{image}(p) \) for some rational map \( p = (p_0, \dots, p_n): I \to \Delta_n \) where \( p_i \) is rational, \( I \subset \mathbb{R} \) is a union of closed intervals and  \( p(\partial I) \subset \partial \Delta_n \),
    \item all the \( n+1 \) coordinates of the maximum likelihood estimator \( \Phi \) are rational functions in the data \( u \).
\end{itemize}
There are two intriguing questions to ask about statistical models with rational MLE: the first one is about which \emph{form} they take; the second one is more concerned with the \emph{classification} of the statistical models, i.e. can we divide these models into easier to understand classes? An answer to the first question was given by June Huh. He showed that if \( \Phi \) is rational, then each of its coordinates is an alternating product of linear forms with a numerator and denominator of the same degree, see \cite{huh2013varieties, huh2013maximum, duarte2021discrete}. For the second question, Arthur Bik and Orlando Marigliano classified all one-dimensional discrete statistical models with rational MLE using \emph{fundamental models} \cite{bik2022classifying}.

This thesis continues the work of Bik and Marigliano. In the first half, we present their classification results on how fundamental models serve as the building blocks of one-dimensional discrete models with rational MLE. In the second half, we establish and extend their finding that there are only finitely many fundamental models within the probability simplices \( \Delta_n  \) for \( n \leq 4 \). Due to the complexity of the problem, the cases \( n \geq 5 \) were left open. We make progress for \( n = 5 \) by reducing the number of cases to check from 300,000 to 12,000. Additionally, this thesis introduces new results on the number of fundamental models in \( \Delta_6 \), with a maximum degree of eleven, and provides an algorithm for solving non-trivial hyperfield linear systems, which is essential to all the computational work presented.

The outline of this thesis is as follows: 
\begin{itemize}
    \item Chapter 2 provides a classification of statistical models using fundamental models.
    \item Chapter 3 introduces chipsplitting games and establishes the connection to fundamental models via chipsplitting outcomes.
    \item Chapter 4 develops tools for analyzing valid chipsplitting outcomes, and Chapter 5 applies these tools to prove that the degree of valid outcomes with positive support sizes up to three is bounded.
    \item Chapter 6 extends the tools from Chapter 4 to prove the boundedness of degree for valid outcomes with positive support size four, and Chapter 7 uses these tools to establish a bound on the degree for positive support size four outcomes.
    \item Chapter 8 introduces the final tool to tackle outcomes with positive support size five, and Chapter 9 completes the proof for positive support size five outcomes.
    \item Chapter 10 presents new techniques to reduce the number of cases that need to be analyzed to prove that the degree of valid outcomes with support size six is bounded.
    \item Chapter 11 computes the number of fundamental outcomes.
    \item Chapter 12 concludes with a discussion on future research directions and the implications of the findings.
\end{itemize}


The source code for the computations discussed in this thesis is available at \cite{ducrepo}.

% \chapter{Classification with Fundamental Models}

In this chapter we present the classification of one-dimensional discrete statistical models with rational maximum likelihood estimator (MLE) using fundamental models. The classification is due to Arthur Bik and Orlando Marigliano~\cite{bik2022classifying}. 

\begin{center}
    \textbf{Problem statement:} Can we find a class of easy to understand models that serve as building blocks for all one-dimensional discrete statistical models with rational MLE?
\end{center}
The answer to this question are \emph{reduced} and \emph{fundamental models}.

\section{Parametrization}

It turns out that one-dimensional discrete statistical models with rational MLE admit the following parametrization.

\begin{proposition}\label{prop:parametrization}
    Let \( \mathcal{M} \) be a one-dimensional discrete statistical models with rational maximum likelihood estimator. Then, there exists a map of the form
    \begin{gather*}
        p: [0,1] \to \Delta_n, \quad \theta \mapsto (w_k \theta^{i_k} (1-\theta)^{j_k})_{k=0}^n \\
        i_k, j_k \in \mathbb{Z}_{\geq 0}, \;  w_k \in \mathbb{R}_{> 0} \quad \forall k = 0, \dots, n
    \end{gather*}
    such that \( \mathcal{M} = \mathrm{image}(p) \).
\end{proposition}

We introduce some notation to simplify the proof of Proposition \ref{prop:parametrization}.
Let \( \mathcal{M} \subset \Delta_n \) be a one-dimensional discrete statistical model parametrized by rational functions \( p_0 =  \frac{g_0}{h_0}, \dots, p_n =  \frac{g_n}{h_n} \). Define \( b \) to be the least common multiple of \( h_0, \dots, h_n \) and \( a_i \coloneqq b p_i \). Since \( \sum p_k = 1 \), we can multiply by \( b \) to obtain \( \sum a_k = b \). We see that the polynomials \( a_0, \dots, a_n, b \) determine the statistical model \( \mathcal{M} \), and have no common factors. The log-likelihood function is then given by
\begin{align*}
    \ell(p) &= \sum u_i \log p_i \\
    &= \sum u_i \log \frac{a_i}{b} \\
    &= \sum u_i \log a_i - \sum u_i \log b.
\end{align*}
To find the maximum likelihood estimator, we need find all critical points of the log-likelihood function. This is equivalent to finding the roots of the gradient of the log-likelihood function
\begin{align}\label{eq:score-equations}
    \ell(p(\theta))' &= \sum u_k \frac{a_k'}{a_k} - \sum u_k \frac{b'}{b} = 0.
\end{align}
These equations are called the \emph{score equations} in algebraic statistics, and the number of complex solutions to these equations for general data \( u \in \mathbb{C}^{n + 1} \) is called the \emph{maximum likelihood degree} of the statistical model. This ML degree has an important meaning in algebraic statistics, as it determines the complexity of the model. We have the following relationship between the ML estimator and the ML degree.

\begin{proposition}\label{prop:rational-mle}
    Having rational maximum likelihood estimator can be expressed equivalently by saying that the maximum likelihood degree of the statistical model is one.
\end{proposition}

\begin{proof}
   Refer to \cite{duarte2021discrete} for a proof.
\end{proof}

To prove Proposition \ref{prop:parametrization}, we need the following lemma.

\begin{lemma}\label{lem:two-complex-factors}
    If \( \mathcal{M} \) has rational MLE, then there are exactly two distinct complex linear factors in \( a_0, \dots, a_n \), and \( b \).
\end{lemma}

\begin{proof}
    We prove the lemma in three steps:
    \begin{itemize}
        \item Let \( f \) be the product of all distinct complex linear factors in \( a_0, \dots, a_n, b \).  If we multiply the score equations \eqref{eq:score-equations} by \( f \), we get 
        \begin{align*}
            f \cdot \ell(p(\theta))' &= \sum u_k f \frac{a_k'}{a_k} - \sum u_k f \frac{b'}{b} = 0. 
        \end{align*}
        Note that every linear factor of \( a_k \) with multiplicity \( m \) occurs in \( a_k' \) with multiplicity \( m-1 \); thus every summand of \( \frac{a_k'}{a_k} \) is of the form \( \frac{\lambda}{(x-\xi)} \), where \( \lambda \in \mathbb{R} \) and \( x-\xi \) is some linear factor of \( a_k \); hence \( f \cdot  \frac{\lambda}{(x-\xi)}  \) is of degree \( \mathrm{deg}(f) - 1\), and therefore \( f \cdot \ell(p(\theta))' \) is of degree \( \mathrm{deg}(f) - 1\).

        \item We claim that the roots of \( \ell(p(\theta))' \) are the same as the roots of \( f \cdot \ell(p(\theta))' \). Assume we have shown this claim.  By Proposition \ref{prop:rational-mle} the ML degree is one. So, \( \ell(p(\theta))' \) has one root. Thus, \( f \cdot \ell(p(\theta))' \) has one root, and therefore \( f \cdot \ell(p(\theta))' \) is of degree one. This implies that \( \mathrm{deg}(f) = 2 \) with the previous step. Thus, there are exactly two distinct complex linear factors in \( a_0, \dots, a_n \), and \( b \).
        
        \item It remains to show that the roots stay the same. Clearly, every root of \( \ell(p(\theta))' \) is a root of \( f \cdot \ell(p(\theta))' \). Conversely, we want to show that no new roots are introduced when multiplying by \( f \), i.e. roots of \( f \) are not roots of \(  f \cdot \ell(p(\theta))' \). To do so, we rewrite 
        \begin{gather*}
            f \cdot \ell(p(\theta))' = \sum_{k=0}^n u_k f \frac{a_k'}{a_k} - \sum_{k=0}^n u_k f \frac{b'}{b} = \sum_{k=0}^{n + 1} v_k f \frac{c_k'}{c_k}\\
            v_k = u_k, \; c_k = a_k \quad \text{for } k = 0, \dots, n,\\ \quad v_{n+1} = - \sum_{k=0}^n u_k, \; c_{n+1} = b.
        \end{gather*}
        Let \( q \) be a complex linear factor of \( f \). We define polynomials \( r_0, \dots, r_{n+1} \) and \( r \) such that \( c_k = q^{l_k}r_k \), \( f = q r \), and \( r_0, \dots, r_{n+1}, r \) do not have \( q \) as a factor. Then, we have for \(  k = 0, \dots, n+1 \) that
        \begin{gather*}
            f \frac{c_k'}{c_k} = q r \cdot \frac{l_k q^{l_k - 1} q'r_k +  q^{l_k}r_k'}{q^{l_k}r_k} = q r\frac{l_k q' }{q} + q r\frac{r_k'}{r_k} \equiv rl_k q' \pmod q.
        \end{gather*}
        Thus, we obtain 
        \begin{align*}
            f \cdot \ell(p(\theta))' \equiv rq'\sum_{k=0}^{n + 1} v_k l_k \equiv rq' \sum_{k=0}^{n } v_k(l_k - l_{n+1}) \pmod q.
        \end{align*}
        Note that by definition of \( l_k \), a value of \( l_k = 0 \) means that \( q \) is not a factor of \( c_k \). By definition of \( f \), at least one \( l_k > 0 \). On the other hand, not all \( l_k \) can be positive since \( a_0, \dots, a_n, b \) share no common factors. Hence, not all \( l_k - l_{n+1} = 0 \) vanish. Hence, for generic data \( u \) we assume \( \sum_{k=0}^{n } v_k(l_k - l_{n+1}) \neq 0 \). This with \( q'r \not \equiv 0 \pmod q \) implies that \( q \) is not a complex linear factor of \( f \cdot \ell(p(\theta))' \). We showed that the roots of \( f \) are not roots of \( f \cdot \ell(p(\theta))' \).
    \end{itemize}
\end{proof}

Equipped with the lemma, we can now prove Proposition \ref{prop:parametrization}.

\begin{proof}
    We want to show the following parametrization of \( \mathcal{M} \):
    \begin{align*}
        p: [0,1] \to \Delta_n, \quad \theta \mapsto (w_k \theta^{i_k} (1-\theta)^{j_k})_{k=0}^n
    \end{align*}
    First, we show that \( I \) is a single closed real interval and not a union of closed intervals. For the sake of contradiction assume that \( I = \bigcup_{k} I_k \) is a union of closed disjoint intervals. By definition of \( \mathcal{M} \) we know that \( p(\partial I) \subset \partial \Delta_n \). Thus, there exist \( \theta_1, \theta_2 \in \partial I_0 \) and \( \theta_3, \theta_4 \in \partial I_1 \) with \( p_i(\theta_1) = p_i(\theta_2) =  0 \) and \( p_j(\theta_3) = p_j(\theta_4) = 0 \) for some \( i,j = 0, \dots, n \). Note that \( \theta_1, \theta_2 \) are roots of \( \frac{a_i}{b} \) and  \( \theta_3, \theta_4 \) are roots of \( \frac{a_j}{b} \). By Lemma \ref{lem:two-complex-factors} exactly two distinct complex linear factors occur in \( a_0, \dots, a_n, b \). Hence, \( \theta_3 = \theta_1 \) or \( \theta_3 = \theta_2 \). Contradiction for \( I_0 \) and \( I_1 \) are disjoint.

    The previous argument shows that \( I = [\alpha, \beta ]\) is a real single closed interval. Thus, the roots of \( a_0, \dots, a_n, b \) are real and take values in \( \partial I = \left\{ \alpha, \beta \right\} \). By a suitable parametrization, we can assume without loss of generality that \( I = [0,1] \). We can now write the polynomials \( a_0, \dots, a_n, b \) as
    \begin{align*}
        a_k(\theta) &= w_k \theta^{i_k} (1-\theta)^{j_k} \\
        b(\theta) &= w \theta^{i} (1-\theta)^{j}
    \end{align*}
    with \( w_k, w \in \mathbb{R}_{>0} \), and \( i_k, j_k, i, j \in \mathbb{Z}_{\geq 0} \) for all \( k = 0, \dots, n \). Since \( a_0, \dots, a_n, b \) share no common factors, there exists some \( i_k = 0 \) if \( i > 0 \); however this would contradict \(0 < w_k \leq a_0(0) + \dots + a_n(0) = b(0) = 0\). So \( i = 0 \). Similarly, \( j = 0 \). Finally, we divide \( p \) by \( w \) to obtain \( b \equiv 1 \).
\end{proof}

\begin{corollary}
    Any one-dimensional {discrete} {statistical} {models} with rational MLE can be represented by \( (w_k, i_k, j_k)_{k=0}^n \) for \( w_k \in \mathbb{R}_{>0} \) and \( i_k, j_k \in \mathbb{Z}_{\geq 0} \).
\end{corollary}

From now on, we only consider one-dimensional discrete statistical models with rational MLE; we call them \emph{models} for short.

\begin{definition}
    The degree \( \mathrm{deg}(\mathcal{M}) \) of a model \( \mathcal{M} \) represented by \( (w_k, i_k, j_k)_{k=0}^n \) is defined as \( \mathrm{max}\left\{ i_k + j_k : k = 0, \dots, n \right\} \).
\end{definition}

\begin{remark}\label{rem:equivalent-models}
    We view two models \( (w_k,i_k,j_k)_{k=0}^n \) and \( (w_k',i_k',j_k')_{k=0}^n \) as the same model if they are equal up to a permutation of the coordinates.
\end{remark}

\begin{example}
    The sequence \( ((1,0,2), (2,1,1), (1,2,0)) \) represents the binomial model with two trials. It has degree two. Its parametrization is given by \( \theta \mapsto ((1-\theta)^2, 2\theta(1-\theta),\theta^2) \). Also see Figure \ref{fig:binom-discrete-model} for a visualization of the binomial model within the probability simplex \( \Delta_2 \).

    Note that we view the sequences \( ((1,0,2), (2,1,1), (1,2,0)) \) or \( ((2,1,1), (1,0,2), (1,2,0)) \) as the same model as \( ((2,1,1), (1,2,0), (1,0,2)) \) since the order of the coordinates does not matter.
\end{example}

\begin{definition}
    Let \( \mathcal{M} \) be a model represented by \( (w_k, i_k, j_k)_{k=0}^n \). The set of exponent pairs \( (i_k, j_k)_{k=0}^n \) is called the support of \( \mathcal{M} \), denoted by \( \mathrm{supp}(\mathcal{M}) \).
\end{definition}

This was our first step towards understanding the structure of models. The next step is to introduce the concept of reduced models.

\section{Reduced Models}

Models in this section refer to one-dimensional discrete statistical models with rational MLE.

\begin{definition}
    We call a model represented by \( (w_k, i_k, j_k)_{k=0}^n \) \emph{reduced} if \( (i_k, j_k) \neq \mathbf 0 \) for all \( k = 0, \dots n \), and \( (i_k, j_k) \neq (i_l, j_l) \) for all \( k \neq l \).
\end{definition}

Due to \( (i_k, j_k) \neq (i_l, j_l) \), we can use functions to represent reduced models.

\begin{remark}\label{rem:representation-of-models-by-functions}
    A reduced model \( \mathcal{M} \) represented by \( (w_k, i_k, j_k)_{k=0}^n \) can also be identified by a function \( f: \mathbb{Z}^2 \to \mathbb{R}_{\geq 0}, (i, j) \mapsto w \), where \( w = w_k \) if \( (i_k, j_k) = (i, j) \) and \( w = 0 \) otherwise. The support of \( f \) is the set of all pairs \( (i, j) \) with \( f(i, j) > 0 \). It coincides with the support of \( \mathcal{M} \).
\end{remark}


Reduced models are our first building blocks for the classification of models. This statement is justified by the following two propositions. They show that every non-reduced model can be transformed into a reduced model by a sequence of linear embeddings.

\begin{proposition}\label{prop:linear-embedding-1}
    Let \( n \in \mathbb{N}_{>0} \).
    Let \( \mathcal{M} \) be a model represented by \( (w_k, i_k, j_k)_{k=0}^n \). If \( (i_l, j_l) = \mathbf{0} \) for some index \( l \), then there exist a model \( \mathcal{M}' \), \( \lambda \in [0,1] \) and \( k = 0, \dots, n \) such that
    \begin{align*}
        \mathcal{M} = \Psi_{\lambda,k}(\mathcal{M}'),
    \end{align*}
    where \( \Psi_{\lambda, k}: \Delta_{n-1} \to \Delta_n \) is defined as \(  p_i \mapsto \begin{cases}
        \lambda p_i & \text{if } k \neq i, \\
        1-\lambda & \text{if } k = i.
    \end{cases} \)
\end{proposition}

\begin{proof}
    Let \( (i_l, j_l) = \mathbf{0} \) for some index \( l \). If \( w_l = 1 \), then \( w_m = 0 \) for all \( m \neq l \); this contradicts \( w_m > 0 \) by Proposition \ref{prop:parametrization}. Set \( \lambda = 1 - w_l > 0 \) and \( k = l \). Define the model \( \mathcal{M}' \) represented by 
    \begin{align*}
        \left(\frac{w_h}{1-w_l}, i_h, j_h\right)^n_{h=0, h \neq l}.
    \end{align*}
    Then, \( \mathcal{M} = \Psi_{\lambda,k}(\mathcal{M}') \).
\end{proof}

\begin{proposition}\label{prop:linear-embedding-2}
    Let \( n \in \mathbb{N}_{>0} \).
    Let \( \mathcal{M} \) be model represented by \( (w_k, i_k, j_k)_{k=0}^n \). If \( (i_m, j_m) = (i_l, j_l)  \) for \( m \neq l \), then there exist a model \( \mathcal{M}' \), \( \lambda \in [0,1] \) and \( k,h = 0, \dots, n \) such that
    \begin{align*}
        \mathcal{M} = \Psi_{\lambda,k,h}(\mathcal{M}'),
    \end{align*}
    where \( \Psi_{\lambda, k,h}: \Delta_{n-1} \to \Delta_n \) is defined as \(  p_i \mapsto \begin{cases}
         p_i & \text{if } i \notin \left\{ k,h \right\}, \\
        \lambda p_k & \text{if } k = i, \\
        (1-\lambda) p_k & \text{if } h = i. \\
    \end{cases} \)
\end{proposition}

\begin{proof}
    Define \( \lambda = \frac{w_m}{w_m + w_l} \), \( k = m \), and \( h = l \). Define the model \( \mathcal{M}' \) represented by 
    \begin{align*}
        \left( w_g + \delta_{gm}w_l, i_g, j_g  \right)^n_{g=0, g \neq l}.
    \end{align*}
    Then, \( \mathcal{M} = \Psi_{\lambda,k}(\mathcal{M}') \).
\end{proof}

By repeatedly applying the two propositions, we can transform any model into a reduced model. 

\begin{corollary}\label{cor:reduced-models}
    If \( \Delta_n \) contains a model of degree \( d \), then there also exists a reduced model of degree \( d \) in \( \Delta_m \) for some \( m \leq n \).
\end{corollary}


\section{Fundamental Models}

As before, models refer to one-dimensional discrete statistical models with rational MLE. The main building blocks for the classification of models are \emph{fundamental models}; we will see that reduced models come from fundamental models.

\begin{definition}
    We call a model represented by \( (w_k, i_k, j_k)_{k=0}^n \) \emph{fundamental} if it is reduced and the equation \( p_0 + \dots p_n \equiv 1 \) for given \( (i_k, j_k)_{k=0}^n \) uniquely determines the weights \( (w_k)_{k=0}^n \).
\end{definition}

\begin{example}
    The binomial model with two trials is fundamental. Given \( (i_0, j_0) = (0,2) \), \( (i_1, j_1) = (1,1) \), and \( (i_2, j_2) = (2,0) \), the equation \( p_0 + p_1 + p_2 = w_0\theta^2 + w_1\theta(1-\theta) + w_2(1-\theta)^2 \equiv 1 \) uniquely determines the weights \( w_0 = 1, w_1 = 2, w_2 = 1 \). To see this observe that this equation is equivalent to \( w_0\theta^2 + w_1\theta - w_1 \theta^2 + w_2 -w_22\theta + w_2\theta^2 = 1\) which is equivalent to solving \( w_2 - 1 + \theta(w_1 - 2w_2) + \theta^2(w_0 - w_1 + w_2) = 0 \) for all \( \theta \in \mathbb{R} \).
\end{example}

\begin{example}\label{ex:prob-simplex-0}
    Consider the probability simplex \( \Delta_0 \). It only contains the model \( 1 \) which is fundamental.
\end{example}

\begin{example}\label{ex:prob-simplex-1}
    Now, consider the probability simplex \( \Delta_1 \). It only contains the models \( \theta \mapsto (\theta, 1-\theta) \) and \( \theta \mapsto (1-\theta, \theta) \) which are equivalent. They are fundamental.
\end{example}

We will see that fundamental models like the ones above are building blocks for all reduced models by \emph{composition}.

\begin{definition}
    Let \( \mathcal{M} \) and \( \mathcal{M}' \) be reduced models which are represented by functions \( f,g : \mathbb{Z}^2 \to \mathbb{R}_{\geq 0} \), see Remark \ref{rem:representation-of-models-by-functions}. Let \( \mu \in (0,1) \). The \emph{composite} \( \mathcal{M} *_\mu \mathcal{M}' \) of \( \mathcal{M} \) and \( \mathcal{M}' \) is the reduced model represented by the function 
    \begin{align*}
        (i,j) \mapsto \mu f(i,j) + (1-\mu) g(i,j).
    \end{align*}
\end{definition}


% \begin{proposition}
%     Let \( \mathcal{M} \) be a reduced model. If \( \mathcal{M} \) is not the composite of two reduced models whose supports are proper subsets of \( \mathrm{supp}(\mathcal{M}) \), then \( \mathcal{M} \) is fundamental.
% \end{proposition}

% \begin{proof}
%     Let \( S \coloneqq \mathrm{supp}(\mathcal{M}) \) and let \( \mathcal{M} \) be represented by \( (v_k, i_k, j_k)_{k=0}^n \). The set of all reduced models with support equal to \( S \) corresponds to the set \( A \) of all real \( (w_k)_{k=0}^n \) that satisfy 
%     \begin{align*}
%         \sum_{k=0}^n w_k t^{i_k}(1-t)^{j_k} \equiv 1, \quad w_k \in \mathbb{R}.
%     \end{align*}
%     This set \( A \) contains \( v \). It is an affine-linear half-space, and its dimension coincides with the dimension of the linear space  \( \mathrm{lin}\{ t^{i_k}(1-t)^{j_k} : k=0, \dots, n\} \) since there exists an open ball around \( \mathbf v \) containing only positive vectors.

%     By assumption \( \mathcal{M} \) is the composite of two reduced models \( \mathcal{M}_1 \) and \( \mathcal{M}_2 \) with supports \( S_1 \) and \( S_2 \) which are proper subsets of \( S \).
% \end{proof}

We are about to show that every reduced model is the composite of finitely many fundamental models.

\begin{proposition}\label{prop:composition-fundamental}
    Let \( \mathcal{M} \) be a reduced model. Then \( \mathcal{M} \) is the composite of finitely many fundamental models.
\end{proposition}

\begin{proof}
    For \( \Delta_0 \) and \( \Delta_1 \) we know that they only contain fundamental models, see Examples \ref{ex:prob-simplex-0} and \ref{ex:prob-simplex-1}. 
    
    Assume we are given \( \Delta_n \) with \( n \geq 2 \), and let \( \mathcal{M} \) be a model that is not fundamental. We aim to show that \( \mathcal{M} \) can be expressed as a composite of two models, \( \mathcal{M}' \) and \( \mathcal{M}'' \), whose supports are proper subsets of \( \mathrm{supp}(\mathcal{M}) \). Assume this is indeed the case. Then, by applying the same argument to \( \mathcal{M}' \) and \( \mathcal{M}'' \), we can recursively decompose each non-fundamental model into models with smaller supports. Since \( \mathrm{supp}(\mathcal{M}) \) is finite, this recursive decomposition must eventually terminate, yielding a decomposition of \( \mathcal{M} \) into fundamental models. Thus, we have shown that any reduced model is the composite of a finite number of fundamental models. 

    Let us prove that \( \mathcal{M} \) is the composite of two models whose supports are proper subsets of \( \mathrm{supp}(\mathcal{M}) \). Since \( \mathcal{M} \) is not fundamental, the equation \( p_0 + \dots + p_n = 1 \) has distinct solutions \( \mathbf w, \mathbf w' \in \mathbb{R}^{n+1}_{> 0} \). Define \( \mathbf v \coloneqq \mathbf w - \mathbf w' \neq \mathbf 0 \). Then, 
    \begin{align*}
        \sum_{k=0}^n v_k \theta^{i_k}(1-\theta)^{j_k} = 0 \quad \forall \theta \in (0,1).
    \end{align*}
    Observe that there exist strictly positive and negative coefficients \( v_k \). Define 
    \begin{align*}
        \lambda &\coloneqq \min \left\{ \frac{w_k}{\lvert v_k \rvert} : k = 0, \dots, n, \; v_k < 0 \right\}, \\
        u_k &\coloneqq w_k + \lambda v_k \quad \text{for } k = 0, \dots, n, \\
        S_1 &\coloneqq \left\{ (i_k, j_k) : k=0, \dots, n, \; u_k \neq 0 \right\}.
    \end{align*}
    Note that \( \lambda > 0 \) since all the coefficients \( w_k \) are strictly positive by definition. Also observe that \( u_k \geq 0 \) if \( v_k \geq 0 \). Moreover, by definition \( \frac{w_k}{\lvert v_k \rvert} \geq \lambda \) for all \( k \geq 0 \). Hence, if \( v_k < 0 \), we also have \( \frac{u_k}{v_k} = \frac{w_k}{v_k} + \lambda  \leq 0\). Multiplying by \( v_k < 0 \) we obtain \( u_k \geq 0 \). All in all, we have \( u_k \geq 0 \) for all \( k = 0, \dots, n \). Moreover, \( u_k = 0 \) if and only if \( v_k < 0 \) and \( \lambda = \frac{w_k}{\lvert v_k \rvert} \). This shows that \( S_1 \subsetneq \mathrm{supp}(\mathcal{M}) \). Since \( u_0 + \dots u_n = 1 \), we have found a reduced model \( \mathcal{M}' \) represented by \( (u_k, i_k, j_k)_{(i_k,j_k) \in S_1} \).

    For the second model, we define
    \begin{align*}
        \mu &\coloneqq \min \left\{ \frac{w_k}{u_k} : k = 0, \dots, n, \; u_k \neq 0 \right\}, \\
        t_k &\coloneqq \frac{w_k - \mu u_k}{1 - \mu} \quad \text{for } k = 0, \dots, n, \\
        S_2 &\coloneqq \left\{ (i_k, j_k) : k=0, \dots, n, \; t_k \neq 0 \right\}.
    \end{align*}
    Similarly, \( \mu > 0 \). We have \( \mu < 1 \) because some \( v_k \) is positive implying \( u_k > w_k \). By definition, we have \( t_k \geq 0 \), and \( t_k = 0 \) if and only if \( u_k \neq 0 \) and \( \mu = \frac{w_k}{u_k} \). This shows that \( S_2 \subsetneq  \mathrm{supp}(\mathcal{M}) \) and \( S_1 \cup S_2 = \mathrm{supp}(\mathcal{M}) \). Since \( t_0 + \dots + t_n = 1 \), we have found a reduced model \( \mathcal{M}'' \) represented by \( (t_k, i_k, j_k)_{(i_k,j_k) \in S_2} \).

    Finally, we see that \( w_k = \mu u_k + (1-\mu) t_k\). This shows that \( \mathcal{M} = \mathcal{M}' *_\mu \mathcal{M}'' \).
\end{proof}

By applying the previous proposition with Corollary \ref{cor:reduced-models}, we obtain the following corollary.

\begin{corollary}\label{cor:fundamental-models-ksmlkdf}
    If \( \Delta_n \) contains a non-fundamental model of degree \( d \), then there exists a fundamental model of degree \( d \) in \( \Delta_m \) for some \( m < n \).
\end{corollary}

\begin{example}
For the two-dimensional probability simplex \( \Delta_2 \), we can classify all models. Again, models refer to one-dimensional discrete statistical models with rational MLE. Note that the model \( \mathcal{M} \) parametrized by \( \theta \mapsto (\theta, 1-\theta) \) satisfies \( \mathcal{M} *_\mu \mathcal{M} = \mathcal{M} \) for all \( \mu \). Since \( \Delta_1 \) only contains the model \( \theta \mapsto (\theta, 1-\theta) \), we can conclude that \( \Delta_2 \) only contains fundamental models or models that are not reduced.

To find all the fundamental models in \( \Delta_2 \), we need to check for all sets \( S = \left\{ (i_k,j_k)\right\}_{k=0}^2 \subset \mathbb{Z}^2_{>0} \) of size three if the equation \( p_0 + p_1 + p_2 = \sum_{k=0}^2 w_k \theta^{i_k}(1-\theta)^{j_k} = 1 \) has a unique solution \( (w_0, w_1, w_2) \). As we can see, a priori infinitely many sets \( S \) need to be checked. However, as we will see in the next section, only those sets \( S \) with \( \max\left\{ i+j : (i,j) \in S \right\} \leq 2n -1 = 3 \) need to be considered. Clearly, this reduces the number of sets \( S \) to be checked to a finite number.

We compute that only the following supports uniquely determine the weights \( (w_0, w_1, w_2) \):
\begin{align*}
    \{ (0,3), (1,1), (3,0) \} , \{ (0,2), (1,1), (2,0) \}, \{ (0,1), (1,1), (2,0) \}, \{ (0,2),(1,0),(1,1) \}.
\end{align*}
They correspond to the fundamental models \( ((1-\theta)^3, 3\theta(1-\theta), \theta^3) \), \( ((1-\theta)^2, 2\theta(1-\theta), \theta^2) \), \( (1-\theta, \theta(1-\theta), \theta^2) \), and \( ((1-\theta)^2, \theta, \theta(1-\theta)) \). The last model is equivalent to the second last model by a parametrization \( \theta \mapsto 1-\theta \) and permutation of the coordinates.

\begin{figure}[H]
    \centering
    \includegraphics[width=0.8\textwidth]{assets/fundamental-models-delta-2.png}
    \caption{From left to right the illustration depicts the models parametrized \( ((1-\theta)^3, 3\theta(1-\theta), \theta^3) \), \( ((1-\theta)^2, 2\theta(1-\theta), \theta^2) \), \( (1-\theta, \theta(1-\theta), \theta^2) \), and \( ((1-\theta)^2, \theta, \theta(1-\theta)) \). The illustration is taken from \cite{bik2022classifying}.}
\end{figure}

We just computed all fundamental models of degree three or less in \( \Delta_2 \). We will see shortly that these are all models in the probability simplex \( \Delta_2 \). Of course, \( \Delta_2 \) contains non-reduced models, too. These are models that come from linear embeddings \( \Psi_{\lambda,k} \) and \( \Psi_{\lambda,k,h} \), see Proposition \ref{prop:linear-embedding-1} and Proposition \ref{prop:linear-embedding-2}. There are infinitely many of them, and for \( \lambda = \frac{1}{3} \) we obtain the models \( \theta \mapsto (\frac{2}{3}\theta, \frac{1}{3}, \frac{2}{3}(1 - \theta)) \) and \( \theta \mapsto (1-\theta, \frac{1}{3}\theta, \frac{2}{3}\theta) \).

\begin{figure}[H]
    \centering
    \includegraphics[width=0.66\textwidth]{assets/non-red-models-delta-2.png}
    \caption{This illustration depicts two non-reduced models in \( \Delta_2 \) for \( \lambda = \frac{1}{3} \). They are parametrized by \( \theta \mapsto (\frac{2}{3}\theta, \frac{1}{3}, \frac{2}{3}(1 - \theta)) \) and \( \theta \mapsto (1-\theta, \frac{1}{3}\theta, \frac{2}{3}\theta) \). All other non-reduced models can be obtained by varying \( \lambda \). The illustration is taken from \cite{bik2022classifying}.} 
\end{figure}
\end{example}

Let us summarize the results of this section. It is the first part of our classification theorem.

\begin{theorem}
    Every one-dimensional discrete statistical model with rational MLE in \( \Delta_n \) is the image of a reduced model in \( \Delta_m \) under a linear embedding \( \Delta_m \to \Delta_n \) for some \( m \leq n \).

    Moreover, every reduced model \( \mathcal{M} \subset \Delta \) can be written as a composite of finitely many fundamental models
    \begin{align*}
        \mathcal{M} = \mathcal{M}_1 *_{\mu_1} ( \dots *_{\mu_{m-2}}( \mathcal{M}_{m-1} *_{\mu_{m-1}} \mathcal{M}_m) )
    \end{align*}
    for some \( m < n \) and \( \mu_1, \dots, \mu_m \in (0,1) \).
\end{theorem}

\begin{proof}
    See Proposition \ref{prop:composition-fundamental}, Proposition \ref{prop:linear-embedding-1}, and Proposition \ref{prop:linear-embedding-2}.
\end{proof}

\section{On the Finiteness of Fundamental Models}

We have established the first part of our classification theorem, namely that fundamental models are building blocks for all models. The second part is showing that there are only finitely many fundamental models in \( \Delta_n \) given \( n \in \mathbb{N} \). Artuhr Bik and Orlando Marigliano proved that there are only finitely many fundamental models in \( \Delta_n \) for \( n \leq 4 \) \cite{bik2022classifying}. We will later make significant progress towards proving the case \( n = 5 \). For \( n \geq 6 \) no attempt has been made yet to the best of our knowledge.

Arthur Bik and Orlando Marigliano first proved the following proposition.

\begin{theorem}\label{thm:degree-fundamental-models}
    Let \( \mathcal{M} \) be a one dimensional discrete statistical model with rational MLE in \( \Delta_n \). For \(n \leq 4 \) we have \( \mathrm{deg}(\mathcal{M}) \leq 2n - 1\).
\end{theorem}

Given Theorem \ref{thm:degree-fundamental-models} it is easy to show the second part of our classification.

\begin{theorem}\label{thm:finiteness-fundamental-models}
    There are only finitely many fundamental models in \( \Delta_n \) for all \( n \leq 4 \).
\end{theorem}

\begin{proof}
    By Theorem \ref{thm:degree-fundamental-models} we know that the degree of a fundamental model is at most \( 2n - 1 \). Since the number of supports of a fundamental model of degree \( 2n - 1 \) is finite, there are only finitely many fundamental models in \( \Delta_n \) for all \( n \leq 4 \).
\end{proof}

We will now spend the rest of this thesis on proving Theorem \ref{thm:degree-fundamental-models}. The idea is to use the building blocks of fundamental models that we have established so far. Namely, it suffices to show the theorem for fundamental models.

\begin{theorem}\label{thm:degree-fundamental-models-reduced}
    Let \(N \in \mathbb{N} \). If \( \mathrm{deg}(\mathcal{M}) \leq 2n - 1\) for all fundamental models in $\Delta_n$ and $n \leq N$, then $\mathrm{deg}(\mathcal{M'}) \leq 2n - 1$ holds for all models in $\Delta_n$.
\end{theorem}

\begin{proof}
    Let $N \in \mathbb{N}$ and $n \leq N$.
    Assume there is some non-fundamental model $\mathcal{M}'$ in $\Delta_n$ of degree greater than $2n - 1$. By Corollary \ref{cor:fundamental-models-ksmlkdf} there exists a fundamental model $\mathcal{M}$ in $\Delta_m$ for some $m < n$ of degree greater than $2m - 1$. This contradicts the assumption that the degree of fundamental models is at most $2n' - 1$ for all $n' \leq N$.
\end{proof}

Counting all \emph{fundamental} models in $\Delta_n$ for $n \leq 4$ is our guiding objective. As a first step, we introduce a combinatorial game that aids in counting fundamental models. We know that every reduced model can be represented by the sequence of triples $(w_k, i_k, j_k)^{n}_{k=0}$, where $w_k \in \mathbb{R}{>0}$ and $i_k, j_k \in \mathbb{Z}_{\geq 0}$. The model can be visualized in a directed graph with vertices in $\mathbb{Z}^2$, where we can place values $w_k$ on vertices $(i_k, j_k)$. Each vertex $(i,j)$ is connected by directed edges to $(i+1, j)$ and $(i, j+1)$. 

\begin{figure}\label{fig:binom-discrete-model-visual}
    \centering
    % https://tikzcd.yichuanshen.de/#N4Igdg9gJgpgziAXAbVABwnAlgFyxMJZABgBoBmAXVJADcBDAGwFcYkQA6EAX1PU1z5CKMgCZqdJq3Zde-bHgJEAjBQkMWbRJx58QGBUKJll6qVpDLd8wUpSrxNDdO2jr+gYuHJRap+fYrOQ9DO2RVU39NGXcDW29fSMlo7Vk9OK8iX0dklx1gjKMUcj9cizSbTOLSHOdy2M8i5BKkupiCxrCS4jMU-PTOhNIeqLyKkPiVYd6xhtDvMhGy9okYKABzeCJQADMAJwgAWyQyEBwIJGVg-aOTmnOkUWuD48RfM4vEAFZn27f7z4AFl+r1UHyQwL0N1BAKQ5BBcNhiGUPyhL0uXyRykhu3RiAAbFicSBoRCkfiEQSkQB2SnYrHwtF-akMyks8HIq5M0ElDnKJ7cy5gh7IgW4v6ApEADkpUpplMxHIAnArpbKscQ6acRcoueLXkqNZTeSKKZRuEA
    \begin{tikzcd}
        . \arrow[r]           & . \arrow[r]           & . \arrow[r]           & .           \\
        1 \arrow[u] \arrow[r] & . \arrow[u] \arrow[r] & . \arrow[u] \arrow[r] & . \arrow[u] \\
        . \arrow[r] \arrow[u] & 2 \arrow[u] \arrow[r] & . \arrow[u] \arrow[r] & . \arrow[u] \\
        . \arrow[u] \arrow[r] & . \arrow[r] \arrow[u] & 1 \arrow[r] \arrow[u] & . \arrow[u]
    \end{tikzcd}
    \caption{The binomial model with two trials visualized in a directed graph with vertices in $\left\{0,1,2,3 \right\}^2$.}
\end{figure}


Surprisingly, we can derive a combinatorial game from this graph by defining a specific set of rules. This game, called the \emph{chipsplitting game}, will be rigorously introduced in the next chapter. After that, we will explore the game's properties and show how it can be used to count fundamental models in $\Delta_n$ for $n \leq 4$.


\chapter{Coherence, Homogeneity and the Reconstruction Theorem}\label{chapter:reconstruction}

The Reconstruction Theorem was originally stated in the context of regularity structures by Hairer \cite{hairer2014theory}. Later, it was revisited by Caravenna and Zambotti \cite{caravenna2021hairer}, where the Reconstruction Theorem was framed in the theory of distributions. In this chapter, we will closely follow the spirit of Caravenna and Zambotti with the advantage being that it allows for an easily accessible and self-contained treatment of the Reconstruction Theorem.

\section{A First Peek at the Reconstruction Theorem}\label{chapter:first-peek-at-reconstruction}

\emph{Problem:} Given a family of distributions $(F_x)_{x \in \mathbb{R}^d}$ we would like to find a distribution $f \in \mathcal{D'}$ that is locally well approximated by $F_x$ around $x$ for every $x \in \mathbb{R}^d$. 

\vspace{0.5cm}

We can think of $(F_x)_{x \in \mathbb{R}^d}$ as a family of local candidate approximations for an unknown distribution $f$. The \emph{Reconstruction Theorem} reconstructs a function $f$ that is well approximated by $F_x$ at any point $x \in \mathbb{R}^d$. Our objective in this section is to find the conditions under which finding a reconstruction $f$ is possible. 

First, we only consider a measurable family of distributions $(F_x)_{x \in \mathbb{R}^d}$ which we call \emph{germs} --- a notion first introduced in \cite{caravenna2021hairer}.

\begin{definition}[Germ]
    A family of distributions $(F_x)_{x \in \mathbb{R}^d}$ is called a \emph{germ} if for all test functions $\psi \in \mathcal{D}$ the map $x \mapsto F_x(\psi)$ is measurable.
\end{definition}

Next, being \emph{locally well approximated} by a germ $(F_x)_{x \in \mathbb{R}^d}$ means that there exists a test function $\psi \in \mathcal{D}, \int \psi(x)\, \mathrm{d}x \neq 0$ such that for all compact sets $K \subset \mathbb{R}^d$ we have 
\begin{align}\label{peek:well-approximated}
    \lim_{\epsilon \to 0} |(f - F_x)(\psi^\epsilon_x)| = 0 \quad \text{uniformly for $x \in K$ }.
\end{align} 

The reconstruction theorem states that \emph{under some condition} we can find a reconstruction $f \in \mathcal{D}'$ that satisfies \eqref{peek:well-approximated}.

\begin{conjecture}\label{peek:conjecture}
    Let $(F_x)_{x \in \mathbb{R}^d}$ be a germ that satisfies some yet unknown condition \emph{\texttt{???}}. Then, there exists a reconstruction $f \in \mathcal{D}'$ and $\gamma > 0$ such that for every test function $\psi \in \mathcal{D}$ there exists $C < \infty$ with
    \begin{gather}\label{peek:eq:conjecture}
        |(f-F_x)(\psi^\epsilon_x)| \leq C \epsilon^{\gamma} \\
        \text{uniformly for $x$ in compact sets and $\epsilon \in (0,1]$} \nonumber.
    \end{gather}
    
\end{conjecture}

Note that the distribution $f$ is indeed a reconstruction of the germ $(F_x)_{x \in \mathbb{R}^d}$ in the sense of \eqref{peek:well-approximated} because $|(f-F_x)(\psi^\epsilon_x)| \leq C  \epsilon^{\gamma} \to 0$ as $\epsilon \to 0$.

\vspace{0.4cm} 

We would like to find a condition \texttt{???} that leads to the above conjecture. Let $(F_x)_{x \in \mathbb{R}^d}$ be a germ. Equation \eqref{eq:starting-point} will be our starting point for our search of \texttt{???}. For any $x \in \mathbb{R}^d$ and any test funtion $\psi$, the distribution $F_x$ evaluated for $\psi$ can be approximated by the mollified distribution $F_x(\psi * \rho^\epsilon)$ for some mollifier $\rho$, i.e.
\begin{align*}
    \lim_{\epsilon \to 0}F_x(\psi * \rho^\epsilon) \overset{\eqref{lemma:mollified-distribution}}{=} \lim_{\epsilon \to 0} \int F_x(\rho_y^\epsilon) \psi(y)\, \mathrm{d}y \overset{\eqref{eq:starting-point}}{=} F_x(\psi).
\end{align*}
This observation inspires us to replace $F_x$ under the integral by $F_y$ so that we obtain the map $f_\epsilon: \psi \mapsto \int F_y(\rho_y^\epsilon) \psi(y)\, \mathrm{d}y$. The motivation for $f_\epsilon$ is that we hope for 
\begin{align*}
    \lim_{\epsilon \to 0}f_\epsilon = \mathcal{R}f \quad \text{where } \mathcal{R}f \text{ is approximated by $F_x$ around any $x \in \mathbb{R}^d$}.
\end{align*}
\begin{definition}[Approximating distributions]\label{def:approximating-distributions}
        Let $(F_x)_{x \in \mathbb{R}^d}$ be a germ and $\epsilon_n = 2^{-n}$ for $n \in \mathbb{N}$. The \emph{approximating distribution} $f_n \in \mathcal{D}'$ is defined as 
        \begin{align*}
                f_n: \psi \mapsto \int_{\mathbb{R}^d} F_y(\rho_y^{\epsilon_n}) \psi(y)\, \mathrm{d}y
        \end{align*}
        for some mollifier $\rho$. 
\end{definition}
It is now our task to find $\texttt{???}$ such that (H1) the limit $\lim_{n \to \infty}f_n$ exists, and (H2) that this limit satisfies \eqref{peek:eq:conjecture}. This is an easier task since now we are only required to find a promising condition $\texttt{???}$ such that (H1) and (H2) hold. We further simplify this problem by ignoring (H2) for the beginning. So, the question becomes: \emph{Under which condition does $f_n$ converge?}

To discuss this question in more depth, we write $f_n$ as a telescopic sum $f_n = f_1 + \sum^{n-1}_{k=1}g_k$ with $g_k = f_{k+1} - f_k$. So, the limit $\lim_{n \to \infty} f_n$ exists if and only if $\sum^\infty_{k=1}g_k < \infty$. By definition of $f_{k}$, the term $g_k$ can be written as  
\begin{align*}
        g_k(\psi) = \int_{\mathbb{R}^d} F_y(\rho_y^{\epsilon_{k+1}} - \rho_y^{\epsilon_k}) \psi(y)\, \mathrm{d}y.
\end{align*}
Here, we encounter our very first obstacle. What is an appropriate choice for our mollifier $\rho$? It turns out that if we can write the \emph{difference} of two mollifiers as a \emph{convolution} of two nice test functions $\hat \varphi$ and $\check \varphi$ , i.e. $ \rho^{\epsilon_{k+1}}_y - \rho^{\epsilon_k}_y = (\hat \varphi^{\epsilon_k} * \check \varphi^{\epsilon_k})_y$, 
we can write with the help of Corollary \ref{cor:minosokoad}
\begin{align*}
    g_k(\psi) = \int_{\mathbb{R}^d} F_z((\hat \varphi^{\epsilon_k} * \check \varphi^{\epsilon_k})_z) \psi(z)\, \mathrm{d}z
    = \iint_{\mathbb{R}^{d \times d}} F_z(\hat \varphi^{\epsilon_k}_y) \check \varphi^{\epsilon_k}(y-z) \psi(z) \, \mathrm{d}y\, \mathrm{d}z.
\end{align*}
We are intentionally vague about what \emph{nice test functions} are in this context; we will discuss them in depth in Chapter \ref{chapter:step-1-tweaking}, where these nice test functions are called \emph{tweaked test functions}.

Taking a closer look at $F_z(\hat \varphi^{\epsilon_k}_y)$ \label{sec:motiadsd}, we recognize a problem when we let $k$ approach zero: the support of $\hat \varphi^{\epsilon_k}_y$ shrinks to some small compact set around $y$, but $F_z$ is a local candidate approximation around $z$; so $F_z$ cannot capture the behaviour around the point of interest $y$. We circumvent this problem with the triangle inequality: $F_z(\hat \varphi^{\epsilon_k}_y) = F_y(\hat \varphi^{\epsilon_k}_y) + \left(F_z(\hat \varphi^{\epsilon_k}_y) - F_y(\hat \varphi^{\epsilon_k}_y)\right)$. 
Therefore, we can write $g_k$ as 
\begin{align*}
        g_k(\psi) = \iint F_y(\hat \varphi^{\epsilon_k}_y) \check \varphi^{\epsilon_k}(y-z) \psi(z) \, \mathrm{d}y\, \mathrm{d}z 
        + \iint (F_z - F_y)(\hat \varphi^{\epsilon_k}_y) \check \varphi^{\epsilon_k}(y-z) \psi(z) \, \mathrm{d}y\, \mathrm{d}z .
\end{align*}
Remember that we are interested in finding a condition \texttt{???} for the germ $(F_x)_{x \in \mathbb{R}^d}$  such that $\sum^\infty_{k=1} g_k < \infty$. 

To find \texttt{???} we let us guide by a closely related problem in another branch of mathematics: rough differential equations. There, one would like to make sense of an integral $I_t$  of the form $I_t = \int^t_0 X_s \, \mathrm{d}Y_s$ where $X_s$ and $Y_s$ are paths of low regularity. For instance, let $G \in \mathcal{V}^p$ and $F \in \mathcal{V}^q$ with $\frac{1}{p} + \frac{1}{q} > 1$, where $\mathcal{V}^j$ is the space of all functions with finite $j$-variation for $j \in \left\{ p,q \right\}$. Then, there exists a canonical integration theory for this setting (the so called \emph{Young} regime \cite{Young1936AnIO}) such that the integral $I_t = \int^t_0 G \, \mathrm{d}F$ is defined. The idea is that for very small $|t-s|$ we approximate 
\begin{align*}
    \int^t_s G \, \mathrm{d}F \approx G(s)(F(t) - F(s)) \eqqcolon A_{s,t}.
\end{align*}
We use this approximation to give a meaning to the integral $\int^t_0 G \, \mathrm{d}F$.
The \emph{Sewing Lemma} \cite{GUBINELLI200486}, an analytical tool, which let integrals of low regularity to be defined in a meaningful sense, allows us to sew the approximations $A_{s,t}$ together to obtain an integral as a Riemann-type sum
\begin{align}\label{sewing-lemma-integral}
    I_t = \int^t_0 G \, \mathrm{d}F \coloneqq \lim\limits_{|\pi| \to 0} \sum\limits_{i=0}^{\# \pi - 1} A_{t_i,t_{i+1}}
\end{align}
for arbitrary partitions\footnote{Here, a partition of $[0,t]$ is an ordered set $\pi = \left\{ 0 = t_0 < t_1 < ... < t_k = t  \right\}$, $\# \pi = k$ and $|\pi| = \max\limits_{i=0,...,\# \pi - 1} |t_{i+1} - t_{i}|$.} $\pi$ of $[0,t]$ with $|\pi| \to 0$ as $n \to 0$.
\begin{lemma}[Sewing Lemma \cite{broux2021sewing}]
    Let $\gamma > 1$ and $\Delta = \left\{ (s,t) :  0 \leq s \leq t \leq T\right\}$ for some fixed $T > 0$. Let $A: \Delta \to \mathbb{R}$ be a continuous function such that there exists $C <\infty$ with
    \begin{gather}
        \delta A_{s,u,t} \coloneqq |A_{s,t} - A_{s,u} - A_{u,t}| \leq C (\max\{|u-s|,|t-u|\})^\gamma \label{sewing-lemma-condition}\\
        \text{uniformly for $0 \leq s \leq u \leq t \leq T$}. \nonumber
    \end{gather} 
    Then, there exists a unique function $I: [0,T] \to \mathbb{R}$ and $\tilde C < \infty$  such that $I_0 = 0$ and 
    \begin{gather*}
        |I_t - I_s - A_{s,t}| \leq \tilde C|t-s|^\gamma\\
        \text{uniformly over $0 \leq s \leq t \leq T$.}
    \end{gather*}  
    Furthermore, $I$ is the limit of Riemann-type sums as in \eqref{sewing-lemma-integral}.  
\end{lemma}
The connection to the Reconstruction Theorem can be seen in the following way: From a distributional viewpoint, we approximate the integral $I_t$ by $F_x$: 
\begin{align*}
    G(x)\int^t_0 \psi \, \mathrm{d}F \eqqcolon F_x(\psi)  \leadsto I_t(\psi) \coloneqq \int^t_0 G \psi \, \mathrm{d}F
\end{align*}
If we let $\psi = 1_{[s,t]}$, we get
\begin{align*}
    F_s(1_{[s,t]}) = G(s) \int^t_s \mathrm{d}F = G(s)(F(t) - F(s)) = A_{s,t}.
\end{align*}
If we further assume that $F_s(1_{[s,t]})$ satisfies \eqref{sewing-lemma-condition}, we have by the Sewing Lemma
\begin{align*}
    (F_x - F_u)((1_{[0,1]})^{y-u}_u) = \frac{(F_x - F_u)(1_{[u,y]})}{y-u} &= \frac{(G(x) - G(u))(F(y) - F(u))}{y-u} \\
    &= \frac{\delta A_{x,u,y}}{y-u}\\
    &\leq C \frac{ (|u-x| + |y-u|)^\gamma}{y-u}.
\end{align*} 
Hence the germ $(F_x)_{x \in \mathbb{R}^d}$ satisfies 
\begin{align*}
    (F_x - F_u)((1_{[0,1]})^{\epsilon}_u) \leq C \epsilon^{-1}(|u-x| + \epsilon)^\gamma
\end{align*}
for $\epsilon = y-u$ as long as $A_{s,t} = F_s(1_{[s,t]})$ satisfies the Sewing Lemma condition \eqref{sewing-lemma-condition}. This inspires us to define a property coined \emph{coherence} --- an optimal condition for the Reconstruction Theorem that was found by Caravenna and Zambotta in \cite{caravenna2021hairer}. Coherence states that a germ satisfies
\begin{gather}\label{pre-condition-coherence}
    |(F_z - F_y)(\varphi^\epsilon_y)| \leq C\epsilon^{a}(|z-y| + \epsilon)^{c - a}  \\ \text{uniformly for $z,y$ in compact sets and $\epsilon \in (0,1]$} \nonumber.
\end{gather}
for some test function $\varphi$, constants  $c$ and $a$.
The precise definition will occur in Chapter \ref{chapter:coherence}. In our previous example $A_{s,t} = F_s(1_{[s,t]})$, we have $a = -1$ and $c = \gamma - 1$ for our coherence condition. 

Returning to our problem of finding a bound for $g_k$ (recall that we aim to show that $\sum^\infty_{k=1} g_k < \infty$):
\begin{align*}
    g_k(\psi) = \iint F_y(\hat \varphi^{\epsilon_k}_y) \check \varphi^{\epsilon_k}(y-z) \psi(z) \, \mathrm{d}y\, \mathrm{d}z 
    + \iint (F_z - F_y)(\hat \varphi^{\epsilon_k}_y) \check \varphi^{\epsilon_k}(y-z) \psi(z) \, \mathrm{d}y\, \mathrm{d}z,
\end{align*}
we now have the condition of \emph{coherence} \eqref{pre-condition-coherence} to control $g_k$. It is not difficult to show that the second term in $g_k$ can be bounded with coherence. As we let $k \to \infty$, we have $\epsilon_k \to 0$. Moreover by coherence, $(F_z - F_y)(\hat \varphi^{\epsilon_k}_y) \leq C \epsilon_k^{a} (|z-y| + \epsilon_k)^{c-a}$. If $|z-y| < \epsilon_k$, then $(F_z - F_y)(\hat \varphi^{\epsilon_k}_y) \leq C2^{c-y} \epsilon_k^c \to 0$ as $k \to \infty$; this is great news because the remaining part of the second term $\check \varphi^{\epsilon_k}(y-z) \psi(z)$ is easily bounded.

Regarding the first term, we want $F_y(\hat \varphi^{\epsilon}_y) \to 0$ as $\epsilon \to 0$. One way to achieve this is by imposing a condition which we will call \emph{homogeneity}: if $F_y(\hat \varphi^{\epsilon}_y) \leq B \epsilon^{\beta}$ for some constant $B < \infty$ , we will say that the germ $(F_x)_{x \in \mathbb{R}^d}$ has homogeneity bound $\beta$. If $\beta > 0$, then   $F_y(\hat \varphi^{\epsilon}_y) \leq B \epsilon^\beta \to 0$ as $\epsilon \to 0$. Hence, the first term in $g_k$ can be controlled thanks to homogeneity; this condition in turn with coherence will allow us to show $\sum^\infty_{k=1} g_k < \infty$. 

It seems that we need a germ to satisfy the coherence and homogeneity condition. Fortunately, we get the homogeneity for free if a germ is coherent\footnote{A germ is said to be coherent if it satisfies the coherence condition in \eqref{pre-condition-coherence}. The precise definition will be given in Chapter \ref{chapter:coherence}.}. So, requiring a germ to be coherent is all we need to get the Reconstruction Theorem going! It gets even better: so far we showed with the help of coherence that a limiting sequence $f_n$ exists that converges to some $f$ which we \emph{might} call our reconstruction. However, it is not clear if $f$ is a reconstruction in the sense of \eqref{peek:eq:conjecture}. We will see that coherence suffices to show that $f$ is indeed a reconstruction.

\section{Coherence and Homogeneity}\label{chapter:coherence}

In this chapter we will rigorously introduce the notion of \emph{coherence} and \emph{homogeneity}. We will later see that coherence is sufficient and even necessary for the Reconstruction Theorem. Moreover, homogeneity follows from coherence.

We gave a heuristic motivation for the coherence condition in Chapter \ref{chapter:first-peek-at-reconstruction}, where we started with the Sewing Lemma and ended up with the following definition for a germ to be \emph{coherent}.

\begin{definition}[$\gamma$-coherent germs]\label{definition:coherence}
   Let $\gamma \in \mathbb{R}$. A germ $(F_x)_{x \in \mathbb{R}^d}$ is called \emph{$\gamma$-coherent} if there exists a test function $\varphi \in \mathcal{D}$ with $\int \varphi(x) \, \mathrm{d}x \neq 0$ such that for every compact set $K \subset \mathbb{R}^d$ there exists a non-positive real number $\alpha_K \leq \min\left\{ 0, \gamma \right\}$ and a constant $C < \infty$ with
   \begin{gather}\label{coherence}
        |(F_z - F_y)(\varphi^\lambda_y)| \leq C\lambda^\alpha(|z-y| + \lambda)^{\gamma - \alpha}  \\ \text{uniformly for $z,y \in K$, $|y-z| \leq 2$  and $\lambda \in (0,1]$} \nonumber.
   \end{gather}
\end{definition}
Note that we require $|y-z| \leq 2$ to hold which appears rather arbitrary. Indeed, one could also define coherence with $|y-z| \leq R$ for any $R \in \mathbb{R}$ instead; they are both equivalent. We can even drop the constraint $|y-z| \leq 2$ entirely, see Proposition \ref{proposition:cutoff}. In the end, we choose $|y-z| \leq 2$ because it is convenient for our purpose of proving the Reconstruction Theorem.

Sometimes it is useful to explicitly mention the family $(\alpha_K)$. So, we say that  $(F_x)_{x \in \mathbb{R}^d}$ is $(\bm{\alpha}, \gamma)$-coherent if $\bm \alpha = (\alpha_K)$ and $\alpha_K$ is the exponent required for the coherence condition \eqref{coherence} to hold for the compact set $K$. 

Fix $K, \varphi, \alpha, \gamma$. The \emph{semi-norm $\vertiii{\cdot}^{\mathrm{coh}}_{K,\varphi,\alpha,\gamma}$} is the smallest constant $C \in \mathbb{R} \cup \left\{ \infty \right\}$ such that the coherence condition \eqref{coherence} holds for $K, \varphi, \alpha, \gamma$. Concretely, we define
\begin{align*}
    \vertiii{F}^{\mathrm{coh}}_{K,\varphi,\alpha,\gamma} = \sup \left\{ \frac{(F_z - F_y)(\varphi^\lambda_y)}{\lambda^\alpha(|z-y| + \epsilon)^{\gamma - \alpha}} : y,z \in K, |z-y| \leq 2, \lambda \in (0,1] \right\}.
\end{align*}

We briefly discuss the meaning of coherence. For some constant $C' < \infty$ we rewrite the inequality \eqref{coherence} in the coherence assumption as
\begin{align}\label{EspressoHouse}
    |(F_z - F_y)(\varphi^\epsilon_y)| \leq C' \begin{cases}
        \epsilon^\gamma \quad & \text{if $|z-y| \leq \epsilon$} \\
        \epsilon^{\alpha} |z-y|^{\gamma - \alpha} \quad & \text{otherwise}
    \end{cases}.
\end{align}
\begin{itemize}
    \item First, note that $\epsilon^{\gamma} \leq \epsilon^{\alpha}$ because of $\epsilon \in (0,1]$ and $\gamma \geq \alpha$. As $|z-y|$ decreases to $\epsilon$, the difference between the two distributions $F_z$ and $F_y$ (evaluated at $\varphi^\epsilon_y$) changes from magnitude $\epsilon^\alpha$ to $\epsilon^{\gamma}$. This change becomes very dramatic when $\gamma > 0$ and $\alpha < 0$. Then, $\epsilon^{\alpha}$ diverges while $\epsilon^{\gamma}$ vanishes as $\epsilon \to 0$.
    \item Second, observe that the right hand side of \eqref{EspressoHouse} shrinks as $\alpha \nearrow 0$ for fixed $\gamma, y$ and $z$. In other words, the larger $\alpha$ (remember that $\alpha < 0$), the better the estimate gets. Hence, without loss of generality we assume that the map $K \mapsto \alpha_K$ is \emph{monotone}, i.e. 
    \begin{align}\label{alpha-monotone}
        K \subset K' \implies \alpha_K \geq \alpha_{K'}.
    \end{align}
    This is achieved by choosing the exponents $\alpha_K$ in the following way: for balls $K = B(0,n)$ of radius $n \in \mathbb{N}$ choose $\alpha_K = \min\left\{ \alpha_{B(0,i)}  : 1\leq i \leq n\right\}$; otherwise for general compact sets $K$ choose $\alpha_K = \min\left\{ \alpha_{B(0,i)}  : 1\leq i \leq n\right\}$ with $n \in \mathbb{N}$ such that  $B(0,n) \supset K$. This ensures that the family of exponents $(\alpha_K)$ is montone. It will play an important role in the proof of the Reconstruction Theorem in case $\gamma < 0$, see Chapter \ref{chapter:step6gammaNegative}.
\end{itemize}

To conclude, for fixed $y \in \mathbb{R}^d$ and distribution $F_y$
\begin{enumerate}
    \item we know more about distributions $F_z$ if $|z - y | \leq \epsilon$ because then $|(F_z - F_y)(\varphi^\epsilon_y)|$ is smaller than for $|z - y| > \epsilon$, and
    \item in case of $\gamma > 0$ and $\alpha < 0$ we obtain even more information about $F_z$ where $|z - y | \leq \epsilon$  as $\epsilon$ decreases.
\end{enumerate}

Next, we discuss how we need to utilize the coherence condition $\eqref{coherence}$ to get the most out of it. Remember, we want to control the $(**)-$part of $g_k$ which is $ \iint (F_z - F_y)(\hat \varphi^{\epsilon_k}_y) \check \varphi^{\epsilon_k}(y-z) \psi(z) \, \mathrm{d}y\, \mathrm{d}z$. As we found out, the coherence property gives us the most information when $|z - y| \leq \epsilon_k$. This means that we need to carefully select $\check \varphi$ such that its support $\mathrm{supp}(\check \varphi)$ has diameter smaller or equal $\epsilon_k$ as this implies $|z-y| \leq \epsilon_k$. When we then apply the coherence condition $\eqref{coherence}$, we can bound 
\begin{align*}
    \iint &(F_z - F_y)(\hat \varphi^{\epsilon_k}_y) \check \varphi^{\epsilon_k}(y-z) \psi(z) \, \mathrm{d}y\, \mathrm{d}z 
    \\ & \leq \sup_{|y-z|\leq \epsilon_k}|(F_z -F_y)(\hat \varphi^{\epsilon_k}_y)| \iint  \check \varphi^{\epsilon_k}(y-z) \psi(z) \, \mathrm{d}y\, \mathrm{d}z    \\
    &\Downarrow \text{coherence condition}
    \\ &\leq C' \epsilon_k^{\gamma} \cdot \left\{ \text{constant} \right\}.
\end{align*}
As we sum $\sum^\infty_{k=1} g_k$, we want $\sum^\infty_{k=1} C' \epsilon_k^{\gamma} \cdot \left\{ \text{constant} \right\}$, which is a geometric sum (since $\epsilon_k \coloneqq 2^{-k}$), to be finite. That is the case when $\gamma > 0$. Thus, we can bound one part of $\sum^\infty_{k=1} g_k$, and the coherence condition \eqref{coherence} helped us successfully to show that the approximating distributions $f_n$ do converge.

To control the $(*)$-part of $g_k$, we have introduced the notion of \emph{homogeneity bound}. We could demand that the germ $(F_x)_{x \in \mathbb{R}^d}$ needs to satisfy the homogeneity bound on top of the coherence condition \eqref{coherence}, but luckily we get it for free when the germ $(F_x)_{x \in \mathbb{R}^d}$ is $\gamma$-coherent. The following lemma is definition and lemma at the same time.

\begin{lemma}[Homogeneity bound]
   Let $(F_x)_{x \in \mathbb{R}^d}$ be a $\gamma$-coherent germ. Then, for every compact set $K \subset \mathbb{R}^d$ there exists a real number $\beta < \gamma$ and a constant $B < \infty$ such that the \emph{homogeneity bound} holds, i.e. 
   \begin{gather*}\label{homogeneity}
                |F_y(\varphi^\epsilon_y)| \leq B\epsilon^\beta \quad
                \text{uniformly for $y \in K$ and $\epsilon \in (0,1]$} \tag{\texttt{HOMB}}.
   \end{gather*}
   We say the germ $(F_x)_{x \in \mathbb{R}^d}$ has \emph{local homogeneity bound} $\bm \beta = (\beta_K)$ if $\beta_K$ is the exponent such that \eqref{homogeneity} holds for the compact set $K \subset \mathbb{R}^d$. We say the germ $(F_x)_{x \in \mathbb{R}^d}$ has \emph{global homogeneity bound} $\beta$ if $\beta_K = \beta$ for all compact sets $K \subset \mathbb{R}^d$.
\end{lemma}

\begin{proof}
    We know how to bound $|(F_y - F_z)(\varphi^\epsilon_y)|$ by the coherence condition \eqref{coherence}. If we can bound $|F_z(\varphi^\epsilon_y)|$, then we can easily obtain 
    \begin{align*}
        |F_y(\varphi^\epsilon_y)| \leq |(F_y - F_z)(\varphi^\epsilon_y) + F_z(\varphi^\epsilon_y)| \leq \left\{ \mathrm{constant} \right\} \cdot \epsilon^{\beta}
    \end{align*}
    for some $\beta$.
    
    Fix any compact set $K \subset \mathbb{R}^d$ and $z \in K$. We use the coherence condition to estimate $|(F_y - F_z)(\varphi^\epsilon_y)| \leq  C \epsilon^{\alpha}(|z-y|+\epsilon)^{\gamma - \alpha} \leq \{ C (\mathrm{diam}(K) + 1)^{\gamma - \alpha} \} \cdot  \epsilon^{\alpha}$ uniformly for $y \in K$ and $\epsilon \in (0,1]$ (where $\mathrm{diam}(K) \coloneqq \sup_{y,z \in K}|y-z| $).

    To estimate $|F_z(\varphi^\epsilon_y)|$, we know there exist $\tilde C < \infty$ and $r \in \mathbb{N}_0$ such that $|F_z(\varphi^\epsilon_y)| \leq \tilde C \lVert \varphi^\epsilon_y \rVert_{C^r}$ for all $y \in K$ and $\epsilon \in (0,1]$ because $F_z$ is a distribution. Also, we have $\lVert \partial^k\varphi^\epsilon_y \rVert_\infty \leq \epsilon^{-|k|- d} \lVert \partial^k\varphi \rVert_\infty \leq \epsilon^{-r - d} \lVert \varphi \rVert_{C^r}$. Thus, $\lVert \varphi^\epsilon_y \rVert_{C^r} \leq \epsilon^{-r-d}\lVert \varphi \rVert_{C^r}$ follows. In the end, we obtain 
    $|F_z(\varphi^\epsilon_y)| \leq \{ \tilde C  \lVert \varphi \rVert_{C^r}  \} \cdot \epsilon^{-r-d}$.

    All we have to do is to chose $B \coloneqq C (\mathrm{diam}(K) + 1)^{\gamma - \alpha} +  \tilde C  \lVert \varphi \rVert_{C^r}  $ and $\beta \leq \min \left\{ \alpha, -r-d, \gamma \right\}$.
\end{proof}

Similar to $(\alpha_K)$, the family $(\beta_K)$ is \emph{monotone} in the sense that 
\begin{align}\label{beta-monotone}
    K \subset K' \implies \beta_K \geq \beta_{K'}.
\end{align} 


We are ready to state a preliminary version of the reconstruction theorem. 

\begin{theorem}[Preliminary reconstruction theorem]\label{peek:prelim-reconstruction-theorem}
    Let $\gamma \in \mathbb{R}$.
   Let $F = (F_x)_{x \in \mathbb{R}^d}$ be a $\gamma$-coherent germ. Then, there exists a distribution $f \in \mathcal{D}'$ such that for every test function $\psi \in \mathcal{D}$ and compact set $K \subset \mathbb{R}^d$ there exists a constant $C < \infty$ with  
   \begin{gather*}
           |(f-F_x)(\psi^\epsilon_x)| \leq C \begin{cases}
                   \epsilon^\gamma \quad &\text{if $\gamma \neq 0$} \\
                   1+|\log\epsilon| & \text{if $\gamma = 0$}
           \end{cases} \\ \text{uniformly for $x \in K$ and $\epsilon \in (0,1]$}.
   \end{gather*}
   If $\gamma > 0$, the distribution $f$ is unique, and we say $f = \mathcal{R}F$ (in words: $f$ is the \emph{reconstruction of $F$}).
\end{theorem}






\section{The Reconstruction Theorem in Detail}

We state the Reconstruction Theorem of Hairer \cite{hairer2014theory} in the language of {distribution theory}.

\begin{theorem}[Reconstruction theorem]\label{theorem:reconstruction-theorem}
   Let $\gamma \in \mathbb{R}$ be a real number. Let $(F_x)_{x \in \mathbb{R}^d}$ be a $(\bm{\alpha}, \gamma)$-coherent germ with local homogeneity bounds $\bm \beta$. Then, there exists a distribution $f \in \mathcal{D}'$ such that for every compact set $K \subset \mathbb{R}^d$ and all $r \in \mathbb{N}$, $r> \max \left\{ -\alpha_{\bar K_2}, -\beta_{\bar K_2} \right\}$ we have TO-DO: there exists $C > 0$ 
   \begin{align*}\label{reconstruction-theorem}
        |(f-F_x)(\psi^\epsilon_x)| \leq C \cdot \begin{cases}
            \epsilon^\gamma \quad & \text{if $\gamma \neq 0$}\\
            1 + |\log \epsilon| \quad & \text{if $\gamma = 0$}
        \end{cases}\tag{\texttt{REC}}
        \\ \text{uniformly for $\psi \in \mathcal{B}_r$, $x \in K$, $\epsilon \in (0,1]$.} 
   \end{align*}
\end{theorem}

\begin{remark} 
   We gather some remarks about the reconstruction theorem.
\begin{enumerate}
   \item As usual, $\varphi$ denotes the test function defined in the coherence property \eqref{coherence}.
   \item $C$ is a constant, which must \emph{not} depend on $\psi,x$ and $\epsilon$. Precisely, it is given by \begin{align*}
           C = \mathrm{const}(\alpha_{\bar K_2}, \gamma, r, d, \varphi) \cdot \vertiii{F}^{\mathrm{coh}}_{\bar K_2, \varphi, \alpha_{\bar K_2}, \gamma}.
   \end{align*}
   The constant $\mathrm{const}(\alpha_{\bar K_2}, \gamma, r, d, \varphi) \in \mathbb{R}$ depends on $\alpha_{\bar K_2}, \gamma, r, d$ and $\varphi$.
   \item For $\gamma > 0$, $f = \mathcal{R}F$ is unique and we call it the \emph{reconstruction} of $F = (F_x)_{x \in \mathbb{R}^d}$. Moreover, the map $F \mapsto \mathcal{R}F$ is linear.
   \item For $\gamma \leq 0$, the distribution $f$ need not be unique, but for any fixed $\alpha \leq \min\left\{ 0, \gamma \right\}$, we can choose $f$ in such a way that the map $F \mapsto \mathcal{R}F$ is linear on the vector space of $(\alpha,\gamma)$-coherent germs with global homogeneity bound $\beta$.
   \item The \hyperref[peek:prelim-reconstruction-theorem]{preliminary reconstruction theorem} found in the previous section is a special case of the reconstruction theorem stated here. Proof. TO-DO ...
\end{enumerate}
\end{remark}

In the subsequent chapters, we will prove that the Reconstruction Theorem given that a germ is coherent (see Theorem \ref{theorem:reconstruction-theorem}). So, coherence is a \emph{sufficient} condition. But, there is more to that: coherence is also \emph{necessary} for a germ to be reconstructable in the sense of Theorem \ref{theorem:reconstruction-theorem}. We say that coherence is an \emph{optimal} condition.

\begin{theorem}[Coherence is necessary]\label{theorem:coherence-is-necessary}
   Fix any $\gamma \in \mathbb{R}$.  Let $(F_x)_{x \in \mathbb{R}^d}$ be a germ. Let $f \in \mathcal{D}'$ be a distribution such that for every compact set $K \subset \mathbb{R}^d$ there exists $C < \infty$ and $r \in \mathbb{N}$ with
   \begin{gather}\label{thm:coh-necessary}
        |(f-F_y)(\psi^\lambda_y)| \leq C \lambda^\gamma \\
        \text{for all $y \in K, \lambda \in (0,1]$ and $\psi \in \mathcal{B}_r$. \nonumber  }
   \end{gather}
   Then, $(F_x)_{x \in \mathbb{R}^d}$ is $\gamma$-coherent.  
\end{theorem}

\begin{proof}
   To show that $(F_x)_{x \in \mathbb{R}^d}$ is $\gamma$-coherent, it suffices to show that there exists $\alpha \leq \min\left\{ 0, \gamma \right\}$ and a constant $\hat C < \infty$ such that the following slightly modified coherence property holds 
   \begin{gather*}
       |(F_x - F_y)(\psi^\lambda_y)| \leq \hat C \lambda^\alpha(|x-y| + \lambda)^{ \gamma - \alpha} \\
       \text{$\forall x,y \in K$ with $|x-y| \leq \frac{1}{2}, \lambda \in (0, \frac{1}{2}]$ and $\psi \in \mathcal{B}_r$}.
   \end{gather*}  
   Together with TO-DO CUTOFF PROPERTY we then see that $(F_x)_{x \in \mathbb{R}^d}$ is $\gamma$-coherent.  

   Fix a compact set $K \subset \mathbb{R}^d$. Let $x,y \in K$ with $|x-y| \leq \frac{1}{2}$, $\lambda \in (0, \frac{1}{2}]$ and $\psi \in \mathcal{B}_r$. We begin with
   \begin{align*}
    |(F_x - F_y)(\psi^\lambda_y)| \leq |(F_x - f)(\psi^\lambda_y)| + |(f - F_y)(\psi^\lambda_y)| \overset{\eqref{thm:coh-necessary}}{\leq} |(F_x - f)(\psi^\lambda_y)| + C \lambda^\gamma.
   \end{align*}
   Next, estimating $|(F_x - f)(\psi^\lambda_y)|$ is a nontrivial task because $\psi_y^\lambda$ is centered around $y$ and not $x$. We overcome this obstacle by doing the substitution $\psi^\lambda_y \leadsto \xi^{\lambda_1}_x$, where 
   \begin{gather*}
       \xi \coloneqq \psi^{\lambda_2}_w, \quad w \coloneqq \frac{y-x}{|x-y| + \lambda}, \\
    \lambda_1 \coloneqq |x-y| + \lambda, \;\text{ and } \; \lambda_2 \coloneqq \frac{\lambda}{|x-y|  + \lambda}.
   \end{gather*}
   We quickly verify that this substitution is correct
   \begin{align*}
    \xi^{\lambda_1}_x = \frac{\psi\left(
        \lambda_2^{-1}\left(\frac{\cdot - x}{|x-y| + \lambda} - w\right)
    \right) }{\left((|x-y| + \lambda)\frac{\lambda}{|x-y| + \lambda}\right)^d}
    =
    \frac{\psi\left(
        \frac{\cdot - x - (y-x)}{\lambda}
    \right) }{\lambda^d} = \lambda^{-d}\psi\left( \frac{\cdot - y}{\lambda} \right) = \psi^\lambda_{y}.
   \end{align*}
   Hence, 
   \begin{align}
    |(F_x - f)(\psi^\lambda_y)| = |(F_x - f)\left( \xi^{\lambda_1}_x \lVert \xi \rVert_{C^{r}}^{-1} \right)| \cdot  \lVert \xi \rVert_{C^{r}} &\overset{\eqref{thm:coh-necessary}}{\leq} C\lambda_1^\gamma\lVert \xi \rVert_{C^{r}}. \label{SponsoredByGilette}
   \end{align}
   To justify that we can indeed use $\eqref{thm:coh-necessary}$, observe that $\lambda_1 \in (0, 1]$ and $\xi^{\lambda_1}_x \lVert \xi \rVert_{C^{r}}^{-1} \in \mathcal{B}_{r}$. The latter observation comes from $\lambda_2 + |w| = 1$ and $\mathrm{supp}(\psi) \subset B(0,1)$; both imply that $\xi = \psi^{\lambda_2}_w$ is supported in $B(0,1)$, and the scaling factor $\lVert \xi \rVert_{C^{r}}^{-1}$ ensures that the $C^r$ norm is always one.
   
   Additionally, $\lVert \xi \rVert_{C^{r}} = \max_{k \leq r}\lVert \partial^k \psi^{\lambda_2}_w \rVert_\infty = \max_{k \leq r} \lambda_2^{-d-k} \lVert \partial^k \psi \rVert_{\infty} \leq \lambda_2^{-d - r}$. So,
   \begin{align*}
    |(F_x - f)(\psi^\lambda_y)| \leq C\lambda_1^\gamma \lambda_2^{-d-r} &= C(|x-y| + \lambda)^\gamma\left(\frac{\lambda}{|x-y|  + \lambda}\right)^{-d-r} \\
    &\leq C  (|x-y| + \lambda)^{\gamma - \alpha} \lambda^\alpha,
   \end{align*}
    where we define $\alpha = \min\left\{ -d-r , \gamma \right\}$. 
    
    At last, 
    \begin{align*}
        |(F_x - F_y)(\psi^\lambda_y)| &\leq C  (|x-y| + \lambda)^{\gamma - \alpha} \lambda^\alpha + C \lambda^\gamma \\
        &\Downarrow \text{where $\lambda^\gamma = \lambda^{\gamma - \alpha} \lambda^\alpha \leq (|x-y| + \lambda)^{\gamma - \alpha} \lambda^\alpha$} \\
        &\leq 2C  (|x-y| + \lambda)^{\gamma - \alpha} \lambda^\alpha.
    \end{align*}
\end{proof}

We slightly modify the previous proof to prove that the constraint $|z-y| \leq 2$ in the coherence condition (see Defintion \ref{definition:coherence}) can be dropped, i.e. if \eqref{coherence} holds uniformly for any $y,z \in K$ with $|z-y| \leq 2$, then it also holds for any $\tilde y, \tilde z \in K$ with $|\tilde z- \tilde y| > 2$ (possibly with another multiplicate constant $C$). Hence, we could also define coherence by
\begin{gather}\label{better-coherence}
    |(F_z - F_y)(\varphi^\lambda_y)| \leq C\lambda^\alpha(|z-y| + \lambda)^{\gamma - \alpha}  \\ \text{uniformly for $z,y \in K$ and $\lambda \in (0,1]$} \nonumber
\end{gather}
since it is equivalent to Defintion \ref{definition:coherence} as the next proposition shows. 

\begin{proposition}\label{proposition:cutoff}
    Let $F$ be a $\gamma$-coherent germ as in Definition \ref{definition:coherence}. Then, it satisfies \eqref{better-coherence} for any compact set $K$ provided the multiplicative constant $C$ is adjusted.
\end{proposition}

\begin{proof}
    Let $F$ be a $\gamma$-coherent germ and $\varphi$ be as in Definition \ref{definition:coherence}. Fix a compact set $K \subset \mathbb{R}^d$. Assume $y,z \in K$ with $|y-z| > 2$. Let $A$ be a finite family of points in $\mathbb{R}^d$ such that $K$ is covered by $A$ and for each point $x \in K$ there exists $a_x \in A$ with $|x-a| < 2$. Such $A$ exists because $K$ is compact. Then, we have $|(F_z - F_y)(\varphi^\lambda_z)| \leq |(F_z - F_{a_z})(\varphi^\lambda_z)| + |(F_{a_z} - F_y)(\varphi^\lambda_z)|$. The first summand is bounded by \eqref{coherence}. Bounding the second summand is nontrivial since $\varphi$ is centered around $z$ and not $a_z$ or $y$. This is the same situation as in the proof of Theorem \ref{theorem:coherence-is-necessary}. So, we write
    \begin{align*}
        |(F_{a_z} - F_y)(\varphi^\lambda_z)| \leq |(F_{a_z} - f)(\varphi^\lambda_z)| + |(f - F_y)(\varphi^\lambda_z)|,
    \end{align*}
    where $f$ is the reconstruction of the germ $F$; note that $f$ exists by the Reconstruction Theorem and the Reconstruction Theorem only requires $F$ to be coherent in the sense of Definition \ref{definition:coherence}. Next, we use the same substitution as in \eqref{SponsoredByGilette} to obtain an upper bound for both summands. These upper bounds only depend on $|a_z - z|$, $|y - z|$ and $\lambda$, which ends the proof.  
\end{proof}

Next, we show uniqueness of the reconstruction. 

\begin{theorem}[Uniqueness]\label{theorem:uniqueness-reconstruction}
   Let $F = (F_x)_{x \in \mathbb{R}^d}$ be a germ and $\varphi \in \mathcal{D}$ be a test function with $\int \varphi(x) \mathrm{d}x \neq 0$. Let $K \subset \mathbb{R}^d$ be a compact set, and let $f, g \in \mathcal{D}'$ be any two distributions such that
   \begin{align*}
       \lim_{\lambda \to 0} |(f-F_x)(\varphi^\lambda_x)| &= 0 \quad \text{uniformly for $x$ in $K$} \\ 
       \quad \lim_{\lambda \to 0} |(g-F_x)(\varphi^\lambda_x)| &= 0 \quad \text{uniformly for $x$ in $K$}
   \end{align*} 
    Then, $f(\psi) = g(\psi)$ for all test functions $\psi \in \mathcal{D}(K)$.  
\end{theorem}

ss

\begin{proof}
    Define $F, \varphi, K, f$ and $g$ as in the theorem. Next, we define $T \coloneqq f - g$, fix $\psi \in \mathcal{D}(K)$ and show $T(\psi) = 0$.
   
    We assume that $\int \varphi(x) \mathrm{d} x = 1$ (otherwise we replace $\varphi$ by $(\int \varphi(x) \mathrm{d}x)^{-1}\varphi$). Then, the family $(\varphi^\lambda)_{\lambda \in (0,1]}$ is a mollifier, and thus $T(\psi) = \lim_{\lambda \to 0}T(\psi * \varphi^\lambda)$. This allows us to estimate 
    \begin{align*}
        |T(\psi * \varphi^\lambda)| = \left|\int T(\varphi^\lambda_x) \psi(x) \mathrm{d}x\right| \leq \lVert \psi \rVert_{L^1} \, \sup_{x \in K}|T(\varphi^\lambda_x)|,
    \end{align*}
    where for the last inequality we recall that $\psi$ has compact support in $K$. Using the triangle inequality, we bound 
    $$
        |T(\varphi^\lambda_x)| = |(f-g)(\varphi^\lambda_x)| \leq |(f-F_x)(\varphi^\lambda_x)| + |(g-F_x)(\varphi^\lambda_x)|.
    $$ 
    Taking the limit $\lambda \to 0$ proves the uniqueness theorem. 
\end{proof}

\chapter{On the Finiteness of Valid Integral Outcomes}

Let us devote the remaining chapters to the study of Theorem \ref{thm:outcome-degree-support-size}, which for the sake of convenience we restate below.

\begin{theorem*}
    The following upper bound 
    \begin{align*}
        \mathrm{deg}(\mathbf w) \leq 2 \cdot |\mathrm{supp}^+(\mathbf w)| - 3
    \end{align*}
    holds for valid integral outcomes \( \mathbf w \) with \( |\mathrm{supp}^+(\mathbf w)| \leq 5 \). 
\end{theorem*}

From now on, valid outcomes \( \mathbf{w} \) refer to \emph{integral} valid outcomes in \( \mathbb{Z}^{V_d} \) for \emph{finite} \( d \in \mathbb{N} \). 

\section{Invertibility Criterion}

Let \( d \in \mathbb{N} \).
One of the most important tools in the study of valid outcomes is the \emph{invertibility criterion} first introduced in \cite{bik2022classifying}. By Theorem \ref{thm:pascal-outcome} we can characterize \emph{valid} outcomes as the roots of all Pascal forms on \( \mathbb{Z}^{V_d} \). In the previous chapter we have already found two bases for the space of Pascal forms, namely \((\mathrm{row}(0), \dots, \mathrm{row}(d)) \) and \((\mathrm{col}(0), \dots, \mathrm{col}(d)) \) (see Definition \ref{def:row-col}). Let us introduce a \emph{new} basis for the space of Pascal forms.

\begin{definition}
    Let \( k = 0, \dots, d \) and \( \mathbf e_k \in \mathbb{R}^{d+1} \) be the \( k \)-th unit vector. We define \( \mathrm{diag}(k) \) to be the unique Pascal form \( \sum c_{i,j}x_{i,j} \) such that \( c_{k,d-k} = \mathbf e_k \).
\end{definition}

\begin{example}
    Fix the degree \( d = 7 \). We visualized \( \mathrm{diag}(3) \) by
    \begin{verbatim}
        .
        .   .
        .   .   .
        1   1   1   1
        4   3   2   1   .
       10   6   3   1   .   .
       20  10   4   1   .   .   . 
       35  15   5   1   .   .   .   .
    \end{verbatim}
\end{example}

\begin{proposition}
    For all integers \( k = 0, \dots, d \) we have:
    \begin{align*}
        \mathrm{diag}(k)  &= \sum_{(i,j) \in V_d}\binom{d - i - j}{k-i} x_{i,j}.
    \end{align*} 
    Note that \( \binom{a}{b} = 0 \) for \( b < 0 \) or \( b > a \).
\end{proposition}

\begin{proof}
    Note that for all \( (i,j) \in V_d \) with \( i+j = d \) we have \( \binom{d - i - j}{k-i} = 1 \) if and only if \( k= i \), and in all other cases \( k \neq i \) the binomial coefficient is zero. Thus, it remains to show that \( \sum_{(i,j) \in V_d}\binom{d - i - j}{k-i} x_{i,j} \) is a Pascal form. We have 
    \begin{align*}
        \binom{d-i-j}{k-i} = \binom{d-i-1-j}{k-i-1} + \binom{d-i-j-1}{k-i}.
    \end{align*}
    for all \( (i,j) \in V_{d-1} \) because \( \binom{a+1}{b+1} = \binom{a}{b+1} + \binom{a}{b}\).
\end{proof}

\begin{proposition}
    Let \( p \) be a Pascal form on \( \mathbb Z^{V_d} \). There exist unique coefficients \( \mu_0, \dots, \mu_d \in \mathbb{Z} \) such that 
    \( p = \mu_0 \mathrm{diag}(0) + \dots + \mu_d \mathrm{diag}(d) \).
\end{proposition}

\begin{proof}
    Let \( p = \sum c_{i,j}x_{i,j} \). Choose \( \mu_k = c_{k,d-k} \) for \( k=0, \dots, d \). Since \( p \) is a Pascal form, the coefficients \( c_{i,j} \) satisfy the Pascal recurrence relation. Thus, the coefficients \( \mu_k \) are uniquely determined.
\end{proof}

The invertibility criterion uses the diagonal basis \( (\mathrm{diag}(0), \dots, \mathrm{diag}(d)) \) to determine whether a given outcome is valid.

\begin{definition}
    Let \( E \subset \left\{ 0, \dots, d \right\} \) and \( S \subset V_d \) with \( \lvert E \rvert = \lvert S \rvert \neq 0 \). The \emph{pairing matrix} of \( (E,S) \) is definded as \( A^{(d)}_{E,S} \coloneqq \begin{bmatrix} \binom{d-i-j}{k-i} \end{bmatrix}_{k \in E, (i,j) \in S} \).
\end{definition}

\begin{example}
    Let \( d = 2 \), \( S = \left\{ (1,1), (0,0) \right\} \) and \( E = \left\{ 0,1 \right\} \). Then the pairing matrix is
    \begin{align*}
        A^{(d)}_{E,S}  = \begin{bmatrix}
            \binom{2-1-1}{0-1} & \binom{2-0-0}{0-2} \\
            \binom{2-1-1}{1-1}  & \binom{2-0-0}{1-2}
        \end{bmatrix} = \begin{bmatrix}
            0 & 0 \\
            1 & 0
        \end{bmatrix}.
    \end{align*}

    Now, assume \( \mathbf{w} \) is an outcome with support in \( S \). Since it is an outcome, we have \( \mathrm{diag}(k)(\mathbf{w}) = 0 \) for all \( k = 0, 1,2,3 \). Thus, 
    \begin{align*}
        A^{(d)}_{E,S} \mathbf w = \mathbf 0.
    \end{align*}
    We make the following observation: if the matrix \( A^{(d)}_{E,S} \) were invertible (it is not for the given example), then we would have \( \mathbf w = \mathbf 0 \). This is the invertibility criterion. 
\end{example}

\begin{proposition}[Invertibility Criterion]
    Let \( \mathbf{w} \) be an outcome with \( \mathrm{supp}(\mathbf w) \subset S \).
    If \( A^{(d)}_{E,S} \) is invertible, then \( \mathbf{w} = \mathbf 0 \).
\end{proposition}

\begin{proof}[Proof by Contraposition]
    Let \( \mathbf{w} \neq \mathbf 0 \). Its support is non-empty. Then, \( \mathbf w' \coloneqq (w_{i,j})_{(i,j) \in S} \neq \mathbf 0 \). So, \( A^{(d)}_{E,S} \cdot \mathbf w' = \mathbf 0 \). The kernel of the pairing matrix is non-trivial. Hence, the pairing matrix \( A^{(d)}_{E,S} \) is not invertible.
\end{proof}

Given a non-zero outcome \( \mathbf{w} \) we try to construct sets \( S \supset \mathrm{supp}(\mathbf w) \) and \( E \) such that the pairing matrix \( A_{E,S}^{(d)} \) is invertible. If we succeed, then \( \mathbf{w} \) is not a \emph{valid} outcome since the initial configuration is the only valid outcome with support in \( S \). If we fail, then \( \mathbf{w} \) may be or may not be a valid.

\section{Divide and Conquer}

The invertibility criterion is a powerful tool to determine whether a given outcome is valid. However, it is not always easy to find suitable sets \( S \) and \( E \) such that the pairing matrix is invertible. We will now introduce a method to construct such sets.
\chapter{Valid Outcomes of Positive Support Size \( \leq 3 \)}

We have all the tools ready to show the following three theorems. 

\begin{theorem}\label{thm:outcome-degree-support-size-232323}
    No valid integral outcomes of positive support size one exists.
\end{theorem}

\begin{theorem}\label{thm:outcome-degree-support-size-232323343}
    For valid integral outcomes \( \mathbf w \) with \( |\mathrm{supp}^+(\mathbf w)| = 2 \) we have \( \mathrm{deg}(\mathbf w) = 1 \).
\end{theorem}

\begin{theorem}
    For valid integral outcomes \( \mathbf w \) with \( |\mathrm{supp}^+(\mathbf w)| = 3 \) we have \( \mathrm{deg}(\mathbf w) \leq 3 \).
\end{theorem}

This proves our Main Theorem \ref{thm:outcome-degree-support-size} for the case of positive support size three or less, i.e. 
\begin{align*}
    \mathrm{deg}(\mathbf w) \leq 2 \cdot |\mathrm{supp}^+(\mathbf w)| - 3
\end{align*}
for all valid integral outcomes \( \mathbf w \) with \( |\mathrm{supp}^+(\mathbf w)| \leq 3 \). The proof of all the theorems were first presented in \cite{bik2022classifying}. 

We start with the proof of the first theorem.

\begin{proof}[Proof of Theorem \ref{thm:outcome-degree-support-size-232323}]
    Let \( \mathbf{w} \in \mathbb{Z}^{V_d} \) be a valid integral outcome. Since it is valid, we either have an empty negative support or a negative support that only contains \( (0,0) \). If the negative support is empty, then \( \mathbf{w} = \mathbf 0 \) by Proposition \ref{prop:outcome-zero}. Hence, we assume \( w_{0,0} < 0 \).

    Now, consider the Pascal form \( \mathrm{diag}(0) = \sum c_{i,j} x_{i,j} \). We have \( c_{0, 0} = c_{0, 1} = \dots = c_{0, d} = 1 \) and \( c_{i,j} = 0 \) for everything else. Similarly, we have for the Pascal form \( \mathrm{diag}(d) = \sum c'_{i,j} x_{i,j} \) that \( c'_{\cdot, 0} = \mathbf 1 \) and \( c'_{i,j} = 0 \) for everything else. Since outcomes are roots of Pascal forms, we have 
    \begin{align*}
        \mathrm{diag}(0)(\mathbf w) = \mathrm{diag}(d)(\mathbf w) = 0.
    \end{align*}
    Since \( w_{0,0} < 0 \) we must have \( w_{0,j} > 0 \) and \( w_{i, 0} > 0 \) for some \( i,j > 0 \). Hence, \( \mathbf{w} \) has positive support size at least two.
\end{proof}

Next, we prove the second theorem.

\begin{proof}[Proof of Theorem \ref{thm:outcome-degree-support-size-232323343}]
    Let \( \mathbf{w} \in \mathbb{Z}^{V_d} \) be an integral outcome with positive support size two and degree \( d \).
    By the previous proof, we see that 
    \begin{align*}
        \mathrm{supp}^+({\mathbf{w}}) = \left\{  (0,j), (i,0) \right\}.
    \end{align*}
    Without loss of generality, we assume \( i = d \). We want to show that \( j = d \). Consider the Pascal form \( \mathrm{row}(d) = \sum c_{i,j} x_{i,j} \), which has only nonzero coefficients \( c_{i,j} \) for \( i + j = d \).
    
    \begin{itemize}
        \item If the degree \( d \) is odd, we have \( c_{d,0} = 1 \) and \( c_{0,d} = -1 \). Since \( \mathrm{row}(d)(\mathbf{w}) = 0 \), we must have \( j = d \).

        \item If the degree \( d \) is even, we have \( c_{d,0} = c_{0,d} = 1 \). Thus, \( \mathrm{row}(d)(\mathbf w) \neq 0 \) for all \( j = 0, \dots, d \). Hence, valid outcomes with positive support size two do not exist for even degrees.
    \end{itemize}

    From now on, we assume 
    \begin{align*}
        \mathrm{supp}^+({\mathbf{w}}) = \left\{  (0,d), (d,0) \right\}.
    \end{align*}

    For sake of contradiction, let \( d \geq 2 \) (we can even assume that \( d \) is odd, but we do not need it). Then, we can divide the support 
    \begin{align*}
        \mathrm{supp}({\mathbf{w}}) = \left\{  (0,0) , (0,d), (d,0) \right\},
    \end{align*}
    via \( \lambda = (2,1,\dots,1) \) to obtain \( S_1 = \left\{ (0,0), (0,d) \right\} \), \( S_k = \emptyset \), and \( S_l = \left\{ (d,0) \right\} \). By Proposition \ref{prop:impossible-support-232423}, the pairing matrix induced by \( S_1 \) and \( E_1 = \left\{ 0,1 \right\} \) is invertible. For \( S_l \) we apply Proposition \ref{prop:impossible-support-23233243243423} and Remark \ref{rem:generality-jfknwejn} to get that the induced pairing matrix is invertible. By Corollary \ref{cor:invertibility-criterion-nooos}, the outcome \( \mathbf{w} \) is zero. This is a contradiction to the assumption that the positive support size is two. Hence, the degree \( d \) equals one.
\end{proof}



\chapter{Proof Continued for \texorpdfstring{\(\gamma \leq 0\)}{gamma <= 0}}\label{chapter:proof-gamma-negative}

The main idea of the proof for \(\gamma > 0\) was that \(f_n\) converges to some reconstruction \(f = \mathcal{R}f\) if \(\gamma > 0\), where \(f_n = f_1 + \sum\limits^{n-1}_{k=1} g_k'  + \sum\limits^{n-1}_{k=1} g_k''\), see~\eqref{approximating-distributions-alternative}.   
If however \(\gamma \leq 0\), the series \(\sum g_k''(\psi)\) need not converge. We fix this by ignoring \(\sum g_k''(\psi)\); the approximating distribution then reads \(f_1 + \sum\limits^{n-1}_{k=1} g_k'(\psi)\). We set 
\begin{align*}
    f^K(\psi) = f_1(\psi) + \lim_{n\to \infty}\sum^{n-1}_{k=1} g_k'(\psi),
\end{align*}
which is well-defined because \(\sum g_k'(\psi)\) converges for all \(\gamma \in \mathbb{R}\) (see Chapter~\ref{chapter:step-3}). Next, \(f^K\) is a distribution on \(\bar K_1\) because TO-DO

In the next steps, we will show that \(f^K\) satisfies~\eqref{reconstruction-theorem}, i.e.
\begin{gather*}
    |(f^K - F_x)(\psi_x^\lambda)| \leq \mathfrak{C} \vertiii{F}^{\mathrm{coh}}_{\bar K_{2},\varphi,\alpha,\gamma} \begin{cases}
        \lambda^\gamma  &\text{if \(\gamma < 0\)}\\
        1 + |\log(\lambda)| \quad &\text{if \(\gamma = 0\) } 
    \end{cases}
    \\
    \text{uniformly for \(\psi \in \mathcal{B}_r\), \(x \in K\) and \(\lambda \in(0,1]\).   }
\end{gather*}
where the constant \(\mathfrak{C}\) is given by TO-DO and TO-Do. We will then spend another chapter to build a \emph{global} distribution \(f \in \mathcal{D}'\) out of the local distributions \(f^K \in \mathcal{D}(\bar K_1)\) such that \(f\) satisfies~\eqref{reconstruction-theorem}, as well.

\section{Step 5: \texorpdfstring{\(f^K\) is a local reconstruction}{fK is a local reconstruction}}
 
We have the same setup as in Chapter~\ref{setup}: Step 0. Let \(K \subset \mathbb{R}^d\) be a compact set, \(x \in K\), \(\psi \in \mathcal{B}_r\) and \(\lambda \in (0,1]\). Then, we have
\begin{align*}
    |(f^K - F_x)(\psi_x^\lambda)|
    &= |((f_1 + \lim_{n\to \infty}\sum_{k=1}^{n-1}g_k') - F_x)(\psi_x^\lambda)| \\
    &= |(f_1 + \lim_{n\to \infty}\sum_{k=1}^{n-1}g_k')(\psi_x^\lambda) - \lim_{n\to \infty} F_x(\psi_x^\lambda* \rho^{\epsilon_n})| \\
    &= \lim_{n \to \infty} |\underbrace{f_1(\psi_x^\lambda) + \left\{ \sum_{k=1}^{n-1}g_k'(\psi_x^\lambda) \right\} - F_x(\psi_x^\lambda* \rho^{\epsilon_n})}_{\coloneqq \bar f_n(\psi_x^\lambda)}|.
\end{align*}  
Next, we write the above expression as a telescopic sum 
\begin{align*}
    |(f^K - F_x)(\psi_x^\lambda)| = 
    \lim_{n \to \infty} |\bar f_n(\psi_x^\lambda)| \leq 
    | \left ( \lim_{n \to \infty} \bar f_n(\psi_x^\lambda) \right ) - \bar f_N(\psi_x^\lambda)|
    + |\bar f_N(\psi_x^\lambda)|
\end{align*}
where \(N\) is chosen such that \(\epsilon_N \leq \lambda < \epsilon_{N - 1}\). The first summand is estimated by Lemma~\ref{technical-lemma-2}
\begin{align*}
    | \left( \lim_{n \to \infty} \bar f_n(\psi_x^\lambda)  \right) - \bar f_N(\psi_x^\lambda)| &\leq
    \sum_{k \geq N} |(\bar f_{k+1} - \bar f_{k})(\psi_x^\lambda)|\\ &= 
    \sum_{k \geq N} |g_k'(\psi^\lambda_x) - F_x(\psi^\lambda_x*(\rho^{\epsilon_{k+1}}-\rho^{\epsilon_k}))| \\
    &\Downarrow \text{Lemma~\ref{technical-lemma-2}} \\
    &\leq \left \{ \frac{\hat C 4^{\gamma - \alpha + d} \lVert \check \varphi \rVert_{L^1} }{1-2^{-\alpha - r}} \right \} \lambda^{\gamma}.
\end{align*}
The second summand \(|\bar f_N(\psi_x^\lambda)|\) is also bounded by Lemma~\ref{technical-lemma-2}
\begin{align*}
    |\bar f_N(\psi_x^\lambda)| &\leq |\bar f_1(\psi_x^\lambda)| + \sum^{N-1}_{k=1} |(\bar f_{k+1}  - \bar f_k)(\psi_x^\lambda)|\\
    &= |\bar f_1(\psi_x^\lambda)| + \sum^{N-1}_{k=1} |g_k'(\psi^\lambda_x) - F_x(\psi^\lambda_x*(\rho^{\epsilon_{k+1}}-\rho^{\epsilon_k}))| \\
    &\Downarrow \parbox{25em}{Lemma~\ref{technical-lemma-2} (use the case \(\epsilon_k \geq \epsilon_{N-1} > \lambda\))}\\
    &\leq  |\bar f_1(\psi_x^\lambda)| + \sum^{N-1}_{k=1} 4^{d+\gamma-\alpha}\hat C \lVert \check \varphi \rVert_{L^1} \epsilon^\gamma_k.
\end{align*}
Next, we observe
\begin{align*}
    |\bar f_1(\psi_x^\lambda)| &= |f_1(\psi_x^\lambda) - F_x(\psi_x^\lambda* \rho^{\epsilon_1})| \\
    &\Downarrow \text{use~\eqref{lemma:mollified-distribution}} \\
    &= \left | \int_{\mathbb{R}^d} F_z(\rho_z^{\epsilon_1}) \psi^\lambda_x(z)  \mathrm{d}z -  \int F_x(\rho_z^{\epsilon_1}) \psi^\lambda_x(z) \mathrm{d} z \right| \\
    &= \left | \int_{\mathbb{R}^d} (F_z - F_x)(\rho_z^{\epsilon_1}) \psi^\lambda_x(z)  \mathrm{d}z\right| \\
    &\Downarrow  \text{Recall \(\rho = \hat \varphi^2 * \hat \varphi\) and use~\eqref{lemma:mollified-distribution}} \\
    &= \left| \iint (F_z - F_x)(\hat \varphi^{\epsilon_1}_y) \hat \varphi^{2\epsilon_1}(y-z)\psi^\lambda_x(z) \; \mathrm{d}y \, \mathrm{d}z \right|.
\end{align*}
The tweaked test function \(\hat \varphi\) has a compact support in \(B(0, \frac{1}{2})\); hence \(\hat \varphi^{2\epsilon}\) is supported in \(B(0, \epsilon_1)\). Thus, the integral is nonzero if \(|y-z| \leq \epsilon_1 = \frac{1}{2}\). Additionally, we have \(|x-z|\leq \lambda\) because \(\psi^\lambda_x(z)\). So, we estimate
\begin{align*}
    |\bar f_1(\psi_x^\lambda)| &\leq  \sup_{\substack{z \in B(x, \lambda) \\ |y-z| \leq \frac{1}{2}}} \left| (F_z - F_x)(\hat \varphi^{\epsilon_1}_y) \right|  \cdot \lVert \hat \varphi^{2\epsilon_1} \rVert_{L^1} \lVert  \psi^{\lambda}_x \rVert_{L^1}.
\end{align*}
Moreover, \(z \in \bar K_1\) (recall that \(x \in K\) and \(\lambda \in (0,1]\)), \(y \in \bar K_{\frac{3}{2}}\) and \(|x-y| \leq |x-z| + |z-y| \leq \frac{3}{2}\). Hence, we use the triangle inequality and the coherence condition to obtain 
\begin{align*}
    \sup_{\substack{z \in B(x, \lambda) \\ |y-z| \leq
     \frac{1}{2}}} \left| (F_z - F_x)(\hat \varphi^{\epsilon_1}_y) \right| 
     &\leq \sup_{\substack{y, z \in \bar K_{3/2} \\ |y-z| \leq \frac{1}{2}}} \left| (F_z - F_y)(\hat \varphi^{\epsilon_1}_y) \right| + \sup_{\substack{x,y \in \bar K_{3/2} \\ |x-y| \leq \frac{3}{2}}} \left| (F_y - F_x)(\hat \varphi^{\epsilon_1}_y) \right| \\
     &\overset{\eqref{coherence-hat}}{\leq} \hat C \epsilon_1^\alpha(|z-y| + \epsilon_1)^{\gamma - \alpha} + \hat C \epsilon_1^\alpha(|y-x| + \epsilon_1)^{\gamma - \alpha}\\
     &\leq \hat C \left(\frac{3}{2}\right)^{\gamma - \alpha } + \hat C \left(\frac{5}{2}\right)^{\gamma - \alpha } \\
     &\leq 2\hat C \cdot  3^{\gamma - \alpha}.
\end{align*}
We bound
\begin{align*}
    |\bar f_1(\psi_x^\lambda)| \leq \left\{ 2\hat C \cdot  3^{\gamma - \alpha} \right\} \lVert \hat \varphi^{2\epsilon_1} \rVert_{L^1} \lVert  \psi^{\lambda}_x \rVert_{L^1} &= \left\{ 2\hat C \cdot  3^{\gamma - \alpha} \right\} \lVert \hat \varphi \rVert_{L^1} \lVert  \psi \rVert_{L^1} \\
    &\leq \left\{ 2\hat C \cdot  3^{\gamma - \alpha} \right\}  \lVert \hat \varphi \rVert_{L^1} \cdot \sup_{\psi \in \mathcal{B}_r} \lVert  \psi \rVert_{L^1} \\
    &\leq \left\{ 2\hat C \cdot  3^{\gamma - \alpha} \right\}  \lVert \hat \varphi \rVert_{L^1} \cdot 2^d \cdot \underbrace{\sup_{\psi \in \mathcal{B}_r} \lVert  \psi \rVert_{\infty}}_{\leq 1} \\
    &\leq 2^{d+1}\hat C \cdot  3^{\gamma - \alpha}  \lVert \hat \varphi \rVert_{L^1}.
\end{align*}
Then, we have \(|\bar f_N(\psi_x^\lambda)| \leq 2^{d+1}\hat C \cdot  3^{\gamma - \alpha}  \lVert \hat \varphi \rVert_{L^1} + \sum^{N-1}_{k=1} 4^{d+\gamma-\alpha}\hat C \lVert \check \varphi \rVert_{L^1} \epsilon^\gamma_k\). Also observe that 
\(\lVert \check \varphi \rVert_{L^1} = \int |\check \varphi(x)| \mathrm{d}x \leq \int|\hat \varphi^{\frac{1}{2}}(x)| + |\hat \varphi^{2}(x)| \mathrm{d}x = 2\int |\hat \varphi(x)| \mathrm{d}x = 2 \lVert \hat \varphi \rVert_{L^1}\). So, we get 
\begin{align*}
|\bar f_N(\psi_x^\lambda)| \leq 4^{d + \gamma - \alpha + 1} \hat C \lVert \hat \varphi \rVert_{L^1} \sum^{N-1}_{k=0}\epsilon^\gamma_k.
\end{align*}
Note that \(\sum^{N-1}_{k=0}\epsilon^\gamma_k\) is a geometric sum which we explicitly compute
\begin{align*}
    \sum^{N-1}_{k=0}\epsilon^\gamma_k = 
    \sum^{N-1}_{k=0}2^{-\gamma k} \leq \begin{dcases}
        \frac{\lambda^\gamma}{1-2^\gamma} \quad &\text{if \(\gamma < 0\) } \\
        \frac{\log(\frac{2}{\lambda})}{\log 2} &\text{if \(\gamma = 0\) }
    \end{dcases}.
\end{align*} 
Finally,
\begin{align*}
    |(f^K - F_x)(\psi_x^\lambda)| &\leq  \frac{\hat C 4^{\gamma - \alpha + d} \lVert \check \varphi \rVert_{L^1} }{1-2^{-\alpha - r}} \lambda^{\gamma} + 4^{d + \gamma - \alpha + 1} \hat C \lVert \hat \varphi \rVert_{L^1}\begin{dcases}
        \frac{\lambda^\gamma}{1-2^\gamma} \quad &\text{if \(\gamma < 0\) } \\
        \frac{\log(\frac{2}{\lambda})}{\log 2} &\text{if \(\gamma = 0\) }
    \end{dcases}.
\end{align*}
If \(\gamma < 0\), then 
\begin{align}
    |(f^K - F_x)(\psi_x^\lambda)| &\;\leq \left\{  \hat C\lVert \hat \varphi \rVert_{L^1}  \frac{4^{\gamma - \alpha + d + 1}}{1 - 2^{-\min\left\{ \alpha + r, - \gamma \right\}}} \right\} \lambda^\gamma \nonumber \\
    &\overset{\eqref{chatnorml1}}{\leq} \left\{ 
        \frac{r^2 2^{\alpha(-r-1)} 4^{d+ \gamma -\alpha + 6} \lVert \varphi \rVert_{L^1}}{1-2^{- \min \left\{ \alpha + r, |\gamma| \right\}} (1+R_{\varphi})^\alpha |\int \varphi(x) \, \mathrm{d}x|^2 }
        \vertiii{F}^{\mathrm{coh}}_{\bar K_{3/2}, \varphi, \alpha, \gamma}.
     \right\}
     \lambda^\gamma. \label{MacbookAirGammaNegative}
\end{align} 
Otherwise if \(\gamma = 0\), then we know \(\log(\frac{2}{\lambda}) (\log{2})^{-1} \leq 2(1 + |\log \lambda|)\). Thus
\begin{align*}
    |(f^K - F_x)(\psi_x^\lambda)| \leq \left\{ 
        \frac{r^2 2^{\alpha(-r-1)} 4^{d -\alpha + 6} \lVert \varphi \rVert_{L^1}}{1-2^{- \min \left\{ \alpha + r, |\gamma| \right\}} (1+R_{\varphi})^\alpha |\int \varphi(x) \, \mathrm{d}x|^2 }
        \vertiii{F}^{\mathrm{coh}}_{\bar K_{3/2}, \varphi, \alpha, \gamma}
     \right\}
     (1 + |\log \lambda|).
\end{align*} 
This shows that \(f^K\) is a local reconstruction. 



% ----------------------------------------



\section{Step 6: \texorpdfstring{Localization}{Localization}}\label{chapter:step6gammaNegative}


Similar to the case \(\gamma > 0\), we need to build a global distribution \(f\) from the local reconstructions \(f^K\). For that, we make use of a localization argument. First, we construct a partition of unity. Fix some test function \(\eta \in \mathcal{D}(B(0, \frac{1}{4}))\) such that \(\eta \geq 0\) on \(B(0, \frac{1}{4})\) and \(\eta \geq 1\) on \(B(0, \frac{1}{8})\). Define 
\begin{align*}
    \xi(x) = \frac{\eta(x)}{\sum\limits_{z \in E} \eta_z(x)} \quad \text{ where } E = \frac{1}{4\sqrt{d}} \mathbb{Z}^d.
\end{align*}
Note that \(\xi_y \in \mathcal{D}(B(y, \frac{1}{4}))\) for every \(y \in \mathbb{R}^d\) and \(\sum\limits_{y \in E} \xi_y \equiv 1\). We call \((\xi_y)_{y \in E}\) a \emph{partition of unity subordinated to the cover \(B(y, \frac{1}{4})_{y \in E}\)}. Define \(B_y = B(y, \frac{1}{4})\). Note that \(B_y\) has diameter \(\frac{1}{2}\). The global reconstruction \(f\) is then defined as
\begin{align*}
    f(\psi) = \sum_{y \in E} f^{B_y}(\xi_y \psi), \quad \forall \psi \in \mathcal{D}.
\end{align*}

Now, we show that \(f\) satisfies the Reconstruction Theorem. Fix a compact set \(K \subset \mathbb{R}^d\), and define 
\begin{align}\label{tropicalGeometryie}
    \alpha = \alpha_{\bar K_2}, \quad \beta = \beta_{\bar K_2}, \quad r > \max\left\{ -\alpha_{\bar K_2}, -\beta_{\bar K_2} \right\}.
\end{align}
Let \(\psi \in \mathcal{B}_r\), \(x \in K\) and \(\lambda \in (0,1]\). For \(\gamma < 0\) we have
\begin{align}\label{AbbaHunter}
    |(f-F_x)(\varphi^\lambda_x)| = |\sum_{y \in E} (f^{B_y} - F_x)(\xi_y \varphi^\lambda_x)| \leq \sum_{y \in E} | (f^{B_y} - F_x)(\xi_y \psi^\lambda_x) |.
\end{align} 
To justify the first equality, note that \(F_x(\varphi^\lambda_x) = F_x(\sum_{y \in E} \xi_y \varphi^\lambda_x) = \sum_{y \in E}F_x(\xi_y \varphi^\lambda_x)\) because \(\sum_{y \in E}\xi_y \equiv 1\).

Next, we make sure that we only sum over a finite number of \(y \in E\). Note that \(\xi_y\) has compact support in \(B(y, \frac{1}{4})\) and \(\psi^\lambda_x\) has compact support in \(B(x, \lambda)\). So, \(\xi_y \psi^\lambda_x \not \equiv 0\) only if \(|y-x| \leq |y - z| + |z-x| \leq \frac{1}{4} + \lambda \leq \frac{5}{4}\). There are at most \((2 \cdot \frac{5}{4} \cdot 4 \sqrt{d} + 1)^d \leq (11 \sqrt{d})^d\) many points \(y \in E\) that satisfy this. 

Then, we write 
\begin{align*}
    | (f^{B_y} - F_x)(\xi_y \psi^\lambda_x) | = | (f^{B_y} - F_x)(\zeta_x^\lambda) |
\end{align*}
for \(\zeta(z) = \xi_y(x + \lambda z) \psi(z)\). We would like to apply~\eqref{MacbookAirGammaNegative} for the compact set \(B_y\) and \(\zeta / \lVert \zeta \rVert_{C^r} \in \mathcal{B}_r\). 
Here, we need to be careful about \(\alpha\), \(\beta\) and \(r\) because we must check that~\eqref{MacbookAirGammaNegative} still holds if we choose \(\alpha\), \(\beta\) and \(r\) as in~\eqref{tropicalGeometryie}. 
Let \(\Gamma = \left\{ y_1,\ldots,y_n \right\} \subset \mathbb{R}^d\) such that \(y_i \cap K \not = \emptyset\) for all \(1 \leq i \leq n\). We have \(\bigcup_{y \in \Gamma}B_y \subset \enlarg{K}{\frac{1}{2}}\) because each ball \(B_y\) has diameter \(\frac{1}{2}\). So, the \(\frac{3}{2}\)-enlargement of \(\bigcup_{y \in \Gamma}B_y\) is contained in \(\enlarg{K}{2}\).
By~\eqref{alpha-monotone} and \(\eqref{beta-monotone}\), we know that the maps \(K \mapsto \alpha_K\) and \(K \mapsto \beta_K\) are monotone. So, we have \(\alpha_{\enlarg{K}{2}} \leq \alpha_{\overline{(\bigcup_{y \in \Gamma}B_y)}_{3/2}}\) and \(\beta_{\enlarg{K}{2}} \leq \beta_{\overline{(\bigcup_{y \in \Gamma}B_y)}_{3/2}}\). This, together with Step 5 (\(\gamma > 0\)), shows that~\eqref{MacbookAirGammaNegative} remains true for \(\alpha, \beta\) and \(r\). Applying~\eqref{MacbookAirGammaNegative} then yields
\begin{align*}
    | (f^{B_y} - F_x)(\zeta_x^\lambda) | = | (f^{B_y} - F_x)(\zeta_x^\lambda / \lVert \zeta \rVert_{C^r}) | \lVert \zeta \rVert_{C^r} &\leq \{ \mathrm{constant} \} \cdot \lVert \zeta \rVert_{C^r} \vertiii{F}^{\mathrm{coh}}_{\enlarg{K}{2}, \varphi, \alpha, \gamma} \lambda^\gamma \\
    &\Downarrow \text{Leibniz Rule and \(\sum_{k=0}^r \binom{r}{k} = 2^r\) }\\
    &\leq  2^r \{ \mathrm{constant} \}\lVert \xi \rVert_{C^r} \lVert \psi \rVert_{C^r} \vertiii{F}^{\mathrm{coh}}_{\enlarg{K}{2}, \varphi, \alpha, \gamma} \lambda^\gamma.
\end{align*} 
Continuing the estimate~\eqref{AbbaHunter} 
\begin{align*}
    |(f-F_x)(\varphi^\lambda_x)| &\overset{\eqref{AbbaHunter}}{\leq} \sum_{y \in E} | (f^{B_y} - F_x)(\xi_y \psi^\lambda_x) | \\ 
    &\;\leq \left\{ (11\sqrt{d})^d \left\{ \mathrm{constant} \right\} \lVert \xi \rVert_{C^r}\lVert \psi \rVert_{C^r}\vertiii{F}^{\mathrm{coh}}_{\enlarg{K}{2}, \varphi, \alpha, \gamma} \right\}\lambda^\gamma.
\end{align*}
The constant before \(\lambda^\gamma\)  then reads
\begin{align}\label{holy-molly}
   \left\{ 2^r \lVert \xi \rVert_{C^r} (11 \sqrt d)^d \frac{r^2 2^{-(r+1) \alpha} 4^{d + \gamma - \alpha + 6}}{1-2^{- \min\left\{ \alpha + r, -\gamma \right\}} |\int \varphi(x) \, \mathrm{d}x|^2  (1 + R_\varphi)^{\alpha}} \lVert \varphi \rVert_{L^1} \right\} \quad \text{in the case \(\gamma < 0\) }.
\end{align}
The proof for the case \(\gamma = 0\) is done similarly and gives the constant 
\begin{align}\label{holy-mollyl}
    \left\{ 2^r \lVert \xi \rVert_{C^r} (11 \sqrt d)^d \frac{r^2 2^{-(r+1) \alpha} 4^{d - \alpha + 6}}{1-2^{- \alpha - r} |\int \varphi(x) \, \mathrm{d}x|^2  (1 + R_\varphi)^{\alpha}} \lVert \varphi \rVert_{L^1} \right\} \quad \text{in the case \(\gamma = 0\) }.
\end{align}
This ends the proof of the Reconstruction Theorem for \(\gamma \leq 0\). 
\chapter{Valid Outcomes of Positive Support
Size Four}

We are now ready to prove that for every valid integral outcome \( \mathbf{w} \) with \( |\mathrm{supp}^+(\mathbf{w})| = 4 \), we have \( \mathrm{deg}(\mathbf{w}) \leq 5 \). As in previous chapters, outcomes are characterized as roots of Pascal forms. We define the two systems of Pascal forms that valid outcomes must satisfy: \( \Phi_1 \coloneqq \left\{ \mathrm{col}(i), \mathrm{row}(i), \mathrm{diag}(i), \mathrm{diag}(d-i) \right\}_{i=1}^3 \), \( \Phi_2 \coloneqq \left\{ \mathrm{row}(d-i), \mathrm{col}(d-i) \right\}_{i=0}^3 \), and \( \Phi \coloneqq \Phi_1 \cup \Phi_2 \).

By Proposition \ref{prop:contracted-part-1}, we can write all hyperfield forms induced by Pascal forms \( p \) in \( \Phi_1 \) as \( \mathrm{sign}(p) = \hat p \)
for some linear form \( \hat p \in H[\mathbf{x}, \mathbf{y}, \mathbf{z}, \mathbf{b}, \mathbf{c}, \mathbf{d}, \mathbf{e}] \) if \( d \geq 11 \). This linear form is independent of the degree \( d \). To make notations consistent later, we set \( \hat p^{\mathrm{even}} \coloneqq  \hat p^{\mathrm{odd}}  \coloneqq \hat p\).

Similarly, by Proposition \ref{prop:contracted-part-2}, we can write all hyperfield forms induced by Pascal forms \( p \) in \( \Phi_2 \) as \( \mathrm{sign}(p) = \begin{cases}
    \hat p^{\mathrm{even}} & \text{ if } d \text{ is even} \\
    \hat p^{\mathrm{odd}} & \text{ if } d \text{ is odd}
\end{cases} \), where \( \hat p^{\mathrm{even}}, \hat p^{\mathrm{odd}} \in H[\mathbf{x}, \mathbf{y}, \mathbf{z}, \mathbf{b}, \mathbf{c}, \mathbf{d}, \mathbf{e}] \) if \( d \geq 12 \). These linear forms \( \hat p^{\mathrm{even}}, \hat p^{\mathrm{odd}}  \) are independent of the degree \( d \).

\begin{definition}\label{def:sdjsndjknsdj}
    We define the following three solution sets:
    \begin{enumerate}
        \item     Define \( \Gamma_d \) to be the set of all valid hyperfield configurations \( \mathbf{s} \in H^{V_d} \) of degree \( d \) such that \( \mathrm{sign}(p)(\mathbf{s}) = H \) for all \( p \in \Phi \).

        \item     Define \( \Gamma^{\mathrm{even}} \) to be the set of all valid contracted hyperfield configurations \( \mathbf{s} \in H^{\Xi} \) such that \( \hat p^{\mathrm{even}}(\mathbf{s}) = H \) for all \( p \in \Phi \).

        \item     Define \( \Gamma^{\mathrm{odd}} \) to be the set of all valid contracted hyperfield configurations \( \mathbf{s} \in H^{\Xi} \) such that \( \hat p^{\mathrm{odd}}(\mathbf{s}) = H \) for all \( p \in \Phi \).
    \end{enumerate}
\end{definition}

By Proposition \ref{prop:hyperfield-criterion}, valid chipsplitting outcomes of degree \( d \) have supports in \( \Gamma_d \). This is the reason why we have defined \( \Gamma_d \) in the first place. 

\begin{proposition}\label{prop:sign-sikjsfnf3223423432}
    Let \( d \geq 12 \). If \( d \) is even, then \( \Gamma_d = \mathrm{contr}_d^{-1}(\Gamma^{\mathrm{even}}) \) holds. If \( d \) is odd, then \( \Gamma_d = \mathrm{contr}_d^{-1}(\Gamma^{\mathrm{odd}}) \) holds.
\end{proposition}

\begin{proof}
    Let \( d \geq 12 \) be even. Let \( \mathbf{s} \in {H}^{V_d} \) be a hyperfield configuration and \( p \in \Phi \). Then, we have \( \mathrm{sign}(p)(\mathbf{s}) = \hat p^{\mathrm{even}}(\mathrm{contr}_d(\mathbf{s})) \)
    by definition of \( \hat p^{\mathrm{even}} \). If \( \mathbf{s} \in \Gamma_d \), then \( H = \mathrm{sign}(p)(\mathbf{s}) = \hat p^{\mathrm{even}}(\mathrm{contr}_d(\mathbf{s})) \). Hence, \( \mathrm{contr}_d(\mathbf{s}) \) is contained in \( \Gamma^{\mathrm{even}} \). If \( \mathrm{contr}_d(\mathbf{s}) \in \Gamma^{\mathrm{even}} \) holds, using the equation above we also see that \( \mathbf{s} \in \Gamma_d \). This shows that \( \Gamma_d = \mathrm{contr}_d^{-1}(\Gamma^{\mathrm{even}}) \).

    The second statement for odd degrees \( d \) follows analogously.
\end{proof}

\begin{corollary}\label{cor:validwunfwufneuiw}
    Let \( d \geq 12 \) and \( \mathbf{w} \in \mathbb{Z}^{V_d} \) be a valid outcome. Then, \( \mathrm{contr}_d(\mathrm{sign}(\mathbf{w})) \in \Gamma^{\mathrm{even}} \cup \Gamma^{\mathrm{odd}} \) holds.
\end{corollary}

\begin{proof}
    Define \( \mathbf{s} \coloneqq \mathrm{sign}(\mathbf{w}) \). By Proposition \ref{prop:sign-sikjsfnf322} we have \( \mathbf{s} \in \Gamma_d \). If \( d \) is even, then \( \mathrm{contr}_d(\mathbf{s}) \in \Gamma^{\mathrm{even}} \) by the previous proposition. If \( d \) is odd, then \( \mathrm{contr}_d(\mathbf{s}) \in \Gamma^{\mathrm{odd}} \) by the previous proposition. This shows the claim.
\end{proof}

This corollary allows us to exclude certain outcomes as valid outcomes. Assume we have some contracted hyperfield configuration \( \xi \in H^{\Xi} \) that is not a root of some of the linear forms \( \hat p^{\mathrm{even}}, \hat p^{\mathrm{odd}} \) for \( p \in \Phi \). Then, any chipsplitting configuration \( \mathbf{w} \in \mathbb{Z}^{V_d} \) with \( \mathrm{contr}_d( \mathrm{sign}(\mathbf{w})) = \xi \) is not a valid outcome.

\begin{proposition}\label{prop:jasndkjsnjsnkjs}
    Let \( \mathbf{s} \in H^{V_d} \) be a valid hyperfield configuration of degree \( d \) with positive support size four or less. If \( d\geq 12 \), then \( \mathbf{s} \notin \Gamma_d \).
\end{proposition}

\begin{proof}
    Let \( d \geq 12 \). For computing \( \Gamma_d \) we could use Algorithm \ref{alg:hyperfield_criterion:efficient} for all \( d = 12, 13, 14, \dots \) and so on, which is not feasible since we would compute solutions sets for many infinitely many degrees \( d \). Instead, we show that \( \Gamma^{\mathrm{even}} \cup \Gamma^{\mathrm{odd}} \) is empty. By Proposition \ref{prop:sign-sikjsfnf3223423432}, \( \Gamma_d \) is empty as well for all \( d\geq 12 \).

    To show that \(  \Gamma^{\mathrm{even}} \) is empty, we can just use Algorithm \ref{alg:hyperfield_criterion:efficient} and Remark \ref{rem:fiuhwiu3} with \(A \coloneqq \left\{ \hat p^{\mathrm{even}} \mid p \in \Phi \right\}\). Similarly, we compute \( \Gamma^{\mathrm{odd}} = \emptyset \) with \(A \coloneqq \left\{ \hat p^{\mathrm{odd}} \mid p \in \Phi \right\}\) and Algorithm \ref{alg:hyperfield_criterion:efficient}. The results are in Appendix TODO. This shows the claim. TODO: show that \( A \) is non-trivial.
\end{proof}

\begin{theorem}\label{thm:main-result-32432432432nkdnjkfd}
    For valid integral outcomes \( \mathbf w \) with \( |\mathrm{supp}^+(\mathbf w)| = 4 \) we have \( \mathrm{deg}(\mathbf w) \leq 5 \).
\end{theorem}

\begin{proof}
    Let \( d \geq 6 \).
    Let \( \mathbf{w} \in \mathbb{Z}^{V_d} \) be a valid outcome with \( |\mathrm{supp}^+(\mathbf w)| = 4 \) and degree \( d \). We have \( \mathrm{sign}(\mathbf{w}) \in \Gamma_d \). By the previous proposition, there is no such \( \mathrm{sign}(\mathbf{w}) \) for \( d \geq 12 \). By Proposition \ref{prop:jdngkjrenj3nw}, the degree of \( \mathrm{sign}(\mathbf{w}) = d \) is six or seven. So, we just need to check eight cases. Of these eight cases, we can exclude all of them by applying Algorithm \ref{alg:hyperfield_criterion:is_zero}. The result of this algorithm is that only the zero outcome is possible for all these cases. This shows that the degree of \( \mathbf{w} \) is at most five.
\end{proof}

\printbibliography

% \appendix
% \chapter{More Monticello Candidates}

\end{document}
