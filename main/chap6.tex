
\chapter{Valid Outcomes of Positive Support Size Four}

We want to show that for all valid integral outcomes \( \mathbf w \) with \( |\mathrm{supp}^+(\mathbf w)| = 4 \) we have
\begin{align*}
    \mathrm{deg}(\mathbf w) \leq 2 \cdot |\mathrm{supp}^+(\mathbf w)| - 3 = 5.
\end{align*}
In the previous chapter, we showed that this inequality holds for all valid integral outcomes \( \mathbf w \) with \( |\mathrm{supp}^+(\mathbf w)| \leq 3 \) using the \emph{Invertibility Criterion}. For this chapter, we will need to introduce a new criterion, the \emph{Hyperfield Criterion}.

\section{Hyperfield Criterion}

Let us define the {sign hyperfield}. For some set \( A \), the set \( 2^A \) denotes the power set of \( A \).

\begin{definition}
    Let \( H \coloneqq \left\{ -1, 0, 1 \right\} \). We define the addition \( + : H \times H \to 2^H \setminus \left\{ \emptyset \right\} \) on \( H \) as follows
    \begin{align*}
        0 + x = \left\{ x \right\}  \quad \forall x \in H, \quad 1 + 1 = \left\{ 1 \right\}, \quad 1 + (-1) = H, \quad (-1) + (-1) = \left\{ -1 \right\}.
    \end{align*}
    Multiplication \( \times : H \times H \to H \) is defined as usual. We call \( H \) the \emph{sign hyperfield}.
\end{definition}

Often, for singleton sets \( \left\{ x \right\} \), we will write \( x \) instead of \( \left\{ x \right\} \). So, 
\begin{align*}
    1 + 1 = 1 \qquad \qquad \text{or} \qquad \qquad (-1) + 0 = -1.
\end{align*}

\begin{remark}
    The tuple \( (H, + , \cdot, 0, 1) \) is called a \emph{hyperfield}. A hyperfield satisfies the following properties:
    \begin{enumerate}
        \item  The maps \( + \) and \( \cdot \) are symmetric;
        \item \( (H \setminus \left\{ 0 \right\}, \cdot, 1) \) is a group;
        \item \( 0 \cdot x = 0 \) and \( 0 + x = x \) hold for all \( x \in H \);
        \item \( \bigcup_{q \in x+y}(q + z) = \bigcup_{q \in x + y}(x + q) \) hold for all \( x,y,z \in H \);
        \item \( a \cdot (x + y) = (a \cdot x) + (a \cdot y) \) hold for all \( a,x,y \in H \).
        \item An inverse element \( y  \in H\) exists for every \( x \in H\) such that the set \( x + y \) contains \( 0 \). This inverse element \( y \) is unique for every \( x \) and is denoted by \( -x \).
    \end{enumerate}

    Refer to \cite{bik2022classifying} Section 6.1 or \cite{baker2018matroids} for more details.
\end{remark}

Next, we define polynomials over the sign hyperfield.

\begin{definition}
    A polynomial in \( n \) variables \( x_1, \dots, x_n \) over \( H \) is a formal sum
    \begin{align*}
        f= \sum_{\mathbf{k} \in \mathbb{Z}^n_{\geq 0}} \lambda_{\mathbf{k}} \mathbf{x}^{\mathbf{k}}, \quad \lambda_{\mathbf{k}} \in H,
    \end{align*}
    where only a finite number of coefficients \( \lambda_{\mathbf{k}} \) are non-zero, and \( \mathbf{x}^{\mathbf{k}} = x_1^{k_1} \cdots x_n^{k_n} \). The set of all polynomials in \( n \) variables over \( H \) is denoted by \( H[x_1, \dots, x_n] \).

    Let \( \mathbf{x} \in H \). Then, we define 
    \begin{align*}
        f(\mathbf{x}) \coloneqq \sum_{\mathbf{k} \in \mathbb{Z}^n_{\geq 0}} \lambda_{\mathbf{k}} \mathbf{x}^{\mathbf{k}} \subset H.
    \end{align*}

    We say that \( f \) \emph{vanishes} at \( \mathbf{x} \in H \) if \( 0 \in f(\mathbf{x}) \). In this case, \( \mathbf{x} \) is a \emph{hyperfield root} of \( f \).
\end{definition}

Any \emph{real} polynomial can be turned into a polynomial over the sign hyperfield by replacing the coefficients with elements of \( H \). We can then evaluate the polynomial at any point in \( H \).

\begin{definition}
    Let \( f = \sum \lambda_{\mathbf{k}} \mathbf{x}^{\mathbf{k}} \in \mathbb{R}[\mathbf{x}] \) be a polynomial over \( \mathbb{R} \). We call 
    \begin{align*}
        \mathrm{sign}(f) \coloneqq \sum_{\mathbf{k} \in \mathbb{Z}^n_{\geq 0}} \mathrm{sign}(\lambda_{\mathbf{k}}) \mathbf{x}^{\mathbf{k}} \in H[\mathbf{x}]
    \end{align*}
    the polynomial over \( H \) induced by \( f \).
\end{definition}

For sake of simplicity, we also write for any real vector \( \mathbf{w} \in \mathbb{R}^n \):
\begin{align*}
    \mathrm{sign}(\mathbf{w}) \coloneqq (\mathrm{sign}(w_1), \dots, \mathrm{sign}(w_n)).
\end{align*}

\begin{example}
    Let \( d =5 \). Consider the Pascal forms on \( \mathbb{Z}^{V_d} \) generated by \( \mathrm{diag}(0) \), \(\mathrm{diag}(1)\), \(\mathrm{diag}(2), \mathrm{diag}(3), \mathrm{diag}(4) \) and \( \mathrm{diag}(5) \). The polynomial over \( H \) induced by these forms can be depicted as follows:
    \begin{verbatim}
+            ·            ·            ·            ·            ·
+ ·          + +          · ·          · ·          · ·          · ·
+ · ·        + + ·        + + +        · · ·        · · ·        · · ·
+ · · ·      + + · ·      + + + ·      + + + +      · · · ·      · · · ·
+ · · · ·    + + · · ·    + + + · ·    + + + + ·    + + + + +    · · · · ·
+ · · · · ·  + + · · · ·  + + + · · ·  + + + + · ·  + + + + + ·  + + + + + +
    \end{verbatim}
    Dots represent zero, and \( + \) represents one. Similarly, consider \( \mathrm{col}(0), \dots, \mathrm{col}(5) \):
    \begin{verbatim}
·            ·            ·            ·            ·            +
· ·          · ·          · ·          · ·          + +          · -
· · ·        · · ·        · · ·        + + +        · - -        · · +
· · · ·      · · · ·      + + + +      · - - -      · · + +      · · · -
· · · · ·    + + + + +    · - - - -    · · + + +    · · · - -    · · · · +
+ + + + + +  · - - - - -  · · + + + +  · · · - - -  · · · · + +  · · · · · -
    \end{verbatim}
    A minus sign \( - \) represents \( -1 \). For \( \mathrm{row}(0), \dots, \mathrm{row}(5) \) we have
    \begin{verbatim}
+            -            +            -            +            -
+ ·          - +          + -          - +          + -          · +
+ · ·        - + ·        + - +        - + -        · - +        · · -
+ · · ·      - + · ·      + - + ·      · + - +      · · + -      · · · +
+ · · · ·    - + · · ·    · - + · ·    · · - + ·    · · · - +    · · · · -
+ · · · · ·  · + · · · ·  · · + · · ·  · · · + · ·  · · · · + ·  · · · · · +
    \end{verbatim}
\end{example}

\begin{definition}
    A hyperfield Pascal form is just a polynomial over \( H \) induced by a Pascal form.
\end{definition}

The reason we introduced the sign hyperfield is that it allows us to neglect the concrete values of the coefficients of a polynomial and focus on their signs. This makes reasoning about roots easier, which is helpful since we saw in earlier chapters that chipsplitting outcomes are roots of Pascal forms.

\begin{proposition}\label{prop:sign-sikjsfnf}
    Let \( f \in \mathbb{R}[\mathbf{x}] \) be a real polynomial. Let \( \mathbf{w} \in \mathbb{R}^n \) be a root of \( f \). Then, \( \mathrm{sign}(\mathbf{w}) \) is a root of \( \mathrm{sign}(f) \).
\end{proposition}

\begin{proof}
    Define \( \mathbf{s} \coloneqq \mathrm{sign}(\mathbf{w}) \). Write \( f = \sum \lambda_{\mathbf{k}} \mathbf{x}^{\mathbf{k}} \) with real coefficients \( \lambda_{\mathbf{k}} \). If \( \lambda_{\mathbf{k}} \mathbf{w}^{\mathbf{k}} = 0 \) for all \( \mathbf{k} \in \mathbb{Z}^n_{\geq 0} \), then clearly the sign of \( \lambda_{\mathbf{k}} \mathbf{w}^{\mathbf{k}} \) is zero; hence the sign of \( f \) is the singleton set \( \left\{ 0 \right\} \) when evaluated at \( \mathbf{s} \). So, \( \mathbf{s} \) is a root of \( \mathrm{sign}(f) \).

    Now, suppose that there exists some \( \mathbf{k} \in \mathbb{Z}^n_{\geq 0} \) such that \( \lambda_{\mathbf{k}} \mathbf{w}^{\mathbf{k}} \neq 0 \). Assume \(  \lambda_{\mathbf{k}} \mathbf{w}^{\mathbf{k}} > 0 \). Then, there also exists some \( \mathbf{j} \in \mathbb{Z}^n_{\geq 0} \) such that we have \( \lambda_{\mathbf{j}} \mathbf{w}^{\mathbf{j}} < 0 \); otherwise \( f(\mathbf{w}) > 0 \) which is a contradiction to \( \mathbf{w} \) being a root of \( f \). Thus, \( \mathrm{sign}(f)(\mathbf{s}) \) has summands of both signs, and hence \( \mathrm{sign}(f)(\mathbf{s}) = H \). So \( 0 \in  \mathrm{sign}(f)(\mathbf{s}) \) holds. Therefore, \( \mathbf{s} \) is a root of \( \mathrm{sign}(f) \).
\end{proof}

Taking the contrapositive of the above proposition, we get the \emph{Hyperfield Criterion}.

\begin{proposition}[Hyperfield Criterion]
    Let \( \mathbf{s} = (s_{i,j})_{(i,j) \in V_d} \in H^{V_d} \). Let \( \mathbf{w} \in \mathbb{Z}^{V_d} \) be a chipsplitting configuration with \( \mathrm{sign}(\mathbf{w}) = \mathbf{s} \). If \( \mathbf{s} \) is not a root of a hyperfield Pascal form, then \( \mathbf{w} \) is not a chipsplitting outcome.
\end{proposition}

\begin{proof}
    Follows from Proposition \ref{prop:sign-sikjsfnf} and Theorem \ref{thm:pascal-outcome}.
\end{proof}

We call a vector \( \mathbf{s} \in H^{V_d} \) a hyperfield configuration. For completeness, we state standard definitions for hyperfield configurations \( \mathbf{s} \in H^{V_d} \) similar to Definition \ref{def:chip-terminology}.