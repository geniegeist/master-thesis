\chapter{Computation of Fundamental Models}

In the final chapter, we compute the number of fundamental models. The implementation details are publicly available in the repository \cite{ducrepo}. The results of these computations are summarized in the following table.

\begin{figure}[H]
    \centering
    \[
    \begin{array}{|c|ccccccccccc|}
    \hline
    n \setminus d & 1 & 2 & 3 & 4 & 5 & 6 & 7 & 8 & 9 & 10 & 11 \\
    \hline
    2 & 1 &   &   &   &   &    &    &    &    &     &     \\
    3 &   & 3 & 1 &   &   &    &    &    &    &     &     \\
    4 &   &   & 12 & 4 & 2 &    &    &    &    &     &     \\
    5 &   &   &    & 82 & 38 & 10 & 4  &    &    &     &     \\
    6 &   &   &    &    & 602 & 254 & 88 & 24 & 2  &     &     \\
    7 &   &   &    &    &     & 6710 & 2421 & 643 & 198 & 32  & 4   \\
    \hline
    \end{array}
    \]
    \caption{The number of fundamental outcomes for each positive support size \( n \) and degree \( d \). For \( n = 2, 3, 4, 5 \) there are no additional columns beyond those shown as we have proven that the degree is bounded.}
    \label{table:computed-fundamental-models}
    \end{figure}


To construct this table, we employed the following methodology. First, we calculated the set of all supports of valid outcomes for a fixed degree \( d\in \mathbb{N} \) using Algorithm \ref{alg:hyperfield_criterion:efficient}. For each chipsplitting support generated, we mapped it back to a statistical model and computed the rank of the corresponding linear system to determine whether it yields a unique solution. If the system is of full rank, the associated statistical model is fundamental, as defined in Definition \ref{def:fundamental-model}. By using this approach, we computed all valid outcomes for positive support sizes \( n = 1, \dots, 7 \) and degree \( d \leq 11 \). This extends the results of \cite{bik2022classifying} for \( n =7 \).