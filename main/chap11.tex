\chapter{Discussion}

This thesis establishes a connection between the classification of discrete statistical models (Theorem \ref{thm:degree-fundamental-models}) and a combinatorial puzzle related to chipsplitting games (Theorem \ref{thm:outcome-degree-support-size}). Specifically, the puzzle investigates whether the degree of a valid chipsplitting outcome can grow indefinitely while its support size remains fixed. For outcomes with positive support sizes up to five, we prove that the degree cannot grow indefinitely, providing a definitive negative answer.

For outcomes with a positive support size of six, significant progress was made toward a similar conclusion. By employing systematic reductions, the number of cases requiring analysis was reduced from approximately 300,000 to 12,000, indicating that a negative answer may hold for this case as well. Furthermore, computational evidence strengthens this hypothesis; our calculations in Table TODO reveal that for positive support size $n=6$, there are no valid outcomes of degree $d \in \{12,\dots,TODO\}$. This pattern suggests that the degree of valid outcomes with positive support size six is bounded above by nine.

To construct Table TODO, we employed the following methodology. First, we calculated the set of all supports of valid outcomes for a fixed degree \( d\in \mathbb{N} \) using Algorithm \ref{alg:hyperfield_criterion:efficient}. For each chipsplitting support generated, we mapped it back to a statistical model and solved the corresponding linear system to determine whether it yields a fundamental model, as defined in Definition \ref{def:fundamental-model}. The full implementation details and code are available in the repository TODO.

By using this approach, we computed all valid outcomes for positive support sizes \( n = 1, \dots, 7 \) and degree \( d \leq 11 \). This extends the results of Bik and Marigliano \cite{bik2022classifying} by one additional support size, made possible by a more efficient implementation.