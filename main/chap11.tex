\chapter{Discussion}

This thesis establishes a connection between the classification of discrete statistical models (Theorem \ref{thm:degree-fundamental-models}) and a combinatorial puzzle related to chipsplitting games (Theorem \ref{thm:outcome-degree-support-size}). Specifically, the puzzle investigates whether the degree of a valid chipsplitting outcome can grow indefinitely while its support size remains fixed. For outcomes with positive support sizes up to five, we prove that the degree cannot grow indefinitely, providing a definitive negative answer.

For outcomes with a positive support size of six, significant progress was made toward a similar conclusion. By employing systematic reductions, the number of cases requiring analysis was reduced from approximately 300,000 to 12,000, indicating that a negative answer may hold for this case as well. Furthermore, computational evidence strengthens this hypothesis; our calculations in Table \ref{table:computed-fundamental-models} reveal that for positive support size $n=6$, there are no valid outcomes of degree $d \in \left\{ 10, \dots, TODO \right\}$. This pattern suggests that the degree of valid outcomes with positive support size six is bounded above by nine.


\begin{figure}[H]
    \centering
    \[
    \begin{array}{|c|ccccccccccc|}
    \hline
    n \setminus d & 1 & 2 & 3 & 4 & 5 & 6 & 7 & 8 & 9 & 10 & 11 \\
    \hline
    2 & 1 &   &   &   &   &    &    &    &    &     &     \\
    3 &   & 3 & 1 &   &   &    &    &    &    &     &     \\
    4 &   &   & 12 & 4 & 2 &    &    &    &    &     &     \\
    5 &   &   &    & 82 & 38 & 10 & 4  &    &    &     &     \\
    6 &   &   &    &    & 602 & 254 & 88 & 24 & 2  &     &     \\
    7 &   &   &    &    &     & 6710 & 2421 & 643 & 198 & 32  & 4   \\
    \hline
    \end{array}
    \]
    \caption{The number of fundamental outcomes for each positive support size \( n \) and degree \( d \). For \( n = 2, 3, 4, 5 \) there are no additional columns beyond those shown as we have proven that the degree is bounded.}
    \label{table:computed-fundamental-models}
    \end{figure}
    

To construct Table \ref{table:computed-fundamental-models}, we employed the following methodology. First, we calculated the set of all supports of valid outcomes for a fixed degree \( d\in \mathbb{N} \) using Algorithm \ref{alg:hyperfield_criterion:efficient}. For each chipsplitting support generated, we mapped it back to a statistical model and solved the corresponding linear system to determine whether it yields a fundamental model, as defined in Definition \ref{def:fundamental-model}. The full implementation details and code are available in the repository TODO.

By using this approach, we computed all valid outcomes for positive support sizes \( n = 1, \dots, 7 \) and degree \( d \leq 11 \). This extends the results of Bik and Marigliano \cite{bik2022classifying} by one additional support size, made possible by a more efficient implementation.
