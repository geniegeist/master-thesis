\chapter{On the Finiteness of Valid Integral Outcomes}

Let us devote the remaining chapters to the study of Theorem \ref{thm:outcome-degree-support-size}, which for the sake of convenience we restate below.

\begin{theorem*}
    The following upper bound 
    \begin{align*}
        \mathrm{deg}(\mathbf w) \leq 2 \cdot |\mathrm{supp}^+(\mathbf w)| - 3
    \end{align*}
    holds for valid integral outcomes \( \mathbf w \) with \( |\mathrm{supp}^+(\mathbf w)| \leq 5 \). 
\end{theorem*}

From now on, valid outcomes \( \mathbf{w} \) refer to \emph{integral} valid outcomes in \( \mathbb{Z}^{V_d} \) for \emph{finite} \( d \in \mathbb{N} \). 

\section{Invertibility Criterion}

Let \( d \in \mathbb{N} \).
One of the most important tools in the study of outcomes is the \emph{invertibility criterion} first introduced in \cite{bik2022classifying}. By Theorem \ref{thm:pascal-outcome} we can characterize outcomes as the roots of all Pascal forms on \( \mathbb{Z}^{V_d} \). In the previous chapter we have already found two bases for the space of Pascal forms, namely \((\mathrm{row}(0), \dots, \mathrm{row}(d)) \) and \((\mathrm{col}(0), \dots, \mathrm{col}(d)) \) (see Definition \ref{def:row-col}). Let us introduce a \emph{new} basis for the space of Pascal forms.

\begin{definition}
    Let \( k = 0, \dots, d \) and \( \mathbf e_k \in \mathbb{R}^{d+1} \) be the \( k \)-th unit vector. We define \( \mathrm{diag}(k) \) to be the unique Pascal form \( \sum c_{i,j}x_{i,j} \) such that \( c_{k,d-k} = \mathbf e_k \).
\end{definition}

\begin{example}
    Fix the degree \( d = 7 \). We visualized \( \mathrm{diag}(3) \) by
    \begin{verbatim}
        .
        .   .
        .   .   .
        1   1   1   1
        4   3   2   1   .
       10   6   3   1   .   .
       20  10   4   1   .   .   . 
       35  15   5   1   .   .   .   .
    \end{verbatim}
\end{example}

\begin{proposition}
    For all integers \( k = 0, \dots, d \) we have:
    \begin{align*}
        \mathrm{diag}(k)  &= \sum_{(i,j) \in V_d}\binom{d - i - j}{k-i} x_{i,j}.
    \end{align*} 
    Note that \( \binom{a}{b} = 0 \) for \( b < 0 \) or \( b > a \).
\end{proposition}

\begin{proof}
    Note that for all \( (i,j) \in V_d \) with \( i+j = d \) we have \( \binom{d - i - j}{k-i} = 1 \) if and only if \( k= i \), and in all other cases \( k \neq i \) the binomial coefficient is zero. Thus, it remains to show that \( \sum_{(i,j) \in V_d}\binom{d - i - j}{k-i} x_{i,j} \) is a Pascal form. We have 
    \begin{align*}
        \binom{d-i-j}{k-i} = \binom{d-i-1-j}{k-i-1} + \binom{d-i-j-1}{k-i}.
    \end{align*}
    for all \( (i,j) \in V_{d-1} \) because \( \binom{a+1}{b+1} = \binom{a}{b+1} + \binom{a}{b}\).
\end{proof}

\begin{proposition}
    Let \( p \) be a Pascal form on \( \mathbb Z^{V_d} \). There exist unique coefficients \( \mu_0, \dots, \mu_d \in \mathbb{Z} \) such that 
    \( p = \mu_0 \mathrm{diag}(0) + \dots + \mu_d \mathrm{diag}(d) \).
\end{proposition}

\begin{proof}
    Let \( p = \sum c_{i,j}x_{i,j} \). Choose \( \mu_k = c_{k,d-k} \) for \( k=0, \dots, d \). Since \( p \) is a Pascal form, the coefficients \( c_{i,j} \) satisfy the Pascal recurrence relation. Thus, the coefficients \( \mu_k \) are uniquely determined.
\end{proof}

Given some set of vertices \( S \subset V_d \) the invertibility criterion uses the diagonal basis \( (\mathrm{diag}(0), \dots, \mathrm{diag}(d)) \) to determine whether a nonzero outcome with support in \( S \) exists.

\begin{definition}
    Let \( E \subset \left\{ 0, \dots, d \right\} \) and \( S \subset V_d \) with \( \lvert E \rvert = \lvert S \rvert \neq 0 \). The \emph{pairing matrix} of \( (E,S) \) is definded as \( A^{(d)}_{E,S} \coloneqq \begin{bmatrix} \binom{d-i-j}{k-i} \end{bmatrix}_{k \in E, (i,j) \in S} \).
\end{definition}

\begin{example}
    Let \( d = 2 \), \( S = \left\{ (1,1), (0,0) \right\} \) and \( E = \left\{ 0,1 \right\} \). Then the pairing matrix is
    \begin{align*}
        A^{(d)}_{E,S}  = \begin{bmatrix}
            \binom{2-1-1}{0-1} & \binom{2-0-0}{0-2} \\
            \binom{2-1-1}{1-1}  & \binom{2-0-0}{1-2}
        \end{bmatrix} = \begin{bmatrix}
            0 & 0 \\
            1 & 0
        \end{bmatrix}.
    \end{align*}

    Now, assume \( \mathbf{w} \) is an outcome with support in \( S \). Since it is an outcome, we have \( \mathrm{diag}(k)(\mathbf{w}) = 0 \) for all \( k = 0, 1,2,3 \). Thus, 
    \begin{align*}
        A^{(d)}_{E,S} \mathbf w = \mathbf 0.
    \end{align*}
    We make the following observation: if the matrix \( A^{(d)}_{E,S} \) were invertible (it is not for the given example), then we would have \( \mathbf w = \mathbf 0 \); so in this case the initial configuration \( \mathbf{0} \) is the \emph{only} outcome with support in \( S \). This is the invertibility criterion. 
\end{example}

\begin{proposition}[Invertibility Criterion]
    Let \( \mathbf{w} \) be an outcome with \( \mathrm{supp}(\mathbf w) \subset S \).
    If \( A^{(d)}_{E,S} \) is invertible, then \( \mathbf{w} = \mathbf 0 \).
\end{proposition}

\begin{proof}[Proof by Contraposition]
    Let \( \mathbf{w} \neq \mathbf 0 \). Its support is non-empty. Then, \( \mathbf w' \coloneqq (w_{i,j})_{(i,j) \in S} \neq \mathbf 0 \). So, \( A^{(d)}_{E,S} \cdot \mathbf w' = \mathbf 0 \). The kernel of the pairing matrix is non-trivial. Hence, the pairing matrix \( A^{(d)}_{E,S} \) is not invertible.
\end{proof}

Given a non-zero configuration \( \mathbf{w} \) we try to construct sets \( S \supset \mathrm{supp}(\mathbf w) \) and \( E \) such that the pairing matrix \( A_{E,S}^{(d)} \) is invertible. If we succeed, then \( \mathbf{w} \) is \emph{not} an outcome since the initial configuration is the only valid outcome with support in \( S \). 

\section{Divide and Conquer}

The invertibility criterion is a powerful tool to determine whether a given configuration is an outcome. However, it is not always easy to find suitable sets \( S \) and \( E \) such that the pairing matrix is invertible. We will now introduce a method to construct such sets.

\subsection*{Divide}

Instead of finding one large set \( S \) with \( \mathrm{supp}(\mathbf w) \subset S \), we divide \( S \) into smaller sets \( S_1, \dots, S_l \). These smaller sets \( S_1, \dots, S_k \) will be implicitly defined by integers \( \lambda_1, \dots, \lambda_l \in \mathbb{N} \) as we will shortly see. We choose \( l \in \mathbb{N} \) and integers \( \lambda_1, \dots, \lambda_l \in \mathbb{N} \) such that for all \( i=1, \dots, d \) we have
\begin{align*}
    \lvert S_i \rvert &\in \left\{ 0, \lambda_i \right\} \\
    S_i &\coloneqq \left\{ (i,j) \in \mathrm{supp}(\mathbf w) : i = c_{k-1}, \dots, c_k - 1 \right\} \\
    c_i &\coloneqq \lambda_1 + \dots + \lambda_i, \\
    \lambda_1 + \dots + \lambda_l &= d+1.
\end{align*}

Such decomposition will always work when \( \lvert \left\{ (i,j) \in \mathrm{supp}(\mathbf{w}) : i \geq d-k \right\} \rvert \leq k+1 \) for all \( k = 0, \dots, d \). This is because we can always choose \( \lambda_1  \) minimal such that \( \lvert S_1 \rvert \in \left\{ 0, \lambda_1 \right\} \). We repeat this process until \( c_l = d+1 \). This decomposition is illustrated in the following example.

\begin{example}\label{ex:decomposition-nsjkfnje}
    Fix the degree \( d=6 \). Assume we have some configuration \( \mathbf w \in \mathbb{Z}^{V_6} \) with support in the positions marked with an \texttt{*} below.
    \begin{verbatim}
        · 
        · · 
        * · · 
        · · · · 
        · · · · · 
        · · · · * · 
        * · * · · * *
    \end{verbatim}
    The first column contains two non-zero entries. So we see \( \lambda_1 = 2 \). Then, we conclude that \( \lambda_2 = \lambda_3 = \lambda_4 = \lambda_5 = \lambda_6 = 1\); otherwise \(     \lvert S_i \rvert \notin \left\{ 0, \lambda_i \right\}     \) for \( i>1 \).
\end{example}


Next with \( S_i \) defined, we define for all \( i=1, \dots, l \) the sets
\begin{align*}
    E_i \coloneqq \begin{cases}
        \left\{ c_{i-1}, \dots, c_i - 1 \right\} & \text{if } \lvert S_i \rvert = \lambda_i, \\
        \emptyset & \text{if } \lvert S_i \rvert = 0.
    \end{cases}
\end{align*}
We see that \( \lvert E_i \rvert = \lvert S_i \rvert \) for all \( i = 1, \dots, l \). 

\subsection*{Conquer}

Given some support \( S \), we divide it into smaller sets \( S_1, \dots, S_l \) as described above. We also define sets \( E_1, \dots, E_l \). Write \( E \coloneqq E_1 \cup \dots \cup E_l \).

\begin{proposition}
    We have 
    \begin{align*}
        A_{E,S}^{(d)} = 
        \begin{bmatrix}
            A_{E_1,S_1}^{(d)} & 0 & \dots & 0 \\
            A_{E_2,S_1}^{(d)} & A_{E_2,S_2}^{(d)} & \dots & 0 \\
            \vdots & \vdots & \ddots & \vdots \\
            A_{E_l,S_1}^{(d)} & A_{E_l,S_2}^{(d)} & \dots & A_{E_l,S_l}^{(d)}
        \end{bmatrix}.
    \end{align*}
\end{proposition}

\begin{proof}
    We have 
    \begin{align*}
        A_{E,S}^{(d)} = 
        \begin{bmatrix}
            A_{E_1,S_1}^{(d)} & A_{E_1,S_2}^{(d)} & \dots & A_{E_1,S_l}^{(d)} \\
            A_{E_2,S_1}^{(d)} & A_{E_2,S_2}^{(d)} & \dots & A_{E_2,S_l}^{(d)} \\
            \vdots & \vdots & \ddots & \vdots \\
            A_{E_l,S_1}^{(d)} & A_{E_l,S_2}^{(d)} & \dots & A_{E_l,S_l}^{(d)}
        \end{bmatrix}.
    \end{align*}
    Let \( x,y = 1, \dots , l \) such that \( x < y \). Let \( k \in E_x \) and \( (i,j) \in S_y \).
    Then, \( k \leq c_x - 1 < c_x \leq c_{y - 1} \leq i \); so \( k - i < 0 \). Thus, \( \binom{d-i-j}{k-i} = 0 \). This implies that the upper off-diagonal blocks are zero.
\end{proof}

\begin{corollary}
    The pairing matrix \( A^{(d)}_{E,S} \) is invertible if and only if \( A^{(d)}_{E_1,S_1}, \dots,  A^{(d)}_{E_l,S_l} \) are invertible.
\end{corollary}

\begin{corollary}[Invertibility Criterion, Divide and Conquer]
    Let \( \mathbf{w} \) be an outcome with \( \mathrm{supp}(\mathbf w) \subset S \).
    If \( A^{(d)}_{E_1,S_1}, \dots,  A^{(d)}_{E_l,S_l} \) are invertible, then \( \mathbf{w} = \mathbf 0 \).
\end{corollary}

\begin{example}
    We continue Example \ref{ex:decomposition-nsjkfnje}. 
    \begin{verbatim}
        · 
        · · 
        * · · 
        · · · · 
        · · · · · 
        · · · · * · 
        * · * · · * *
    \end{verbatim}
    With \( \lambda = (2,1,1,1,1,1) \) we obtain the following decomposition
    \begin{align*}
        S_1 = \left\{ (0,0), (0,4) \right\}, S_2 = \left\{ (2,0) \right\}, S_3 = \emptyset, S_4 = \left\{ (4,1) \right\}, S_5 = \left\{ (5,0) \right\}, S_6 = \left\{ (6,0) \right\},
    \end{align*}
    and
    \begin{align*}
        E_1 = \left\{ 0, 1 \right\}, E_2 = \left\{ 2 \right\}, E_3 = \emptyset, E_4 = \left\{ 4 \right\}, E_5 = \left\{ 5 \right\}, E_6 = \left\{ 6 \right\}.
    \end{align*}
    Then, the pairing matrix reads 
    \begin{align*}
        A^{(d)}_{E,S} = \begin{bmatrix}
            1 & 1 & 0 & 0 & 0 & 0 \\
            6 & 2 & 0 & 0 & 0 & 0 \\
            * & * & 1 & 0 & 0 & 0 \\
            * & * & * & 1 & 0 & 0 \\
            * & * & * & * & 1 & 0 \\
            * & * & * & * & * & 1
        \end{bmatrix},
    \end{align*}
    where \( * \) stands for arbitrary entries. The matrix is invertible, so no nonzero outcome with support in \( S = \left\{ (0,0), (0,4), (2,0), (4,1), (5,0), (6,0) \right\} \) exists.
\end{example}

\section{Impossible Supports}

Now, we show that specific supports cannot be the supports of valid integral outcomes. Hence, the title of this section \emph{Impossible Supports}. For instance, can we have an outcome whose support is only contained in \( S = \left\{ (0,0), (0,i) \right\} \) for some \( i \in \mathbb{N} \)? We will show that this is not possible.



\begin{proposition}
    Let \( d \in \mathbb{N} \), and \( i=0, \dots, d \). If \( S = \left\{ (0,i) \right\} \) and \( E = \left\{ 0 \right\} \), then \( A^{(d)}_{E,S} \) is invertible.
\end{proposition}

\begin{proof}
    We have \( A^{(d)}_{E,S} = \begin{bmatrix}
        1
    \end{bmatrix} \), which is invertible.
\end{proof}

\begin{proposition}
    Let \( d \in \mathbb{N} \). Assume \( i,j=0, \dots, d \) with \( i < j \). If \( S = \left\{ (0,i), (0,j) \right\} \) and \( E = \left\{ 0,1 \right\} \), then \( A^{(d)}_{E,S} \) is invertible.
\end{proposition}

\begin{proof}
    We have \( A^{(d)}_{E,S} = \begin{bmatrix}
        1 & 1 \\ d-i & d-j
    \end{bmatrix} \), which is invertible.
\end{proof}

\begin{proposition}\label{prop:impossible-support-2}
    Let \( d \in \mathbb{N} \). Assume \( i,j,k=0, \dots, d \) with \( i < j < k \). If \( S = \left\{ (0,i), (0,j), (0,k) \right\} \) and \( E = \left\{ 0,1,2 \right\} \), then \( A^{(d)}_{E,S} \) is invertible.
\end{proposition}

\begin{proof}
    We have 
    \begin{align*}
        A^{(d)}_{E,S} = \begin{bmatrix}
            \binom{d-i}{0} & \binom{d-j}{0} & \binom{d-k}{0} \\
            \binom{d-i}{1} & \binom{d-j}{1} & \binom{d-k}{1} \\
            \binom{d-i}{2} & \binom{d-j}{2} & \binom{d-k}{2}
        \end{bmatrix} = \begin{bmatrix}
            1 & 1 & 1 \\
            d-i & d-j & d-k \\
            \frac{(d-i)(d-i-1)}{2} & \frac{(d-j)(d-j-1)}{2} & \frac{(d-k)(d-k-1)}{2}
        \end{bmatrix}.
    \end{align*}
    We substitute 
    \begin{align*}
        x &= d - i, \\
        y &= d - j, \\
        z &= d - k.
    \end{align*}
    Then, we see
    \begin{align*}
        \begin{bmatrix}
            1 & 0 & 0 \\
            0 & 1 & 0 \\
            0 & 1 & 2
        \end{bmatrix}A^{(d)}_{E,S} = \begin{bmatrix}
            1 & 1 & 1 \\
            x & y & z \\
            x^2 & y^2 & z^2
        \end{bmatrix}.
    \end{align*}
    The matrix on the right-hand side is invertible because it is a Vandermonde matrix. Thus, the pairing matrix \( A^{(d)}_{E,S} \) is invertible.
\end{proof}

\begin{proposition}
    Let \( d \in \mathbb{N} \). Assume \( i,j,=0, \dots, d \) with \( i < j \). Moreover, let \( k=0, \dots, d-1 \). If \( S = \left\{ (0,i), (0,j), (1,k) \right\} \), \( E = \left\{ 0,1,2 \right\} \), and \( i+j \neq 2k + 1 \), then \( A^{(d)}_{E,S} \) is invertible.
\end{proposition}

\begin{proof}
    We have 
    \begin{align*}
        A^{(d)}_{E,S} = \begin{bmatrix}
            \binom{d-i}{0} & \binom{d-j}{0} & \binom{d-k-1}{-1} \\
            \binom{d-i}{1} & \binom{d-j}{1} & \binom{d-k-1}{0} \\
            \binom{d-i}{2} & \binom{d-j}{2} & \binom{d-k-1}{1}
        \end{bmatrix} = \begin{bmatrix}
            1 & 1 & 0 \\
            d-i & d-j & d-k-1 \\
            \frac{(d-i)(d-i-1)}{2} & \frac{(d-j)(d-j-1)}{2} & \frac{(d-k-1)(d-k-2)}{2}
        \end{bmatrix}.
    \end{align*}
    We substitute 
    \begin{align*}
        x &= d - i, \\
        y &= d - j, \\
        z &= d - k - 1.
    \end{align*}
    Then, we see 
    \begin{align*}
        \begin{bmatrix}
            1 & 0 & 0 \\
            0 & 1 & 0 \\
            0 & 1 & 2
        \end{bmatrix}
        A^{(d)}_{E,S} \begin{bmatrix}
            1 & 1 & 0 \\
            0 & -1 & 0 \\
            0 & 0 & 1
        \end{bmatrix}
        \begin{bmatrix}
            1 & 0 & 0 \\
            0 & \frac{1}{x-y} & 0 \\
            0 & 0 & 1
        \end{bmatrix} = 
        \begin{bmatrix}
            1 & 0 & 0 \\
            x & 1 & 1 \\
            x^2 & x+y & 2z+1
        \end{bmatrix}.
    \end{align*}
    We see that the determinant is nonzero because \( x + y \neq 2z + 1 \) by \( i+j \neq 2k+1 \).
\end{proof}

Without loss of generality, we may assume
\begin{align*}
    S &\subset \left\{ (i,j) \in V_d \mid i < \lvert S \rvert \right\}\\
    E &= \left\{ 0,1, \dots, \lvert S \rvert - 1 \right\}
\end{align*}
because the matrices \( A^{(d)}_{E,S} = A^{(d-\rho)}_{E - \rho, S - \rho} \) are equal, where \( \rho \coloneqq \min \left\{ E \cup \left\{ i \mid (i,j) \in S \right\} \right\} \) and \( E - \rho \coloneqq \left\{ (i - \rho, j) \mid (i,j) \in E \right\} \). This assumption allows us to apply the previous propostions to more general \( S \) and \( E \).

\begin{example}
    Assume we have a configuration with support \( S = \left\{ (0,i), (0,j), (0,k) \right\} \) for \( 0 \leq i < j< k \leq d \). By Proposition \ref{prop:impossible-support-2}, we know that no valid integral nonzero outcome can have this support.

    Now, let us consider a configuration \( \mathbf{w} \) with support \( \mathrm{supp}(\mathbf{w}) \subset S \) such that \( S \) can be decomposed into \( S_1, \dots, S_l \) as described before. Let \( \ell = 1, \dots, l \). If \( S_\ell = \left\{ (x,i), (x,j), (x,k) \right\} \) for \( 0 \leq i < j< k \leq d \) and \( x \in \mathbb{N} \), then \( \mathbf{w} = \mathbf 0 \) by Proposition \ref{prop:impossible-support-2} and the previous comment on the generality of \( S \) and \( E \). 

    For instance, this configuration is not an outcome
    \begin{verbatim}
        *
        .  .
        .  .  *  
        .  .  .  .  
        .  .  .  *  .  
        .  .  .  .  .  .  
        .  .  .  *  .  .  .  
        *  .  .  *  .  .  .  .  
    \end{verbatim}
    where \( * \) denotes a non-zero entry. This is because for \( \lambda = (3,3,1,1) \) we have \( S_2 = \left\{ (3,0), (3,1), (3,3) \right\} \).
\end{example}