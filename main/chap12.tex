\chapter{Discussion}

This thesis establishes a connection between the classification of discrete statistical models (Theorem \ref{thm:degree-fundamental-models}) and a combinatorial puzzle related to chipsplitting games (Theorem \ref{thm:outcome-degree-support-size}). Specifically, the puzzle investigates whether the degree of a valid chipsplitting outcome can grow indefinitely while its support size remains fixed. For outcomes with positive support sizes up to five, we prove that the degree cannot grow indefinitely, providing a definitive negative answer.

For outcomes with a positive support size of six, progress was made toward a similar conclusion. By employing systematic reductions, the number of cases requiring analysis was reduced from approximately 300,000 to 12,000, indicating that a negative answer may hold for this case as well. With additional computational resources, one can reduce the number of cases even further, potentially leading to a number of cases that can be analyzed using the techniques described in this thesis. With even greater computational power, one could potentially extend the results to support size seven.

We would like to conclude this thesis by discussing some possible directions for future research. First, it would be interesting to investigate a better criterion for determining fixed-contractables forms. This could potentially lead to a reduction in the number of cases that need to be analyzed for positive support size six. Second, investigating a larger contraction size than five could provide further insights into the degree of valid outcomes of positive support size six. Finally, finding a larger non-trivial system would allow us to exclude even more cases.