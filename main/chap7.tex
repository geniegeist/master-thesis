\chapter{Valid Outcomes of Positive Support
Size Four}

We have all the tools to prove the main result of this chapter, namely for all valid integral outcomes \( \mathbf w \) with \( |\mathrm{supp}^+(\mathbf w)| = 4 \) we have
\begin{align*}
    \mathrm{deg}(\mathbf w) \leq 5.
\end{align*}
As in previous chapters, we characterize outcomes as roots of Pascal forms. So, we define the following two systems of Pascal forms that valid outcomes must be roots of:
\begin{gather*}
    \Phi_1 \coloneqq \{ 
        \mathrm{col}(1), \mathrm{col}(2), \mathrm{col}(3), \mathrm{row}(1), \mathrm{row}(2), \mathrm{row}(3),\\
         \mathrm{diag}(1), \mathrm{diag}(2), \mathrm{diag}(3), \mathrm{diag}(d-1), \mathrm{diag}(d-2), \mathrm{diag}(d-3) 
     \},
\end{gather*}
and 
\begin{gather*}
    \Phi_2 \coloneqq \{ 
        \mathrm{col}(d), \mathrm{col}(d-1), \mathrm{col}(d-2), \mathrm{col}(d-3), \\
        \mathrm{row}(d), \mathrm{row}(d-1), \mathrm{row}(d-2), \mathrm{row}(d-3) 
     \}.
\end{gather*}
We also define \( \Phi \coloneqq \Phi_1 \cup \Phi_2 \).

By Proposition \ref{prop:contracted-part-1}, we can write all hyperfield forms induced by Pascal forms \( p \) in \( \Phi_1 \) as 
\begin{align*}
    \mathrm{sign}(p) = \hat p
\end{align*}
for some linear form \( \hat p \in H[\mathbf{x}, \mathbf{y}, \mathbf{z}, \mathbf{b}, \mathbf{c}, \mathbf{d}, \mathbf{e}] \) if \( d \geq 11 \). This linear form is independent of the degree \( d \). To make notations consistent later, we set \( \hat p^{\mathrm{even}} \coloneqq  \hat p^{\mathrm{odd}}  \coloneqq \hat p\).

Similarly, by Proposition \ref{prop:contracted-part-2}, we can write all hyperfield forms induced by Pascal forms \( p \) in \( \Phi_2 \) as
\begin{align*}
    \mathrm{sign}(p) = \begin{cases}
        \hat p^{\mathrm{even}} & \text{ if } d \text{ is even} \\
        \hat p^{\mathrm{odd}} & \text{ if } d \text{ is odd}
    \end{cases}
\end{align*}
for some linear forms \( \hat p^{\mathrm{even}}, \hat p^{\mathrm{odd}} \in H[\mathbf{x}, \mathbf{y}, \mathbf{z}, \mathbf{b}, \mathbf{c}, \mathbf{d}, \mathbf{e}] \) if \( d \geq 12 \). These linear forms \( \hat p^{\mathrm{even}}, \hat p^{\mathrm{odd}}  \) are independent of the degree \( d \).

\begin{definition}\label{def:sdjsndjknsdj}
    We define the following three solution sets:
    \begin{enumerate}
        \item     Define \( \Gamma_d \) to be the set of all valid hyperfield configurations \( \mathbf{s} \in H^{V_d} \) of degree \( d \) such that \( \mathrm{sign}(p)(\mathbf{s}) = H \) for all \( p \in \Phi \).

        \item     Define \( \Gamma^{\mathrm{even}} \) to be the set of all valid contracted hyperfield configurations \( \mathbf{s} \in H^{\Xi} \) such that \( \hat p^{\mathrm{even}}(\mathbf{s}) = H \) for all \( p \in \Phi \).

        \item     Define \( \Gamma^{\mathrm{odd}} \) to be the set of all valid contracted hyperfield configurations \( \mathbf{s} \in H^{\Xi} \) such that \( \hat p^{\mathrm{odd}}(\mathbf{s}) = H \) for all \( p \in \Phi \).
    \end{enumerate}
\end{definition}

By Proposition \ref{prop:hyperfield-criterion}, valid chipsplitting outcomes of degree \( d \) have supports in \( \Gamma_d \). This is the reason why we have defined \( \Gamma_d \) in the first place. 

\begin{proposition}\label{prop:sign-sikjsfnf3223423432}
    Let \( d \geq 12 \). Then, the following holds:
    \begin{enumerate}
        \item If \( d \) is even, then \( \Gamma_d = \mathrm{contr}_d^{-1}(\Gamma^{\mathrm{even}}) \).
       \item If \( d \) is odd, then \( \Gamma_d = \mathrm{contr}_d^{-1}(\Gamma^{\mathrm{odd}}) \).
    \end{enumerate}
\end{proposition}

\begin{proof}
    Let \( d \geq 12 \) be even. Let \( \mathbf{s} \in {H}^{V_d} \) be a hyperfield configuration and \( p \in \Phi \). Then, we have 
    \begin{align*}
        \mathrm{sign}(p)(\mathbf{s}) = \hat p^{\mathrm{even}}(\mathrm{contr}_d(\mathbf{s})).
    \end{align*}
    by definition of \( \hat p^{\mathrm{even}} \). If \( \mathbf{s} \in \Gamma_d \), then \( H = \mathrm{sign}(p)(\mathbf{s}) = \hat p^{\mathrm{even}}(\mathrm{contr}_d(\mathbf{s})) \). Hence, \( \mathrm{contr}_d(\mathbf{s}) \) is contained in \( \Gamma^{\mathrm{even}} \). If \( \mathrm{contr}_d(\mathbf{s}) \in \Gamma^{\mathrm{even}} \) holds, using the equation above we also see that \( \mathbf{s} \in \Gamma_d \). This shows that \( \Gamma_d = \mathrm{contr}_d^{-1}(\Gamma^{\mathrm{even}}) \).

    The second statement for odd degrees \( d \) follows analogously.
\end{proof}

\begin{corollary}\label{cor:validwunfwufneuiw}
    Let \( d \geq 12 \). Let \( \mathbf{w} \in \mathbb{Z}^{V_d} \) be a valid outcome. Then, we have 
    \begin{align*}
        \mathrm{contr}_d(\mathrm{sign}(\mathbf{w})) \in \Gamma^{\mathrm{even}} \cup \Gamma^{\mathrm{odd}}.
    \end{align*}
\end{corollary}

\begin{proof}
    Define \( \mathbf{s} \coloneqq \mathrm{sign}(\mathbf{w}) \). By Proposition \ref{prop:sign-sikjsfnf322} we have \( \mathbf{s} \in \Gamma_d \). If \( d \) is even, then \( \mathrm{contr}_d(\mathbf{s}) \in \Gamma^{\mathrm{even}} \) by the previous proposition. If \( d \) is odd, then \( \mathrm{contr}_d(\mathbf{s}) \in \Gamma^{\mathrm{odd}} \) by the previous proposition. This shows the claim.
\end{proof}

This corollary allows us to exclude certain outcomes as valid outcomes. Assume we have some contracted hyperfield configuration \( \xi \in H^{\Xi} \) that is not a root of some of the linear forms \( \hat p^{\mathrm{even}}, \hat p^{\mathrm{odd}} \) for \( p \in \Phi \). Then, any chipsplitting configuration \( \mathbf{w} \in \mathbb{Z}^{V_d} \) with \( \mathrm{contr}_d( \mathrm{sign}(\mathbf{w})) = \xi \) is not a valid outcome.

\begin{proposition}\label{prop:jasndkjsnjsnkjs}
    Let \( \mathbf{s} \in H^{V_d} \) be a valid hyperfield configuration of degree \( d \) with positive support size four or less. If \( d\geq 12 \), then \( \mathbf{s} \notin \Gamma_d \).
\end{proposition}

\begin{proof}
    Let \( d \geq 12 \). For computing \( \Gamma_d \) we could use Algorithm \ref{alg:hyperfield_criterion:efficient} for all \( d = 12, 13, 14, \dots \) and so on, which is not feasible since we would compute solutions sets for many infinitely many degrees \( d \). Instead, we show that \( \Gamma^{\mathrm{even}} \cup \Gamma^{\mathrm{odd}} \) is empty. By Proposition \ref{prop:sign-sikjsfnf3223423432}, \( \Gamma_d \) is empty as well for all \( d\geq 12 \).

    To show that \(  \Gamma^{\mathrm{even}} \) is empty, we can just use Algorithm \ref{alg:hyperfield_criterion:efficient} and Remark \ref{rem:fiuhwiu3} with \(A \coloneqq \left\{ \hat p^{\mathrm{even}} \mid p \in \Phi \right\}\). Similarly, we compute \( \Gamma^{\mathrm{odd}} = \emptyset \) with \(A \coloneqq \left\{ \hat p^{\mathrm{odd}} \mid p \in \Phi \right\}\) and Algorithm \ref{alg:hyperfield_criterion:efficient}. The results are in Appendix TODO. This shows the claim. TODO: show that \( A \) is non-trivial.
\end{proof}

\begin{theorem}\label{thm:main-result-32432432432nkdnjkfd}
    For valid integral outcomes \( \mathbf w \) with \( |\mathrm{supp}^+(\mathbf w)| = 4 \) we have \( \mathrm{deg}(\mathbf w) \leq 5 \).
\end{theorem}

\begin{proof}
    Let \( d \geq 6 \).
    Let \( \mathbf{w} \in \mathbb{Z}^{V_d} \) be a valid outcome with \( |\mathrm{supp}^+(\mathbf w)| = 4 \) and degree \( d \). We have \( \mathrm{sign}(\mathbf{w}) \in \Gamma_d \). By the previous proposition, there is no such \( \mathrm{sign}(\mathbf{w}) \) for \( d \geq 12 \). By Proposition \ref{prop:jdngkjrenj3nw}, the degree of \( \mathrm{sign}(\mathbf{w}) = d \) is six or seven. So, we just need to check eight cases. Of these eight cases, we can exclude all of them by applying Algorithm \ref{alg:hyperfield_criterion:is_zero}. The result of this algorithm is that only the zero outcome is possible for all these cases. This shows that the degree of \( \mathbf{w} \) is at most five.
\end{proof}