\chapter{Valid Outcomes of Positive Support Size Six}

We want to prove that all valid outcomes \( \mathbf w \) with \( |\mathrm{supp}^+(\mathbf w)| = 6 \) have \( \mathrm{deg}(\mathbf w) \leq 9 \). This thesis makes a contribution towards a proof by reducing the number of cases to check. In theory, we could use the same approach as in the previous chapters; specifically, we begin by computing 
\(
\Gamma^{\mathrm{even}} \cap \left\{ \mathbf{s} \in H^{\Xi} : \lvert \mathrm{supp}^+(\mathbf{s}) \rvert = 6 \right\}
\)
and 
\(
\Gamma^{\mathrm{odd}} \cap \left\{ \mathbf{s} \in H^{\Xi} : \lvert \mathrm{supp}^+(\mathbf{s}) \rvert = 6 \right\},
\)
albeit with a slight modification to these sets. While we previously considered contractions of size four, we now consider contractions of size five, i.e. we work with contraction variables as depicted below. 
\begin{figure}[H]
    \begin{align*}
        \begin{array}{cccccccccccccccccccc}
            y_{0,4} & & & & & & & & & & & & \\
            y_{0,3} & y_{1,4} & & & & & & & & & & & \\
            y_{0,2} & y_{1,3} & y_{2,4} & & & & & & & & & & \\
            y_{0,1} & y_{1,2} & y_{2,3} & y_{3,4} & & & & & & & & & \\
            y_{0,0} & y_{1,1} & y_{2,2} & y_{3,3} & y_{4,4} & & & & & & & & \\
            c_0 & y_{1,0} & y_{2,1} & y_{3,2} & y_{4,3} & d_0 & & & & & & & \\
            c_0 & c_1 & y_{2,0} & y_{3,1} & y_{4,2} & d_1 & e_0 & & & & & & \\
            c_0 & c_1 & c_2 & y_{3,0} & y_{4,1} & d_2 & e_1 & d_0 & & & & & \\
            c_0 & c_1 & c_2 & c_3 &  y_{4,0}  & d_3 & e_2 & d_1 & e_0 & & & & \\
            c_0 & c_1 & c_2 & c_3 &  c_4  & d_4 & e_3 & d_2 & e_1 & d_0 & & & \\
            c_0 & c_1 & c_2 & c_3 &  c_4  & * & e_4 & d_3 & e_2 & d_1 & e_0 & & \\
            c_0 & c_1 & c_2 & c_3 &  c_4  & * & * & d_4 & e_3 & d_2 & e_1 & d_0 & \\
            c_0 & c_1 & c_2 & c_3 &  c_4  & * & * & * & e_4 & d_3 & e_2 & d_1 & e_0 \\
            x_{0,4} & x_{1,4} & x_{2,4} & x_{3,4} &  x_{4,4}  & b_4 & b_4 & b_4 & b_4 & z_{0,4} & z_{1,4} & z_{2,4} & z_{3,4} & z_{4,4} \\
            x_{0,3} & x_{1,3} & x_{2,3} & x_{3,3} & x_{4,3} & b_3 & b_3 & b_3 & b_3 & b_3 & z_{0,3} & z_{1,3} & z_{2,3} & z_{3,3} & z_{4,3} \\
            x_{0,2} & x_{1,2} & x_{2,2} & x_{3,2} & x_{4,2} & b_2 & b_2 & b_2 & b_2 & b_2 & b_2 & z_{0,2} & z_{1,2} & z_{2,2} & z_{3,2} & z_{4,2}\\
            x_{0,1} & x_{1,1} & x_{2,1} & x_{3,1} & x_{4,1} & b_1 & b_1 & b_1 & b_1 & b_1 & b_1 & b_1 & z_{0,1} & z_{1,1} & z_{2,1} & z_{3,1} & z_{4,1} \\
            x_{0,0} & x_{1,0} & x_{2,0} & x_{3,0} & x_{4,0} & b_0 & b_0 & b_0 & b_0 & b_0 & b_0 & b_0 & b_0 & z_{0,0} & z_{1,0} & z_{2,0} & z_{3,0} & z_{4,0}
        \end{array}
    \end{align*}  
\end{figure}
From this point forward, when we refer to \( \mathbf{s} \in H^{\Xi} \), we will mean contractions of size five. We increase the contraction size to gain more information about potential supports, which we hope will prove useful when handling cases manually.

\begin{definition}
    Define \(  \Gamma^{\mathrm{even}}_6 \) and \( \Gamma^{\mathrm{odd}}_6 \) analogously to Definition \ref{def:sdjsndjknsdj} with 
    \begin{gather*}
        \Phi_1 = \left\{ \mathrm{col}(i), \mathrm{row}(i), \mathrm{diag}(i), \mathrm{diag}(d-i) \right\}_{i=0}^4 \text{ and } \Phi_2 = \left\{ \mathrm{col}(d-i), \mathrm{row}(d-i) \right\}_{i=0}^4.
    \end{gather*}
\end{definition}

\begin{proposition}
    We have \( \lvert \Gamma^{\mathrm{even}}_6 \rvert  = 150032\) and \( \lvert \Gamma^{\mathrm{odd}}_6 \rvert  = 154177\).
\end{proposition}

\begin{proof}
    This is verified by a computer program, which is available on GitHub \cite{ducrepo} under tile file \texttt{chapter07\_intro.ipynb}.
\end{proof}

The number of cases to check has increased by a factor of hundred compared to the previous chapters. If we were able to reduce the cases, we could apply the same techniques as in the previous chapters. We will spend the remaining chapter to reduce the number of cases to check to around 12,000 cases in the hope to make the proof computationally feasible. Due to time constraints of this thesis, we were not able to attempt a complete proof.

\section{Finding Fixed-Contractable Pascal Forms}

To reduce the number of cases, we generate \emph{fixed-contractable} linear combinations of hyperfield Pascal forms.

\begin{definition}
    Let \( d \in \mathbb{N}_{\geq 15} \), \( k = 0, \dots ,4 \), and \( p \) be a hyperfield linear form on \( H^{V_d} \). We say \( p \) is \emph{contractable} for \( d \) on \( b_k \) if we can write
    \begin{align*}
        p = \sum_{(i,j) \in V_d \setminus \left\{ (5,k), \dots, (d-k-5, k) \right\}} \lambda_{i,j} x_{i,j}  +\lambda b_k \quad \text{for} \quad b_k \coloneqq x_{5,k} + \dots + x_{d-5-k,k}
    \end{align*}
    for \( \lambda_{i,j}, \lambda \in H \). In other words, we can write \( p = \hat p \) for some linear form \( \hat p \in H[\mathbf{x}, b_k] \). 
    Similarly, we define \( p \) is \emph{contractable} on \( c_k \), \( d_k \), and \( e_k \) if we can write \( p = \hat p \) for some linear form \( \hat p \in H[\mathbf{x}, c_k] \), \( \hat p \in H[\mathbf{x}, d_k] \), and \( \hat p \in H[\mathbf{x}, e_k] \), respectively.
\end{definition}

\begin{remark}
    Clearly, \( p \) is contractable for \( d \) if and only if it is contractable for \( d \) on \( b_k \), \( c_k \), \( d_k \), and \( e_k \) for all \( k = 0, \dots, 4 \).
\end{remark}

\begin{definition}
    Let \( d \in \mathbb{N}_{\geq 15} , i \in \left\{ 0,\dots,4,d-4, \dots, d \right\}\), and $p = \sum \lambda_{i,j} x_{i,j}$ be a hyperfield linear form on \( H^{V_d} \). We define the $i$-th $b$-row of $p$ as \( p_{b_{i}} \coloneqq \begin{bmatrix} \lambda_{5,i} & \dots & \lambda_{d-i-5,i} \end{bmatrix} \in H^{d -  i - 9} \). Similarly, we define $p_{c_{i}}$, $p_{d_{i}}$ and $p_{e_{i}}$ to denote the $i$-th $b$-column, $d$-diagonal and $e$-diagonal of $p$, respectively.
  \end{definition}


\begin{proposition}\label{prop:nwfiewnfiuwneufni2un2}
    Let \( d \in \mathbb{N}_{\geq 15} , i \in \{ 0, \dots, 4 \} \), and $T$ be a formal linear combination of
    \begin{align*}
        \{ \mathrm{sign}(\mathrm{row}(j)), \mathrm{sign}(\mathrm{col}(j)), \mathrm{sign}(\mathrm{diag}(j)) \mid j \in \{ 0, \dots, 4\} \cup \{ t-4, \dots, t \} \}.
    \end{align*}
Then, the following statements hold for all realizations \( p_d \) of \( T \):
  \begin{enumerate}
  \item The $c$-column of $(p_d)_{c_i}$ only depends on $\{ \mathrm{sign}(\mathrm{row}(k)), \mathrm{sign}(\mathrm{diag}(k)) \}_{k = 0}^4$.
  \item The $b$-row of $(p_d)_{b_i}$ only depends on $\{ \mathrm{sign}(\mathrm{col}(k)), \mathrm{sign}(\mathrm{diag}(d-k)) \}_{k = 0}^4$.
  \item The $d$-diagonal of $(p_d)_{d_i}$ only depends on \(\{ \mathrm{sign}(\mathrm{row}(d-k)), \mathrm{sign}(\mathrm{col}(d-k)) \}_{k = 0}^4 \). A similar statement holds for $e$-diagonals.
  \end{enumerate}
  \end{proposition}
  
  \begin{proof}
   This follows immediately from the definition of $\mathrm{row}, \mathrm{col}$, and $\mathrm{diag}$.
  \end{proof}

  \begin{proposition}
    Let \( d \in \mathbb{N}_{\geq 15} \) and \( i = 0, \dots 4 \).
    The following statements hold:
    \begin{enumerate}
    \item Let $T \in \{\mathrm{sign}(\mathrm{row}(k)), \mathrm{sign}(\mathrm{diag}(k)) \}_{k = 0, \dots, 4}$. The $c_{i}$-column of $p_d$ is a constant vector. 
  
    \item Let $T \in \{\mathrm{sign}( \mathrm{col}(k)),\mathrm{sign}(\mathrm{diag}(d-k)) \}_{k = 0, \dots, 4}$. The $b_{i}$-row of $p_d$ is a constant vector. 
  
    \item Let $T \in \{ \mathrm{sign}(\mathrm{row}(d-k)), \mathrm{sign}(\mathrm{col}(d-k)) \}_{k = 0, \dots, 4}$. The $d_{i}$-diagonal of $p_d$ is a constant vector; similarly for the $e_{i}$-diagonal. 
    \end{enumerate}
  \end{proposition}
  
  \begin{proof}
    This also follows easily from the definition of $\mathrm{row}, \mathrm{col}$, and \(\mathrm{diag} \). 
  \end{proof}
  
  Let us investigate how realizations of formal form change when increasing the degree. We fix the following notations:
\begin{itemize}
    \item Let \( T = \sum_{i=0}^{4}  \lambda_{i} \mathrm{row}(i)  \) or \( T = \sum_{i=0}^{4}  \lambda_{t-i} \mathrm{row}(t-i) \);
    \item Write the realizations of \( T \) as \( p \coloneqq \sum p_{i,j}x_{i,j}  \coloneqq p_d  \) and \( q \coloneqq \sum q_{i,j}x_{i,j} \coloneqq p_{d+1} \) for some degree \( t = d \);
    \item Define \( r \coloneqq \max\left\{ i \mid \lambda_i \neq 0 \right\} \) if \(  \max\left\{ i \mid \lambda_i \neq 0 \right\} \leq 4 \), otherwise \( r \coloneqq \min\left\{ i \mid \lambda_i \neq 0 \right\} \); 
    \item Write \( \mathrm{row}(r) = \sum r_{i,j}x_{i,j} \).
\end{itemize}  


\begin{lemma}\label{prop:row_extend_d}
    Let \( d \in \mathbb{N}_{\geq 9} \). Then, \( q_{i,j} = p_{i,j} \) holds for all \( (i,j) \in V_d \).
\end{lemma}
  
\begin{proof}
    Let \( d \in \mathbb{N} \), \( \ell = 0, \dots, 4,t-4,\dots,t \) and \(T_{\ell} =  \mathrm{row}(\ell) \). We denote its realizations in \( d \) and \( d + 1 \) by \( p_d^{(\ell)} = \sum p_{i,j}^{(\ell)}x_{i,j} \) and \( p_{d+1}^{(\ell)} = \sum q_{i,j}^{(\ell)}x_{i,j} \), respectively. By Proposition \ref{prop:pascal-formulas}, we see that \( p_{i,j}^{(\ell)} = q_{i,j}^{(\ell)} \) for all \( (i,j) \in V_d \). Next, assume \( T =  \sum_{i=0}^{4}  \lambda_{i} \mathrm{row}(i) + \sum_{i=0}^{4}  \lambda_{t-i} \mathrm{row}(t-i)  \). Then, \( p_{i,j} = \sum_{\ell} \lambda_\ell p_{i,j}^{(\ell)} = \sum_{\ell} \lambda_\ell q_{i,j}^{(\ell)} = q_{i,j} \) for all \( (i,j) \in V_d \).
\end{proof}
  
\pagebreak
\begin{example}
    Let \( T = \mathrm{row}(3) + \mathrm{row}(2) \). We visualize \( p_8 \) and \( p_9 \):
    \begin{verbatim}
                            -48
-28                         -28 20
-14 14                      -14 14 -6
 -5  9 -5                    -5  9 -5  1
  .  5 -4  1                  .  5 -4  1  . 
  2  2 -3  1  .               2  2 -3  1  .  .
  2  . -2  1  .  .            2  . -2  1  .  .  .
  1 -1 -1  1  .  .  .         1 -1 -1  1  .  .  .  .
  . -1  .  1  .  .  .  .      . -1  .  1  .  .  .  .  .
  .  .  1  1  .  .  .  .  .   .  .  1  1  .  .  .  .  .  .
    \end{verbatim}
  \end{example}
  
\begin{lemma}\label{lemma:sign_row_propagation}
    Let \( d \in \mathbb{N}_{\geq 9} \). If there exists \( k \in \left\{ 0, \dots, r \right\} \) such that $\mathrm{sign}(r_{i,d-i}) = \mathrm{sign}(p_{i,d-i})$ for all \( i = k, \dots, r \), then \( \mathrm{sign}(q_{i,d+1-i}) = \mathrm{sign}(p_{i,d-i}) \) for all \( i = k, \dots, r\).
\end{lemma}

\begin{proof}
    Without loss of generality, we assume that \( \lambda_r > 0 \).
    First, we see that \( q_{r, \cdot} \coloneqq (q_{r,j})_{j=0}^{d-r} = \lambda_r \cdot \mathbf{1} \) and \( q_{i,\cdot} =  \mathbf{0} \) for all \( i > r \). By the Pascal property, we have \( q_{r-1,d+1-(r-1)} = q_{r-1,d-(r-1)} - q_{r,d+1-r} = q_{r-1,d-(r-1)} - \lambda_r \).
    This shows \( q_{r-1,d+1-(r-1)} < q_{r-1,d-(r-1)} = p_{r-1,d-(r-1)} < 0 \), where the last inequality follows from the assumption $\mathrm{sign}(r_{i,d-i}) = \mathrm{sign}(p_{i,d-i})$. Thus, we have \( \mathrm{sign}(q_{r-1,d+1-(r-1)}) = \mathrm{sign}(q_{r-1,d-(r-1)}) = \mathbf -1\). 
    
    Next, use the Pascal property \( q_{r-2,d+1-(r-2)} = q_{r-2,d-(r-2)} - q_{r-1,d+1-(r-1)} \). Note \( q_{r-2,d+1-(r-2)} > 0 \) as \( q_{r-2,d-(r-2)} > 0 \) and \( q_{r-1,d+1-(r-1)} < 0 \). Hence, we have \( \mathrm{sign}(q_{r-2,d+1-(r-2)}) = \mathrm{sign}(q_{r-2,d-(r-2)}) = 1 \). We continue this argument for \( r-3, r-4, \dots, k \). This shows that \( \mathrm{sign}(q_{i,d+1-i}) = \mathrm{sign}(q_{i,d-i}) = \mathrm{sign}(p_{i,d-i}) \) for all \( i = k, \dots, r\).
\end{proof}

\begin{lemma}\label{lemma:same_sign_propagation_easy}
    Let \( d \in \mathbb{N}_{\geq 15} \) and \( r \leq 4 \). If there exists \( k \in \left\{ 0, \dots, r \right\} \) such that for all \( i = k, \dots, r\) we have \(  \mathrm{sign}(\mathrm{row}(r))_{c_i} = \mathrm{sign}(p)_{c_i} \),
    then for all \( i = k, \dots, r\) we either have \( p_{c_i} > 0, q_{c_i} > 0 \) or \( p_{c_i} < 0, q_{c_i} < 0 \).
\end{lemma}
  
\begin{proof}
    Let \( i=k, \dots, r \). If we show \( \mathrm{sign}(p_{i,d-i}) = \mathrm{sign}(p_{i,d-i-5}) \), then $\mathrm{sign}(r_{i,d-i}) = \mathrm{sign}(p_{i,d-i})$ holds by assumption \( \mathrm{sign}(\mathrm{row}(r))_{c_i} = \mathrm{sign}(p)_{c_i} \). So, we can use Lemma \ref{lemma:sign_row_propagation} to prove the statement. 
    
    We see that \( \mathrm{sign}(p_{r,d-r}) = \mathrm{sign}(p_{r,d-r-5}) \) since \( \mathrm{sign}(\mathrm{row}(r))_{c_r} = \mathrm{sign}(p)_{c_r} \) and \( \mathrm{sign}(r_{r,d-r}) = \mathrm{sign}(r_{r, d - r - 5}) \). For \( r - 1 \), we then use the Pascal property. We repeat this argument for \( r-2, r-3, \dots, k \). This shows that \( \mathrm{sign}(p_{i,d-i}) = \mathrm{sign}(p_{i,d-i-5}) \) for all \( i = k, \dots, r \).
\end{proof}

\begin{proposition}\label{prop:fixed-contraction-homo-row}
    Let \( d \in \mathbb{N}_{\geq 15} \) and \( r \leq 4 \). If there exists \( k \in \left\{ 0, \dots, r \right\} \) such that for all \( i = k, \dots, r\) we have \(  \mathrm{sign}(\mathrm{row}(r))_{c_i} = \mathrm{sign}(p)_{c_i} \),
    then  \( \mathrm{sign}(T) \) is fixed-contractable on \( c_i \) for all \( i = k, \dots, r \).
\end{proposition}

\begin{proof}
    Let \( i = k, \dots, r \) and \( d \in \mathbb{N}_{\geq 15} \)
    First, it is easy to see that \( p \) is contractable on \( c_i \) because \( \mathrm{row}(r) \) is contractable and \( \mathrm{sign}(\mathrm{row}(r))_{c_i} = \mathrm{sign}(p)_{c_i} \). By Lemma \ref{lemma:same_sign_propagation_easy} the sign does not change when increasing the degree \( d \leadsto d+1 \). Hence, the contractability of \( p \) on \( c_i \) is preserved for all degrees greater or equal to \( d \). Therefore, there exists one \( \hat p \in H[\mathbf{x}, c_i] \) for all \( d' \geq d \) such that \( \hat p = p_{d'} \).
\end{proof}

There exist similar propositions for contractability on \( d_i \) and \( e_i \).

\begin{proposition}\label{prop:i3jhr23h923h8}
    Let \( d \in \mathbb{N}_{\geq 15} \) and \( r \geq d-4 \). If there exists \( k \in \left\{ r, \dots, d \right\} \) such that for all \( i = r, \dots, k\) we have \(  \mathrm{sign}(\mathrm{row}(r))_{d_i} = \mathrm{sign}(p)_{d_i} \),
then  \( \mathrm{sign}(T) \) is fixed-contractable on \( d_i \) for all \( i = r, \dots, k\). The analogous statement holds for \( e_i \).
\end{proposition}

\begin{proof}
    We use the same proof as before, but now the sign of the entire diagonal $d_{i}$ changes whenever we increase the dimension by one. Fortunately, the contractability on $d_{i}$ is not affected by this. 
\end{proof}

We state analogous propositions for \( \mathrm{col}(\cdot) \) of Proposition \ref{prop:fixed-contraction-homo-row} and \ref{prop:i3jhr23h923h8} but skip the proofs since they are similar. Let \(  T = \sum_{i=0}^{4}  \lambda_{i} \mathrm{col}(i) + \sum_{i=0}^{4}  \lambda_{t-i} \mathrm{col}(t-i) \).

\begin{proposition}\label{prop:fixed-contraction-homo-col}
    Let \( d \in \mathbb{N}_{\geq 15} \) and \( r \leq 4 \). If there exists \( k \in \left\{ 0, \dots, r \right\} \) such that for all \( i = k, \dots, r\) we have \(  \mathrm{sign}(\mathrm{col}(r))_{b_i} = \mathrm{sign}(p)_{b_i} \),
    then  \( \mathrm{sign}(T) \) is fixed-contractable on \( b_i \) for all \( i = k, \dots, r \).
\end{proposition}

\begin{proposition}\label{prop:23e232sdada2kmkl}
    Let \( d \in \mathbb{N}_{\geq 15} \) and \( r \geq d-4 \). If there exists \( k \in \left\{ r, \dots, d \right\} \) such that for all \( i = r, \dots, k\) we have \(  \mathrm{sign}(\mathrm{col}(r))_{d_i} = \mathrm{sign}(p)_{d_i} \),
then  \( \mathrm{sign}(T) \) is fixed-contractable on \( d_i \) for all \( i = r, \dots, k\). The analogous statement holds for \( e_i \).
\end{proposition}

Here is an analogous version of Lemma \ref{prop:row_extend_d} but for \( \mathrm{diag}(\cdot) \).

\begin{lemma}\label{prop:diag_extend_d}
    Let \( d \in \mathbb{N}_{\geq 9} \). Then, \( q_{i,j+1} = p_{i,j} \) holds for all \( (i,j) \in V_d \).
\end{lemma}

\begin{proof}
    Just use Lemma \ref{prop:row_extend_d} and symmetries \( \sigma \in S_3 \).
\end{proof}

\begin{example}
    Consider \( T = \mathrm{diag}(3) + \mathrm{diag}(2) \). Then, \( p_8 \) is represented by the triangle on the left and \( p_9 \) is represented by the triangle on the right.
    \begin{verbatim}
  .                           .  
  .  .                        .  .  
  1  1  1                     1  1  1   
  4  3  2  1                  4  3  2  1  .
 10  6  3  1  .              10  6  3  1  .  .
 20 10  4  1  .  .           20 10  4  1  .  .  .
 35 15  5  1  .  .  .        35 15  5  1  .  .  .  .
 56 21  6  1  .  .  .  .     56 21  6  1  .  .  .  .  .
 84 28  7  1  .  .  .  .  .  84 28  7  1  .  .  .  .  .  .
                            120 36  8  1  .  .  .  .  .  .  . 
\end{verbatim}
\end{example}

Not surprisingly, we have analogous propositions for \( \mathrm{diag}(\cdot) \) of Proposition \ref{prop:fixed-contraction-homo-row}. Consider the formal linear combination \(  T = \sum_{i=0}^{4}  \lambda_{i} \mathrm{diag}(i) + \sum_{i=0}^{4}  \lambda_{t-i} \mathrm{diag}(t-i) \).

\begin{proposition}\label{prop:fixed-contraction-homo-diag}
    Let \( d \in \mathbb{N}_{\geq 15} \). The following statements hold:
    \begin{enumerate}
        \item  Assume \( r \leq 4 \). If there is \( k \in \left\{ 0, \dots, r \right\} \) such that for all \( i = k, \dots, r\) we have \(  \mathrm{sign}(\mathrm{diag}(r))_{c_i} = \mathrm{sign}(p)_{c_i} \),
        then  \( \mathrm{sign}(T) \) is fixed-contractable on \( c_i \) for all \( i = k, \dots, r \).
        \item Assume \( r \geq d-4 \).
        If there exists \( k \in \left\{ r, \dots, d \right\} \) such that for all \( i = r, \dots, d\) we have \(  \mathrm{sign}(\mathrm{diag}(r))_{b_i} = \mathrm{sign}(p)_{b_i} \),
        then  \( \mathrm{sign}(T) \) is fixed-contractable on \( b_i \) for all \( i = r, \dots, d \).
    \end{enumerate}
\end{proposition}

\begin{proof}
    The proofs are analogous to the proof of Proposition \ref{prop:fixed-contraction-homo-row}.
\end{proof}

%\begin{definition}
%    Let \( B \) be a set of formal linear combinations. Define the set of fixed-contractable forms $\mathrm{FC}(B)$ of \( B \) to be 
%    \begin{align*}x
%    \mathrm{FC}(B) \coloneqq B \cap \left\{ p \mid \text{\( p \) is fixed-contractable} \right\}.
%    \end{align*}
%\end{definition}

We now provide more Propositions that will later help us to automatically prove that formal forms are fixed-contractable.

\begin{proposition}\label{prop:row_homo_diag}
    Let \( d \in \mathbb{N}_{\geq 15} \), \( T = \sum_{i=0}^4 \lambda_i \mathrm{row}(i)\) with realization \( p_d \), and \( T' = T - \mathrm{diag}(0) \) with realization \( p'_d \). If \( (p_d)_{c_0} \geq \mathbf{2} \), then we have \( (p'_{d'})_{c_0} \geq \mathbf 1 \) for all \( d' \geq d \).
\end{proposition}

\begin{proof}
     We see that \( (\mathrm{diag}(0))_{c_0} = \mathbf 1 \) is a constant vector for all degrees. Note that \( (p_{d'})_{c_0} \geq \mathbf 2 \) for all \( d' \geq d \) by Lemma \ref{prop:row_extend_d}. So, we have \( (p_{d'} - \mathrm{diag}(0))_{c_0} \geq \mathbf 1 \) for all \( d' \geq d \).
\end{proof}


\begin{example}
    Let \( T' = \mathrm{row}(1) + \mathrm{row}(2) - \mathrm{diag}(0)\).
    Then, \((p'_d)_{c_0} \geq \mathbf 1 \) for all dimensions \( d \geq 15 \). Let us visualize \( \mathrm{row}(1) + \mathrm{row}(2) \) for \( d = 18 \):
    \begingroup
    \fontsize{8pt}{10pt}\selectfont
    \begin{verbatim}
      135 
      119  -16 
      104  -15    1 
       90  -14    1    . 
       77  -13    1    .    . 
       65  -12    1    .    .    . 
       54  -11    1    .    .    .    . 
       44  -10    1    .    .    .    .    . 
       35   -9    1    .    .    .    .    .    . 
       27   -8    1    .    .    .    .    .    .    . 
       20   -7    1    .    .    .    .    .    .    .    . 
       14   -6    1    .    .    .    .    .    .    .    .    . 
        9   -5    1    .    .    .    .    .    .    .    .    .    . 
        5   -4    1    .    .    .    .    .    .    .    .    .    .    . 
        2   -3    1    .    .    .    .    .    .    .    .    .    .    .    . 
        .   -2    1    .    .    .    .    .    .    .    .    .    .    .    .    . 
       -1   -1    1    .    .    .    .    .    .    .    .    .    .    .    .    .    . 
       -1    .    1    .    .    .    .    .    .    .    .    .    .    .    .    .    .    . 
        .    1    1    .    .    .    .    .    .    .    .    .    .    .    .    .    .    .    .
    \end{verbatim}
    \endgroup
    As we can see, it is \( c_0 \)-contractable since its \( c_0 \)-column is positive. Subtracting \( \mathrm{diag}(0) \) from \( \mathrm{row}(1) + \mathrm{row}(2) \) will not change the sign of the \( c_0 \)-column.
  \end{example}

\begin{proposition}\label{prop:row_homo_zero_diag}
     Let \( d \in \mathbb{N}_{\geq 15} \), \( T = \sum_{i=0}^4 \lambda_i \mathrm{row}(i)\) with realization \( p_d \), and \( T' = T + \mathrm{diag}(0) \) with realization \( p'_d \). If \( (p_d)_{c_0} \geq \mathbf{0} \), then we have \( (p'_{d'})_{c_0} \geq \mathbf 1 \) for all \( d' \geq d \).
\end{proposition}
  
  
\begin{proof}
    We see that \( (\mathrm{diag}(0))_{c_0} = \mathbf 1 \) is a constant vector for all degrees. Note that \( (p_{d'})_{c_0} \geq \mathbf 0 \) for all \( d' \geq d \) by Lemma \ref{prop:row_extend_d}. So, we have \( (p_{d'} + \mathrm{diag}(0))_{c_0} \geq \mathbf 1 \) for all \( d' \geq d \).
\end{proof}
  
  
  
\begin{example}
    Let \( T' = \mathrm{row}(2) + \mathrm{row}(3) - \mathrm{diag}(0) \).
    Then, \( (p'_d)_{c_0} < \mathbf{0} \) for all \( d \geq 15 \). Moreover, we can use Proposition \ref{prop:fixed-contraction-homo-row} to show \( (p'_d)_{c_1} > \mathbf{0} \) for all \( d \geq 15 \).
  \end{example}

\begin{proposition}\label{prop:col_homo_d_zero_diag}
Let \( T = \sum_{i=d-4}^d \lambda_i \mathrm{col}(i)\). Assume that \( (p_d)_{d_0} \geq \mathbf{0} \) for some degree \( d \in \mathbb{N}_{\geq 15} \). Then \( (p_{d'} - \mathrm{col}(d'))_{d_0} \geq \mathbf 1 \) for all degrees \( d' \geq d \)
\end{proposition}
  
\begin{proof}
We see that \( (\mathrm{col}(d))_{d_0} = -\mathbf 1 \) is a constant vector for all degrees. Note that \( (p_{d'})_{d_0} \geq \mathbf 0 \) for all \( d' \geq d \) by Lemma \ref{prop:diag_extend_d}. So, we have \( (p_{d'} - \mathrm{col}(d'))_{d_0} \geq \mathbf 1 \) for all dimensions \( d' \geq d \).
\end{proof}
  



\section{An Extended Trivial System}

To compute \(  \Gamma^{\mathrm{even}}_6 \), we defined the system \( \Phi = \Phi_1 \cup \Phi_2 \). We proved that this system is non-trivial. If we can find a system \( \Psi \) that is a superset of \( \Phi \) and is also non-trivial, then we can reduce the number of cases to check. 

\begin{proposition}
    Define \( \Psi = \Phi \cup \left\{ \mathrm{diag}(i) - \mathrm{diag}(j) \mid (i,j) \in Z \right\} \), where 
    \begin{gather*}
        Z \coloneqq \{ (0,1), (0,2), (0,3), (0,4), (0,d-1), (0,d-2), (0,d-3), (0,d-4),\\ (1,2), (1,3), (1, d), (1,d-4), (1,d-2), (1,d-3), (2,d), (2,d-1),\\ (2,d-3), (2, d-4), (3,d), (3,d-1), (3,d-2), (3, d-4)
        (d-4,d), (d-3,d),\\ (d-2,d), (d-2,d-1), (d-1,d-2), (d-1,d-3), (d-1,d), (1, d-1)\}.
    \end{gather*}
    The system \( \Psi \) is non-trivial.
\end{proposition}

\begin{proof}
    Let \( T \in \Psi \). The case \( T \in \Phi \) has already been covered in previous chapters.

    \begin{itemize}
        \item Let \( T = \mathrm{diag}(1) - \mathrm{diag}(d-1) \) be a formal linear combination. Let \( d \in \mathbb{N}_{\geq 15} \) be odd and \( \mathbf{w} \) be a root of \( \Psi \). Then, the realization \( p_d = \sum \lambda_{i,j}x_{i,j}\) of \( T \) for \( t = d \) satisfies \( \lambda_{0,0} = 0 \) and \( \lambda_{0,k} < {0} \) for all \( k = 1, \dots, d \).
    
        If \( \mathbf{w} \) is a trivial root of \( p_d \), then it satisfies \( w_{0,k} = 0 \) for all \( k = 1, \dots, d \). Then, \( \mathrm{diag}(0)(\mathbf{w}) < 0 \); this is a contradiction because \( \mathbf{w} \) is supposed to be a root of \( \mathrm{diag}(0) \).
            
            Let \( d \in \mathbb{N}_{\geq 15} \) be even and \( \mathbf{w} \) be a root of \( \Psi \). The realization \( p_d = \sum \lambda_{i,j}x_{i,j} \) satisfies \( \lambda_{i,d-i} \neq 0 \) if and only if \( i \in \left\{ 0, d \right\} \). If \( \mathbf{w} \) is a trivial solution of \( p_d \), then it satisfies \( w_{d,0} > 0 \) since it is a root of \( \mathrm{diag}(d) \). However, \( \mathrm{col}(d-1)(\mathbf{w}) < 0 \), which is a contradiction since \( \mathbf{w} \) is a root of \( \mathrm{col}(d-1) \).
        
        \item In every other case, it is easy to see that the realization \( p = \sum \lambda_{i,j} x_{i,j} \coloneqq p_d \) of \( T \) satisfies \( \lambda_{0,0} \neq 0 \), \( \mathrm{supp}^+(p) \neq \emptyset \), and \( \mathrm{supp}^-(p) \neq \emptyset \) for any degree \( d \in \mathbb{N}_{\geq 15} \). Thus, any root \( \mathbf{w} \) of \( p \) satisfies \( \mathrm{supp}^+(p) \cap \mathrm{supp}^+(w) \neq \emptyset \) or \( \mathrm{supp}^-(p) \cap \mathrm{supp}^+(w) \neq \emptyset \).
    \end{itemize}
    

    
\end{proof}

\begin{proposition}
    Let \( T \in \Psi \). Then, \( T \) is fixed-contractable.
\end{proposition}

\begin{proof}
    Let \( T \in \Phi \). This case has already been covered in previous chapters. Let \( T \notin \Phi \), i.e. \( T = \mathrm{diag}(i) - \mathrm{diag}(j) \). Then, use Proposition \ref{prop:fixed-contraction-homo-diag} if \( i,j \leq 4 \) or \( i,j \geq d-4 \). Otherwise, the claim follows immediately from Proposition \ref{prop:nwfiewnfiuwneufni2un2}.
\end{proof}


\begin{definition}
 Define \( \Gamma^{\mathrm{even}} \) to be the set of all valid contracted hyperfield configurations \( \mathbf{s} \in H^{\Xi} \) such that \( \hat p^{\mathrm{even}}(\mathbf{s}) = H \) for all \( p \in \Psi \), and \( \Gamma^{\mathrm{odd}} \) to be the set of all valid contracted hyperfield configurations \( \mathbf{s} \in H^{\Xi} \) such that \( \hat p^{\mathrm{odd}}(\mathbf{s}) = H \) for all \( p \in \Psi \). 

\end{definition}

\begin{definition}
    We define \( \Gamma^{\mathrm{even}}_6 \coloneqq \Gamma^{\mathrm{even}} \cap \left\{ \mathbf{s} \in H^{\Xi} : \lvert \mathrm{supp}^+(\mathbf{s}) \rvert = 6 \right\} \). 
    Additionally, let us define \( \Gamma^{\mathrm{odd}}_6 \coloneqq \Gamma^{\mathrm{odd}} \cap \left\{ \mathbf{s} \in H^{\Xi} : \lvert \mathrm{supp}^+(\mathbf{s}) \rvert = 6 \right\} \).

\end{definition}

\begin{proposition}
    We have \( \lvert \Gamma^{\mathrm{even}}_6 \rvert  = 106806\) and \( \lvert \Gamma^{\mathrm{odd}}_6 \rvert  = 110272\).
\end{proposition}

\begin{proof}
    This is verified by computer.
\end{proof}

We excluded around 100,000 cases; there are still around 217,000 cases left to check.

\section{Reducing Cases with Fixed-Contractable Forms}

The final reduction step relies on using fixed-contractable forms as a filter. 
    Let \( G \) be a set of formal linear combinations that contains the expressions
    \begin{gather*}
        \mathrm{col}(i_1) + \mathrm{row}(i_2), 
        \mathrm{col}(i_1) - \mathrm{col}(i_2), 
        \mathrm{col}(i_1) - \mathrm{diag}(i_2), 
        \mathrm{row}(i_1) + \mathrm{row}(i_2), 
        \mathrm{row}(i_1) - \mathrm{row}(i_2), \\
        \mathrm{row}(i_1) - \mathrm{col}(i_2), 
        \mathrm{row}(i_1) - \mathrm{diag}(i_2), 
        \mathrm{diag}(i_1) - \mathrm{diag}(i_2), 
        \mathrm{diag}(i_1) + \mathrm{row}(i_2) + \mathrm{col}(i_3), \\
        \mathrm{row}(i_1) - \mathrm{diag}(i_2) + \mathrm{col}(i_3), 
        \mathrm{col}(i_1) + \mathrm{row}(i_2) + \mathrm{col}(i_3), 
        \mathrm{col}(i_1) + \mathrm{row}(i_2) - \mathrm{col}(i_3), \\
        \mathrm{col}(i_1) - \mathrm{row}(i_2) - \mathrm{col}(i_3), 
        \mathrm{col}(i_1) + \mathrm{col}(i_2) - \mathrm{col}(i_3), 
        \mathrm{diag}(i_1) - \mathrm{diag}(i_2) + \mathrm{col}(i_3) + \mathrm{row}(i_4), \\
        \mathrm{row}(i_1) + \mathrm{row}(i_2) + \mathrm{col}(i_3) + \mathrm{diag}(i_4), 
        \mathrm{row}(i_1) + \mathrm{row}(i_2) + \mathrm{col}(i_3) - \mathrm{diag}(i_4), \\
        \mathrm{row}(i_1) + \mathrm{row}(i_2) - \mathrm{col}(i_3) - \mathrm{diag}(i_4), 
        \mathrm{diag}(i_1) - \mathrm{diag}(i_2) + \mathrm{col}(i_3) + \mathrm{row}(i_4) + \mathrm{col}(i_5), \\
        \mathrm{diag}(i_1) - \mathrm{diag}(i_2) + \mathrm{col}(i_3) + \mathrm{row}(i_4) - \mathrm{col}(i_5), 
        \mathrm{diag}(i_1) - \mathrm{diag}(i_2) + \mathrm{col}(i_3) - \mathrm{row}(i_4) - \mathrm{col}(i_5), \\
        \mathrm{diag}(i_1) - \mathrm{diag}(i_2) + \mathrm{diag}(i_3) + \mathrm{row}(i_4) + \mathrm{col}(i_5), 
        \mathrm{diag}(i_1) - \mathrm{diag}(i_2) + \mathrm{diag}(i_3) + \mathrm{row}(i_4) - \mathrm{col}(i_5), \\
        \mathrm{diag}(i_1) - \mathrm{diag}(i_2) + \mathrm{col}(i_3) + \mathrm{col}(i_4) - \mathrm{col}(i_5),
        \mathrm{row}(i_1) + \mathrm{row}(i_2) + \mathrm{row}(i_3) + \mathrm{row}(i_4), \\
        \mathrm{row}(i_1) + \mathrm{row}(i_2) + \mathrm{row}(i_3) - \mathrm{row}(i_4), 
        \mathrm{row}(i_1) + \mathrm{row}(i_2) - \mathrm{row}(i_3) - \mathrm{row}(i_4), \\
        \mathrm{row}(i_1) - \mathrm{row}(i_2) - \mathrm{row}(i_3) - \mathrm{row}(i_4), 
        \mathrm{col}(i_1) + \mathrm{col}(i_2) + \mathrm{col}(i_3) + \mathrm{col}(i_4), \\
        \mathrm{col}(i_1) + \mathrm{col}(i_2) + \mathrm{col}(i_3) - \mathrm{col}(i_4), 
        \mathrm{col}(i_1) + \mathrm{col}(i_2) - \mathrm{col}(i_3) - \mathrm{col}(i_4), \\
        \mathrm{col}(i_1) - \mathrm{col}(i_2) - \mathrm{col}(i_3) - \mathrm{col}(i_4), 
        \mathrm{diag}(i_1) + \mathrm{diag}(i_2) + \mathrm{diag}(i_3) + \mathrm{diag}(i_4), \\
        \mathrm{diag}(i_1) + \mathrm{diag}(i_2) + \mathrm{diag}(i_3) - \mathrm{diag}(i_4), 
        \mathrm{diag}(i_1) + \mathrm{diag}(i_2) - \mathrm{diag}(i_3) - \mathrm{diag}(i_4), \\
        \mathrm{diag}(i_1) - \mathrm{diag}(i_2) - \mathrm{diag}(i_3) - \mathrm{diag}(i_4),  
        \mathrm{col}(i_1) + \mathrm{col}(i_2) + \mathrm{col}(i_3) + \mathrm{col}(i_4) - \mathrm{col}(i_5), \\
        \mathrm{col}(i_1) + \mathrm{col}(i_2) + \mathrm{col}(i_3) - \mathrm{col}(i_4) - \mathrm{col}(i_5), 
        \mathrm{col}(i_1) + \mathrm{col}(i_2) - \mathrm{col}(i_3) - \mathrm{col}(i_4) - \mathrm{col}(i_5), \\
        \mathrm{row}(i_1) + \mathrm{row}(i_2) + \mathrm{row}(i_3) + \mathrm{row}(i_4) - \mathrm{row}(i_5), 
        \mathrm{row}(i_1) + \mathrm{row}(i_2) + \mathrm{row}(i_3) - \mathrm{row}(i_4) - \mathrm{row}(i_5), \\
        \mathrm{row}(i_1) + \mathrm{row}(i_2) - \mathrm{row}(i_3) - \mathrm{row}(i_4) - \mathrm{row}(i_5)
    \end{gather*}  
    for all \( i_1, \dots, i_5 \in \left\{ 0,1,2,3,4,t-4,t-3,t-2,t-1,t \right\} \). 

    \begin{remark}
The choice of linear combinations in \( G \) is arbitrary; the more forms we include, the more cases we can exclude later. Common sense suggests not every formal linear combination in \( G \) is fixed-contractable; so we develop an algorithm to computationally prove that some forms are fixed-contractable. 
\end{remark}

\subsection*{Step 1: Realization}

The first step of the algorithm is to realize all the forms in \( G \) at a fixed degree \( D \) and check if it is contractable at degree \( D \). We choose \( D = 40 \); the higher the degree, the more likely it is that a form is fixed-contractable from that degree onwards. 


\begin{algorithm}[H]
    \caption{Realize}
    \label{alg:realize}
    \begin{algorithmic}[1]
    \Require a set of formal forms $G$ and degree \( D \)
\Ensure $G' \subset G$ is a set containing all forms whose realization at degree \( D \) is contractable.
\Function{\texttt{realize}}{}
    \State \( G' \gets \emptyset \)
    \For{$T \in G$}
        \State \( p \gets  \) realization of \( T \) at degree \( D \)
        \If{\texttt{is\_contractable}($p$)}
            \State $G' \gets G' \cup \left\{ T \right\}$
        \EndIf        
    \EndFor
    \State \Return $G'$
\EndFunction
\end{algorithmic}
\end{algorithm}

Computing \( G' \) took around two hours on a MacBook Air with an M1 chip. The details can be found in the source code \cite{ducrepo} under the file \texttt{chapter07\_step1\_realize.ipynb}.

\subsection*{Step 2: Automatic Proof of Fixed-Contractability}

The set \( G' \) contains formal linear combinations of Pascal forms whose realizations at degree \( D \) are contractable. However, this does not imply that they are fixed-contractable. We need to prove that they are fixed-contractable. For that purpose, we develop an algorithm that automatically proves fixed-contractability. Its pseudocode is given below.

\begin{algorithm}[H]
    \caption{Automatic Proof}
    \label{alg:autoproof}
    \begin{algorithmic}[1]
    \Require $G'$
\Ensure $G'' \subset G'$ is a set containing fixed-contractable forms
\Function{\texttt{prove}}{}
    \State \( G'' \gets \emptyset \)
    \For{$T \in G'$}
        \If{\texttt{prove\_fixed\_contractable}($T$)}
            \State $G'' \gets G'' \cup \left\{ T \right\}$
        \EndIf        
    \EndFor
    \State \Return $G''$
\EndFunction
\end{algorithmic}
\end{algorithm}
The function \texttt{prove\_fixed\_contractable}($T$) just checks if Proposition \ref{prop:fixed-contraction-homo-row}, \ref{prop:i3jhr23h923h8}, \ref{prop:fixed-contraction-homo-col}, \ref{prop:23e232sdada2kmkl}, \ref{prop:fixed-contraction-homo-diag}, \ref{prop:row_homo_diag}, \ref{prop:row_homo_zero_diag}, or Proposition \ref{prop:col_homo_d_zero_diag} can be applied to \( T \); if they can, we proved that \( T \) is fixed-contractable. For details, we refer to the implementation found in \cite{ducrepo} under the file \texttt{chapter07\_step2\_prove.ipynb}. 


\subsection*{Step 3: Filtering Invalid Cases}

Once the filter set \( G'' \) has been constructed, we proceed as follows: For each \( \mathbf{s} \in \Gamma^{\mathrm{even}}_6 \), we check whether for all \( T \in G'' \) its realization \( p \coloneqq p_D \) with \( D = 40 \) satisfies \( 0 \in \hat p^{\mathrm{even}}(\mathbf{s}) \). If there exists any \( T \) for which this condition fails, \( \mathbf{s} \) is excluded. By applying this method, we were able to reduce the number of cases from 106,806 to just 6,700. 
\begin{algorithm}[H]
    \caption{Apply Filter (even)}
    \label{alg:filter}
    \begin{algorithmic}[1]
\Function{\texttt{filter\_even}}{}
    \For{$\mathbf{s} \in \Gamma^{\mathrm{even}}_6$}
        \For{$T \in G''$}
        \State \( p \gets \) realization of \( T \) at degree \( D = 40 \)
        \If{ \(  0 \notin \hat p^{\mathrm{even}}(\mathbf{s}) \)}
            \State exclude \( \mathbf{s} \) from further consideration
        \EndIf        
        \EndFor       
    \EndFor
\EndFunction
\end{algorithmic}
\end{algorithm}
The process is also repeated for \( \mathbf{s} \in \Gamma^{\mathrm{odd}}_6 \) with \( D = 41 \), resulting in 8737 cases. 
In total, this amounts to 15,437 cases. 
The computations required three hours on a MacBook Pro equipped with an M3 chip. Further details can be found in the source code \cite{ducrepo} under the file \texttt{chapter07\_step3\_apply\_filter.ipynb}.


\section{Carrying Out the Proof}

To finalize the proof, it is necessary to check the \( 15,437 \) cases. 
One approach is to reduce the number of cases further by considering a larger set \( G \), 
which, however, requires more computational power. 
Once the number of cases is sufficiently small, we can apply the techniques developed in previous chapters, 
such as the Invertibility Criterion, the Hyperfield Criterion, and the Hexagon Criterion, to eliminate all remaining cases. Then, we deal with each case individually, i.e. applying the Invertibility Criterion with hand-picked \( \lambda \).
If it is not possible to rule out a particular case \( \mathbf{s} \in H^{\Xi} \), 
it may be necessary to develop a new technique to resolve it, or one may be on the right track to finding a counterexample.

Due to time constraints, we were unable to complete the proof. 
The next step would have been to apply the Invertibility Criterion, the Hyperfield Criterion, 
and the Hexagon Criterion to the remaining cases exactly like in Chapter 9.
Depending on how many cases remain after applying these criteria, 
we might also need to compute an even larger set \( G \). 