\chapter{Valid Outcomes of Positive Support Size Six}

We want to prove that for all valid integral outcomes \( \mathbf w \) with \( |\mathrm{supp}^+(\mathbf w)| = 6 \) we have
\begin{align*}
    \mathrm{deg}(\mathbf w) \leq 9.
\end{align*}
In principle we could use the same idea as in the previous chapters to find all valid outcomes of positive support size six, i.e. we first start with computing 
\begin{align*}
    \Gamma^{\mathrm{even}} \cap \left\{ \mathbf{s} \in H^{\Xi} : \lvert \mathrm{supp}^+(\mathbf{s}) \rvert = 6 \right\}
        \text{ and }
         \Gamma^{\mathrm{odd}} \cap \left\{ \mathbf{s} \in H^{\Xi} : \lvert \mathrm{supp}^+(\mathbf{s}) \rvert = 6 \right\}.
\end{align*}

We make a slight modification to these sets. In the previous chapters, we have considered contractions of size four. We can generalize this concept and consider contraction size of five, i.e. we have the contraction variables as depicted in Figure \ref{fig:sjdnfkjewnjhwurhi3u2h}. From now on, when we write \( \mathbf{s} \in H^{\Xi} \) we mean contractions of size five.

\begin{definition}
    We define 
    \begin{gather*}
        \Gamma^{\mathrm{even}}_6 \coloneqq \Gamma^{\mathrm{even}} \cap \left\{ \mathbf{s} \in H^{\Xi} : \lvert \mathrm{supp}^+(\mathbf{s}) \rvert = 6 \right\}
        \text{ and }
        \Gamma^{\mathrm{odd}}_6 \coloneqq \Gamma^{\mathrm{odd}} \cap \left\{ \mathbf{s} \in H^{\Xi} : \lvert \mathrm{supp}^+(\mathbf{s}) \rvert = 6 \right\},
    \end{gather*}
    where \( \Gamma^{\mathrm{even}}  \) and \(  \Gamma^{\mathrm{odd}} \) are defined analogously to Definition \ref{def:sdjsndjknsdj}, but now they contain hyperfield configurations of contraction size five for 
    \begin{gather*}
        \Phi_1 \coloneqq \{ 
            \mathrm{col}(1), \mathrm{col}(2), \mathrm{col}(3), \mathrm{col}(3), \mathrm{row}(1), \mathrm{row}(2), \mathrm{row}(3),  \mathrm{row}(4),\\
             \mathrm{diag}(1), \mathrm{diag}(2), \mathrm{diag}(3), \mathrm{diag}(4), \mathrm{diag}(d-1), \mathrm{diag}(d-2), \mathrm{diag}(d-3), \mathrm{diag}(d-4) 
         \},
    \end{gather*}
    and 
    \begin{gather*}
        \Phi_2 \coloneqq \{ 
            \mathrm{col}(d), \mathrm{col}(d-1), \mathrm{col}(d-2), \mathrm{col}(d-3), \mathrm{col}(d-4), \\
            \mathrm{row}(d), \mathrm{row}(d-1), \mathrm{row}(d-2), \mathrm{row}(d-3), \mathrm{row}(d-4) 
         \}.
    \end{gather*}
\end{definition}


\begin{figure}[H]
    \begin{align*}
        \begin{array}{cccccccccccccccccccc}
            y_{0,4} & & & & & & & & & & & & \\
            y_{0,3} & y_{1,4} & & & & & & & & & & & \\
            y_{0,2} & y_{1,3} & y_{2,4} & & & & & & & & & & \\
            y_{0,1} & y_{1,2} & y_{2,3} & y_{3,4} & & & & & & & & & \\
            y_{0,0} & y_{1,1} & y_{2,2} & y_{3,3} & y_{4,4} & & & & & & & & \\
            c_0 & y_{1,0} & y_{2,1} & y_{3,2} & y_{4,3} & d_0 & & & & & & & \\
            c_0 & c_1 & y_{2,0} & y_{3,1} & y_{4,2} & d_1 & e_0 & & & & & & \\
            c_0 & c_1 & c_2 & y_{3,0} & y_{4,1} & d_2 & e_1 & d_0 & & & & & \\
            c_0 & c_1 & c_2 & c_3 &  y_{4,0}  & d_3 & e_2 & d_1 & e_0 & & & & \\
            c_0 & c_1 & c_2 & c_3 &  c_4  & d_4 & e_3 & d_2 & e_1 & d_0 & & & \\
            c_0 & c_1 & c_2 & c_3 &  c_4  & * & e_4 & d_3 & e_2 & d_1 & e_0 & & \\
            c_0 & c_1 & c_2 & c_3 &  c_4  & * & * & d_4 & e_3 & d_2 & e_1 & d_0 & \\
            c_0 & c_1 & c_2 & c_3 &  c_4  & * & * & * & e_4 & d_3 & e_2 & d_1 & e_0 \\
            x_{0,4} & x_{1,4} & x_{2,4} & x_{3,4} &  x_{4,4}  & b_4 & b_4 & b_4 & b_4 & z_{0,4} & z_{1,4} & z_{2,4} & z_{3,4} & z_{4,4} \\
            x_{0,3} & x_{1,3} & x_{2,3} & x_{3,3} & x_{4,3} & b_3 & b_3 & b_3 & b_3 & b_3 & z_{0,3} & z_{1,3} & z_{2,3} & z_{3,3} & z_{4,3} \\
            x_{0,2} & x_{1,2} & x_{2,2} & x_{3,2} & x_{4,2} & b_2 & b_2 & b_2 & b_2 & b_2 & b_2 & z_{0,2} & z_{1,2} & z_{2,2} & z_{3,2} & z_{4,2}\\
            x_{0,1} & x_{1,1} & x_{2,1} & x_{3,1} & x_{4,1} & b_1 & b_1 & b_1 & b_1 & b_1 & b_1 & b_1 & z_{0,1} & z_{1,1} & z_{2,1} & z_{3,1} & z_{4,1} \\
            x_{0,0} & x_{1,0} & x_{2,0} & x_{3,0} & x_{4,0} & b_0 & b_0 & b_0 & b_0 & b_0 & b_0 & b_0 & b_0 & z_{0,0} & z_{1,0} & z_{2,0} & z_{3,0} & z_{4,0}
        \end{array}
    \end{align*}  
    \caption{Contraction variables for \( d = 16 \) are depicted.}\label{fig:sjdnfkjewnjhwurhi3u2h}
\end{figure}


\begin{proposition}
    The following holds:
    \begin{enumerate}
        \item We have \( \lvert \Gamma^{\mathrm{even}}_6 \rvert  = 150032\).
        \item We have \( \lvert \Gamma^{\mathrm{odd}}_6 \rvert  = 154177\).
    \end{enumerate}
\end{proposition}

\begin{proof}
    This is verifed by computer. 
    
    TODO. 
\end{proof}

The number of cases to check has increased by a factor of hundred compared to the previous chapters. If we would be able to reduce the cases immensely, we could in theory apply the same techniques as in the previous chapters to prove \( \mathrm{deg}(\mathbf w) \leq 9 \). We will spend the remaining part of this chapter to find a way to reduce the number of cases to check to around 10,000 cases; this is a reduction by a factor of 30. We hope that this reduction is sufficient to make the problem computationally feasible for future proofs. Due to time constraints of this thesis, we will not be able to prove \( \mathrm{deg}(\mathbf w) \leq 9 \) for all valid outcomes of positive support size six.