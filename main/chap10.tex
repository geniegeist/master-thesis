\chapter{Valid Outcomes of Positive Support Size Six}

We want to prove that for all valid integral outcomes \( \mathbf w \) with \( |\mathrm{supp}^+(\mathbf w)| = 6 \) we have
\begin{align*}
    \mathrm{deg}(\mathbf w) \leq 9.
\end{align*}
The original paper by Bik and Marigliano \cite{bik2022classifying} did not attempt to prove this result. 
In principle we could use the same idea as in the previous chapters to find all valid outcomes of positive support size six, i.e. we first start with computing 
\begin{align*}
    \Gamma^{\mathrm{even}} \cap \left\{ \mathbf{s} \in H^{\Xi} : \lvert \mathrm{supp}^+(\mathbf{s}) \rvert = 6 \right\}
        \text{ and }
         \Gamma^{\mathrm{odd}} \cap \left\{ \mathbf{s} \in H^{\Xi} : \lvert \mathrm{supp}^+(\mathbf{s}) \rvert = 6 \right\}.
\end{align*}

We make a slight modification to these sets. In the previous chapters, we have considered contractions of size four. We can generalize this concept and consider contraction size of five, i.e. we have the contraction variables as depicted in Figure \ref{fig:sjdnfkjewnjhwurhi3u2h}. From now on, when we write \( \mathbf{s} \in H^{\Xi} \) we mean contractions of size five.

\begin{definition}
    We define 
    \begin{gather*}
        \Gamma^{\mathrm{even}}_6 \coloneqq \Gamma^{\mathrm{even}} \cap \left\{ \mathbf{s} \in H^{\Xi} : \lvert \mathrm{supp}^+(\mathbf{s}) \rvert = 6 \right\}
        \text{ and }
        \Gamma^{\mathrm{odd}}_6 \coloneqq \Gamma^{\mathrm{odd}} \cap \left\{ \mathbf{s} \in H^{\Xi} : \lvert \mathrm{supp}^+(\mathbf{s}) \rvert = 6 \right\},
    \end{gather*}
    where \( \Gamma^{\mathrm{even}}  \) and \(  \Gamma^{\mathrm{odd}} \) are defined analogously to Definition \ref{def:sdjsndjknsdj}, but now they contain hyperfield configurations of contraction size five for 
    \begin{gather*}
        \Phi_1 \coloneqq \{ 
            \mathrm{col}(1), \mathrm{col}(2), \mathrm{col}(3), \mathrm{col}(3), \mathrm{row}(1), \mathrm{row}(2), \mathrm{row}(3),  \mathrm{row}(4),\\
             \mathrm{diag}(1), \mathrm{diag}(2), \mathrm{diag}(3), \mathrm{diag}(4), \mathrm{diag}(d-1), \mathrm{diag}(d-2), \mathrm{diag}(d-3), \mathrm{diag}(d-4) 
         \},
    \end{gather*}
    and 
    \begin{gather*}
        \Phi_2 \coloneqq \{ 
            \mathrm{col}(d), \mathrm{col}(d-1), \mathrm{col}(d-2), \mathrm{col}(d-3), \mathrm{col}(d-4), \\
            \mathrm{row}(d), \mathrm{row}(d-1), \mathrm{row}(d-2), \mathrm{row}(d-3), \mathrm{row}(d-4) 
         \}.
    \end{gather*}
\end{definition}


\begin{figure}[H]
    \begin{align*}
        \begin{array}{cccccccccccccccccccc}
            y_{0,4} & & & & & & & & & & & & \\
            y_{0,3} & y_{1,4} & & & & & & & & & & & \\
            y_{0,2} & y_{1,3} & y_{2,4} & & & & & & & & & & \\
            y_{0,1} & y_{1,2} & y_{2,3} & y_{3,4} & & & & & & & & & \\
            y_{0,0} & y_{1,1} & y_{2,2} & y_{3,3} & y_{4,4} & & & & & & & & \\
            c_0 & y_{1,0} & y_{2,1} & y_{3,2} & y_{4,3} & d_0 & & & & & & & \\
            c_0 & c_1 & y_{2,0} & y_{3,1} & y_{4,2} & d_1 & e_0 & & & & & & \\
            c_0 & c_1 & c_2 & y_{3,0} & y_{4,1} & d_2 & e_1 & d_0 & & & & & \\
            c_0 & c_1 & c_2 & c_3 &  y_{4,0}  & d_3 & e_2 & d_1 & e_0 & & & & \\
            c_0 & c_1 & c_2 & c_3 &  c_4  & d_4 & e_3 & d_2 & e_1 & d_0 & & & \\
            c_0 & c_1 & c_2 & c_3 &  c_4  & * & e_4 & d_3 & e_2 & d_1 & e_0 & & \\
            c_0 & c_1 & c_2 & c_3 &  c_4  & * & * & d_4 & e_3 & d_2 & e_1 & d_0 & \\
            c_0 & c_1 & c_2 & c_3 &  c_4  & * & * & * & e_4 & d_3 & e_2 & d_1 & e_0 \\
            x_{0,4} & x_{1,4} & x_{2,4} & x_{3,4} &  x_{4,4}  & b_4 & b_4 & b_4 & b_4 & z_{0,4} & z_{1,4} & z_{2,4} & z_{3,4} & z_{4,4} \\
            x_{0,3} & x_{1,3} & x_{2,3} & x_{3,3} & x_{4,3} & b_3 & b_3 & b_3 & b_3 & b_3 & z_{0,3} & z_{1,3} & z_{2,3} & z_{3,3} & z_{4,3} \\
            x_{0,2} & x_{1,2} & x_{2,2} & x_{3,2} & x_{4,2} & b_2 & b_2 & b_2 & b_2 & b_2 & b_2 & z_{0,2} & z_{1,2} & z_{2,2} & z_{3,2} & z_{4,2}\\
            x_{0,1} & x_{1,1} & x_{2,1} & x_{3,1} & x_{4,1} & b_1 & b_1 & b_1 & b_1 & b_1 & b_1 & b_1 & z_{0,1} & z_{1,1} & z_{2,1} & z_{3,1} & z_{4,1} \\
            x_{0,0} & x_{1,0} & x_{2,0} & x_{3,0} & x_{4,0} & b_0 & b_0 & b_0 & b_0 & b_0 & b_0 & b_0 & b_0 & z_{0,0} & z_{1,0} & z_{2,0} & z_{3,0} & z_{4,0}
        \end{array}
    \end{align*}  
    \caption{Contraction variables for \( d = 16 \) are depicted.}\label{fig:sjdnfkjewnjhwurhi3u2h}
\end{figure}

We increased the contraction size because we gain more information about possible supports, which we hope would become useful when dealing cases by hand.

\begin{proposition}
    The following holds:
    \begin{enumerate}
        \item We have \( \lvert \Gamma^{\mathrm{even}}_6 \rvert  = 150032\).
        \item We have \( \lvert \Gamma^{\mathrm{odd}}_6 \rvert  = 154177\).
    \end{enumerate}
\end{proposition}

\begin{proof}
    This is verifed by computer. 
    
    TODO. 
\end{proof}

The number of cases to check has increased by a factor of hundred compared to the previous chapters. If we would be able to reduce the cases immensely, we could in theory apply the same techniques as in the previous chapters to prove \( \mathrm{deg}(\mathbf w) \leq 9 \). We will spend the remaining part of this chapter to find a way to reduce the number of cases to check to around 10,000 cases; this is a reduction by a factor of 30. We hope that this reduction is sufficient to make the problem computationally feasible for future proofs. Due to time constraints of this thesis, we will not be able to prove \( \mathrm{deg}(\mathbf w) \leq 9 \) for all valid outcomes of positive support size six.

\section{An Extended Trivial System}

We have the system \( \Phi = \Phi_1 \cup \Phi_2 \). We proved that this system is trivial. If we can find a system \( \Psi \) that is a superset of \( \Phi \) and is also trivial, then we can reduce the number of cases to check. 

\begin{proposition}
    Define \( \Psi' = \Phi \cup \left\{ \mathrm{diag}(i) - \mathrm{diag}(j) \mid (i,j) \in Z \right\} \), where 
    \begin{gather*}
        Z \coloneqq \{ (0,1), (0,2), (0,3), (0,4), (0,d-1), (0,d-2), (0,d-3), (0,d-4),\\ (1,2), (1,3), (1, d), (1,d-4), (1,d-2), (1,d-3),\\ (2,d), (2,d-1), (2,d-3), (2, d-4),\\ (3,d), (3,d-1), (3,d-2), (3, d-4)\\
        (d-4,d), (d-3,d), (d-2,d), (d-2,d-1), (d-1,d-2), (d-1,d-3), (d-1,d)\}.
    \end{gather*}
    The system \( \Psi' \) is trivial.
\end{proposition}

\begin{proof}
    It is easy to see that every \( p = \sum \lambda_{i,j} x_{i,j} = \mathrm{diag}(q) - \mathrm{diag}(r) \) for \( (q,r) \in Z \) satisfies \( \lambda_{0,0} \neq 0 \), \( \mathrm{supp}^+(p) \neq \emptyset \), and \( \mathrm{supp}^-(p) \neq \emptyset \). Thus, any root \( \mathbf{w} \) of \( p \) satisfies \( \mathrm{supp}^+(p) \cap \mathrm{supp}^+(w) \neq \emptyset \) or \( \mathrm{supp}^-(p) \cap \mathrm{supp}^+(w) \neq \emptyset \).
\end{proof}

Next, we need to show the following proposition.

\begin{proposition}
    Let \( p \in \left\{ \mathrm{diag}(i) - \mathrm{diag}(j) \mid (i,j) \in Z \right\} \). Let \( d \geq 15 \). As in Proposition \ref{prop:contracted-part-1}, we can write \( \mathrm{sign}(p) = \hat p^{\mathrm{even}} \) and  \( \mathrm{sign}(p) = \hat p^{\mathrm{odd}} \) for some linear form \( \hat p^{\mathrm{even}}, \hat p^{\mathrm{odd}} \in H[\mathbf{x}, \mathbf{y}, \mathbf{z}, \mathbf{b}, \mathbf{c}, \mathbf{d}, \mathbf{e}] \) for even and odd degree \( d \), respectively. These linear forms \(  \hat p^{\mathrm{even}}, \hat p^{\mathrm{odd}} \) are independent of \( d \).
\end{proposition}

\begin{proof}
    This will later be proved by computer. For now, we assume this is true. Do not worry it will not lead to circular reasoning.
\end{proof}

\begin{proposition}
    Define \( \Psi \coloneqq \Psi \cup \left\{ \mathrm{diag}(1) - \mathrm{diag}(d-1) \right\}\). The system \( \Psi \) is trivial.
\end{proposition}

\begin{proof}
    TODO
\end{proof}

\begin{proposition}
    Let \( p = \mathrm{diag}(1) - \mathrm{diag}(d-1)\). We can write \( \mathrm{sign}(p) = \hat p^{\mathrm{even}} \) and  \( \mathrm{sign}(p) = \hat p^{\mathrm{odd}} \) for some linear form \( \hat p^{\mathrm{even}}, \hat p^{\mathrm{odd}} \in H[\mathbf{x}, \mathbf{y}, \mathbf{z}, \mathbf{b}, \mathbf{c}, \mathbf{d}, \mathbf{e}] \) for even and odd degree \( d \), respectively. These linear forms \(  \hat p^{\mathrm{even}}, \hat p^{\mathrm{odd}} \) are independent of \( d \).
\end{proposition}

\begin{proof}
    Will be proven by computer.
\end{proof}

We redefine the following sets.

\begin{definition}
    We define the following three solution sets:
    \begin{enumerate}
        \item     Define \( \Gamma^{\mathrm{even}} \) to be the set of all valid contracted hyperfield configurations \( \mathbf{s} \in H^{\Xi} \) such that \( \hat p^{\mathrm{even}}(\mathbf{s}) = H \) for all \( p \in \Psi \).

        \item     Define \( \Gamma^{\mathrm{odd}} \) to be the set of all valid contracted hyperfield configurations \( \mathbf{s} \in H^{\Xi} \) such that \( \hat p^{\mathrm{odd}}(\mathbf{s}) = H \) for all \( p \in \Psi \).
    \end{enumerate}
\end{definition}

\begin{definition}
    We define 
    \begin{gather*}
        \Gamma^{\mathrm{even}}_6 \coloneqq \Gamma^{\mathrm{even}} \cap \left\{ \mathbf{s} \in H^{\Xi} : \lvert \mathrm{supp}^+(\mathbf{s}) \rvert = 6 \right\}
        \text{ and }
        \Gamma^{\mathrm{odd}}_6 \coloneqq \Gamma^{\mathrm{odd}} \cap \left\{ \mathbf{s} \in H^{\Xi} : \lvert \mathrm{supp}^+(\mathbf{s}) \rvert = 6 \right\}.
    \end{gather*}
\end{definition}


\begin{proposition}
    The following holds:
    \begin{enumerate}
        \item We have \( \lvert \Gamma^{\mathrm{even}}_6 \rvert  = 106806\).
        \item We have \( \lvert \Gamma^{\mathrm{odd}}_6 \rvert  = 110272\).
    \end{enumerate}
\end{proposition}

We exluded around 100,000 cases; there are still around 217,000 cases left to check.

\section{Contractable Pascal Forms}

To further reduce the cases, we want to generate a set \( G \) of hyperfield Pascal forms that are \emph{fixed-contractable}. We will see that of the 217,000 cases we can exclude cases that are not a hyperfield root of some \( p \in G \).

\begin{definition}
    Let \( p \) be a Hyperfield linear form. We say \( p \) is \emph{contractable} if we can write \( \mathrm{sign}(p) = \hat p^{\mathrm{even}} \) and  \( \mathrm{sign}(p) = \hat p^{\mathrm{odd}} \) for some linear form \( \hat p^{\mathrm{even}}, \hat p^{\mathrm{odd}} \in H[\mathbf{x}, \mathbf{y}, \mathbf{z}, \mathbf{b}, \mathbf{c}, \mathbf{d}, \mathbf{e}] \) for even and odd degree \( d \), respectively. 
\end{definition}

\begin{definition}
    Let \( i = 0,1,2,3,4 \).
    Let \( p \) be a Hyperfield linear form. We say \( p \) is \emph{contractable} on \( b_i \) if we can write
    \begin{align*}
        p = \sum_{(i,j) \in V_d \setminus \left\{ (5,i), \dots, (d-i-5, i) \right\}} \lambda_{i,j} x_{i,j}  +\lambda b_i
    \end{align*}
    where \( b_i \coloneqq x_{5,i} + \dots + x_{d-5-i,i} \). We similarly define \( p \) is \emph{contractable} on \( c_i \), \( d_i \), and \( e_i \).
\end{definition}

\begin{definition}
    Let \( p \) be a Hyperfield linear form. We say \( p \) is \emph{fixed-contractable} if it is contractable and the linear forms \( \hat p^{\mathrm{even}} \) and \( \hat p^{\mathrm{odd}} \) are independent of \( d \). 

    We similarly define \( p \) is \emph{fixed-contractable} on \( b_i \), \( c_i \), \( d_i \), and \( e_i \).
\end{definition}

\begin{definition}
    Let $p(x) = \sum \lambda_{ij} x_{ij}$ be any hyperfield linear form. For $i = 0,1,2,3,4$ we write \( p_{c_{i}} \coloneqq \begin{bmatrix} \lambda_{i, 5} & \dots & \lambda_{i,d-i-5} \end{bmatrix} \).
  We call this the $i$-th $c$-column of $p$.
  
   Similarly, we define $p_{b_{i}}$, $p_{d_{i}}$ and $p_{e_{i}}$ to denote the $i$-th $b$-row, $d$-diagonal and $e$-diagonal of $p$, respectively.
  \end{definition}


\begin{proposition}
    Let $p$ be a linear combination of $\{ \mathrm{row}(i), \mathrm{col}(i), \mathrm{diag}(i) : i \in \{ 0, \dots, 4\} \cup \{ d-4, \dots, d \} \}$. Then, the following statements hold:
  \begin{itemize}
  \item The $c$-columns of $\mathrm{sign}(p)$ only depend on $\{ \mathrm{sign}(\mathrm{row}(i)), \mathrm{sign}(\mathrm{diag}(i)) \}_{i = 0, \dots, 4}$.
  \item The $b$-rows of $\mathrm{sign}(p)$ only depend on $\{ \mathrm{sign}(\mathrm{col}(i)), \mathrm{sign}(\mathrm{diag}(d-i)) \}_{i = 0, \dots, 4}$.
  \item The $d$-diagonals and $e$-diagonals of $\mathrm{sign}(p)$ only depend on 
  \begin{align*}
    \{ \mathrm{sign}(\mathrm{row}(d-i)), \mathrm{sign}(\mathrm{col}(d-i)) \}_{i = 0, \dots, 4}.
  \end{align*}
  \end{itemize}
  \end{proposition}
  
  \begin{proof}
   This follows easily from the definition of $\mathrm{row}, \mathrm{col}, \mathrm{diag}$.
  \end{proof}

  \begin{proposition}
    The following statements hold:
    \begin{itemize}
    \item Let $p \in \{\mathrm{sign}(\mathrm{row}(i)), \mathrm{sign}(\mathrm{diag}(i)) \}_{i = 0, \dots, 4}$. For any $i = 0, \dots, 4$ the $c_{i}$-column of $p$ is a constant vector, i.e. $p_{c_{i}} \in \{ -1, 0, 1 \}$. 
  
    \item Let $p \in \{\mathrm{sign}( \mathrm{col}(i)),\mathrm{sign}(\mathrm{diag}(d-i)) \}_{i = 0, \dots, 4}$. For any $i = 0, \dots, 4$ the $b_{i}$-row of $p$ is a constant vector, i.e. $p_{b_{i}} \in \{ -1, 0, 1 \}$. 
  
    \item Let $p \in \{ \mathrm{sign}(\mathrm{row}(d-i)), \mathrm{sign}(\mathrm{col}(d-i)) \}_{i = 0, \dots, 4}$. For any $i = 0, \dots, 4$ the $d_{i}$-diagonal of $p$ is a constant vector, i.e. $p_{d_{i}} \in \{ -1, 0, 1 \}$; similarly for the $e_{i}$-diagonal. 
    \end{itemize}
  \end{proposition}
  
  \begin{proof}
    This also follows easily from the definition of $\mathrm{row}, \mathrm{col}, \mathrm{diag}$. 
  \end{proof}
  

\begin{proposition}\label{prop:row_extend_d}
    Let \( p : \mathbb{Z}^{V_d} \to \mathbb{Z},  q: \mathbb{Z}^{V_{d+1}} \to \mathbb{Z}  \) be Pascal forms that can be expressed as \( p = q = \sum_{i=0}^{d}  \lambda_{i} \mathrm{row}(i) \). Let \( (p_{ij})_{(i,j) \in V_d} \) denote the coefficients of the linear form \( p: x \mapsto \sum_{(i,j) \in V_d} p_{ij}x_{ij} \), and let \( (q_{ij})_{(i,j) \in V_{d+1}} \) denote the coefficients of the linear form \( q: x \mapsto \sum_{(i,j) \in V_{d+1}} q_{ij}x_{ij} \). Then, \( q_{ij} = p_{ij} \) for all \( (i,j) \in V_d \).
    
    In other words, if the visualization of the coefficients \( (p_{ij})_{(i,j) \in V_d} \) of the linear form \( p: x \mapsto \sum_{(i,j) \in V_d} p_{ij}x_{ij} \) on the grid \( V_d \) looks like this
    \begin{align*}
      \begin{matrix}
        p_{0,d} & & & \\
        \vdots & p_{1,d-1} & & &    \\
        \vdots & \vdots & \ddots & &    \\
        \vdots & \vdots & \vdots & \ddots &    \\
        \vdots & \vdots & \vdots & \vdots & \ddots &  &\\
        \vdots & \vdots & \vdots & \vdots &  \vdots  & \ddots  \\
        \vdots & \vdots & \vdots & \vdots &  \vdots  & & \ddots\\
        p_{0,0} & p_{1,0} & \hdots & \hdots &  \hdots  & \hdots & \hdots & p_{d,0} \\
      \end{matrix}.
    \end{align*}
    Then, the visualization of the coefficients \( (q_{ij})_{(i,j) \in V_d} \) of the linear form \( q: x \mapsto \sum_{(i,j) \in V_d} q_{ij}x_{ij} \) on the grid \( V_{d+1} \) looks like this
    \begin{align*}
      \begin{matrix}
        q_{0,{d+1}} & & & \\
        p_{0,d} & q_{1, d} & & \\
        \vdots & p_{1,d-1} & q_{2, d-1} & &    \\
        \vdots & \vdots & \ddots & \ddots &    \\
        \vdots & \vdots & \vdots & \ddots & \ddots    \\
        \vdots & \vdots & \vdots & \vdots & \ddots & \ddots &\\
        \vdots & \vdots & \vdots & \vdots &  \vdots  & \ddots & \ddots  \\
        \vdots & \vdots & \vdots & \vdots &  \vdots  & & \ddots & \ddots \\
        p_{0,0} & p_{1,0} & \hdots & \hdots &  \hdots  & \hdots & \hdots & p_{d,0} & q_{d+1, 0} \\
      \end{matrix}.
    \end{align*}
  \end{proposition}
  
  \begin{proof}
    First, the statement follows immediately for \( p=q= \lambda \mathrm{row}(i) \) for all \( i = 0, \dots, d \) from the definition of \( \mathrm{row}(i) \). 
    
    Now, assume \(  p = q = \sum_{i=0}^{d}  \lambda_{i} \mathrm{row}(i)  \). Let \( (p^{(l)}_{ij})_{(i,j) \in V_d} \) and \( (q^{(l)}_{ij})_{(i,j) \in V_{d+1}} \) denote the coefficients of the linear forms \( \mathrm{row}(l) \). Since we know that \( p^{(l)}_{ij} = q^{(l)}_{ij} \) for all \( l = 0, \dots, d \) and \( (i,j) \in V_d \), we find that \( q_{ij} = q^{(l)}_{ij} = \sum p^{(l)}_{ij} = p_{ij} \) for all \( (i,j) \in V_d \).
  \end{proof}
  
  \begin{corollary}
    The same statement holds for \( p = q = \sum_{i=0}^{d}  \lambda_{i} \mathrm{col}(i) \).
  \end{corollary}
  
  \begin{proof}
    Use symmetry.
  \end{proof}
  

\begin{example}
    Consider \( p = q = \mathrm{row}(3) + \mathrm{row}(2) \) in \( \mathbb{Z}^{V_8} \) and \( \mathbb{Z}^{V_9} \), respectively. Then, \( p \) is represented by 
    \begin{verbatim}
      -28
      -14    14
       -5     9    -5
        .     5    -4     1
        2     2    -3     1     .
        2     .    -2     1     .     .
        1    -1    -1     1     .     .     .
        .    -1     .     1     .     .     .     .
        .     .     1     1     .     .     .     .     .
    \end{verbatim}
    and \( q \) is represented by
    \begin{verbatim}
      -48
      -28    20
      -14    14    -6
       -5     9    -5     1
        .     5    -4     1     . 
        2     2    -3     1     .     .
        2     .    -2     1     .     .     .
        1    -1    -1     1     .     .     .     .
        .    -1     .     1     .     .     .     .     .
        .     .     1     1     .     .     .     .     .     .
    \end{verbatim}
  \end{example}
  