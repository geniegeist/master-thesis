\chapter{Valid Outcomes of Positive Support Size \( \leq 3 \)}

We have all the tools ready to show the following three theorems. 

\begin{theorem}\label{thm:outcome-degree-support-size-232323}
    No valid integral outcomes of positive support size one exists.
\end{theorem}

\begin{theorem}\label{thm:outcome-degree-support-size-232323343}
    For valid integral outcomes \( \mathbf w \) with \( |\mathrm{supp}^+(\mathbf w)| = 2 \) we have \( \mathrm{deg}(\mathbf w) = 1 \).
\end{theorem}

\begin{theorem}
    For valid integral outcomes \( \mathbf w \) with \( |\mathrm{supp}^+(\mathbf w)| = 3 \) we have \( \mathrm{deg}(\mathbf w) \leq 3 \).
\end{theorem}

This proves our Main Theorem \ref{thm:outcome-degree-support-size} for the case of positive support size three or less, i.e. 
\begin{align*}
    \mathrm{deg}(\mathbf w) \leq 2 \cdot |\mathrm{supp}^+(\mathbf w)| - 3
\end{align*}
for all valid integral outcomes \( \mathbf w \) with \( |\mathrm{supp}^+(\mathbf w)| \leq 3 \). The proof of all the theorems were first presented in \cite{bik2022classifying}. 

We start with the proof of the first theorem.

\begin{proof}[Proof of Theorem \ref{thm:outcome-degree-support-size-232323}]
    Let \( \mathbf{w} \in \mathbb{Z}^{V_d} \) be a valid integral outcome. Since it is valid, we either have an empty negative support or a negative support that only contains \( (0,0) \). If the negative support is empty, then \( \mathbf{w} = \mathbf 0 \) by Proposition \ref{prop:outcome-zero}. Hence, we assume \( w_{0,0} < 0 \).

    Now, consider the Pascal form \( \mathrm{diag}(0) = \sum c_{i,j} x_{i,j} \). We have \( c_{0, 0} = c_{0, 1} = \dots = c_{0, d} = 1 \) and \( c_{i,j} = 0 \) for everything else. Similarly, we have for the Pascal form \( \mathrm{diag}(d) = \sum c'_{i,j} x_{i,j} \) that \( c'_{\cdot, 0} = \mathbf 1 \) and \( c'_{i,j} = 0 \) for everything else. Since outcomes are roots of Pascal forms, we have 
    \begin{align*}
        \mathrm{diag}(0)(\mathbf w) = \mathrm{diag}(d)(\mathbf w) = 0.
    \end{align*}
    Since \( w_{0,0} < 0 \) we must have \( w_{0,j} > 0 \) and \( w_{i, 0} > 0 \) for some \( i,j > 0 \). Hence, \( \mathbf{w} \) has positive support size at least two.
\end{proof}

Next, we prove the second theorem.

\begin{proof}[Proof of Theorem \ref{thm:outcome-degree-support-size-232323343}]
    Let \( \mathbf{w} \in \mathbb{Z}^{V_d} \) be an integral outcome with positive support size two and degree \( d \).
    By the previous proof, we see that 
    \begin{align*}
        \mathrm{supp}^+({\mathbf{w}}) = \left\{  (0,j), (i,0) \right\}.
    \end{align*}
    Without loss of generality, we assume \( i = d \). We want to show that \( j = d \). Consider the Pascal form \( \mathrm{row}(d) = \sum c_{i,j} x_{i,j} \), which has only nonzero coefficients \( c_{i,j} \) for \( i + j = d \).
    
    \begin{itemize}
        \item If the degree \( d \) is odd, we have \( c_{d,0} = 1 \) and \( c_{0,d} = -1 \). Since \( \mathrm{row}(d)(\mathbf{w}) = 0 \), we must have \( j = d \).

        \item If the degree \( d \) is even, we have \( c_{d,0} = c_{0,d} = 1 \). Thus, \( \mathrm{row}(d)(\mathbf w) \neq 0 \) for all \( j = 0, \dots, d \). Hence, valid outcomes with positive support size two do not exist for even degrees.
    \end{itemize}

    From now on, we assume 
    \begin{align*}
        \mathrm{supp}^+({\mathbf{w}}) = \left\{  (0,d), (d,0) \right\}.
    \end{align*}

    For sake of contradiction, let \( d \geq 2 \) (we can even assume that \( d \) is odd, but we do not need it). Then, we can divide the support 
    \begin{align*}
        \mathrm{supp}({\mathbf{w}}) = \left\{  (0,0) , (0,d), (d,0) \right\},
    \end{align*}
    via \( \lambda = (2,1,\dots,1) \) to obtain \( S_1 = \left\{ (0,0), (0,d) \right\} \), \( S_k = \emptyset \), and \( S_l = \left\{ (d,0) \right\} \). By Proposition \ref{prop:impossible-support-232423}, the pairing matrix induced by \( S_1 \) and \( E_1 = \left\{ 0,1 \right\} \) is invertible. For \( S_l \) we apply Proposition \ref{prop:impossible-support-23233243243423} and Remark \ref{rem:generality-jfknwejn} to get that the induced pairing matrix is invertible. By Corollary \ref{cor:invertibility-criterion-nooos}, the outcome \( \mathbf{w} \) is zero. This is a contradiction to the assumption that the positive support size is two. Hence, the degree \( d \) equals one.
\end{proof}