\chapter{Valid Outcomes of Positive Support
Size Five}

In this chapter, we prove that for all valid integral outcomes \( \mathbf w \) with \( |\mathrm{supp}^+(\mathbf w)| = 5 \) we have
\begin{align*}
    \mathrm{deg}(\mathbf w) \leq 7.
\end{align*}
Tools like the Invertibility Criterion, Hyperfield Criterion, and the Hexagon Criterion will be used to prove this result.

\section{Case \( d = 8, \dots, 41 \)}

First, we show similar to Proposition \ref{prop:jdngkjrenj3nw} that no outcome of degree \( d = 8, \dots, 41 \) exists with \( |\mathrm{supp}^+(\mathbf w)| = 5 \).

\begin{proposition}
    Let $A = \{ \mathrm{diag}(i) \}_{i=0}^d \cup \{ \mathrm{row}(i)\}^d_{i=0} \cup \{ \mathrm{col}(i) \}^d_{i=0}$ for some degree \( d \in \mathbb{N} \). Let \( \mathbf{x} \in H^{V_d} \) be nonzero with \( \mathrm{supp}^-(\mathbf{x}) = \left\{ (0,0) \right\} \). Then, the number of solutions \( \lvert S_5(A) \rvert \) for \( d = 8, \dots, 41 \) is depicted in Table \ref{tab:solutions324324324}.
\end{proposition}

\begin{proof}
    We compute the set of \( S_5(A) \) for \( d = 8, \dots, 40 \), and \( 41 \) using the implementation of Algorithm \ref{alg:solve} which is included in the appendix TODO.
\end{proof}

\begin{table}
    \centering
    \begin{tabular}{|c|c|}
    \hline
    degree \( d \) & number of solutions \( \lvert S_5(A) \rvert \) \\ \hline
    8  & 792  \\ \hline
    9  & 882  \\ \hline
    10 & 950  \\ \hline
    11 & 1084 \\ \hline
    12 & 1102 \\ \hline
    13 & 1212 \\ \hline
    14 & 1248 \\ \hline
    15 & 1400 \\ \hline
    16 & 1400 \\ \hline
    17 & 1530 \\ \hline
    18 & 1553 \\ \hline
    19 & 1723 \\ \hline
    20 & 1710 \\ \hline
    21 & 1856 \\ \hline
    22 & 1863 \\ \hline
    23 & 2049 \\ \hline
    24 & 2020 \\ \hline
    25 & 2182 \\ \hline
    26 & 2173 \\ \hline
    27 & 2375 \\ \hline
    28 & 2330 \\ \hline
    29 & 2508 \\ \hline
    30 & 2483 \\ \hline
    31 & 2701 \\ \hline
    32 & 2640 \\ \hline
    33 & 2834 \\ \hline
    34 & 2793 \\ \hline
    35 & 3027 \\ \hline
    36 & 2950 \\ \hline
    37 & 3160 \\ \hline
    38 & 3103 \\ \hline
    39 & 3353 \\ \hline
    40 & 3260 \\ \hline
    41 & 3486 \\ \hline
    \end{tabular}
    \caption{Number of solutions \( \lvert S_5(A) \rvert \) for degrees \( d = 8, \dots, 41 \).}
    \label{tab:solutions324324324}
\end{table}

\begin{proposition}\label{prop:uiwuwinca}
    No outcome of degree \( d = 8, \dots, 41 \) exists with \( |\mathrm{supp}^+(\mathbf w)| = 5 \).
\end{proposition}

\begin{proof}
    We use Algorithm \ref{alg:hyperfield_criterion:is_zero}. The result of this algorithm is that only the zero outcome is possible for all these cases except for the following eight:
    \begin{enumerate}
        \item \( \left\{ (4, 1), (5, 0), (1, 6), (0, 8), (3, 5) \right\} \),
        \item \( \left\{  (3, 0), (0, 4), (5, 1), (1, 7), (4, 4) \right\} \),
        \item \( \left\{  (3, 0), (0, 5), (5, 1), (1, 7), (4, 4) \right\} \),
        \item \( \left\{   (1, 1), (3, 0), (3, 4), (0, 8), (5, 3) \right\} \),
        \item \( \left\{   (3, 0), (0, 6), (5, 1), (1, 7), (4, 4) \right\} \),
        \item \( \left\{   (0, 3), (3, 0), (5, 1), (1, 7), (4, 4) \right\} \),
        \item \( \left\{  (1, 1), (4, 0), (3, 4), (0, 8), (5, 3) \right\} \), and
        \item \( \left\{  (3, 1), (5, 0), (1, 6), (0, 8), (3, 5) \right\} \).
    \end{enumerate}
    Valid configurations that have one of these eight sets as positive support are of degree eight or less. 
    
    Finally, we apply Algorithm \ref{alg:hyperfield_criterion:is_zero_general} to these eight cases with \( E \coloneqq \left\{ 3,4,5,6,7,8 \right\} \). The result is that only the zero outcome is possible. This shows the claim.
\end{proof}

\section{Case \( d \geq 42 \)}

We have proven that no valid outcome of degree \( d = 8, \dots, 41 \) exists with \( |\mathrm{supp}^+(\mathbf w)| = 5 \). Next, we show that no valid outcome of degree \( d \geq 42 \) exists using contractions.

As in Proposition \ref{prop:jasndkjsnjsnkjs}, we compute the sets 
\begin{gather*}
    \Gamma^{\mathrm{even}} \cap \left\{ \mathbf{s} \in H^{\Xi} : \lvert \mathrm{supp}^+(\mathbf{s}) \rvert \leq 5 \right\} \text{ and }
    \Gamma^{\mathrm{odd}} \cap \left\{ \mathbf{s} \in H^{\Xi} : \lvert \mathrm{supp}^+(\mathbf{s}) \rvert \leq 5 \right\} 
\end{gather*}
using Algorithm \ref{alg:hyperfield_criterion:efficient}. The Proposition \ref{prop:jasndkjsnjsnkjs} has already shown that the sets 
\begin{align*}
    \Gamma^{\mathrm{even}} \cap \left\{ \mathbf{s} \in H^{\Xi} : \lvert \mathrm{supp}^+(\mathbf{s}) \rvert \leq 4 \right\} \text{ and }
    \Gamma^{\mathrm{odd}} \cap \left\{ \mathbf{s} \in H^{\Xi} : \lvert \mathrm{supp}^+(\mathbf{s}) \rvert \leq 4 \right\} 
\end{align*}
are empty. So, we just need to check the case \( \lvert \mathrm{supp}^+(\mathbf{s}) \rvert = 5 \).



\begin{definition}
    We define 
    \begin{gather*}
        \Gamma^{\mathrm{even}}_5 \coloneqq \Gamma^{\mathrm{even}} \cap \left\{ \mathbf{s} \in H^{\Xi} : \lvert \mathrm{supp}^+(\mathbf{s}) \rvert = 5 \right\}
        \text{ and }
        \Gamma^{\mathrm{odd}}_5 \coloneqq \Gamma^{\mathrm{odd}} \cap \left\{ \mathbf{s} \in H^{\Xi} : \lvert \mathrm{supp}^+(\mathbf{s}) \rvert = 5 \right\}.
    \end{gather*}
\end{definition}

\begin{proposition}
    The following holds:
    \begin{enumerate}
        \item We have \( \lvert \Gamma^{\mathrm{even}}_5 \rvert  = 1283\).
        \item We have \( \lvert \Gamma^{\mathrm{odd}}_5 \rvert  = 1265\).
    \end{enumerate}
\end{proposition}

\begin{proof}
    See APpendix TODO. \# KEYWORD:LAOL
\end{proof}

\begin{corollary}
    Let \( d\geq 42 \). Let \( \mathbf{w} \in \mathbb{Z}^{V_d} \) be a valid outcome of degree \( d \) with \( |\mathrm{supp}^+(\mathbf w)| = 5 \). Then, \( \mathrm{contr}_d(\mathrm{sign}(\mathbf{w})) \in \Gamma^{\mathrm{even}}_5 \cup \Gamma^{\mathrm{odd}}_5 \).
\end{corollary}

\begin{proof}
    This follows from Corollary \ref{cor:validwunfwufneuiw}.
\end{proof}

Let us pick some \( \mathbf{s} \in \Gamma^{\mathrm{even}}_5 \cup \Gamma^{\mathrm{odd}}_5 \). We want to show that any configuration \( \mathbf{w} \in \mathbb{Z}^{V_d} \) that maps to \( \mathbf{s} \) under \( \mathrm{contr}_d \circ \mathrm{sign} \) must be a zero outcome which is a contradiction. So, we need to check \( 1283 + 1265 = 2548 \) cases.

\begin{remark}
    If you compare this proof with the proof of Theorem \ref{thm:main-result-32432432432nkdnjkfd}, you will see that the number of cases in \( \Gamma^{\mathrm{even}} \cup \Gamma^{\mathrm{odd}} \) to check has increased from zero to \( 2548 \).
\end{remark}

We make the following simplication to the index set \( \Xi \).

\begin{definition}
    Define the index set
    \begin{align*}
        \Xi' \coloneqq \left\{ 0,1,2,3 \right\}^2 \sqcup \left\{ 0,1,2,3 \right\}^2 \sqcup \left\{ 0,1,2,3 \right\}^2 \sqcup \left\{ 0,1,2,3 \right\} \sqcup \left\{ 0,1,2,3 \right\} \sqcup \left\{ 0,1,2,3 \right\}.
    \end{align*}
    Let \( \mathbf{s} \in H^{\Xi} \) be weakly valid. We also define \( \chi: \Xi \to \Xi' \) by setting
    \begin{align*}
        \mathbf{s} = (\mathbf{x}, \mathbf{y}, \mathbf{z}, \mathbf{b}, \mathbf{c}, \mathbf{d}, \mathbf{e}) \mapsto (\mathbf{x}, \mathbf{y}, \mathbf{z}, \mathbf{b}, \mathbf{c}, \underbracket{\mathbf{d} + \mathbf{e}}_{\coloneqq \mathbf{f}}).
    \end{align*}
    The addition \( \mathbf{f} \coloneqq \mathbf{d} + \mathbf{e} \) is done component-wise.
\end{definition}

\begin{definition}
    We define \( \mathrm{contr}_d' \coloneqq \chi \circ \mathrm{contr}_d \).
\end{definition}

\begin{figure}[H]
    \begin{align*}
        \begin{array}{cccccccccccccccccccc}
            y_{0,3} & & & & & & & & & & & & \\
            y_{0,2} & y_{1,3} & & & & & & & & & & & \\
            y_{0,1} & y_{1,2} & y_{2,3} & & & & & & & & & & \\
            y_{0,0} & y_{1,1} & y_{2,2} & y_{3,3} & & & & & & & & & \\
            c_0 & y_{1,0} & y_{2,1} & y_{3,2} & f_0 & & & & & & & & \\
            c_0 & c_1 & y_{2,0} & y_{3,1} & f_1 & f_0 & & & & & & & \\
            c_0 & c_1 & c_2 & y_{3,0} & f_2 & f_1 & f_0 & & & & & & \\
            c_0 & c_1 & c_2 & c_3 & f_3 & f_2 & f_1 & f_0 & & & & & \\
            c_0 & c_1 & c_2 & c_3 &  *  & f_3 & f_2 & f_1 & f_0 & & & & \\
            c_0 & c_1 & c_2 & c_3 &  *  & * & f_3 & f_2 & f_1 & f_0 & & & \\
            c_0 & c_1 & c_2 & c_3 &  *  & * & * & f_3 & f_2 & f_1 & f_0 & & \\
            c_0 & c_1 & c_2 & c_3 &  *  & * & * & * & f_3 & f_2 & f_1 & f_0 & \\
            c_0 & c_1 & c_2 & c_3 &  *  & * & * & * & * & f_3 & f_2 & f_1 & f_0 \\
            x_{0,3} & x_{1,3} & x_{2,3} & x_{3,3} & b_3 & b_3 & b_3 & b_3 & b_3 & b_3 & z_{0,3} & z_{1,3} & z_{2,3} & z_{3,3} \\
            x_{0,2} & x_{1,2} & x_{2,2} & x_{3,2} & b_2 & b_2 & b_2 & b_2 & b_2 & b_2 & b_2 & z_{0,2} & z_{1,2} & z_{2,2} & z_{3,2} \\
            x_{0,1} & x_{1,1} & x_{2,1} & x_{3,1} & b_1 & b_1 & b_1 & b_1 & b_1 & b_1 & b_1 & b_1 & z_{0,1} & z_{1,1} & z_{2,1} & z_{3,1} \\
            x_{0,0} & x_{1,0} & x_{2,0} & x_{3,0} & b_0 & b_0 & b_0 & b_0 & b_0 & b_0 & b_0 & b_0 & b_0 & z_{0,0} & z_{1,0} & z_{2,0} & z_{3,0}
        \end{array}
    \end{align*}  
    \caption{We visualize \( \Xi' \).}
\end{figure}

dsfdsf