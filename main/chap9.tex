\chapter{Valid Outcomes of Positive Support
Size Five}

In this chapter, we prove that for all valid integral outcomes \( \mathbf w \) with \( |\mathrm{supp}^+(\mathbf w)| = 5 \) we have
\begin{align*}
    \mathrm{deg}(\mathbf w) \leq 7.
\end{align*}
Tools like the Invertibility Criterion, Hyperfield Criterion, and the Hexagon Criterion will be used to prove this result. We follow the same proof as Bik and Marigliano in \cite{bik2022classifying}.

\section{Case \( d = 8, \dots, 41 \)}

First, we show similar to Proposition \ref{prop:jdngkjrenj3nw} that no outcome of degree \( d = 8, \dots, 41 \) exists with \( |\mathrm{supp}^+(\mathbf w)| = 5 \).

\begin{proposition}
    Let $A = \{ \mathrm{diag}(i) \}_{i=0}^d \cup \{ \mathrm{row}(i)\}^d_{i=0} \cup \{ \mathrm{col}(i) \}^d_{i=0}$ for some degree \( d \in \mathbb{N} \). Let \( \mathbf{x} \in H^{V_d} \) be nonzero with \( \mathrm{supp}^-(\mathbf{x}) = \left\{ (0,0) \right\} \). Then, the number of solutions \( \lvert S_5(A) \rvert \) for \( d = 8, \dots, 41 \) is depicted in Table \ref{tab:solutions324324324}.
\end{proposition}

\begin{proof}
    We compute the set of \( S_5(A) \) for \( d = 8, \dots, 40 \), and \( 41 \) using the implementation of Algorithm \ref{alg:solve} which is included in the appendix TODO.
\end{proof}

\begin{table}
    \centering
    \begin{tabular}{|c|c|}
    \hline
    degree \( d \) & number of solutions \( \lvert S_5(A) \rvert \) \\ \hline
    8  & 792  \\ \hline
    9  & 882  \\ \hline
    10 & 950  \\ \hline
    11 & 1084 \\ \hline
    12 & 1102 \\ \hline
    13 & 1212 \\ \hline
    14 & 1248 \\ \hline
    15 & 1400 \\ \hline
    16 & 1400 \\ \hline
    17 & 1530 \\ \hline
    18 & 1553 \\ \hline
    19 & 1723 \\ \hline
    20 & 1710 \\ \hline
    21 & 1856 \\ \hline
    22 & 1863 \\ \hline
    23 & 2049 \\ \hline
    24 & 2020 \\ \hline
    25 & 2182 \\ \hline
    26 & 2173 \\ \hline
    27 & 2375 \\ \hline
    28 & 2330 \\ \hline
    29 & 2508 \\ \hline
    30 & 2483 \\ \hline
    31 & 2701 \\ \hline
    32 & 2640 \\ \hline
    33 & 2834 \\ \hline
    34 & 2793 \\ \hline
    35 & 3027 \\ \hline
    36 & 2950 \\ \hline
    37 & 3160 \\ \hline
    38 & 3103 \\ \hline
    39 & 3353 \\ \hline
    40 & 3260 \\ \hline
    41 & 3486 \\ \hline
    \end{tabular}
    \caption{Number of solutions \( \lvert S_5(A) \rvert \) for degrees \( d = 8, \dots, 41 \).}
    \label{tab:solutions324324324}
\end{table}

\begin{proposition}\label{prop:uiwuwinca}
    No outcome of degree \( d = 8, \dots, 41 \) exists with \( |\mathrm{supp}^+(\mathbf w)| = 5 \).
\end{proposition}

\begin{proof}
    We use Algorithm \ref{alg:hyperfield_criterion:is_zero}. The result of this algorithm is that only the zero outcome is possible for all these cases except for the following eight:
    \begin{enumerate}
        \item \( \left\{ (4, 1), (5, 0), (1, 6), (0, 8), (3, 5) \right\} \),
        \item \( \left\{  (3, 0), (0, 4), (5, 1), (1, 7), (4, 4) \right\} \),
        \item \( \left\{  (3, 0), (0, 5), (5, 1), (1, 7), (4, 4) \right\} \),
        \item \( \left\{   (1, 1), (3, 0), (3, 4), (0, 8), (5, 3) \right\} \),
        \item \( \left\{   (3, 0), (0, 6), (5, 1), (1, 7), (4, 4) \right\} \),
        \item \( \left\{   (0, 3), (3, 0), (5, 1), (1, 7), (4, 4) \right\} \),
        \item \( \left\{  (1, 1), (4, 0), (3, 4), (0, 8), (5, 3) \right\} \), and
        \item \( \left\{  (3, 1), (5, 0), (1, 6), (0, 8), (3, 5) \right\} \).
    \end{enumerate}
    Valid configurations that have one of these eight sets as positive support are of degree eight or less. 
    
    Finally, we apply Algorithm \ref{alg:hyperfield_criterion:is_zero_general} to these eight cases with \( E \coloneqq \left\{ 3,4,5,6,7,8 \right\} \). The result is that only the zero outcome is possible. This shows the claim.
\end{proof}

\section{Case \( d \geq 42 \)}

We have proven that no valid outcome of degree \( d = 8, \dots, 41 \) exists with \( |\mathrm{supp}^+(\mathbf w)| = 5 \). Next, we show that no valid outcome of degree \( d \geq 42 \) exists using contractions.

As in Proposition \ref{prop:jasndkjsnjsnkjs}, we compute the sets 
\begin{gather*}
    \Gamma^{\mathrm{even}} \cap \left\{ \mathbf{s} \in H^{\Xi} : \lvert \mathrm{supp}^+(\mathbf{s}) \rvert \leq 5 \right\} \text{ and }
    \Gamma^{\mathrm{odd}} \cap \left\{ \mathbf{s} \in H^{\Xi} : \lvert \mathrm{supp}^+(\mathbf{s}) \rvert \leq 5 \right\} 
\end{gather*}
using Algorithm \ref{alg:hyperfield_criterion:efficient}. The Proposition \ref{prop:jasndkjsnjsnkjs} has already shown that the sets 
\begin{align*}
    \Gamma^{\mathrm{even}} \cap \left\{ \mathbf{s} \in H^{\Xi} : \lvert \mathrm{supp}^+(\mathbf{s}) \rvert \leq 4 \right\} \text{ and }
    \Gamma^{\mathrm{odd}} \cap \left\{ \mathbf{s} \in H^{\Xi} : \lvert \mathrm{supp}^+(\mathbf{s}) \rvert \leq 4 \right\} 
\end{align*}
are empty. So, we just need to check the case \( \lvert \mathrm{supp}^+(\mathbf{s}) \rvert = 5 \).



\begin{definition}
    We define 
    \begin{gather*}
        \Gamma^{\mathrm{even}}_5 \coloneqq \Gamma^{\mathrm{even}} \cap \left\{ \mathbf{s} \in H^{\Xi} : \lvert \mathrm{supp}^+(\mathbf{s}) \rvert = 5 \right\}
        \text{ and }
        \Gamma^{\mathrm{odd}}_5 \coloneqq \Gamma^{\mathrm{odd}} \cap \left\{ \mathbf{s} \in H^{\Xi} : \lvert \mathrm{supp}^+(\mathbf{s}) \rvert = 5 \right\}.
    \end{gather*}
\end{definition}

\begin{proposition}
    The following holds:
    \begin{enumerate}
        \item We have \( \lvert \Gamma^{\mathrm{even}}_5 \rvert  = 1283\).
        \item We have \( \lvert \Gamma^{\mathrm{odd}}_5 \rvert  = 1265\).
    \end{enumerate}
\end{proposition}

\begin{proof}
    See APpendix TODO. \# KEYWORD:LAOL
\end{proof}

\begin{corollary}
    Let \( d\geq 42 \). Let \( \mathbf{w} \in \mathbb{Z}^{V_d} \) be a valid outcome of degree \( d \) with \( |\mathrm{supp}^+(\mathbf w)| = 5 \). Then, \( \mathrm{contr}_d(\mathrm{sign}(\mathbf{w})) \in \Gamma^{\mathrm{even}}_5 \cup \Gamma^{\mathrm{odd}}_5 \).
\end{corollary}

\begin{proof}
    This follows from Corollary \ref{cor:validwunfwufneuiw}.
\end{proof}

Let us pick some \( \mathbf{s} \in \Gamma^{\mathrm{even}}_5 \cup \Gamma^{\mathrm{odd}}_5 \). We want to show that any configuration \( \mathbf{w} \in \mathbb{Z}^{V_d} \) that maps to \( \mathbf{s} \) under \( \mathrm{contr}_d \circ \mathrm{sign} \) must be a zero outcome which is a contradiction. So, we need to check \( 1283 + 1265 = 2548 \) cases.
Some of these cases occur multiple times; when we exclude these, we have \( 2318 \) cases to check:
\begin{align*}
    \lvert \Gamma^{\mathrm{even}}_5 \cup \Gamma^{\mathrm{odd}}_5  \rvert = 2318.
\end{align*}

\begin{remark}
    If you compare this proof with the proof of Theorem \ref{thm:main-result-32432432432nkdnjkfd}, you will see that the number of cases in \( \Gamma^{\mathrm{even}} \cup \Gamma^{\mathrm{odd}} \) to check has increased from zero to \( 2318 \).
\end{remark}

We make the following simplication to the index set \( \Xi \).

\begin{definition}
    Define the index set
    \begin{align*}
        \Xi' \coloneqq \left\{ 0,1,2,3 \right\}^2 \sqcup \left\{ 0,1,2,3 \right\}^2 \sqcup \left\{ 0,1,2,3 \right\}^2 \sqcup \left\{ 0,1,2,3 \right\} \sqcup \left\{ 0,1,2,3 \right\} \sqcup \left\{ 0,1,2,3 \right\}.
    \end{align*}
    Let \( \mathbf{s} \in H^{\Xi} \) be weakly valid. We also define the map
    \begin{gather*}
        \chi: H^\Xi \to H^{\Xi'},\\ \mathbf{s} = (\mathbf{x}, \mathbf{y}, \mathbf{z}, \mathbf{b}, \mathbf{c}, \mathbf{d}, \mathbf{e}) \mapsto (\mathbf{x}, \mathbf{y}, \mathbf{z}, \mathbf{b}, \mathbf{c}, \underbracket{\mathbf{d} + \mathbf{e}}_{\coloneqq \mathbf{f}}).
    \end{gather*}
    The addition \( \mathbf{f} \coloneqq \mathbf{d} + \mathbf{e} \) is done component-wise.
\end{definition}

\begin{definition}
    We define \( \mathrm{contr}_d' \coloneqq \chi \circ \mathrm{contr}_d \).
\end{definition}

\begin{figure}
    \begin{align*}
        \begin{array}{cccccccccccccccccccc}
            y_{0,3} & & & & & & & & & & & & \\
            y_{0,2} & y_{1,3} & & & & & & & & & & & \\
            y_{0,1} & y_{1,2} & y_{2,3} & & & & & & & & & & \\
            y_{0,0} & y_{1,1} & y_{2,2} & y_{3,3} & & & & & & & & & \\
            c_0 & y_{1,0} & y_{2,1} & y_{3,2} & f_0 & & & & & & & & \\
            c_0 & c_1 & y_{2,0} & y_{3,1} & f_1 & f_0 & & & & & & & \\
            c_0 & c_1 & c_2 & y_{3,0} & f_2 & f_1 & f_0 & & & & & & \\
            c_0 & c_1 & c_2 & c_3 & f_3 & f_2 & f_1 & f_0 & & & & & \\
            c_0 & c_1 & c_2 & c_3 &  *  & f_3 & f_2 & f_1 & f_0 & & & & \\
            c_0 & c_1 & c_2 & c_3 &  *  & * & f_3 & f_2 & f_1 & f_0 & & & \\
            c_0 & c_1 & c_2 & c_3 &  *  & * & * & f_3 & f_2 & f_1 & f_0 & & \\
            c_0 & c_1 & c_2 & c_3 &  *  & * & * & * & f_3 & f_2 & f_1 & f_0 & \\
            c_0 & c_1 & c_2 & c_3 &  *  & * & * & * & * & f_3 & f_2 & f_1 & f_0 \\
            x_{0,3} & x_{1,3} & x_{2,3} & x_{3,3} & b_3 & b_3 & b_3 & b_3 & b_3 & b_3 & z_{0,3} & z_{1,3} & z_{2,3} & z_{3,3} \\
            x_{0,2} & x_{1,2} & x_{2,2} & x_{3,2} & b_2 & b_2 & b_2 & b_2 & b_2 & b_2 & b_2 & z_{0,2} & z_{1,2} & z_{2,2} & z_{3,2} \\
            x_{0,1} & x_{1,1} & x_{2,1} & x_{3,1} & b_1 & b_1 & b_1 & b_1 & b_1 & b_1 & b_1 & b_1 & z_{0,1} & z_{1,1} & z_{2,1} & z_{3,1} \\
            x_{0,0} & x_{1,0} & x_{2,0} & x_{3,0} & b_0 & b_0 & b_0 & b_0 & b_0 & b_0 & b_0 & b_0 & b_0 & z_{0,0} & z_{1,0} & z_{2,0} & z_{3,0}
        \end{array}
    \end{align*}  
    \caption{Visualization of \( \Xi' \).}
\end{figure}

\begin{definition}
    Define \( \Lambda \subset H^{\Xi'} \) to be the set 
    \begin{align*}
        \Lambda  \coloneqq \chi( \Gamma^{\mathrm{even}}_5 \cup \Gamma^{\mathrm{odd}}_5 ) = \left\{ \mathbf{s}' \in H^{\Xi'} \mid \mathbf{s}' = \chi(\mathbf{s}) \text{ for some \( \mathbf{s} \in  \Gamma^{\mathrm{even}}_5 \cup \Gamma^{\mathrm{odd}}_5 \)}\right\}.
    \end{align*}
\end{definition}

A simple computation of the set \( \Lambda \) shows that it has \( 2290 \) elements.

\begin{proposition}\label{prop:ieshwu4rhui3w}
    Let \( \mathbf{s}' \in \Lambda \). Then, \( \mathbf{s}' \) has positive support of size five or four.
\end{proposition}

\begin{proof}
    Let \( \mathbf{s} = (\mathbf{x}, \mathbf{y}, \mathbf{z}, \mathbf{b}, \mathbf{c}, \mathbf{d}, \mathbf{e}) \in  \Gamma^{\mathrm{even}}_5 \cup \Gamma^{\mathrm{odd}}_5 \). Since \( \mathbf{d}, \mathbf{e} \) are non-negative, we see from
    \begin{gather*}
        \chi: H^\Xi \to H^{\Xi'}, \mathbf{s} \mapsto \mathbf{s}' = (\mathbf{x}, \mathbf{y}, \mathbf{z}, \mathbf{b}, \mathbf{c}, \mathbf{d} + \mathbf{e})
    \end{gather*}
    that \( \mathbf{f} \in \left\{ 0,1 \right\} \) holds. Hence, \( \mathbf{s}' \) has positive support of size five or four.
\end{proof}

\section{Case \( d \geq 42\) continued: \( \mathbf{s}' \in \Lambda\) with Positive Support Size Four}

\begin{corollary}
    From Proposition \ref{prop:ieshwu4rhui3w}, we see that \( \mathbf{s}' \in \Lambda \) has positive support size four if and only if \( \mathbf{s} = (\mathbf{x}, \mathbf{y}, \mathbf{z}, \mathbf{b}, \mathbf{c}, \mathbf{d}, \mathbf{e}) \) satisfies \( d_i = 1 \) and \( e_j = 1 \) for some \( i,j = 0, 1,2,3 \).
\end{corollary}

\begin{corollary}
    Let \( \mathbf{s} \in \Gamma^{\mathrm{even}}_5 \cup \Gamma^{\mathrm{odd}}_5 \). The element \( \mathbf{s} \) maps to some \( \mathbf{s}' \in \Lambda \) with positive support size four under \( \chi \) if and only if \( \mathrm{supp}(\mathbf{s}) = \left\{ x_{0,0}, x_{0,3}, x_{1,1}, x_{3,0}, d_0, e_0 \right\} \).
\end{corollary}

\begin{proof}
    This is easily verified by computer using Algorithm \ref{alg:iewjr83h8w9}. TODO TAG JAIKAJ.
\end{proof}

\begin{algorithm}
    \caption{Check Configurations for Positive Support}
    \label{alg:iewjr83h8w9}
    \begin{algorithmic}[1]
    \Ensure a subset \( X \subset \Gamma^{\mathrm{even}}_5 \cup \Gamma^{\mathrm{odd}}_5 \) consisting of configurations that map to \( \mathbf{s}' \in \Lambda \) with positive support size four under \( \chi \).
    \State $X \gets \texttt{list()}$
    \State $D \gets \{d_0, d_1, d_2, d_3\}$
    \State $E \gets \{e_0, e_1, e_2, e_3\}$
    \For{\( \mathbf{s} \in \Gamma^{\mathrm{even}}_5 \cup \Gamma^{\mathrm{odd}}_5 \)}
        \If{$\mathrm{supp}(\mathbf{s}) \cap D \neq \emptyset$ \textbf{and} $\mathrm{supp}(\mathbf{s}) \cap E \neq \emptyset$}
            \State \texttt{X.append(s)}
        \EndIf
    \EndFor
    \State \Return \( X \)
\end{algorithmic}
\end{algorithm}

We exclude this one case from the 2290 cases in \( \Lambda \) with the following proposition.

\begin{proposition}
    Let \( d \geq 42 \). Let \( \mathbf{w} \in \mathbb{Z}^{V_d} \) be a weakly valid outcome. Then, we have 
    \begin{align*}
        \mathrm{supp}^+(\mathrm{contr}_d(\mathrm{sign}(\mathbf{w}))) \neq \left\{ x_{0,3}, x_{1,1}, x_{3,0}, d_0, e_0 \right\} 
    \end{align*}
\end{proposition}

\begin{proof}
    Assume for the sake of contradiction that 
    \begin{align*}
        \mathrm{supp}^+(\mathrm{contr}_d(\mathrm{sign}(\mathbf{w}))) = \left\{ x_{0,3}, x_{1,1}, x_{3,0}, d_0, e_0 \right\}.
    \end{align*}
    Then, \( \mathbf{w} \) is an outcome with support 
    \begin{align*}
        \mathrm{supp}(\mathbf{w}) = \left\{ (0,0), (0,3), (1,1), (3,0), (i,d-i), (j, d-j) \right\}
    \end{align*}
    for some even \( i =4,6,8, \dots, d-4 \) and odd \( j = 5,7,9, \dots, d-4 \).

    Let \( \mathbf{u} \in \mathbb{Z}^{V_3} \) be the following outcome 
    \begin{verbatim}
        1
        .  .
        .  3  .
       -1  .  .  1
    \end{verbatim}
    It has support in \( \mathrm{supp}(\mathbf{u}) = \left\{ (0,0), (0,3), (1,1), (3,0) \right\}  \subset  \mathrm{supp}(\mathbf{w})\). Define the outcome \( \mathbf{v} \coloneqq \mathbf{w} +w_{0,0} \mathbf{u} \). Then, \( v_{0,0} = 0 \) and \( \mathbf{v} \neq \mathbf{0} \). However, if we apply the Invertibility Criterion with \( \lambda = \mathbf{1} \) on \( \mathbf{v} \), we see that \( \mathbf{v} \) is zero. This is a contradiction. Hence, \( \mathrm{supp}^+(\mathrm{contr}_d(\mathrm{sign}(\mathbf{w}))) \neq \left\{ x_{0,3}, x_{1,1}, x_{3,0}, d_0, e_0 \right\}  \).
\end{proof}

\section{Case \( d \geq 42\) continued: \( \mathbf{s}' \in \Lambda\) with Positive Support Size Five}

It remains to show the other 2289 cases of \( \mathbf{s}' \in \Lambda \) with positive support size five. From now on, assume that \( \mathbf{s}' \in \Lambda \) has positive support size five. We will show that all these cases are invalid. For that we introduce \emph{relative coordinates} defined below to make use of the Invertibility Criterion.

\begin{definition}
    Let \( d \geq 42 \).
    Let \( M \) be a sentinel value with no further significance. We use it to encode integers from \( 4, \dots, d-7 \). Define the map 
    \begin{align*}
        \mathrm{relcoord}: \left\{ 0, \dots, d \right\} \to \left\{ 0,1,2,3,d-6,d-5,d-4,d-3,d-2,d-1,d,M \right\}, \\
        x \mapsto \begin{cases}
            x & \text{if } x \in \left\{ 0,1,2,3, d-6,d-5,d-4,d-3,d-2,d-1,d \right\}, \\
            M& \text{if } x \in \left\{ 4, \dots, d-7 \right\}.
        \end{cases}.
    \end{align*}
    Define the map \emph{relative set} as follows:
    \begin{align*}
        \mathrm{relset}: \mathbb{Z}^{V_d} &\to 2^{\left\{ 0, \dots, 3, M, d-6, \dots, d \right\} \times \left\{ 0, \dots, 3, M, d-6, \dots, d \right\}}, \\
        \mathbf{w} &\mapsto \left\{ (\mathrm{relcoord}(i), \mathrm{relcoord}(j)) \mid (i,j) \in \mathrm{supp}(\mathbf{w}) \right\}.
    \end{align*}
\end{definition}

\begin{proposition}
    Let \( d \geq 42 \). Let \( \mathbf{w} \in \mathbb{Z}^{V_d} \) be a valid outcome with positive support size five and degree \( d \). Write \( \mathrm{contr}_d'(\mathrm{sign}(\mathbf{w})) = (\mathbf{x},\mathbf{y},\mathbf{z},\mathbf{b},\mathbf{c},\mathbf{f} ) \). Let \( i,j = 0,1,2,3 \). Then, all of the following hold:
    \begin{enumerate}
        \item \( (i,j) \in \mathrm{relset}(\mathbf{w}) \) if \( x_{i,j} \neq 0 \);
        \item \( (i, d-3+j-i) \in \mathrm{relset}(\mathbf{w}) \) if \( y_{i,j} \neq 0 \);
        \item \( (d-3+i-j,j) \in \mathrm{relset}(\mathbf{w}) \) if \( z_{i,j} \neq 0 \);
        \item \( \mathrm{relset}(\mathbf{w}) \cap \left\{ (M,i), (d-6, i), \dots, (d-4-i, i) \right\} \neq \emptyset \) if \( b_i \neq 0 \);
        \item \( \mathrm{relset}(\mathbf{w}) \cap \left\{ (i,M), (i, d-6), \dots, (i,d-4-i) \right\} \neq \emptyset \) if \( c_i \neq 0 \);
        \item \( \mathrm{relset}(\mathbf{w}) \cap \left\{ (M, d-4-i), \dots, (M, d-6), (M,M), (d-6, M), \dots, (d-4-i, M) \right\} \neq \emptyset \) if \( f_i \neq 0 \).
    \end{enumerate}
\end{proposition}

\begin{proof}
    Let \( \mathbf{w} \) be some valid outcome with \( x_{i,j} \neq 0 \). Then, \( w_{i,j} \neq 0 \) with \( i,j = 0, 1,2, 3 \). By definition of \( \mathrm{relcoord} \), \( i \mapsto i \) and \( j \mapsto j \). So \( (i,j) \in \mathrm{relset}(\mathbf{w}) \) since \( (i,j) \in \mathrm{supp}(\mathbf{w}) \).

    Assume \( y_{i,j} \neq 0 \). Then, \( w_{i, d-3+j-i} \neq 0 \). By definition of \( \mathrm{relcoord} \), \( i \mapsto i \) and \( d-3+j-i \mapsto d-3+j-i \). So \( (i,d-3+j-i) \in \mathrm{relset}(\mathbf{w}) \) since \( (i,d-3+j-i) \in \mathrm{supp}(\mathbf{w}) \). The case \( z_{i,j} \neq 0 \) is similar.

    Assume \( b_{i} \neq 0 \). Then, there must exist some nonzero \( w_{k, i} \) for some \( k = 4, \dots, d-4-i \). Clearly, \( k \) maps to some element in \( \left\{ M, d-6, \dots, d-4-i \right\} \). This shows the claim. The case for \( c_i \) is similar.

    Assume \( f_{i} \neq 0 \). Then, there must exist some nonzero \( w_{k, d-i-k} \) for some \( k = 4, \dots, d-4-i \). Clearly, \( k \) and \( d-i-k \)  map to some element in \( \left\{ M, d-6, \dots, d-4-i \right\} \). This shows the claim.
\end{proof}

Relative coordinates help us to apply the Invertibility Criterion. 

\begin{example}\label{ex:siuh438h89}
    Let \( d \geq 42 \). Let \( \mathbf{w} \in \mathbb{Z}^{V_d} \) be some valid configuration with support size six. Now, assume that the relative support set of \( \mathbf{w} \) is 
    \begin{align*}
        \mathrm{relset}(\mathbf{w}) = \left\{ (0,0), (0,d), (1,3), (M,2), (M, d-6), (d-5, M) \right\}.
    \end{align*}
    Let us visualize the configuration \( \mathbf{w} \). Can such a configuration exist?
    \begin{verbatim}
d     X
d-1   .   .
d-2   .   .   .
d-3   .   .   .   .
d-4   .   .   .   .   .
d-5   .   .   .   .   .   .
d-6   .   .   .   .   .   X   X
M     .   .   .   .   .   .   .   .
M     ........................................
M     ..........................................
M     ............................................
M     .   .   .   .   .   .   .   .   .   .   .   .   
M     .   .   .   .   .   .   .   .   .   .   .   .   .   X
M     .   .   .   .   .   .   .   .   .   .   .   .   .   X   .   
3     .   X   .   .   .   .   .   .   .   .   .   .   .   .   .   .
2     .   .   .   .   X   X   X   X   X   X   X   X   .   .   .   .   .     
1     .   .   .   .   .   .   .   .   .   .   .   .   .   .   .   .   .   .
0     X   .   .   .   .   .   .   .   .   .   .   .   .   .   .   .   .   .   .

      0   1   2   3   M   M   M   M   M   M   M   M  d-6 d-5 d-4 d-3 d-2 d-1  d
    \end{verbatim}
    We see that 
    \begin{align*}
        \mathrm{supp}(\mathbf{w}) = \left\{ (0,0), (0,d), (1,3) \right\} \cup \left\{ (i,2), (j,d-6 ) \right\} \cup \left\{ (d-5,k ) \right\}.
    \end{align*}
    for \( i,j,k  = 4, \dots, d-7 \). When \( i = j \), we can apply the Invertibility Criterion (Proposition \ref{prop:impossible-support-2324223423123123} and Proposition \ref{prop:impossible-support-232423}) with \( \lambda = (3,1, \dots,1, 2, 1, \dots, 1) \) to see that \( \mathbf{w} = \mathbf{0} \), which is a contradiction. So assume \( i \neq j \). Then, we use \( \lambda = (3, 1, \dots, 1) \) to see that \( \mathbf{w} = \mathbf{0} \), which is a contradiction. Hence, \( \mathbf{w} \) cannot be an outcome.
\end{example}

Let us generalize this example to elements \(  \mathbf{s}' \in \Lambda \).

\begin{proposition}
    Let \( \mathbf{s}' \in \Lambda \) with positive support size five. Then, all of the following hold:
    \begin{enumerate}
        \item \( \lvert \mathrm{supp}^+(\mathbf{s}') \cap \left\{ b_0,b_1,b_2,b_3 \right\} \rvert \leq 1 \);
        \item \( \lvert \mathrm{supp}^+(\mathbf{s}') \cap \left\{ c_0,c_1,c_2,c_3 \right\} \rvert \leq 1 \);
        \item \( \lvert \mathrm{supp}^+(\mathbf{s}') \cap \left\{ f_0,f_1,f_2,f_3 \right\} \rvert \leq 1 \).
    \end{enumerate}
\end{proposition}

\begin{proof}
    This is verified by computer. Keyword TAG MUSIAS.
\end{proof}

\begin{corollary}
    Let \( \mathbf{s}' \in \Lambda \) with positive support size five. Then, we have
    \begin{align*}
        \lvert \left\{ (x,y) \in \mathrm{relset}(\mathbf{s}') : x = M \right\} \rvert \leq 2 \quad \text{and} \quad         \lvert \left\{ (x,y) \in \mathrm{relset}(\mathbf{s}') : y = M \right\} \rvert \leq 2.
    \end{align*}
\end{corollary}

\begin{proof}
    Follows immediately from the previous proposition.
\end{proof}

\begin{proposition}
    Let \( d\geq 42 \).
    Let \( \mathbf{w} \in \mathbb{Z}^{V_d} \) be a valid configuration with positive support size five and 
    \begin{align*}
        \lvert \left\{ (x,y) \in \mathrm{relset}(\mathbf{w}) : x = M \right\} \rvert = 2.
    \end{align*}
    Denote these elements by \( (M, x) \) and \( (M, y) \) for \( x \neq y \). Write \( (i,x), (i',y) \in \mathrm{supp}(\mathbf{w}) \) with \( i,i' = 4, \dots, d-7 \) for the elements that map to \( (M, x) \) or \( (M, y) \) under \( \mathrm{relcoord} \), respectively. If we successfully apply the Invertibility Criterion for the case \( i = i' \) to show the contradiction \( \mathbf{w} = \mathbf 0 \) with 
    \begin{align*}
        \lambda = (\mathbf{a},1,\dots,1, 2, 1, \dots, 1, \mathbf{b})
    \end{align*}
    for some \(\mathbf{a} \in \mathbb{Z}^{k}_{\geq 1}, \mathbf{b} \in \mathbb{Z}^{h}_{\geq 1} \), and \( 1 \leq k,h \leq 4 \), then we can also apply the Invertibility Criterion for the case \( i \neq i' \) with
    \begin{align*}
        \lambda' = (\mathbf{a},1, \dots, 1, \mathbf{b})
    \end{align*}
    to show the same contradiction.
\end{proposition}

\begin{proof}
    Assume \( i \neq i' \). Then, we can just apply Proposition \ref{prop:impossible-support-23233243243423} as long as \( S'_{l'} \in \left\{ 0, \lambda'_l \right\} \) is satisfied for all \( l' \), see Section \ref{subsec:divide} on the divide and conquer approach of the Invertibility Criterion. Theses conditions are satisfied because the sets \( S_l \) induced by \( \lambda \) satisfy \( S_{l} \in \left\{ 0, \lambda_l \right\} \) by assumption.
\end{proof}

\begin{corollary}
    The previous propostion allows us to specialize to the case \( i = i' \) when \( (i,x), (i',y) \in \mathrm{supp}(\mathbf{w}) \) for \( i,i' = 4, \dots, d-7 \) and \( x \neq y \) occurs. A similar statement holds for the case \( \lvert \left\{ (x,y) \in \mathrm{relset}(\mathbf{w}) : y = M \right\} \rvert = 2 \).
\end{corollary}

\begin{example}
    Returning to Example \ref{ex:siuh438h89}, we see that it suffices to consider the case \( i = j \). The case \( i \neq j \) then follows from the previous corollary.
\end{example}