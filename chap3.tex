\chapter{Coherence and the Reconstruction Theorem}\label{chapter:reconstruction}

The Reconstruction Theorem was originally stated in the context of regularity structures by Hairer~\cite{hairer2014theory}. Later, Caravenna and Zambotti~\cite{caravenna2021hairer} revisited the Reconstruction Theorem and embedded it into the theory of distributions. In this chapter, we closely follow their approach with the advantage being that it allows for an easily accessible and self-contained treatment of the Reconstruction Theorem.

\section{A Peek at the Reconstruction Theorem}\label{chapter:first-peek-at-reconstruction}

We confront ourselves with the following problem.

\textbf{Problem:} Given a family of distributions \({(F_x)}_{x \in \mathbb{R}^d}\), find a distribution \(f \in \mathcal{D'}\) that is locally well approximated by \(F_x\) around any point \(x \in \mathbb{R}^d\).

\vspace{0.2cm}

Think of \({ F= (F_x)}_{x \in \mathbb{R}^d}\) as a family of local approximations of an unknown distribution \(f\). Our goal in this chapter is to find an assumption under which finding \(f\) is possible. Sometimes, \( f \) is denoted by \( \mathcal{R}F \), and we say \( \mathcal{R}F \) is the reconstruction of the germ \( F \). The symbol \( \mathcal{R} \) is to be understood as a reconstruction operator, mapping germs to their reconstruction.

Let's start with the first novel definition by Caravenna and Zambotti~\cite{caravenna2021hairer}.
\begin{definition}[Germ]
    A family of distributions \({(F_x)}_{x \in \mathbb{R}^d}\) is called a \emph{germ} if for all test functions \(\psi \in \mathcal{D}\) the map \(x \mapsto F_x(\psi)\) is measurable.
\end{definition}

For our purpose, a germ \({F = (F_x)}_{x \in \mathbb{R}^d}\) is \emph{locally well approximated} by a distribution \( f = \mathcal{R}F \) if for any test function \(\psi \in \mathcal{D}, \int \psi(x)\, \mathrm{d}x \neq 0\) and compact set \(K \subset \mathbb{R}^d\) we have 
\begin{align}\label{peek:well-approximated}
    \lim_{\epsilon \to 0} |(f - F_x)(\psi^\epsilon_x)| = 0 \quad \text{uniformly for \(x \in K\) }.
\end{align} 

The Reconstruction Theorem states that \emph{under some assumption} we find a reconstruction \(  f = \mathcal{R}F \) of some germ \( F = (F_x)_{x \in \mathbb{R}^d} \) such that~\eqref{peek:well-approximated} holds. This is our conjecture.

\begin{conjecture}\label{peek:conjecture}
    Let \({F = (F_x)}_{x \in \mathbb{R}^d}\) be a germ that satisfies an assumption \emph{\texttt{???}} that we do not know yet. Let \( \gamma > 0 \). Then, there exists a reconstruction \(f \in \mathcal{D}'\) such that for every test function \(\psi \in \mathcal{D}\) there exists \(C < {\infty}\) with
    \begin{gather}\label{peek:eq:conjecture}
        |(f-F_x)(\psi^\epsilon_x)| \leq C \epsilon^{\gamma} \\
        \text{uniformly for \(x\) in compact sets and \( \epsilon \in (0,1] \)} \nonumber. % chktex 9
    \end{gather}
    
\end{conjecture}

Any distribution \(f\) satisfying the above inequality approximates the germ \( F = (F_x)_{x \in \mathbb{R}^d} \) well in the sense of~\eqref{peek:well-approximated} because \(|(f-F_x)(\psi^\epsilon_x)| \leq C  \epsilon^{\gamma} \to 0\) as \(\epsilon \to 0\).

Our task is to find a suitable assumption for the conjecture. Let \({(F_x)}_{x \in \mathbb{R}^d}\) be a germ. Recall~\eqref{eq:starting-point} and set \( F = F_x \), then \( F_x(\psi * \rho^\epsilon) \to F_x(\psi) \) as \( \epsilon \to 0 \) because
\begin{align*}
    F_x(\psi * \rho^\epsilon) \overset{\eqref{lemma:mollified-distribution}}{=} \int F_x(\rho_y^\epsilon) \psi(y)\, \mathrm{d}y \to F_x(\psi) \quad \text{ as } \epsilon \to 0.
\end{align*}
This observation inspires us to replace \(F_x\) under the integral by \(F_y\) so that we obtain the map \(f_\epsilon: \psi \mapsto \int F_y(\rho_y^\epsilon) \psi(y)\, \mathrm{d}y\). The motivation for the newly constructed map \(f_{\epsilon}\) is that we hope that \( f_\epsilon \) converges to \( \mathcal{R}F \) as \( \epsilon \to 0 \).

\begin{definition}[Approximating Distribution]\label{def:approximating-distributions}
        Let \({(F_x)}_{x \in \mathbb{R}^d}\) be a germ and \( \rho \) a mollifier. Set \(\epsilon_n = 2^{-n}\) for \(n \in \mathbb{N}\). The \emph{approximating distribution} \(f_n \in \mathcal{D}'\) is defined as 
        \begin{align*}
                f_n: \psi \mapsto \int_{\mathbb{R}^d} F_y(\rho_y^{\epsilon_n}) \psi(y)\, \mathrm{d}y.
        \end{align*}
\end{definition}
Now, the assumption \(\texttt{???}\) should ensure that \(f_n\) converges. The limit \( \lim_{n \to \infty} f_n \) is then called the reconstruction of the germ \( (F_x)_{x \in \mathbb{R}^d} \); of course we must justify that \( \lim f_n \) is indeed a reconstruction. Ideally, this follows from \texttt{???}. Moving forward, we write 
\begin{align*}
    f_n = f_1 + \sum^{n-1}_{k=1}g_k, \quad \text{where } g_k \coloneqq f_{k+1} - f_k.
\end{align*}
The limit \(\lim_{n \to \infty} f_n\) exists if and only if \(\sum g_k\) converges. By definition, we have
\begin{align*}
        g_k(\psi) = \int_{\mathbb{R}^d} F_y(\rho_y^{\epsilon_{k+1}} - \rho_y^{\epsilon_k}) \psi(y)\, \mathrm{d}y.
\end{align*}
A smart choice of our mollifier \( \rho \) lets us write \( \rho^{\epsilon_{k+1}}_y - \rho^{\epsilon_k}_y = {(\hat \varphi^{\epsilon_k} * \check \varphi^{\epsilon_k})}_y\) for two nice test functions \( \hat \varphi \) and \( \check \varphi \). Finding such a mollifier with test functions \( \hat \varphi \) and \( \check \varphi \) later fills an entire chapter called \emph{Tweaking}. This is one of the most important steps in proving the Reconstruction Theorem. For now, we take the existence of \( \rho \) for granted.
By Corollary~\ref{cor:minosokoad} we write
\begin{align*}
    g_k(\psi) &= \int_{\mathbb{R}^d} F_z({(\hat \varphi^{\epsilon_k} * \check \varphi^{\epsilon_k})}_z) \psi(z)\, \mathrm{d}z \\
    &= \iint_{\mathbb{R}^{d \times d}} F_z(\hat \varphi^{\epsilon_k}_y) \check \varphi^{\epsilon_k}(y-z) \psi(z) \, \mathrm{d}y\, \mathrm{d}z \\
    &= \iint F_y(\hat \varphi^{\epsilon_k}_y) \check \varphi^{\epsilon_k}(y-z) \psi(z) \, \mathrm{d}y\, \mathrm{d}z 
    + \iint (F_z - F_y)(\hat \varphi^{\epsilon_k}_y) \check \varphi^{\epsilon_k}(y-z) \psi(z) \, \mathrm{d}y\, \mathrm{d}z.
\end{align*}
Remember that we want \(\sum g_k\) to converge; the crucial piece is \( F_z - F_y \). To find \texttt{???}, we let us guide by a closely related problem in another branch of mathematics: stochastic analysis. 

In stochastic analysis, one would like to make sense of an integral \(I_t\)  of the form \(I_t = \int^t_0 X_s \, \mathrm{d}Y_s\) where \(X_s\) and \(Y_s\) are paths of low regularity. Consider the following example: let \(G \in \mathcal{V}^p\) and \(F \in \mathcal{V}^q\) with \(\frac{1}{p} + \frac{1}{q} > 1\), where \(\mathcal{V}^j\) is the space of all functions with finite \(j\)-variation for \( j \in \left \{ p,q \right \} \). Then, there exists a canonical integration theory for this setting (the so-called \emph{Young regime}~\cite{Young1936AnIO}) such that \(I_t = \int^t_0 G \, \mathrm{d}F\) is defined. It is based on the approximation idea
\begin{align*}
    \int^t_s G \, \mathrm{d}F \approx G(s)(F(t) - F(s)) \eqqcolon A_{s,t} \quad \text{for very small \(|t-s|\)}.
\end{align*}
The \emph{Sewing Lemma}~\cite{GUBINELLI200486}, an analytical tool, which lets integrals of low regularity to be defined in a meaningful sense, allows us to sew the local approximations \(A_{s,t}\) together to obtain an integral as a Riemann-type sum
\begin{align}\label{sewing-lemma-integral}
    I_t = \int^t_0 G \, \mathrm{d}F \coloneqq \lim\limits_{|\pi| \to 0} \sum\limits_{i=0}^{\# \pi - 1} A_{t_i,t_{i+1}}
\end{align}
for arbitrary partitions\footnote{Here, a partition of \([0,t]\) is an ordered set \(\pi = \left \{ 0 = t_0 < t_1 < \cdots < t_k = t  \right \} \), \( \# \pi = k \) and \(|\pi| = \max\limits_{i=0, \ldots ,\# \pi - 1} |t_{i+1} - t_{i}|\).} \( \pi \) of \([0,t]\) with \(|\pi| \to 0\) as \(n \to 0\).
\begin{lemma}[Sewing Lemma~\cite{broux2021sewing}]\label{first-sewing-lemma}
    Let \(\gamma > 1\) and \( \Delta = \left \{ (s,t) :  0 \leq s \leq t \leq T\right \} \) for some fixed \(T > 0\). Let \(A: \Delta \to \mathbb{R}\) be a continuous function such that there exists \( C <\infty \) with
    \begin{gather}
        \delta A_{s,u,t} \coloneqq |A_{s,t} - A_{s,u} - A_{u,t}| \leq C {(\max \{|u-s|,|t-u|\})}^\gamma \label{sewing-lemma-condition}\\
        \text{uniformly for \(0 \leq s \leq u \leq t \leq T\)}. \nonumber
    \end{gather} 
    Then, there exists a unique function \(I: [0,T] \to \mathbb{R}\) and \(\tilde C < {\infty}\)  such that \(I_0 = 0\) and 
    \begin{gather*}
        |I_t - I_s - A_{s,t}| \leq \tilde C|t-s|^\gamma \\
        \text{uniformly over \(0 \leq s \leq t \leq T\).}
    \end{gather*}  
    Furthermore, \(I\) is the limit of Riemann-type sums as in~\eqref{sewing-lemma-integral}.  
\end{lemma}
The connection to the Reconstruction Theorem is established in the following way: From a distributional viewpoint, we want to give the integral \(I(\psi) = \int G \psi \, \mathrm{d}F\) a meaning for all test functions \( \psi \in \mathcal{D} \). We use the following approximation
\begin{align*}
    F_x(\psi)  \leadsto I(\psi), \quad \text{where } F_x(\psi) \coloneqq G(x)\int \psi \, \mathrm{d}F 
\end{align*}
If we allowed indicator functions as test functions \(\psi = 1_{[s,t]}\), we would get
\begin{align*}
    F_s(1_{[s,t]}) = G(s) \int^t_s \mathrm{d}F = G(s)(F(t) - F(s)) = A_{s,t}.
\end{align*}
If we further assumed that \(A_{s,t} = F_s(1_{[s,t]})\) satisfies~\eqref{sewing-lemma-condition}, we would have 
\begin{align*}
    (F_x - F_u)({(1_{[0,1]})}^{y-u}_u) = \frac{(F_x - F_u)(1_{[u,y]})}{y-u} &= \frac{(G(x) - G(u))(F(y) - F(u))}{y-u} \\
    &= \frac{\delta A_{x,u,y}}{y-u}\\
    &\leq C \frac{ {(|u-x| + |y-u|)}^\gamma}{y-u}
\end{align*} 
by the Sewing Lemma. Hence, the germ \({(F_x)}_{x \in \mathbb{R}^d}\) would satisfy
\begin{align*}
    (F_x - F_u)({(1_{[0,1]})}^{\epsilon}_u) \leq C \epsilon^{-1}{(|u-x| + \epsilon)}^\gamma
\end{align*}
for \(\epsilon = y-u\). 

Our reasoning premised around \( 1_{[s,t]} \) being a test function; clearly it is not smooth. So the argumentation fails. Nevertheless, it inspires us to define a property coined \emph{coherence}. A germ is called \emph{coherent} if \( (F_x)_{x \in \mathbb{R}^d} \) satisfies
\begin{gather}\label{pre-condition-coherence}
    |(F_z - F_y)(\varphi^\epsilon_y)| \leq C\epsilon^{a}{(|z-y| + \epsilon)}^{c - a}  \\ \text{uniformly for \(z,y\) in compact sets and \(\epsilon \in (0,1]\)} \nonumber. % chktex 9
\end{gather}
for some test function \(\varphi\), constants  \(c\) and \(a\).
The precise definition will occur in Chapter~\ref{chapter:coherence}. In our previous example \(A_{s,t} = F_s(1_{[s,t]})\), we have \(a = -1\) and \(c = \gamma - 1\). Note that our naive (and wrong) argumentation is not completely preposterous because in Chapter~\ref{section:sewing-lemma} we fix it by considering a dyadic approximation of the indicator function \( 1_{[s,t]} \).

Returning to our problem of finding an upper bound for \(g_k\)
\begin{align*}
    g_k(\psi) = \iint F_y(\hat \varphi^{\epsilon_k}_y) \check \varphi^{\epsilon_k}(y-z) \psi(z) \, \mathrm{d}y\, \mathrm{d}z 
    + \iint (F_z - F_y)(\hat \varphi^{\epsilon_k}_y) \check \varphi^{\epsilon_k}(y-z) \psi(z) \, \mathrm{d}y\, \mathrm{d}z,
\end{align*}
we now have \emph{coherence}~\eqref{pre-condition-coherence} to control the second summand of \(g_k\). By coherence 
\begin{align*}
    |(F_z - F_y)(\hat \varphi^{\epsilon_k}_y)| \leq C \epsilon_k^{a} {(|z-y| + \epsilon_k)}^{c-a}.
\end{align*}
If \(|z-y| < \epsilon_k\), then \(|(F_z - F_y)(\hat \varphi^{\epsilon_k}_y)| \leq C2^{c-y} \epsilon_k^c \to 0\) as \(k \to \infty\).

Regarding the first summand, we want \(F_x(\hat \varphi^{\epsilon}_x) \to 0\) as \(\epsilon \to 0\). One way to achieve this is by imposing a condition which we will call \emph{homogeneity}: if \(F_x(\hat \varphi^{\epsilon}_x) \leq B \epsilon^{\beta}\) for some constant \(B < \infty\), we will say that the germ \({(F_x)}_{x \in \mathbb{R}^d}\) has homogeneity bound \(\beta\). If a germ has homogeneity bound \(\beta > 0\), then   \(F_y(\hat \varphi^{\epsilon}_y) \leq B \epsilon^\beta \to 0\) as \(\epsilon \to 0\).  

It seems that we want a germ to satisfy the coherence and homogeneity condition. Fortunately, we get homogeneity for free if a germ is coherent. So, requiring a germ to be coherent is all we need to get the Reconstruction Theorem going.

\section{Coherence and Homogeneity}\label{chapter:coherence}

In this section we rigorously define \emph{coherence} and \emph{homogeneity}. We will see that coherence is an optimal condition for the Reconstruction Theorem. Moreover, homogeneity follows from coherence.

In Chapter~\ref{chapter:first-peek-at-reconstruction} we gave a naive motivation for coherence. Here comes the rigorous definition.

\begin{definition}[\(\gamma\)-coherent germs]\label{definition:coherence}
   Let \(\gamma \in \mathbb{R}\). A germ \({(F_x)}_{x \in \mathbb{R}^d}\) is called \emph{\(\gamma\)-coherent} if there exists a test function \(\varphi \in \mathcal{D}\) with \(\int \varphi(x) \, \mathrm{d}x \neq 0\) such that for every compact set \(K \subset \mathbb{R}^d\) there exists a non-positive real number \(\alpha_K \leq \min\left\{ 0, \gamma \right\}\) and a constant \(C < \infty\) with
   \begin{gather}\label{coherence}
        |(F_z - F_y)(\varphi^\lambda_y)| \leq C\lambda^{\alpha_K}{(|z-y| + \lambda)}^{\gamma - \alpha_K}  \\ \text{uniformly for \(z,y \in K\), \(|y-z| \leq 2\)  and \(\lambda \in (0,1]\)} \nonumber. % chktex 9
   \end{gather}
   We say that  \({(F_x)}_{x \in \mathbb{R}^d}\) is \((\bm{\alpha}, \gamma)\)-coherent if \(\bm \alpha = (\alpha_K)\). If \( \alpha_k \) is constant, we say \({(F_x)}_{x \in \mathbb{R}^d}\) is \(({\alpha}, \gamma)\)-coherent.
\end{definition}
\begin{remark}\label{remark:cutoff}
    Note how we require \(|y-z| \leq R\) for \( R = 2 \), which appears rather arbitrary. It turns out to be indifferent what value of \( R \) we plug in. The constraint \(|y-z| \leq R\) can even be entirely dropped, see Proposition~\ref{proposition:cutoff}. In the end, we choose \(R = 2\) because it is convenient for our goals.
\end{remark}

Fix \(K, \varphi, \alpha, \gamma\). The \emph{semi-norm \(\vertiii{\cdot}^{\mathrm{coh}}_{K,\varphi,\alpha,\gamma}\)} is the smallest constant \(C \in \mathbb{R} \cup \left\{ \infty \right\}\) such that the coherence condition~\eqref{coherence} holds for \(K, \varphi, \alpha, \gamma\). Concretely, we define
\begin{align*}
    \vertiii{F}^{\mathrm{coh}}_{K,\varphi,\alpha,\gamma} = \sup \left\{ \frac{(F_z - F_y)(\varphi^\lambda_y)}{\lambda^\alpha{(|z-y| + \lambda)}^{\gamma - \alpha}} : y,z \in K, |z-y| \leq 2, \lambda \in (0,1] \right\}. % chktex 9
\end{align*}

We briefly discuss the intuition behind coherence. For some constant \( C' < \infty \) we rewrite inequality~\eqref{coherence} to
\begin{align}\label{EspressoHouse}
    |(F_z - F_y)(\varphi^\lambda_y)| \leq C' \begin{cases}
        \lambda^\gamma \quad & \text{if \(|z-y| \leq \lambda\)} \\
        \lambda^{\alpha} |z-y|^{\gamma - \alpha} \quad & \text{otherwise}
    \end{cases}.
\end{align}
Observe that \(\lambda^{\gamma} \leq \lambda^{\alpha}\) because \(\lambda \in (0,1] \) and \(\gamma \geq \alpha\). As \(|z-y|\) decreases to \(\lambda\), the magnitude of \(|(F_z - F_y)(\varphi^\lambda_y)|\) shifts from \(\lambda^\alpha\) to \(\lambda^{\gamma}\). This change becomes very dramatic when \(\gamma > 0\) and \(\alpha < 0\) because \( \lambda^{\alpha} \) diverges whereas \(\lambda^{\gamma}\) vanishes.

We make the following important observation.
\begin{remark}[Monotonicity]
    The right-hand side of~\eqref{EspressoHouse} shrinks as \(\alpha \nearrow 0\) for fixed \(\gamma, y\) and \(z\). In other words, the larger \(\alpha\) (remember that \(\alpha < 0\)), the better the estimate becomes. Hence, without loss of generality we assume that the map \(K \mapsto \alpha_K\) is \emph{monotone}; that is for all compact sets \( K, K' \subset \mathbb{R}^d \) we have
\begin{align}\label{alpha-monotone}
    K \subset K' \implies \alpha_K \geq \alpha_{K'}.
\end{align}
This is achieved by choosing the exponents \(\alpha_K\) in the following way: for balls \(K = B(0,n)\) of radius \(n \in \mathbb{N}\) choose \(\alpha_K = \min\left\{ \alpha_{B(0,i)}  : 1\leq i \leq n\right\}\); otherwise for general compact sets \(K\) choose \(\alpha_K = \min\left\{ \alpha_{B(0,i)}  : 1\leq i \leq n\right\}\) with \(n \in \mathbb{N}\) such that  \(B(0,n) \supset K\). This ensures that the family of exponents \((\alpha_K)\) is monotone. It will play an important role in the proof of the Reconstruction Theorem in case \(\gamma < 0\).
\end{remark}

For the upcoming proofs, we need an assumption that controls the regularity of the reconstruction \( \mathcal{R}F \) for some germ \( F = (F_x)_{x \in \mathbb{R}^d} \). This assumption is called \emph{homogeneity} and directly follows from coherence.

\begin{lemma}[Homogeneity Bound]\label{lemma:homogeneity-bound}
   Let \({(F_x)}_{x \in \mathbb{R}^d}\) be a \(\gamma\)-coherent germ. Then, for every compact set \(K \subset \mathbb{R}^d\) there exists a real number \(\beta_K < \gamma\) and a constant \(B < \infty\) such that
   \begin{gather*}\label{homogeneity}
                |F_y(\varphi^\lambda_y)| \leq B\lambda^{\beta_K} \quad
                \text{uniformly for \(y \in K\) and \(\lambda \in (0,1]\)}. % chktex 9
   \end{gather*}
   We say the germ \({(F_x)}_{x \in \mathbb{R}^d}\) has \emph{local homogeneity bound} \(\bm \beta = (\beta_K)\). We say the germ \({(F_x)}_{x \in \mathbb{R}^d}\) has \emph{global homogeneity bound} \(\beta \in \mathbb{R}\) if \(\beta_K = \beta\) for all compact sets \(K \subset \mathbb{R}^d\).
\end{lemma}

\begin{proof}
    Writing
    \begin{align*}
        |F_y(\varphi^\lambda_y)| \leq |(F_y - F_z)(\varphi^\lambda_y)| + |F_z(\varphi^\lambda_y)|
    \end{align*}
    we use coherence to bound the first summand. Precisely, fix any compact set \(K \subset \mathbb{R}^d\) and \(z \in K\). By coherence, we have \(|(F_y - F_z)(\varphi^\lambda_y)| \leq  C \lambda^{\alpha}(|z-y|+\lambda)^{\gamma - \alpha} \leq \{ C (\mathrm{diam}(K) + 1)^{\gamma - \alpha} \} \cdot  \lambda^{\alpha}\) uniformly for \(y \in K\) and \(\lambda \in (0,1]\) (where \(\mathrm{diam}(K) \coloneqq \sup_{y,z \in K}|y-z| \)). % chktex 9

    To estimate \(|F_z(\varphi^\lambda_y)|\), we know there exist \(\tilde C < \infty\) and \(r \in \mathbb{N}_0\) such that \(|F_z(\varphi^\lambda_y)| \leq \tilde C \lVert \varphi^\lambda_y \rVert_{C^r}\) for all \(y \in K\) and \(\lambda \in (0,1]\) because \(F_z\) is a distribution. Also, we have \(\lVert \partial^k\varphi^\lambda_y \rVert_\infty \leq \lambda^{-|k|- d} \lVert \partial^k\varphi \rVert_\infty \leq \lambda^{-r - d} \lVert \varphi \rVert_{C^r}\). Thus, \(\lVert \varphi^\lambda \rVert_{C^r} \leq \lambda^{-r-d}\lVert \varphi \rVert_{C^r}\) follows. In the end, we obtain 
    \(|F_z(\varphi^\lambda_y)| \leq \{ \tilde C  \lVert \varphi \rVert_{C^r}  \} \cdot \lambda^{-r-d}\).

    We choose \(B = C {(\mathrm{diam}(K) + 1)}^{\gamma - \alpha} +  \tilde C  \lVert \varphi \rVert_{C^r}  \) and \(\beta \leq \min \left\{ \alpha, -r-d, \gamma \right\}\). We can further decrease \( \beta \) to ensure \( \beta < \gamma \).
\end{proof}

Similar to \((\alpha_K)\), the family \((\beta_K)\) is \emph{monotone} in the sense that 
\begin{align}\label{beta-monotone}
    K \subset K' \implies \beta_K \geq \beta_{K'}.
\end{align} 
We also introduce a semi-norm that quantifies the homogeneity of a coherent germ \( F  \)
\begin{align}\label{definition:semi-norm-homogeneity}
    \vertiii{F}^{\mathrm{\hom}}_{K, \varphi, \beta} = \sup_{\substack{x \in K \\ \lambda \in (0,1]}} \frac{|F_x(\varphi^\lambda_x)|}{\lambda^\beta} \qquad \text{compact set \( K \subset \mathbb{R}^d \)}.
\end{align}

\section{The Reconstruction Theorem in Detail}

We are ready to state the Reconstruction Theorem.

\begin{theorem}[Reconstruction Theorem~\cite{caravenna2021hairer}]\label{theorem:reconstruction-theorem}
   Let \(\gamma \in \mathbb{R}\) and \({(F_x)}_{x \in \mathbb{R}^d}\) be an \((\bm{\alpha}, \gamma)\)-coherent germ with local homogeneity bounds \(\bm \beta\). Then, there exists a distribution \(f \in \mathcal{D}'(\mathbb{R}^d)\) such that for every compact set \(K \subset \mathbb{R}^d\) and all integers \(r> \max \left\{ -\alpha_{\bar K_2}, -\beta_{\bar K_2} \right\}\), \( \alpha \coloneqq \alpha_{\enlarg{K}{2}} \) we have
   \begin{gather}\label{reconstruction-theorem}
        |(f-F_x)(\psi^\lambda_x)| \leq \{\mathrm{constant} \} \cdot \vertiii{F}^{\mathrm{coh}}_{\bar K_2, \varphi, \alpha, \gamma} \cdot \begin{cases}
            \lambda^\gamma \quad & \text{if \(\gamma \neq 0\)}\\
            1 + |\log \lambda| \quad & \text{if \(\gamma = 0\)}
        \end{cases}
        \\ \text{uniformly for \(\psi \in \mathcal{B}_r\), \(x \in K\), \(\lambda \in (0,1]\).} \nonumber
   \end{gather}
   The multiplicative constant may only depend on \( \alpha, \gamma, r, d \) and \( \varphi \). It is explicitly computed in~\eqref{constant:rec-gamma-bigger-zero},~\eqref{holy-molly} and~\eqref{holy-mollyl} for the cases \( \gamma > 0 \), \( \gamma < 0 \) and \( \gamma = 0 \) respectively.

   If \( \gamma > 0 \), the reconstruction \( f = \mathcal{R}F \) is unique. Moreover, the map \( F \mapsto \mathcal{R}F \) is linear.

   If \( \gamma \leq 0 \) is no longer unique. However, for any \( \alpha \leq 0 \) and \( \gamma \geq \alpha \) we can choose \( \mathcal{R}F \) in such a way that the map \( F \mapsto \mathcal{R}F \) is linear on the vector space of \( (\alpha, \gamma) \)-coherent germs with global homogeneity bound \( \beta \).
\end{theorem}

Remarkably, coherence is not only \emph{sufficient}; that is coherence implies the Reconstruction Theorem. Coherence is also necessary. We say that coherence is an \emph{optimal} condition.

\begin{theorem}[Coherence is necessary]\label{theorem:coherence-is-necessary}
   Fix any \(\gamma \in \mathbb{R}\).  Let \({(F_x)}_{x \in \mathbb{R}^d}\) be a germ. Let \(f \in \mathcal{D}'\) be a distribution such that for every compact set \(K \subset \mathbb{R}^d\) there exists \(C < \infty\) and \(r \in \mathbb{N}\) with
   \begin{gather}\label{thm:coh-necessary}
        |(f-F_y)(\psi^\lambda_y)| \leq C \lambda^\gamma \\
        \text{for all \(y \in K, \lambda \in (0,1]\) and \(\psi \in \mathcal{B}_r\).} \nonumber
   \end{gather}
   Then, \({(F_x)}_{x \in \mathbb{R}^d}\) is \(\gamma\)-coherent.  
\end{theorem}

\begin{proof}
   To show that \({(F_x)}_{x \in \mathbb{R}^d}\) is \(\gamma\)-coherent, we prove that there exists \(\alpha \leq \min\left\{ 0, \gamma \right\}\) and a constant \(C < \infty\) such that
   \begin{gather*}
        |(F_z - F_y)(\varphi^\lambda_y)| \leq C\lambda^\alpha{(|z-y| + \lambda)}^{\gamma - \alpha}  \\ \text{uniformly for \(z,y \in K\), \(|y-z| \leq \frac{1}{2}\)  and \(\lambda \in (0,1]\)} \nonumber. % chktex 9
   \end{gather*}  
   Note that we require \(|y-z| \leq \frac{1}{2}\), but this is equivalent to Definition~\ref{definition:coherence}, see Remark~\ref{remark:cutoff}.

   Fix a compact set \(K \subset \mathbb{R}^d\). Let \(x,y \in K\) with \(|x-y| \leq \frac{1}{2}\), \(\lambda \in (0, \frac{1}{2}]\) and \(\psi \in \mathcal{B}_r\). Then,
   \begin{align*}
    |(F_x - F_y)(\psi^\lambda_y)| \leq |(F_x - f)(\psi^\lambda_y)| + |(f - F_y)(\psi^\lambda_y)| \overset{\eqref{thm:coh-necessary}}{\leq} |(F_x - f)(\psi^\lambda_y)| + C \lambda^\gamma.
   \end{align*}
   Next, estimating \(|(F_x - f)(\psi^\lambda_y)|\) is nontrivial because \(\psi_y^\lambda\) is centered around \(y\) and not \(x\). We overcome this obstacle by substituting \(\psi^\lambda_y \leadsto \xi^{\lambda_1}_x\), where 
   \begin{gather*}
       \xi \coloneqq \psi^{\lambda_2}_w, \quad w \coloneqq \frac{y-x}{|x-y| + \lambda}, \\
    \lambda_1 \coloneqq |x-y| + \lambda, \;\text{ and } \; \lambda_2 \coloneqq \frac{\lambda}{|x-y|  + \lambda}.
   \end{gather*}
   We quickly verify the correctness of the substitution
   \begin{align*}
    \xi^{\lambda_1}_x = \frac{\psi\left(
        \lambda_2^{-1}\left(\frac{\cdot - x}{|x-y| + \lambda} - w\right)
    \right) }{\left((|x-y| + \lambda)\frac{\lambda}{|x-y| + \lambda}\right)^d}
    =
    \frac{\psi\left(
        \frac{\cdot - x - (y-x)}{\lambda}
    \right) }{\lambda^d} = \lambda^{-d}\psi\left( \frac{\cdot - y}{\lambda} \right) = \psi^\lambda_{y}.
   \end{align*}
   Hence, 
   \begin{align}
    |(F_x - f)(\psi^\lambda_y)| = |(F_x - f)\left( \xi^{\lambda_1}_x \lVert \xi \rVert_{C^{r}}^{-1} \right)| \cdot  \lVert \xi \rVert_{C^{r}} &\overset{\eqref{thm:coh-necessary}}{\leq} C\lambda_1^\gamma\lVert \xi \rVert_{C^{r}}. \label{SponsoredByGilette}
   \end{align}
   To justify \(\eqref{thm:coh-necessary}\), observe that 
   \begin{itemize}
       \item  \(\lambda_1 \in (0, 1]\), and 
       \item  \(\xi^{\lambda_1}_x \lVert \xi \rVert_{C^{r}}^{-1} \in \mathcal{B}_{r}\) because \(\lambda_2 + |w| = 1\) and \(\mathrm{supp}(\psi) \subset B(0,1)\); both imply that \(\xi = \psi^{\lambda_2}_w\) is supported in \(B(0,1)\), and the scaling factor \(\lVert \xi \rVert_{C^{r}}^{-1}\) ensures that the \(C^r\)-norm is one.
   \end{itemize}   
   Additionally, \(\lVert \xi \rVert_{C^{r}} = \max_{k \leq r}\lVert \partial^k \psi^{\lambda_2}_w \rVert_\infty = \max_{k \leq r} \lambda_2^{-d-k} \lVert \partial^k \psi \rVert_{\infty} \leq \lambda_2^{-d - r}\). So,
   \begin{align*}
    |(F_x - f)(\psi^\lambda_y)| \leq C\lambda_1^\gamma \lambda_2^{-d-r} &= C(|x-y| + \lambda)^\gamma\left(\frac{\lambda}{|x-y|  + \lambda}\right)^{-d-r} \\
    &\leq C  (|x-y| + \lambda)^{\gamma - \alpha} \lambda^\alpha,
   \end{align*}
    where we define \(\alpha = \min\left\{ -d-r , \gamma \right\}\). Then,
    \begin{align*}
        |(F_x - F_y)(\psi^\lambda_y)| &\leq C  (|x-y| + \lambda)^{\gamma - \alpha} \lambda^\alpha + C \lambda^\gamma \\
        &\Downarrow \text{where \(\lambda^\gamma = \lambda^{\gamma - \alpha} \lambda^\alpha \leq (|x-y| + \lambda)^{\gamma - \alpha} \lambda^\alpha\)} \\
        &\leq 2C  (|x-y| + \lambda)^{\gamma - \alpha} \lambda^\alpha.
    \end{align*}
\end{proof}

We slightly modify the previous proof to show that the constraint \(|z-y| \leq 2\) in the coherence condition (see Definition~\ref{definition:coherence}) can be dropped. If~\eqref{coherence} holds uniformly for any \(y,z \in K\) with \(|z-y| \leq 2\), then it also holds for any \(\tilde y, \tilde z \in K\) with \(|\tilde z- \tilde y| > 2\) (possibly with a different multiplicative constant \(C\)). Hence, coherence is equivalent to
\begin{gather}\label{better-coherence}
    |(F_z - F_y)(\varphi^\lambda_y)| \leq C\lambda^\alpha(|z-y| + \lambda)^{\gamma - \alpha}  \\ \text{uniformly for \(z,y \in K\) and \(\lambda \in (0,1]\)} \nonumber.
\end{gather}

\begin{proposition}\label{proposition:cutoff}
    Let \(F\) be a \(\gamma\)-coherent germ as in Definition~\ref{definition:coherence}. Then, it satisfies~\eqref{better-coherence} for any compact set \(K\) provided the multiplicative constant \(C\) is adjusted.
\end{proposition}

\begin{proof}
    Let \(F\) be a \(\gamma\)-coherent germ and \(\varphi\) be as in Definition~\ref{definition:coherence}. Fix a compact set \(K \subset \mathbb{R}^d\). Assume \(y,z \in K\) with \(|y-z| > 2\). Let \(A\) be a finite family of points in \(\mathbb{R}^d\) such that \(K\) is covered by \(A\) and for each point \(x \in K\) there exists \(a_x \in A\) with \(|x-a| < 2\). Such \(A\) exists because \(K\) is compact. Then, we have \(|(F_z - F_y)(\varphi^\lambda_z)| \leq |(F_z - F_{a_z})(\varphi^\lambda_z)| + |(F_{a_z} - F_y)(\varphi^\lambda_z)|\). The first summand is bounded by~\eqref{coherence}. Bounding the second summand is nontrivial since \(\varphi\) is centered around \(z\) and not \(a_z\) or \(y\). This is the same situation as in the proof of Theorem~\ref{theorem:coherence-is-necessary}. So, we write
    \begin{align*}
        |(F_{a_z} - F_y)(\varphi^\lambda_z)| \leq |(F_{a_z} - f)(\varphi^\lambda_z)| + |(f - F_y)(\varphi^\lambda_z)|,
    \end{align*}
    where \(f\) is the reconstruction of the germ \(F\); note that \(f\) exists by the Reconstruction Theorem and the Reconstruction Theorem only requires \(F\) to be coherent in the sense of Definition~\ref{definition:coherence}. Next, we use the same substitution as in~\eqref{SponsoredByGilette} to obtain an upper bound for both summands. These upper bounds only depend on \(|a_z - z|\), \(|y - z|\) and \(\lambda\), which ends the proof.  
\end{proof}

Next, we show uniqueness of the reconstruction \( \mathcal{R}F \). 

\begin{theorem}[Uniqueness]\label{theorem:uniqueness-reconstruction}
   Let \(F = {(F_x)}_{x \in \mathbb{R}^d}\) be a germ and \(\varphi \in \mathcal{D}\) be a test function with \(\int \varphi(x) \mathrm{d}x \neq 0\). Let \(K \subset \mathbb{R}^d\) be a compact set, and let \(f, g \in \mathcal{D}'\) be any two distributions such that
   \begin{align*}
       \lim_{\lambda \to 0} |(f-F_x)(\varphi^\lambda_x)| &= 0 \quad \text{uniformly for \(x\) in \(K\)} \\ 
       \quad \lim_{\lambda \to 0} |(g-F_x)(\varphi^\lambda_x)| &= 0 \quad \text{uniformly for \(x\) in \(K\)}
   \end{align*} 
    Then, \(f(\psi) = g(\psi)\) for all test functions \(\psi \in \mathcal{D}(K)\).  
\end{theorem}

\begin{proof}
    Define \(F, \varphi, K, f\) and \(g\) as in the theorem. Next, we define \(T \coloneqq f - g\), fix \(\psi \in \mathcal{D}(K)\) and show \(T(\psi) = 0\).
   
    We assume that \(\int \varphi(x) \mathrm{d} x = 1\) (otherwise we replace \(\varphi\) by \((\int \varphi(x) \mathrm{d}x)^{-1}\varphi\)). Then, the family \((\varphi^\lambda)_{\lambda \in (0,1]}\) is a mollifier, and thus \(T(\psi) = \lim_{\lambda \to 0}T(\psi * \varphi^\lambda)\). This allows us to estimate 
    \begin{align*}
        |T(\psi * \varphi^\lambda)| = \left|\int T(\varphi^\lambda_x) \psi(x) \mathrm{d}x\right| \leq \lVert \psi \rVert_{L^1} \, \sup_{x \in K}|T(\varphi^\lambda_x)|,
    \end{align*}
    where for the last inequality we recall that \(\psi\) has compact support in \(K\). Using the triangle inequality, we bound 
    \begin{align*}
        |T(\varphi^\lambda_x)| = |(f-g)(\varphi^\lambda_x)| \leq |(f-F_x)(\varphi^\lambda_x)| + |(g-F_x)(\varphi^\lambda_x)|.
    \end{align*}
    Taking the limit \(\lambda \to 0\) proves the uniqueness. 
\end{proof}
